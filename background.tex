\chapter{Background and Related Work}
\label{bg}
%%%%%%%%%%%%%%%%%%%%%
\section{Definition of key terms}
\subsection{Self-regulation}
\label{bg:def:self-reg}
Animals and flying beings, that live on or above earth, form social communities similar to human societies \cite{SIHQ1995}. In recent years, the study of biological social insects and other animals reveals us that simple individuals of these self-organized  societies can solve various complex and large everyday-problems with a few behavioural rules, relying on their minimum sensing and communication abilities \cite{Camazine+2001}. Some common tasks of these biological societies include: dynamic foraging, building amazing nest structures, maintaining division of labour among workers \cite{Bonabeau+1999}. These tasks are done by colonies  ranging from a few animals to thousands or millions of individuals. Despite their huge colony size, they easily achieve surprising efficiency in those tasks with many common features, e.g. robustness, flexibility, synergy (for an example in ants, see Fig. \ref{fig:self-org-ants}).

Today, these findings have inspired scientists and engineers to use this knowledge of biological self-organization in developing solutions for various problems of artificial systems, such as  routing traffics in telecommunication and vehicle networks, designing control algorithms for large groups of autonomous robots, automating industrial shop-floor tasks and so forth \cite{Garnier+2007}.
%%----- FIG fig:weaver-green-tree-ants---------
\begin{figure}[htp]
  \centering
  \subfloat[Weaver ants]{\label{fig:weaver-ants}\includegraphics[width=6cm, height=4cm]{./photos/ants-hy27c.eps}}                
  \hspace{0.25cm}
  \subfloat[Green-tree ants]{\label{fig:green-tree-ants}\includegraphics[width=6cm, height=4cm]{./photos/ants_spacecollective.org_greenTreeAnt.eps}}
  \caption{(a) During nest construction, weaver ants combine two leaves by pulling them from two sides \protect\citeasnoun{Yahya2000}.
(b) Green-tree ants retrieve a large-prey. http://www.spacecollective.org, last seen on 01/05/2010.}
%
  \label{fig:self-org-ants}
\end{figure}
% FIG: self-org
\begin{figure}[htp]
\centering
\includegraphics[height=8cm, angle=0]
{./images/dia-files/self-org-1}
\caption{Self-organization viewed from four (A-D) inseparable perspectives. Adopted from \protect\citeasnoun{Camazine+2001}.}
\label{fig:self-org-1} % Give a unique label
\vspace*{0.25cm}
\centering
\includegraphics[height=5cm, angle=0]{./images/dia-files/self-org-agent}
\caption{ Three major interfaces of a self-regulated agent.}
\label{fig:self-org-agent} % Give a unique label
\end{figure}

Self-organization, in biological and  other systems, is often characterized in terms of four major ingredients: 1) positive feedback, 2) negative feedback, 3) presence of multiple interactions among individuals and their environment, and 4) amplification of fluctuations  e.g., random walks, errors, random task-switching \cite{Camazine+2001}. An external observer, as if looking through transparent glasses, can recognize a self-organized system by observing the individual interactions of that system from four interlinked perspectives (Fig. \ref{fig:self-org-1}). 

The first perspective is the {\em  positive feedback} or amplification that can be resulted from the execution of simple behavioural ``rules of thumb''. For example, recruitment to a food source through trail laying and trail following in some ants  is due to the positive feedback that attract other ants to follow the trail and to lay more pheromones over time. The second perspective is the {\em negative feedback} that counterbalances positive feedback. This usually occurs to stabilize collective patterns, e.g., crowding at the food sources (saturation), competition between paths to food sources etc. The third perspective is the {\em presence of multiple  interactions} that can be direct \acf{P2P}, broadcast or indirect {\em stigmergic}, e.g. ants' pheromone laying. Finally, the fourth  perspective is the {\em amplification of fluctuations} that comes from various stochastic events. For example, errors in trail following of some ants may lead some foragers to get lost and later on, to find new and unexploited food sources, and recruit other ants to those sources.

%% FIG: fig:honey-bee-nest-comb
\begin{figure}
\centering
\subfloat[Honey-bee nest]{\includegraphics[width=6cm, height=4cm]{./photos/honey-bee-nest-hy103.eps}} 
\hspace{0.25cm}
\subfloat[Construction of honey-combs]{\includegraphics[width=6cm, height=4cm]{./photos/honey-bee-comb-building-knol-google.eps}}
\caption{(a) A Honey-bee colony has built a nest on a tree-branch. From http://www.harunyahya.com, last seen on 01/05/2010. (b) Honey-bees are constructing honey combs. From http://knol.google.com, last seen on 01/05/2010.}
\label{fig:honey-bee-nest}
\end{figure}
% 
In a self-organized system, an individual agent may have limited cognitive, sensing and communication capabilities. But they are collectively capable of solving complex and large problems, e.g. coordinated nest construction of honey-bees (Fig. \ref{fig:honey-bee-nest}), collective defence of school fishes from a predator attack (Fig. \ref{fig:school-of-fish}), ordered homing of bats (Fig. \ref{fig:bats-colony}).  Since the discovery of these collective behavioural patterns of self-organized societies, scientists observed modulation or adaptation of behaviours in the individual level \cite{Garnier+2007}. For example, in order to prevent a life-threatening humidity-drop in the colony, cockroaches maintain a locally sustainable humidity level by increasing their tendency to aggregate, i.e. by regulating their individual aggregation behaviours. 

As shown in Fig. \ref{fig:self-org-agent}, the  self-regulation of an individual agent is depicted through a triangle where its base-arm represents the simple behavioural rules of thumb (in this case, intense aggregation of cockroaches in low humidity). This is supported by two side-arms: local communication and local sensing. This local sensing is sometimes also referred to as sensing or information gathering from the work in progress, e.g.stigmergy and this local communication is an instance of direct communication with neighbours  \cite{Camazine+2001}.

%% FIG: School of Fish
\begin{figure}[htp]
\centering
\subfloat[A moving school of fish]{\includegraphics[width=6cm, height=4cm]{./photos/adventureaquariumschooloffish.eps}} 
\hspace{0.25cm}
\subfloat[A shark attacked a school of fish]{\includegraphics[width=6cm, height=4cm]{./photos/schoo_of_fish_shark-bigtiger.eps}}
\caption{(a) Group cohesion of a school of fish. (b) When a shark attacks a school of fish, they misguide the attacker by swift random movements. From http://www.travelblog.org last seen on 01/05/2010.}
\label{fig:school-of-fish}
\end{figure}
%%
Self-regulation has been studied in many other branches of knowledge. In most places of literature, self-regulation refers to the exercise of control over oneself to bring the self into line with preferred standards \cite{Baumeister+2007}. One of the most notable self-regulatory process is the human body's homoeostatic process where the human body's inner process seeks to return to its regular temperature when it gets overheated or chilled. \citeasnoun{Baumeister+2007} has referred self-regulation to goal-directed behaviour or feedback loops, whereas self-control may be associated with conscious impulse control.

In psychology, self-regulation denotes the strenuous actions to resist temptation or to overcome anxiety. self-regulation is also divided into two categories: 1) conscious and 2) unconscious self-regulation. Conscious self-regulation puts emphasis on conscious and deliberate efforts in self-regulation. On the other hand, unconscious self-regulation refers to the automatic self-regulatory process that is not labour intensive but operate in harmony with unpredictable and unfolding events in the environment. This process uses the available informational input in ways that help to attain an activated goal.
%% FIG: Bats navigation
\begin{figure}[htp]
\centering
\subfloat[A bat colony]{\includegraphics[width=6cm, height=4cm]{./photos/bats_hy2.eps}} 
\hspace{0.25cm}
\subfloat[Navigation of bats]{\includegraphics[width=6cm, height=4cm]{./photos/bats_hy3.eps}}
\caption{(a) One of the largest bat colony with about 50 million bats, (b) These bats can show amazing navigation abilities: they always fly back to their nest on a straight route from wherever they are, reproduced from \protect\citeasnoun{Tuttle1995}.}
\label{fig:bats-colony}
\end{figure}

The concepts of self-regulation is also common in cybernetic theory where self-regulation in inanimate mechanisms shows that they can regulate themselves by making adjustments according to pre-programmed goals or set standards. A common example of this kind can be found in a thermostat that controls a heating and cooling system to maintain a desired temperature in a room. In physics, chemistry, biology and some other branches of natural sciences, the concept of self-regulation is centred around the study of self-organizing individuals. 

Self-regulation has also been studied in the context of human social systems. Here it is originated from the division of social labour that creates self-organized processes with self-regulating effects \cite{Kppers+1990}. Two types of self-regulation have been reported in many places of sociology literature: 1) self-regulation from self-organization and 2) self-regulation from activities of components in a heterarchical organization. It is interesting to note that self-regulation in biological species provides the similar evidences of bottom-up approach of self-regulation of heterarchical organization through the interaction of individuals in the absence, or in parallel, of strict hierarchy \cite{Beer1981}.

From the above discussion, we see that the term {\em self-regulation} carries a wide range of meanings in different branches of knowledge. In psychology and cognitive neuroscience, self-regulation is discussed solely within an individual's perspective whereas, in biology and social sciences, self-regulation is discussed within the context of a group of individuals or society as a whole. In  this thesis, the latter context is more appropriate where  self-regulation focuses on monitoring an individual's own state and environmental changes, in relation to the communal goal and, in turn, this leads to adjust the behaviours of that individual with respect to the changes found.
%%%%======================================================================
\subsection{Communication} 
\label{bg:def:comm}
%%FIG:  Comm defined
\begin{figure}
\centering
\includegraphics[width=8.5cm, angle=0]
{./images/dia-files/comm-defined.eps}
\caption{ General models of communication, adopted from \protect\citeasnoun{West+2003}.}
\label{fig:gen-comm-defined} % Give a unique label
\end{figure}
%%
\textbf{What is Communication?} Defining {\em communication} can be challenging due to the use of this term in several disciplines with somewhat different meanings. This has been potrayed in the writing of Sarah Trenholm \cite{West+2003} who describes communication as a piece of luggage overstuffed with all manners of odd ideas and meanings. \citeasnoun{West+2003} defines communication  as:
\begin{quote}
\ssp
``A social process where individuals employ symbols to establish and interpret meaning in their environment.''
\end{quote}
\sdp
 The notion of being a ``social process'' involves, at least two or more individuals engaged in dynamic and ongoing interactions. Moreover, symbols can be simply some sort of arbitrary labels given to a phenomena and they can represent  concrete objects or an abstract ideas. Encyclopaedia Britannica also defines communication as ``the exchange of meanings between individuals through a common system of symbol''. But since this definition lacks the notion of sociality we find this definition incomplete.In our view, communication process takes place within the context of  symbol or message exchange between two or more parties with a clear intent to influence each others' behaviours.

According to a biological model of communication (Fig. \ref{fig:bio-comm-defined}), communication is a biological process where an  individual (sender) intentionally transmits encoded message though a physical signal and that, on being received and decoded by another individual of same species (receiver), influences receiver's behaviour \cite{Frings1997}. Note that, here individuals are of same species and thus they have a  shared message vocabulary and mechanism of message encoding/decoding. Although  this definition has not included the dynamics of a communication process, it is more precise for low-level biological and artificial systems. It accounts for the behavioural changes during communication process. These changes can be tracked though observing states of individuals.
   
%% FIG bio-comm defined
\begin{figure}
\centering
\includegraphics[width=8cm,height=4cm, angle=0]{./images/dia-files/animal-comm-defined.eps}
\caption{ A biological model of communication, adopted from \protect\citeasnoun{Frings1997}.}
\label{fig:bio-comm-defined} 
\end{figure}
%%
\textbf{Models of Communication.} If we explore the elements of communication, we can grab the whole picture involved in an individual's communication and sensing process. This can be explained through the study of the models  of communication. There exists a plenty of models of communication. For this thesis, here we briefly discuss three prominent models: 1) linear model, 2) interaction model and 3) transaction model.

Fig. \ref{fig:gen-comm-defined} embeds first two models inside a single circle. The linear model is shown by dashed lines. This model was introduced by Claude Shanon and Warren Weaver in 1949. Here communication is a one way process where a {\em message} is sent from a {\em source} to a {\em receiver} through a {\em channel}.

In Fig. \ref{fig:gen-comm-defined}, around the linear model, interaction model has been drawn. This model, proposed by Wilber Schramm in 1954, views communication as a two-way process that uses an additional {\em feedback} element  linking both source and receiver. This feedback is a response given to the source by the receiver to confirm how the message is being understood. Here, during message passing, both source and receiver utilize their individual {\em field of experiences} that describe the overlap of their common experiences, cultures etc. 

Unlike separate filed of experiences and discrete sending and receiving of messages, in transactional model, introduced by Barnlund in 1970, the sending and receiving of messages are done simultaneously. Here, the  field of experiences of source and receiver can overlap to some degrees. In all of the above three models, {\em noise} or a common message distorting element is present in the communication process. This noise can be occurred from the linguistic influences, i.e. message semantics,  physical or bodily influences, cognitive influences or even from biological
or physiological influences e.g., anger or shouting voice while talking.

The above models of communication describe the incremental complexities of message exchanging in the communication process. Surely, the transactional model is the most elegant model that prescribes adjusting the sender's message content while receiving an implicit or explicit feedback in real-time. For example, while advising her son to read a story book, a mother may alter her verbal message as she simultaneously ``reads'' the non-verbal message from her child's face. 

However, in case of a multi-robot system, such sophistication in communication may not be required or realizable by the current state-of-the-art in multi-robot communication technology. In this study, we follow the simple linear model that can meet the necessary communication requirements of a multi-robot system. The feedback has not been considered as we have assumed that all robots of our artificial multi-robot system have a common shared vocabulary such that a message is understood as it is sent.

%% TABLE: COMM-CATEGORIES
\begin{table}
\caption{General characteristics of common communication modes}
\label{table:comm-categories}
\begin{center}
\begin{tabular}{|l|l|l|}
\hline \textbf{Type} & \textbf{Indirect or} & \textbf{Direct or }\\
& \textbf{implicit communication} & \textbf{explicit communication}\\
\hline Centralized & Typically a central entity   & Both global \& local broadcast  \\
Communication & modifies the environment. & communication are commonly\\
 & It facilitates passive forms  &  used. P2P  communication can \\
  &  of communications, i.e.     & also occur. Here, exchange\\
&  communication without   &    of messages  occurs through a\\
& specific target recipient. &  central entity.\\
\hline Decentralized & All individuals are free to & P2P and local-broadcast \\
or Local & modify the environment &  are  most common. \\
Communication & and convey information &  Global broadcast occurs \\
 & to others. & to handle an emergency. \\
 & & All communications are local\\
& & without having a central entity.\\
\hline
\end{tabular}
\end{center}
\end{table}
%%
%% [COMM classification]
\begin{figure}
\centering
\includegraphics[width=10cm, angle=0]
{./dia-files/bio-comm-strategies.eps}
\caption{Common communication strategies observed in social systems.}
\label{fig:comm-strategies} 
\end{figure}
%%
\textbf{Communication modes and strategies.} The communication  structure of a system can broadly be classified into two major categories: centralized communication and decentralized or local communication. A centralized communication system generally has a central entity, e.g. a gateway,  that routes all incoming and outgoing communications. Individual nodes of this system often do not communicate each other directly. But they can send and receive messages through this central gateway.  This central entity can play other roles, such as access control, resource allocation and so on. On the other hand, in local communication, there is no central entity and each node can independently route messages to others.

%%
\begin{figure}
\centering
\includegraphics[width=10cm, angle=0]
{./dia-files/bio-comm-strategies-peers.eps}
%figure caption is below the figure
\caption{Number of recipients involved in various communication strategies.}
\label{fig:comm-strategies-peers}  % Give a unique label
\end{figure}
%%
In both biological and robotic literature two basic types of communication are often discussed: 1) direct or explicit communication and 2) indirect or implicit communication. {\em Direct communication} is an intentional communicative act of message passing that aims at one or more particular receiver(s) \cite{Mataric1998}. It typically exchanges information through physical signals. 

In contrast, {\em indirect communication}, sometimes termed as {\em stigmergy} in biological literature, happens as a form of modifying the environment, e.g. pheromone dropping by ants \cite{Bonabeau+1999}. In an ordinary sense, this is an observed behaviour and many robotic researchers call it as {\em no communication} \cite{Labella2007}. In order to avoid ambiguity, in this dissertation, by the term {\em communication}, we always refer to the {\em direct} communication.

Direct or explicit communication can be limited by a communication range and thus by a number of target recipients. Under both centralized and local communication, nodes can select a certain number of target recipients of their messages. This process specifies {\em to whom} a node intends to communicate. In this thesis, we have denoted this mechanism of target recipient(s) selection as {\em communication strategy}.  

Fig. \ref{fig:comm-strategies} shows the most common communication strategies found in a  social system.  In the simplest case, when only two nodes can communicate we call this P2P communication. When nodes can spread information to a limited number of peers of their locality, the communication takes the form of {\em local broadcast}, i.e. one sender and a few receivers within a certain locality. For example, when a foraging honey-bee gives the information of flower sources to a number of peers through various dances, it conveys this information to a few peers through a local broadcast. However, giving the sample of nectar through tactile or taste to its peers can be considered as an instance of P2P communication. The {\em global broadcast} strategy can be found in almost all social species to handle emergency situations, e.g. emitting alarm signal in danger.

Table \ref{table:comm-categories}  shows the relationship between various communication modes and their ways of adopting different strategies. Fig. \ref {fig:comm-strategies-peers} shows a typical count of average number of peers in various communication strategies. The actual number of peers under local broadcast strategy depends on a particular social system and it changes over time in different levels of interactions among individuals. Sec. \ref{bg:bio-comm} and \ref{bg:mrs-comm} reviews communication in biological social system  and multi-robot system respectively.
%%-------------------------------------------------------------------
\subsection{Division of labour or task-allocation}
\label{bg:def:dol}
Encyclopaedia Britannica serves the definition of {\em division of labour} as the ``separation of a work process into a number of tasks, where each task is performed by a separate person or a group of persons''. Originated from economics and sociology, the term division of labour is widely used in many branches of knowledge. As mentioned by the Scottish philosopher Adam Smith, the founder of modern economics:
\begin{quote}
\ssp 
The great increase of the quantity of work, which in consequence of the division of labour the same number of people are capable of performing, is owing to three different circumstances; firstly, to increase the dexterity in every particular workman; secondly, to the saving of the time which is commonly lost in passing from one species of work to another; and lastly, to the invention of a great number of machines which facilitate and abridge labour, and enable one man to do the work of many.
(Adam Smith (1776) in \citeasnoun{Sendova-Franks+1999})
\end{quote} 
\sdp
In sociology, DOL usually denotes the work specialization \cite{Sayer+1992}. Basically, it answers three major questions:
\begin{enumerate}
\item {\em What task?} This gives us the description of the tasks to be done, service to be rendered or products to be manufactured.
\item {\em Why dividing it to individuals?} This states the underlying social standards for this division, such as task appropriateness based on class, gender, age, skill etc.
\item {\em How to divide it?} This relates to the method or process of separating the whole task into small pieces of tasks that can be performed easily. 
\end{enumerate}
%% DoL: termites, skyscrapper
%% FIG: Termite nest
\begin{figure}[htp]
\centering
\subfloat[A termite nest]{\includegraphics[width=6cm, height=4cm]{./photos/termites_nest.eps}} 
\hspace{0.25cm}
\subfloat[Two Skyscrapers]{\includegraphics[width=6cm, height=4cm]{./photos/skyscraper.eps}}
\caption{(a) A termite colony constructs their nest through bottom-up approach, i.e. without a central planner. (b) Humans construct skyscrapers using a top-down plan. From http://www.harunyahya.com, last seen on 01/05/2010.}
\label{fig:termite-nest}
\end{figure}
%%
From the study of social insects and other biological societies, we can find that two major metrics of DOL have been established in literature: 1) task-specialization and 2) plasticity. {\em Task-specialization} is an integral part of DOL where a worker usually does not perform all tasks, but rather specializes in a set of tasks, according to its morphology, age, or chance \cite{Bonabeau+1999}. This DOL among nest-mates, whereby different activities are performed simultaneously by groups of specialized individuals, is believed to be more efficient than if tasks are performed sequentially by unspecialised individuals.

DOL is also characterized by {\em plasticity} which means that the removal of one class of workers is quickly compensated for by other workers. Thus distributions of workers among different concurrent tasks keep changing according to the external (environmental) and internal conditions of a colony \cite{Garnier+2007}.

In artificial social systems, like multi-agent or multi-robot system, the term ``division of labour'' is often found synonymous to ``task-allocation'' \cite{Shen+2001}. However, some researchers  argued to distinguish these terms due to the origin and particular contextual use of these terms \cite{Labella2007}. Particularly, DOL adopts the biological notion of collective task performance with little or no communication. On the other hand, task allocation follows the meaning of assigning task(s) to particular robot(s) based on individual robot capabilities, typically through explicit communication, such as {\em intentional cooperation} \cite{Parker1998}. Generally, the former is considered by swarm robotic system and latter is done under {\em traditional multi-robot system}. Sec. \ref{bg:mrs:overview} covers both of these approaches and Sec. \ref{bg:mrta} provides critical reviews on DOL under these approaches.

In this dissertation, to define DOL, we have followed the biological approach that emphasizes on having task-specialization and plasticity among workers. However, we do not put any restriction on the use of communication. In fact, we view DOL as a group-level phenomenon which occurs due to the individual agent's self-regulatory task-allocation behaviour. Our model of self-regulated DOL provides us necessary abstractions to the communication and sensing processes of individuals that are discussed in detail in Chapter \ref{afm} and \ref{local-comm}. 
%------------------------------------------------------------------
%%%%%%%%%%%%%%%%%%%%%
\section{Communication in biological social systems}
\label{bg:bio-comm}
Communication plays a central role in self-regulated DOL of biological social systems.In this section, communication among  social insects are briefly reviewed.
%
\subsection{Purposes, modalities and ranges}
Communication in biological societies serves many closely related social purposes. Most P2P communication include: recruitment to a new food source or nest site, exchange of food particles, recognition of individuals, simple attraction, grooming, sexual communication etc. In addition to that colony-level broadcast communication include: alarm signal, territorial and home range signals and nest markers, communication for achieving certain group effect such as, facilitating or inhibiting  a group activity \cite{Holldobler1990}.
\begin{table}
\caption{Common communication modalities in biological social systems}
\label{table:bio-comm-modalities}
\begin{center}
\begin{threeparttable}
\begin{tabular}{|l|l|l|}
\hline \textbf{Modality} & \textbf{Range} & \textbf{Information type}\\
\hline Sound & Long\tnote{a} & Advertising about food  source,  danger etc. \\                                                                                                                                               
\hline Vision & Short\tnote{b}  & Private, e.g. courtship display. \\
\hline Chemical  & Short/long & Various messages, e.g. food location, alarm etc.\\
\hline Tactile & Short & Qualitative info, e.g. quality of flower,\\ & & peer identification etc.\\
\hline Electric & Short/long & Mostly advertising types, e.g. aggression messages.\\
\hline
\end{tabular}
\begin{tablenotes}
\item [a]Depending on the type of species, long range signals can reach from a few metres to several kilometres.
\item [b]Short range typically covers from few mm to about a metre or so.
\end{tablenotes}
\end{threeparttable}
\end{center}
\end{table}

%[Modalities and Ranges]
Biological social insects use different modalities to establish social communication, such as, sound, vision, chemical, tactile,  electric and so forth (Table \ref{table:bio-comm-modalities}).  Sound waves can travel a long distance and thus they are suitable for advertising signals. They are also best for transmitting complicated information quickly \cite{Slater1986}. Visual signals can travel more rapidly than sound, but they are limited by the physical size or line of sight of an animal. They also do not travel around obstacles. Thus they are suitable for short-distance private signals.

%% FIG. Fireflies
\begin{figure}[htp]
\centering
\subfloat[Flashing fireflies]{\includegraphics[width=6cm, height=4cm]{./photos/fire-flies.eps}} 
\hspace{0.25cm}
\subfloat[A firefly emitting light]{\includegraphics[width=6cm, height=4cm]{./photos/firefly-light-under.eps}}
\caption{(a) Flashing lights of fireflies displaying their synchronous behaviours (b) A firefly can produce light to signal other fireflies.\protect\newline From http://www.letsjapan.markmode.com, last seen on 01/06/2010.}
\label{fig:fireflies}
\end{figure}
%
In ants and some other social insects, chemical communication is predominant. Any kind of chemical substance that is used for communication between intra-species or inter-species is termed as {\em semiochemical} \cite{Holldobler1990}. A pheromone is a semiochemical, usually a type of glandular secretion, used for communication within species. One individual releases it as a signal and others respond to it after tasting or smelling. Using pheromones individuals can code quite complicated messages in smells. For example, a typical an ant colony operates with somewhere between 10 and 20 kinds of signals. Most of these are chemical in nature. If wind and other conditions are favourable,  this type of signals emitted by such a tiny species can be detected from several kilometres away. Thus chemical signals are extremely economical of their production and transmission. But they are quite slow to diffuse away. But ants and other social insects manage to create sequential and compound messages either by a graded reaction of different concentrations of same substance or by blends of signals.

Tactile communication is also widely observed in ants and other species typically by using their body antennae and forelegs. It is observed that in ants touch is primarily used  for receiving information rather than informing something. It is usually found as an invitation behaviour in worker recruitment process. When an ant intends to recruit a nest-mate for foraging or other tasks it runs upto a nest-mate and beats her body very lightly with  antennae and forelegs. The recruiter then runs to a recently laid pheromone trail or lays a new one. In this form of communication limited amount of information is exchanged. In underwater environment some fishes and other species also communicate through electric signals where their nerves and muscles work as batteries. They use continuous or intermittent pulses with  different frequencies to learn about environment and to convey their identity and aggression messages.
%%
%%%%%%%%%%%%%%%%%%%%%%%%%%%%%%%%%%%%%%%%%%
\subsection{Signal active space and locality}
%bio-comm-ants-active-space
\begin{figure}
\centering
\includegraphics[width=12cm, angle=0]
{./dia-files/bio-comm-ants-active-space.eps}
%figure caption is below the figure
\caption{Pheromone active space observed in ants, reproduced from \protect\citeasnoun{Holldobler1990}.}
\label{fig:ants-active-space} % Give a unique label
\end{figure}
The concept of active space (AS) is widely used to describe the propagation of signals by species. In a network environment of signal emitters and receivers, active space is defined as the area encompassed by the signal during the course of transmission \cite{Mcgregor2000}. In case of long-range signals, or even in case of short-range signals, this area include several individuals where their social grouping allows them to stay in cohesion. The concept of active space is described somewhat differently in case some social insects. In case of ants, this active space is defined as a zone within which the concentration of pheromone (or any other behaviourally active chemical substances) is at or above threshold concentration \cite{Holldobler1990}. Mathematically this is denoted by a ratio:
\begin{equation}
AS = \frac{\textit{Amount of pheromone emitted (Q)}}{\textit{Threshold concentration at which the receiving ant responds (K)}}
\end{equation}
Here, Q is measured in number of molecules released in a burst or in per unit of time whereas K is measured in molecules per unit of volume. 
Fig. \ref{fig:ants-active-space} shows the use of active spaces of two species of ants: (a) {\em Atta texana} and (b) {\em Myrmicaria eumenoides}.  The former one uses two different concentrations of {\em 4-methyl-3-heptanone} to create attraction and alarm signals, whereas the latter one uses two different chemicals: {\em Beta-pinene} and {\em Limonene} to create similar kinds signals, i.e. alerting and circling.
 
The adjustment of this ratio enables individuals to gain a shorter fade-out time and permits signals to be more sharply pinpointed in time and space by the receivers. In order to transmit the location of the animal in the signal, the rate of information transfer can be increased by either lowering the rate of emission of Q or by increasing K, or both. For alarm and trail systems a lower value of this ratio is used. Thus, according to need, individuals regulate their active space by making it large or small, or by reaching their maximum radius quickly or slowly, or by enduring briefly or for a long period of time. For example, in case of alarm, recruitment and sexual communication signals where encoding the location of an individual is needed, the information in each signal increases as the logarithm of the square of distance over which the signal travels. From the precise study of pheromones it has been found that active space of alarm signal is consists of a concentric pair of hemispheres (Fig. \ref{fig:ants-active-space}). As an ant enters the outer zone, she is attracted inward toward the point source; when she next crosses into the central hemisphere she become alarmed. It is also observed that ants can release pheromones with different active spaces.

% indirect & b/c comm
\begin{figure}
\begin{minipage}[t]{0.48\linewidth}
\centering
\includegraphics[width=6cm, height=4cm, angle=0]
{./photos/ants_group_comm_bioteams_com.eps}
\caption{A group of ants following pheromone-trail. \protect\newline From http://www.bioteams.com, last seen on 01/06/2010.}
\label{fig:ant-indirect} % Give a unique label
\end{minipage}
\hspace{0.5cm}
\begin{minipage}[t]{0.48\linewidth}
\centering
\includegraphics[width=6cm,height=4cm, angle=0]{./photos/honey-bee-waggle-dance-knol-google.eps}
\caption{ A dancing honey-bee (\protect{\em centre}) and its followers. \protect\newline From http://knol.google.com, last seen on 01/06/2010.}
\label{fig:honey-bee-local-bc} % Give a unique label
\end{minipage}
\end{figure}
Active space has strong role in modulating the behaviours of ants. For example, when workers of {\em Acanthomyops claviger} ants produce alarm signal due to an attack by a rival or insect predator, workers sitting a few millimetres away begin to react within seconds. However, those ants sitting a few centimetres away take a minute or longer to react. In many cases, ants and other social insects exhibit modulatory communication within their active space where many individuals involve in many different tasks. For example, while retrieving the large prey, workers of {\em Aphaeonogerter} ants produce chirping sounds (known as \textit{stridulate}) along with releasing poison gland pheromones. These sounds attract more workers and keep them within the vicinity of the dead prey to protect it from their competitors. This communication amplification behaviour can increase the active space to a maximum distance of 2 meters.
\begin{table}
\caption{Common communication strategies in biological social systems}
\label{table:bio-comm-strategy}
\begin{center}
\begin{tabular}{|l|l|}
\hline 
\textbf{Communication strategy} & \textbf{Common modalities used}\\
\hline 
Indirect & Chemical and electric \\
%\hline 
P2P &  Vision and tactile\\
%\hline 
Local broadcast &  Sound, chemical and vision\\
%\hline 
Global broadcast & Sound, chemical and electric\\
\hline
\end{tabular}
\end{center}
\end{table}
%%%%%%%%%%%%%%%%%%%%%
\subsection{Common communication strategies}
\label{bg:bio-comm:strategies}
%%%
% FIG. Honey bee dance language
\begin{figure}
\centering
\subfloat[Honey-bee's waggle dance]{\includegraphics[width=6cm, height=4cm]{./photos/honey-bee-round-dance.eps}} 
\hspace{0.25cm}
\subfloat[Honey-bee's round dance]{\includegraphics[width=6cm, height=4cm]{./photos/honey-bee-waggle-dance.eps}}
\caption{Examples of local broadcast communication of honey-bees: (a) Honey-bees show waggle-dance (making figure of 8) when food is far and (b) they show round-dance without any waggle when food is closer (within about 75m of hive). From \protect\citeasnoun{Slater1986}.}
\label{fig:honey-bee-dances}
\end{figure}
%%
In biological social systems, we can find all different sorts of communication strategies ranging from indirect pheromone trail laying to local and global broadcast of various signals. Sec. \ref{bg:def:comm} discusses the most common four communication strategies in natural and artificial world, i.e. indirect, P2P, local and global broadcast communication strategies. Table \ref{table:bio-comm-strategy} lists the use of various communication modalities under different communication strategies. Here we give a few real examples of those strategies from biological social systems. In biological literature, the pheromone trail laying is one of the most discussed indirect communication strategy among various species of ants. Fig. \ref{fig:ant-indirect} shows a pheromone trail following of a group of foraging ants. This indirect communication strategy effectively helps ants to find a better food source among multiple sources, find shorter distance to a food source, marking nest site and move there etc. \cite{Hughes2008}. Direct P2P communication strategy is also very common among most of the biological species. Fig. \ref{fig:bees-ants-p2p-comm} shows P2P communication of ants and honey-bees. This tactile form of communication is very effective to exchange food item, flower nectar with each-other or this can be useful even in recruiting nest-mates to a new food source or nest-site.
%%%%%%%%%%%%%%%%%%%%%%
\begin{figure}
\centering
\subfloat[Two honey-bees]{\includegraphics[width=6cm, height=4cm]{./photos/honey-bee-p2p-hy23.eps}} 
\hspace{0.25cm}
\subfloat[Two ants ]{\includegraphics[width=6cm, height=4cm]{./photos/ants-p2p-hy14.eps}}
\caption{Example of P2P tactile communication: (a) Honey-bees exchange nectar samples by close contact (b) ants also exchange food or information via tactile communication. \protect\newline  From http://www.harunyahya.com/ last seen 01/05/2010.}
\label{fig:bees-ants-p2p-comm}
\end{figure}
%%
%%-----------------------------------------------------
\subsection{Roles of communication in task-allocation}
\label{bg:bio-comm:comm-role}
Communication among nest-mates and sensing of tasks are the integral parts of the self-regulated DOL process in biological social systems. They create necessary  preconditions for switching from one tasks to another or to attend dynamic urgent tasks. Suitable communication strategies favour individuals to select a better tasks. For example, \citeasnoun{Garnier+2007} reported two worker-recruitment experiments on black garden ants and honey-bees. The scout ants of {\em Lasius niger}  recruit uninformed ants to food source using a well-laid pheromone trails. {\em Apis mellifera} honey-bees also recruit nest-mates to newly discovered distant flower sources through waggle-dances. In the experiments,  poor food sources were given first to both ants and honey-bees. After some time,  rich food sources were introduced  to them. It was found that only honey-bees were able to switch from poor source to a rich source using their sophisticated dance communication.

%%
\begin{table}
\caption{Self-regulation of communication behaviours in biological social systems}
\label{table:bio-comm-task-urgency}
\begin{center}
\begin{tabular}{|l|l|l|}
\hline \textbf{Example event} & \textbf{Strategy} & \textbf{Modulation of communication}\\
&  &  \textbf{upon sensing tasks}\\
\hline Ant's alarm signal &  Global  & High concentration of pheromones\\
by pheromones & broadcast &  increase aggressive alarm-behaviours \\                                                                                                                                               
\hline Honey-bee's  & Local  &  High quality of nectar source increases \\
round dance & broadcast & dancing and foraging bees\\
\hline Ant's tandem run     & P2P & High quality of nest \\
for nest selection & &   increases traffic flow\\
\hline Ant's pheromone   & Indirect & Food source located at shorter distance\\
trail-laying to   & &  gets higher priority as less pheromone \\
food sources & & evaporates and more ants joins\\
\hline
\end{tabular}
\end{center}
\end{table}
%%%%%%%%%%%%%%%%%
Table \ref{table:bio-comm-task-urgency} presents the link between sensing the task and self-regulation of communication behaviours among ants and honey-bees. Here, we can see that communication is modulated based on the perception of  task-urgency irrespective of the communication strategy of a particular species. Under indirect communication strategy of ants, i.e. pheromone trail-laying, we can see that the principles of self-organization, e.g. positive and negative feedbacks take place due to the presence of different amount of pheromones for different time periods. Initially, food source located at shorter distance gets relatively more ants  as the ants take less time to return nest. So, more pheromone deposits can be found in this path as a result of positive feedback process.  Thus, the density of pheromones or the strength of indirect communication link reinforces ants to follow this particular trail.

%%
\begin{figure}
\centering
\includegraphics[width=6cm, angle=-90]
{./images/ch2/honey-bee-dance-stat.eps}
%figure caption is below the figure
\caption{Self-regulation in honey-bee's dance communication behaviours, produced after the results of \protect\possessivecite{Von1967} honey-bee round-dance experiment performed on 24 August 1962.}
\label{fig:honey-bee-dance-stat}  % Give a unique label
\end{figure}
%%
Similarly, perception of task-urgency influences the P2P and broadcast communication strategies. {\em Leptothorax albipennis} ant take lees time in assessing a relatively better nest site and quickly return home to recruit its nest-mates \cite{Pratt+2002}. Here, the quality of nest directly influences its intent to make more ``tandem-runs'' or to do tactile communication with nest-mates. We have already discussed about the influences of the quality of  flower sources to honey-bee dance.  Fig. \ref{fig:honey-bee-dance-stat} shows this phenomena more vividly. It has been plotted using the data from the honey-bee round-dance experiments of \citeasnoun[p. 45]{Von1967}. In this plot, Y1 line refers to the concentration of sugar solution. This solution was kept in a bowl  to attract honey-bees and the amount of this solution was varied from $\frac{3}{16}$M to 2M (taken as 100\%). In this experiment, the variation of this control parameter influenced honey-bees' communication behaviours while producing an excellent self-regulated DOL.

In Fig. \ref{fig:honey-bee-dance-stat} Y2 line represents the number of collector bees that return home. The total number of collectors was 55 (taken as 100\%). Y3 line plots the percent of collectors displaying round dances. We can see that the fraction of dancing collectors is directly proportional to the concentration of sugar solution or the sensing of task-urgency. Similarly, the average duration of dance per bee  is plotted in Y4 line. The maximum dancing period was 23.8s (taken as 100\%). Finally, from Y5 line we can see the outcome of the round-dance communication as the number of newly recruited bees to the feeding place. The maximum number of recruited bees was 18 (taken as 100\%). So, from an overall observation, we can see that bees sense the concentration of food-source  as the task-urgency and they self-regulate their round-dance communication behaviour according to their perception of this task-urgency. Thus, this self-regulated dancing behaviour of honey-bees attracts an optimal number of inactive bees to work.

Broadcast communication is one of the classic ways to handle dynamic and urgent tasks in biological social systems. It can be commonly observed in birds, ants, bees and many other species. Table \ref{table:bio-comm-task-urgency} mentions about the alarm communication of ants. Similar to the honey-bee's dance communication, ants has a rich language of chemical communication that can produce words through blending of different glandular secretions in different concentrations. Fig. \ref{fig:ants-active-space} shows how ants can use different concentrations of chemicals to make different stimulus for other ants. From the study of ants, it is clear to us that taking defensive actions, upon sensing a danger, is one of the highest-priority tasks in an ant colony. Thus, for this highly urgent task, ants almost always use their global broadcast communication strategy through their strong chemical signals and they make sure all individuals can hear about this task.  This gives us a coherent picture of the self-regulation of biological species based on their perception of task-urgency.
\begin{figure}
\centering
\subfloat[Polistes wasps]{\includegraphics[width=6cm, height=4cm]{./photos/Wasps_wikimedia.org_Polistes_nest_3_sjh.eps}}
\hspace{0.25cm}
\subfloat[Polybia wasps]{\includegraphics[width=6cm, height=4cm]{./photos/Polybia_occidentalis_I_JP6646_discoverlife.org.eps}} 
\caption{Colony founding in two types of social wasps (a) {\em Polistes}  founds colony by a few queens independently (b) {\em Polybia occidentalis}  founds colony by swarms. From http://www.discoverlife.org, last seen 01/05/2010.}
\label{fig:social-wasps}
\end{figure}
%%-----------------------------------------------
\subsection{Effect of group size on communication}
\label{bg:bio-comm:group-size}
The performance of cooperative tasks in large group of individuals also depends on the communication and sensing strategies adopted by the group. As introduced in Sec. \ref{intro:comm}, from the study of social wasps,  we can find that depending on the group size, different kinds of information flow occur in different types of social wasps \cite{Jeanne1999}. Polistes independent founders (Fig. \ref {fig:social-wasps}(a)) are species in which reproductive females establish colonies alone or in small groups with about $10^2$ individuals at maturity. Polybia swarm founders (Fig. \ref {fig:social-wasps}(b)) initiates colonies by swarm of workers and queens. They have a large number of individuals, in the order of $10^6$ and 20\% of them can be queen. 
\begin{figure}[htp]
\centering
\includegraphics[width=9cm, angle=0]
{./dia-files/jannae-fig10-info-flow-cmp.eps}
%figure caption is below the figure
\caption{Different patterns of information flow in two types of social wasps: polybia and polistes, reproduced from \protect\citeasnoun{Jeanne1999}.}
\label{fig:wasps-info-flow}  % Give a unique label
\end{figure}
%%
\begin{figure}[htp]
\centering
\includegraphics[width=6cm, angle=0]
{./images/ch2/jeanne-fig9-info-flow.eps}
%figure caption is below the figure
\caption{Information flow in polybia social wasps, reproduced from \protect\citeasnoun{Jeanne1999}.}
\label{figs:sf-wasps-info-flow}  % Give a unique label
\end{figure}
Fig. \ref{fig:wasps-info-flow} compares the occurrence of information flow among independent and swarm founders. In case of swarm founders information about nest-construction or broods food-demand can not reach to foragers directly. Fig. \ref{figs:sf-wasps-info-flow} shows the path of information flow among swarm founders for nest construction. The works of {\em pulp foragers} and {\em water foragers} depend largely on their communication with {\em builders}. On the other hand, in case of independent founders there is no such communication and sensing are present among individuals. In Sec. \ref{intro:comm} we have termed these two types communication and sensing strategies as GSNC (for independent founders) and LSLC (swarm founders).

\citeasnoun{Jeanne1999} explained the above phenomena of selecting different strategies in terms of task-specialization patterns and stochastic properties found in the group. In case of large colonies, many individuals repeatedly performs same tasks as this minimizes their interferences, although they still have a little probability to select a different task randomly. But because of the large group size, the queuing delay in inter-task switching keeps this task-switching probability very low. Thus, in swarm founders, task-specialization becomes very high among individuals. On the other hand, in small group of independent founders, specialization on a specific task is very costly, because this prevents individuals not to do other tasks whose task-urgencies can soon become very high. Thus these individuals tend to become generalist and do not communicate task information with each other.

\begin{figure}[htp]
\centering
\includegraphics[width=9cm, angle=0]
{./images/ch2/jeanne-fig6-group-size.eps}
%figure caption is below the figure
\caption{Productivity of social wasps shown as a function of group size, reproduced from \protect\citeasnoun{Jeanne1999}.}
\label{fig:wasps-group-productivity}  % Give a unique label
\end{figure}
%%
The above interesting findings on GSNC and LSLC in social wasps have been linked up with  the group productivity of wasps. Fig. \ref{fig:wasps-group-productivity} illustrates high group productivity in case of LSLC of swarm founders. The per capita productivity was measured as the number of cells built in the nest in (a) and the weight of dry brood in grams in (b). In case of independent founders this productivity is much lesser (max. 24 cells per queen at the time the first offspring observed) comparing to the thousands of cells produced by swarm founders.  This shows  us the direct link between high productivity of social wasps and their selection of LSLC strategy. These fascinating findings from wasp colonies have motivated us to test these communication and sensing strategies in a fairly large multi-robot system to achieve an effective self-regulated MRTA.
%%%%%%%%%%%%%%%%%%%%%%%%%%%%%%%%%%%%%%%%%%%%%%%%%%%%%%%%%%%%%%%%%%%
\section{Overview of multi-robot systems (MRS)}
\label{bg:mrs:overview}
%%
\subsection{MRS research paradigms}
\label{bg:mrs:paradigms}
%%% Earliest multi-robot system
\begin{figure}
\begin{minipage}[t]{0.48\linewidth}
\centering
\includegraphics[width=6cm, height=4cm, angle=0]
{./photos/Nerd_Herd.eps}
\caption{ The Nerd-Herd. From \protect\citeasnoun{Mataric1994}}
\label{fig:mataric-nerd-herd} % Give a unique label
\end{minipage}
\hspace{0.5cm}
\begin{minipage}[t]{0.48\linewidth}
\centering
\includegraphics[width=6cm,height=4cm, angle=0]{./photos/kube_box_pushing.eps}
\caption{ A group of 10 box pushing robots. From \protect\citeasnoun{Kube1997} }
\label{fig:kube-box-pushing} 
\end{minipage}
\end{figure}
%%
Historically the concept of multi-robot system comes almost after the introduction of behaviour-based robotics paradigm \cite{Brooks1986}. In 1967, using the traditional sense-plan-act or hierarchical approach \cite{Murphy2000}, the first artificially intelligent robot, Shakey, was created at Stanford Research Institute. In late 80s, \citename{Brooks1986} influenced this entire field of mobile robotics by his layered behaviour-based robot-control approach that acted significantly differently than the hierarchical approach.  

At the same time, \citeasnoun{Braitenberg1984} described a set of experiments where increasingly complex vehicles could be built from simple mechanical and electrical components. Around the same time and with similar principles, \citeasnoun{Reynolds1987} developed a distributed behavioural model for a bird in a flock that assumed that a flock is simply the result of the interactions among the individual birds (see Sec.  \ref{bg:mrs:srs}). 

Early researches on multi-robot systems also include the concept of cellular robotic system \cite{Fukuda+1987,Beni1988} multi-robot motion planning \cite{Arai+1989,Premvuti+1990,Wang1989} and architectures for multi-robot cooperation \cite{Asama+1989}. Fig. \ref{fig:mataric-nerd-herd} and \ref{fig:kube-box-pushing} present us the two earliest multi-robot systems in foraging and box-pushing task domains, developed by two pioneers in this filed, \citeasnoun{Mataric1994} and \citeasnoun{Kube1997} respectively.

% FIG: Multi-robot in emergency
\begin{figure}
\centering
\subfloat[A disaster site: buring oil rig]{\includegraphics[width=6cm, height=4cm]{./photos/burning-oil-rig-explosion-fire-photo11.eps}} 
\hspace{0.25cm}
\subfloat[An array of Seaglider underwater robots]{\includegraphics[width=6cm, height=4cm]{./photos/seaglider.eps}}
\caption{Example of multi-robot system working at a disaster site (a) British Petroleum's oil rig sinks in the Gulf of Mexico after explosion.\protect\newline From http:///news.bbc.co.uk, reported on 22/04/2010. 
(b) IRobot's Seaglider fleet that was surveying oil spill in the Gulf of Mexico at depths of up to 1,000 meters. From http:///www.irobot.com, reported on 25/05/2010.}
\label{fig:bp-oil-disaster}
\end{figure}
%% FIG: Robocup
\begin{figure}
\centering
\subfloat[Robot teams at Robocup soccer]{\includegraphics[width=6cm, height=4cm]{./photos/robocup_bbc_41132545_robotbody2.eps}} 
\hspace{0.25cm}
\subfloat[Robot team at Robocup rescue]{\includegraphics[width=6cm, height=4cm]{./photos/rugbot_robocup_rescue_osaka.eps}}
\caption{Example of multi-robot system at sports and rescue (a) Robot dogs playing at Robocup soccer. \protect\newline From http:///news.bbc.co.uk, reported on 11/05/2005. 
(b)Rugbot robot team was competing at Osaka's Robocup rescue league from \protect\citeasnoun{Birk+2006}.}
\label{fig:robocup}
\end{figure}
%%
From the beginning of the behaviour-based paradigm, the biological inspirations influenced many cooperative robotics researchers to examine the social characteristics of insects and animals and to apply them to design multi-robot systems \cite{Arkin1998}. The underlying basic idea is to use the simple local control rules of various social species, such as ants, bees, birds etc., to the development of similar behaviours in robots. In multi-robot literature, there are many examples that demonstrate the ability of multi-robot teams to aggregate, flock, forage, follow trails etc. \cite{Bonabeau+1999}. The dynamics of ecosystem, such as cooperation, has also been applied in multi-robot systems that has presented the emergent cooperation among team members \cite{Mcfarland1994,Martinoli+1996}. 

On the other hand, the study of competitive behaviours among animal and human societies has also been applied in multi-robot systems, such as in multi-robot soccer \cite{Asada+1999}. Fig. \ref{fig:bp-oil-disaster} and \ref{fig:robocup} show us that multi-robot system can be used to develop a wide ranges of useful applications ranging from human disaster recovery to games and entertainment.

As discussed above, there are several research groups who follow different approaches to handle multi-robot research problems. Based on the underlying philosophies and principles, We have classified them into two broad paradigms: 1) traditional multi-robot system and 2) swarm robotic system. Below we have highlighted them  and Sec. \ref{bg:mrs:mrs} and \ref{bg:mrs:srs} present review various issues of these systems. 
%
\subsubsection*{1. Traditional multi-robot system paradigm}
Traditional multi-robot systems follow the organizational, social, knowledge-based and multi-agent based approaches to solve problems of multi-robot system. They do not take inspiration from biological social systems directly. Explicit modelling of environment, tasks, robots can be the main features of these systems. Traditional multi-robot system can be classified into the following two broad categories \cite{Parker2008}.\\
\textbf{A. Organizational and social approaches: }
Organizational and social paradigms are typically based on organization theory derived from human systems that reflect the knowledge from sociology, economics, psychology and other related fields. To solve complex problems this paradigm usually follows the cooperative and collaborative forms of distributed intelligence. In multi-robot systems the example of this paradigm is found in two major formats: 1) the use of roles and value system and 2) market economics.  For example, in multi-robot soccer \cite{Stone+1999} positions played by different robots are usually considered as defined roles. Market economics approach \cite{Dias+2006} uses market economics theory that enables the selection of robots for specific tasks according to their individual capabilities determined by a bidding process.\\
\textbf{B. Knowledge-based and multi-agent based approaches: }
This paradigm, commonly used for developing multi-agent systems, is knowledge-based, ontological and semantic paradigm. Here knowledge is defined as ontology and shared among robots/agents from disparate sources. It reduces the communication overhead by utilizing the shared vocabulary and semantics. Due to low bandwidth, limited power, limited computation, noise and uncertainty in sensing/actuation, the use of this approach is usually restricted in multi-robot systems.
% 
\subsubsection*{2. Swarm robotic paradigm}
In bio-inspired, emergent swarm robotic paradigm, local sensing and local interaction form the basis of collective behaviours of swarms of robots. Today, this paradigm has been emerged as a sub-field of robotics called {\em swarm robotics} \cite{Sahin+2005}. This is a powerful paradigm for those applications that require performing shared common tasks over distributed workspace, redundancy or fault-tolerance without any complex interaction of entities. Some examples include flocking, herding, searching, chaining, formations, harvesting, deployment, coverage etc. 

Although our approximate classification of multi-robot system includes most of the research directions, it is very hard to specifically categorize all diverse researches on multi-robot systems. However, most of the researchers select a suitable paradigm to abstract the problem from a specific perspective with common fundamental challenges of multi-robot system discussed in the later sections.
%---------------------------------------------------------
\subsection{MRS taxonomies}
\label{bg:mrs:taxonomies}
The vast amount of research in multi-robot system makes it necessary to use well-established classifications or taxonomies in order to specify and design a useful multi-robot system. In Section \ref{bg:mrs:paradigms} we see that the main-stream research in multi-robot system can be classified into two distinct paradigms. However, these paradigms have certain assumptions, often unspecified or implicit, regarding the design of robot hardware, software, communication and interaction etc. Thus, multi-robot system taxonomies can be useful for many purposes, e.g. to avoid ambiguities in system specification by reducing the size and complexity of possible design spaces, and to use certain trade-off among various features for achieving overall system performance.

While earlier multi-robot system taxonomies, e.g. one proposed by \citeasnoun{Premvuti+1990}, \citeasnoun{Cao+1997} discuss very fundamental design issues of multi-robot system, recent taxonomies e.g. proposed by \citeasnoun{Dudek+2002}, \citeasnoun{Gerkey+2004}, \citeasnoun{Balch2002}, \citeasnoun{Farinelli+2004} etc. give us the detail design choices for making useful system specifications. We classify these recent taxonomies into two groups: 1) generalized taxonomies and 2) specialized taxonomies. We consider taxonomies of \citeasnoun{Dudek+2002} and \citeasnoun{Parker2008} as generalized taxonomies since they  can be used to specify almost all necessary features of a multi-robot system.   

On the other hand, specialized taxonomies provide multi-robot system specification with respect to particular system features, e.g., the taxonomy of \citeasnoun{Balch2002} is only useful in a multi-robot system with reinforcement learning, the taxonomy of \citeasnoun{Gerkey+2004} gives us the specification of tasks in a MRTA context. Other less common taxonomies e.g., one proposed by \citeasnoun{Farinelli+2004} or another proposed by \citeasnoun{Cao+1997} are centred around the coordination (weak, strong or none), communication (implicit, explicit or none), architecture (centralized or decentralized) etc. Here we briefly describe the major axes of leading taxonomies of multi-robot systems.
%%%
\subsubsection*{Generalized taxonomy of multi-robot systems}
The generalized taxonomy of \citeasnoun{Dudek+2002} provides seven main axes for multi-robot team specification. We have regrouped them into the following three major areas.\\
\textbf{1. Collective or group size, composition and reconfigurability: }  A multi-robot system can be formed by one, two, multiple or virtually infinite number of autonomous robots. Composition refers to the homogeneity of the group members. Robots can be identical in both form and function (hardware and software), homogeneous (consisting of same physical hardware) or physically heterogeneous. Collective reconfigurability refers to the rate at which robots can spatially re-position themselves. It can be completely static, coordinated or dynamically arranged.\\
\textbf{2. Communication range, topology and bandwidth: } The maximum distance between two robots, required for effective communication, can be zero (i.e. they can not communicate directly) or infinite (i.e. all robots can communicate to any other robot) or in-between these two options. Communication topology determines the style of addressing target peers e.g., through broadcast messaging, individual addressing by name or address or, following tree-like hierarchy or redundant graphs. Communication bandwidth provides the measure of costs associated with communication. This can be no cost (i.e. infinite bandwidth) or high cost (i.e. limited or no bandwidth) or something in between these two extreme cases.\\
\textbf{3. Processing abilities: } This refers to the software architectures that can be used for deploying controllers of robots. General models include finite state automata, a push-down automata, neural-networks or Turing machines. 
%%%
\subsubsection*{Specialized taxonomies of multi-robot systems} 
The taxonomy of \citeasnoun{Gerkey+2004} defines three axes of possible tasks and robot capabilities of multi-robot system:\\
\textbf{1. Single-task robots  {\em vs.} multi-task robots}: The former (latter) means a robot can perform one (multiple) tasks at a time.\\
\textbf{2. Single-robot tasks  {\em vs.} multi-robot tasks }: while under the former, each task requires only one robot, under the latter, multi-robots may be required.\\
\textbf{3. Instantaneous assignment {\em vs.} time-executed assignment }: While the former refers a situation when planning for future task-allocations is not possible, under latter, planning is possible.

\citeasnoun{Balch2002} extended this taxonomies of tasks and applied to multi-robot learning cases. His taxonomy of reward include: source of reward (internal or external or both), rewarding time (immediate or delayed), continuity of reward (discrete or continuous), locality of reward (global or local or a combination of both) relation to performance (tied to performance or based-on intuitive state-value).
%%
\subsubsection*{Taxonomy of interaction in multi-robot teams} 
In addition to the above two classes of taxonomies, we present here \possessivecite{Parker2008} notion of {\em interaction} in a robot team in four levels: 1) collective, 2) cooperative 3) collaborative and 4) coordinative. 
%% Fig interaction
\begin{figure}
\centering
\includegraphics[width=9cm, angle=0]
{./images/ch2/parker-interaction-classification.eps}
%figure caption is below the figure
\caption{\small Categorization of types of interactions in multi-robot system, reproduced from \protect\citeasnoun{Parker2008}.}
\label{fig:parker-interaction} % Give a unique label
\end{figure}
%
As seen in Fig. \ref{fig:parker-interaction}, the interactions among a multi-robot team can be viewed along three different axes:
\begin{enumerate}
\item Types of goals of entities: either shared goal such as, cleaning a floor, or, individual goal.
\item Awareness of entities  about others on the team: either aware such as, in cooperative transport, or, unaware such as, in a typical foraging.
\item Influence of the action of one entity to advance the goal of others: such as, one robot's floor cleaning helps other robots not to clean that part of the floor.
\end{enumerate}
Based on this axes interaction can be classified into four categories:\\
% see Fig
\textbf{1. Collective interaction: }
Entities are not aware of others on the team, yet do share goals and their actions are beneficial to team-mates. Most swarm-robotic systems follow this kind of interaction to perform biologically-relevant tasks, such as foraging, swarming, formation keeping and so forth.\\
\textbf{2. Cooperative interaction: }
Entities are aware of others on the team, they share goals and their actions are beneficial to their team-mates. This type of interaction is used to reason about team-mates' capabilities where multiple robots work together, usually in a common shared workspace, such as cleaning a work-site, pushing a box, performing search and rescue, extra-planetary exploration and so forth.\\ 
\textbf{3. Collaborative interaction: }
Having individual goals (and even individual capabilities), entities aware of their team-mates and their actions are beneficial to their team-mates. One example of this kind of interaction is a team of collaborative robots where each must reach a unique goal position by sharing the sensory capabilities to all members such as found in coalition formation \cite{Parker+2006}.\\
\textbf{4. Coordinative interaction: }
Entities are aware of each other, but they do not share a common goal and their actions are not helpful to other team members. For example, in a common workspace, robots try to minimize interference by coordinating their actions as found in multi-robot path planning techniques, traffic control techniques and so on.

Beyond this four most common types of interactions another kind of interaction can be found in adversarial domain where entities effectively work each other such as multi-robot soccer. Here entities have individual goals, they are aware of each other, but their actions have a negative affect on other robots' goals. Most researches do not consider these type team as a typical swarm robotic system which is explained later.  

In this dissertation, we have used the taxonomy of \citeasnoun{Dudek+2002} for specifying our multi-robot system (Sec. \ref{expt-tools:mrs-design}). The taxonomy of \citeasnoun{Parker2008} is used to analyse the dependence of multi-robot system on various levels of interactions in Sec. \ref{bg:mrta}. 
%%
% FIG: Multi-robot indoor and outdoor operations
\begin{figure}
\begin{minipage}[t]{0.48\linewidth}
\centering
\includegraphics[width=6cm, height=4cm, angle=0]
{./photos/centibot_demo3-11.eps}
\caption{ Hundreds of Centibots robots worked at indoor search, navigation and mapping tasks. From \protect\citeasnoun{Ortiz+2005}. }
\label{fig:centibots-indoor}
\end{minipage}
\hspace{0.5cm}
\begin{minipage}[t]{0.48\linewidth}
\centering
\includegraphics[width=6cm,height=4cm, angle=0]{./photos/pioneer_robots_610x455.eps}
\caption{Pioneer robots operating at outdoor uncertain environment of Georgia Tech. Mobile Robot Lab. From http://news.cnet.com, reported on 05/04/2010.}
\label{fig:pioneers-outdoor} % Give a unique label
\end{minipage}
\end{figure}
%-----------------------------------------------------------------
\subsection{Traditional MRS}
\label{bg:mrs:mrs}
%%
%Challenges of multi-robot system
multi-robot system not only shares the problem of controlling a single robot but also it amplifies the problem to several orders of magnitude. Below we list a few major challenges of any multi-robot system:
\begin{itemize}
\item \textbf{Increased uncertainty about environment:}
When multiple robots work in a partially observable world, the environmental view becomes severely restricted due to both in terms of noisy sensor readings and frequent obstacle detections. Thus, in multi-robot system, the uncertainty about the environment increased in may folds.
%
\item \textbf{Increased dynamic changes of the environment:}
Since many robots work in a shared environment, the dynamic movements and physical interferences among the robots become more frequent. So robots are required to change their course of action more frequently.
% 
\item \textbf{Decreased communication throughput:}
Interference in communication is inescapable for a team of robots. Since the typical bandwidth of a communication channel is fixed, adding more robots reduces the effective communication throughout and thus increased latency in robot-robot or robot-computer communications. If the robots are required to coordinate their action then the saturation of the communication channel affects the overall team-performance.
%
\item \textbf{Decreased real-time performance:}
In a functional multi-robot system, autonomous mobile robots need to do some tasks in real-time, e.g. identifying their current poses ({\em localization}) to determine next motions, avoiding obstacles etc. However, when the number of robots increases the real-time performance can be poor due to the above factors.
%
\item \textbf{Increased sensor failures and break-downs:}
This is also common in a multi-robot system that the real-time interaction of large number robots can decrease the life of their hardware as they become subject to more collisions and interferences. Thus overall reliability of the multi-robot system can be decreased gradually.
\end{itemize}
%%
Despite the above big challenges, researchers design and operate multi-robot system successfully using a number of intelligent solutions since the last few decades. In the previous subsections we have seen how the researches are classified into distinctive paradigms and can be specified by precise taxonomies. Here we list a number of typical issues that any multi-robot system typically encounters from its inception to implementation.
\begin{itemize}
\item {\em Motion control}. How to use sensor values to produce real-time motions avoiding obstacles?
\item {\em Localization}. How to find out the self-position in the world so that reaching to a specified target location becomes possible?
\item {\em Navigation/map-building}. How to integrate sensor values to build maps or representation of the environment for further exploration?
\item {\em Task-selection}. How to plan/predict and select a particular high-level task (e.g. find a red object or picking up a stick) provided that a number of tasks present in the environment? 
\item {\em Interaction and communication}.How to interact or communicate with other robots for cooperating, collaborating or coordination in doing tasks?
\item {\em Adaptation/learning} How to remember things so that future robot actions or behaviours become improved?
\end{itemize}
%%
Not all of the above issues are present in all multi-robot system. Many multi-robot systems do not use any form of navigation, communication or learning and yet they do some useful tasks. However it is important to understand how these issues can be solved in a structured, modular and timely manner. For example, conflicts occur if a resource is required by or, a unique single task is distributed to, more than one robot at any given time. Several resources such as bandwidth, space etc. may be needed by more than one robot. The sharing of bandwidth among robots is a great problem in case of applications like multi-robot mapping \cite{Konolige+2003}. As shown in Fig. \ref{fig:centibots-indoor}, in large multi-robot team such as in Centibots system \cite{Ortiz+2005}, task interference and high bandwidth communication between 100s of robots appear as a significant research challenge.

Whatever be the principle characteristics of a multi-robot system, e.g., homogeneity, coupling, communication methods etc., each multi-robot system must address some degree to those problems. For example, usually every multi-robot system adopts a control architecture under a specific paradigm to organize its hardware, software and communication system. Similarly every multi-robot system address the issues of communication, localization, interaction in a way specific to the application and underlying design principles or philosophies. In the following subsections, we have attempted to summarize the key research issues that can potentially influence the selection and implementation of our multi-robot system.
%-----------------------------
\subsection*{Architecture and control}
\label{bg:mrs:arch}
%% FIG: three layer arch
\begin{figure}
\begin{center}
\includegraphics[width=5cm,height=6cm]{./images/ch2/three-layered.eps} % The printed column width is 8.4 cm.
\caption{A typical hybrid robot control architecture, adopted from \protect\citeasnoun{Mataric2007}.} 
\label{fig:three-layer-arch}
\end{center}
\end{figure}
%%
In multi-robot system, two high-level control strategies are very common: 1) centralized and 2) decentralized or distributed. Under a specific control strategy, traditionally three basic system architectures are widely adopted: deliberative, reactive and hybrid \cite{Arkin1998}. Deliberative systems based on central planning are well suited for the centralized control approach. The single controller makes a plan from its {\em sense-plan-act} loop by gathering the sensory information and each robot performs its part.

Reactive systems are widely used in distributed control where each robot executes its own controller maintaining a tight coupling between the system's sensors and actuators, usually through a set of well-designed behaviours. Here, various group behaviour emerges from the interactions of individuals that communicate and cooperate when needed. 

Hybrid systems are usually the mixture of the two above approaches; where each robot can run its own hybrid controller with the help of a plan with necessary information from all other robots. Behaviour-based control architecture can also be considered as a separate category of distributed control architecture, where each robot behaves according to a behaviour-based controller and can learn, adapt and contribute to improve and optimize the group-level behaviour  \cite{Mataric2007}.
%%
%% FIG: ALLIANCE
\begin{figure}
\centering
\includegraphics[width=12cm, angle=0]
{./images/ch2/parker-alliance-arch.eps}
%figure caption is below the figure
\caption{ALLIANCE architecture. From \protect\citeasnoun{Parker1998}.}
\label{fig:parker-alliance-arch} % Give a unique label
\end{figure}
%% FIG: foraging fsa
\begin{figure}
\begin{center}
\includegraphics[width=6cm,height=5cm]{./dia-files/foraging-fsa.eps} % The printed column width is 8.4 cm.
\caption{Finite state machine for foraging task, reproduced from \protect\citeasnoun{Arkin1998}.} 
\label{fig:foraging-fsa}
\end{center}
\end{figure}
%%
Although most of the multi-robot system control architectures share some common characteristics (such as distributed and behaviour-based control strategy) based on their difference of underlying design principles we have put them into three groups:
\begin{enumerate}
\item \textbf{Behaviour-based classical architectures: }
The ALLIANCE architecture \cite{Parker1998} is one of the earliest behaviour-based fully distributed architectures (Fig. \ref{fig:parker-alliance-arch}). This architecture has used the mathematically modelled behaviour sets and motivational system. The primary mechanism for task selection of a robot is to activate the motivational behaviour partly based on the estimates of other robots behaviour. This architecture was designed for heterogeneous teams of robots performing loosely coupled tasks with fault-tolerance and co-operative control strategy. Broadcast of local eligibility  \cite{Werger2001} is another behaviour-based architecture that uses port-attributed behaviour technique through broadcast communication method. It was demonstrated to perform coordinated tasks, such as multi-target observation tasks. Similar to the above two architectures, many other researchers proposed and implemented many variants of behaviour-based architectures. Some of them used the classic three layer (plan-sequence-execute) approach, \cite{Simmons+2002} used a {\em layered architecture}.
%
\item \textbf{Market-based architectures: }
Using the theory of marker economics and well-known Contract Net Protocol \cite{Davis1988+}, these architectures solve the task-allocation problem by auction or bidding process. Major architectures following market-based approaches include MURDOCH \cite{Gerkey+2002}, M+ system \cite{Botelho+1999}, first-piece auction \cite{Zlot+2002}, dynamic role assignments \cite{Chaimowicz2002} among others.
%
\item \textbf{Multi-agent based architectures: }
Some multi-robot system architectures are influenced by multi-agent systems. For example, CHARON is a hierarchical behaviour-based architecture that rely on the notion of agents and modes \cite{Chaimowicz2002}. Similarly CAMPOUT is another distributed behaviour-based architecture that provide high-level functionality by making use of basic low-level behaviours in downward task decomposition of a multi-agent planner \cite{Huntsberger+2003}. It is comprised of five different architectural mechanisms including, behaviour representation, behaviour composition, behaviour coordination, group coordination and communication behaviours.
\end{enumerate}
%% Our approach
In this dissertation, we closely follow the behaviour-based hybrid architecture with an event-driven mechanism for activating behaviours. Our multi-robot control architecture and robot controllers are illustrated in Sec. \ref{expt-tools:arch}.
%%%%%---------------------------------------
\subsection*{Learning}
\label{bg:mrs:learn}
A great deal of research on multi-robot learning has been carried out since the inception of multi-robot system \cite{Mataric+2001,Parker1995}. Learning, identified as the ability to acquire new knowledge or skills and improve one's performance, is useful in multi-robot system due to the necessity of robots to know about itself, its environment and other team-members \cite{Mataric2007}. Learning can improve performance since robot controllers are not perfect by design and robots are required to work in an uncertain environment that all possible states or actions can not be predicted in advance. Beside learning, it is also important to forget learned things that are no longer needed (or correct) as well as, to make room for new things to be learned and stored in a finite memory space of a robot. 

Several learning techniques are available in robotics domain, such as reinforce or unsupervised learning, supervised learning and learning by imitation. Although reinforce learning, or learning based on environmental or peer feedback, is a good option for multi-robot system, it has been found that in large teams the ability to lean in this way is restricted due to large continuous state and action space \cite{Yang+2004}. Several other learning techniques are also available to explore in multi-robot system domain including Markov models, Q-learning, fuzzy logic, neural nets, game theory, probabilistic or Bayesian theory among others. 
%%
\subsection*{Localization and exploration}
\label{bg:mrs:loc}
Mobile robot systems highly rely on precise localization for performing their autonomous activities in indoor or outdoor. Localization is the determination of exact pose (position and orientation) with respect to some relative or absolute coordinate system. This can be done by using proprioceptive sensors that monitor motion of a robot or exteroceptive sensors that provide information of world representation, such as  global positioning system (GPS) or indoor navigation system. Many other methods are also available, such as landmark recognition, cooperative positioning and other visual methods \cite{Arkin+2002}.

Localization issue of multi-robot system also invites researchers to examine specific areas like exploration and map generation. In exploration problem, robots need to minimize the time needed to explore the given area. Many researchers uses various kinds of exploration algorithms for solving this NP-hard problem, such as line-of-sight constrained exploration algorithm , collaborative multi-robot exploration \cite{Burgard+2000}. In mapping problem, mostly inaccurate localization information from teams of robots are accumulated and combined to generate a map by various techniques, such as probabilistic approaches \cite{Thurn+2000}.
% FIG: Multi-robot indoor and outdoor operations
\begin{figure}
\centering
\subfloat[Swarmbot]{\includegraphics[width=6cm, height=4cm]{./photos/jamesMcLurkin_SwarmbotAndSwarm.eps}} 
\hspace{0.25cm}
\subfloat[Robots detecting boundary]{\includegraphics[width=6cm, height=4cm]{./photos/james_fig-BoundaryDetection-Robots.eps}}
\caption{ (a) A Swarmbot and the Swarms 
 (b) Swarmbots detecting boundary using distributed algorithms, from \protect\citeasnoun{Mclurkin+2009}.}
\label{fig:swarmbot-boundary-detection}
\end{figure}
%%
%%
\subsection*{Example research areas of multi-robot system}
\label{bg:mrs:eg}
Researches on multi-robot system have been targeted for numerous application domains that all can not be listed here. Here we have included a few major areas that have received highest attention in the robotics research community. 

Cooperative transport of large objects  by multiple robots was investigated by many researchers such as, following a formal model of cooperative transport in ants \cite{Kube+1993}, box-pushing by six-legged robots \cite{Mataric+1995}. Another kind of object transport problem include clustering objects into piles  \cite{Beckers+1994}, collecting waste or trash cite{Parker1994}, sorting coloured objects  \cite{Melhuish+1998} and so on. It has also been observed that multi-robot teams as micro or mini machines are helpful to improve the control and efficiency of mining and its processing operations \cite{Dunbar+2002}. 

Many researchers address research issues under the requirements of a military or space application, e.g. behaviour-based formation control \cite{Balch+1998}, landmine detection \cite{Franklin+1995}, multiple planetary rovers for various missions \cite{Huntsberger2004}.

Although research on multi-robot system has been becoming more mature since last decades, it is not easy to find many industrial applications relying on multiple autonomous mobile robots. We have found one exceptional application developed by Kiva Systems in the domain of multi-robot material handling in warehouses \cite{Wurman+2008}. Along with this, Sec. {bg:mrs-industry} reviews some possible applications of multi-robot system in automation industry.
%%-----------------------------------------------
\subsection{Swarm robotic system}
\label{bg:mrs:srs}
When many traditional multi-robot systems showed serious  scalability failures, the necessity of adopting a  new paradigm became obvious \cite{Lerman+2006}. Researchers of traditional multi-robot system approach realized  that  executing their time and processing intensive algorithms in large number of real robots ($\geq 10$) could be a nightmare. Adding more robots almost exponentially amplified their inherent  problems e.g. physical and communication interferences with an ever-ending hunger of more CPU powers. 

In early and mid-90s, many researchers found that applying biological principles of swarm intelligence effectively removed and reduced many bottlenecks of traditional multi-robot system.  In 1995, Maja Mataric published that complex group behaviours could be produced by the appropriate combinations of more simple ``basis behaviours'' \cite{Mataric1995}. The idea of using simple biological behaviours, such as avoidance and following, to create complex flocking and foraging behaviours inspired many other researchers to search solution for controlling large multi-robot system in this direction.  The early research of Reynold (1987) helped many others to apply the principles of swam intelligence in multi-robot system. 

The term {\em swarm intelligence} was first coined by  \citeasnoun{Beni2005} in late 80s. It represents the effort of designing distributed problem solving algorithms inspired by the collective behaviours of social insect colonies \cite{Bonabeau+1999}. The idea of using simple robots to create complex patterns or structures was also studied under {\em cellular robotics} \cite{Fukuda+1987}. During recent years, {\em swarm robotics} has emerged as an application of swarm intelligence in multi-robot systems with the emphasis on physical embodiment of entities and realistic interactions among the entities and their environment. These systems of swarms or minimalist robots\footnote{Although both swarm robotics and minimalist robotics follow similar approaches for solving similar problems, the latter does not explicitly relate its origin to swarm intelligence.} can be represented by a common term {\em swarm robotic system}.

\subsubsection*{Advantages of swarm robotic system} 
The simplicity of swarm robotic system inspires researches to build multi-robot system with cheap robotic hardware, to equip them with simple controllers and control them through local informations, without creating any explicit model of the environment or using any sophisticated  controller. The redundancy of robots, parallelism in their task-executions and an overall distributed control architecture, support addition or removal (or failure) of robots in run-time. Moreover the control algorithms are now decoupled from the model of the environment or other robots. Thus this system now becomes more robust, fault-tolerant, scalable and adaptive to unknown and dynamic environment.
%%
\subsubsection*{Distinct features of swarm robotic system}
In order to distinguish swarm robotics from other branches of robotics such as collective robotics, distributed robotics, robot colonies and so forth, \citeasnoun{Sahin+2005} proposed a formal definition and a set of criteria for swarm robotics research. 
\begin{quote}
\ssp
``Swarm robotics is the study of how large number of relatively simple physically embodied agents can be designed such that a desired collective behaviour emerges from the local interactions among agents and between the agents and the environment.'' 
\end{quote}
\sdp
And the notable criteria of swarm robotics research are listed as follows.
\begin{enumerate}
\item \textbf{Autonomous robots: }
that exclude the sensor networks and may include metamorphic robotic system without having no centralized planning and control element.
\item \textbf{Large number of robots: }
usually $\geq$ 10 robots, or at least having provision for scalability if the group size is below this number.
\item \textbf{Mostly homogeneous groups of robots: }
that typically exclude the multi-robot soccer teams having heterogeneous robots.
\item\textbf{ Relatively incapable of inefficient robots: }
so that the task complexity enforces either cooperation among robots or increased performance or robustness without putting no restriction on individual robot's hardware/software complexity.
\item \textbf{Robots with local sensing and communication capabilities: } 
that does not use global coordination channel to coordinate among themselves, rather enforces distributed coordination.
\end{enumerate}
%%%
% FIG: Swarmbots demo
\begin{figure}
\centering
\subfloat[Swarmbots at exploration]{
\includegraphics[width=6cm, height=4cm, angle=0]
{./photos/swarm-bots-crossing-canal.eps}}
\hspace{0.5cm}
\subfloat[Swarmbots at mock rescue]{
\includegraphics[width=6cm,height=4cm, angle=0]{./photos/swarmbots-pulling-child-725972.eps}}
\caption{A group of Swarmbots (a) crossing a rough terrain (b) pulling a child to a safe location, from \protect\citeasnoun{Mondada+2004}.}
\label{fig:swarmbots} % Give a unique label
\end{figure}
%%
\subsubsection*{Classification and application of swarm robotic system}
swarm robotic system can broadly be classified into two distinct classes. The first class consists of simple and relatively inexpensive mobile robots that are fully autonomous and can work in isolation. For example, e-puck robots \cite{Cianci+2004} fall under this class. Other class of robots include self-reconfigurable \cite{Fukuda+1987} and self-assembling robots which can be built by coupling several identical units together, e.g. a robotic snake. In this dissertation, we have limited our focus to the former class of robots alone. Fig. \ref{fig:swarmbots} (a) shows amazing abilities of Swarmbots which are crossing a rough terrain. Fig \ref{fig:swarmbots} (b) shows another interesting demonstrations of swarm robotic system that a team of 18 Swarmbots pulls a child to a safe place.
%% FIG: BOIDS
\begin{figure}
\centering
\subfloat[Separation]{\includegraphics[width=3cm, height=2cm]{./photos/boid-separation.eps}} 
\hspace{0.25cm}
\subfloat[Alignment]{\includegraphics[width=3cm, height=2cm]{./photos/boid-alignment.eps}}
\hspace{0.25cm}
\subfloat[Cohesion]{\includegraphics[width=3cm, height=2cm]{./photos/boid-cohesion.eps}}
\caption{ Reynold's simulated flocking of boids \protect\citeasnoun{Reynolds1987}. (a) Separation: steer to avoid crowding local flockmates, (b) Alignment: steer towards the average heading of local flockmates and (c) Cohesion: steer to move toward the average position of local flock-mates. \protect\newline  From http://www.red3d.com/cwr/boids/, last seen on 01/06/2010.}
\label{fig:boid-rules}
\end{figure}
%%
\begin{figure}
\centering
\subfloat[A flock of birds]{\includegraphics[width=6cm, height=4cm]{./photos/bird-flocking.eps}} 
\hspace{0.25cm}
\subfloat[Simulated boids]{\includegraphics[width=6cm, height=4cm]{./photos/boid-flocking.eps}}
\caption{ (a) Real flocks of birds, from http://www.travelblog.org, and (b) simulated flock of birds, from http://www.red3d.com/cwr/boids, last seen on 01/06/2010.}
\label{fig:flocking-birds}
\end{figure}
%%
\subsubsection*{Modelling swarm robotic system}
Modelling both the behaviour of an individual robot controller and the overall system-level collective behaviours have become interesting issues. This is due to the fact that, in this kind of system, collective behaviours, e.g. flocking or aggregation, can significantly be changed by of just changing one or few parameters of individual robot controllers. Modelling swarm robotic system can give us an early insight about a target system before its implementation before doing any time-consuming simulation or expensive real-experiment.

A plenty of approaches are available for modelling swarm robotic system \cite{Gazi+2006}. Most common modelling approaches include: behaviour-based approach, probabilistic models, potential-function based approach, asynchronous swarm models, multi-agent based swarm models. Behaviour-based approach can be found in the study of \citeasnoun{Reynolds1987} who has simulated the flocking behaviours of birds. Fig. \ref{fig:boid-rules} illustrates how three simple rules can produce a coordinated flocking motion (Fig. \ref{fig:flocking-birds}(b)). From biological observation of flocking birds  it is obvious that collective behaviours can be generated through a local control and interaction rules. Similar to this study, many other researchers tries to apply behaviours based local strategy for formation control \cite{Balch+1998}, aggregation \cite{Mataric1995}, sorting \cite{Melhuish+1998}, foraging \cite{Liu+2007}, cooperative transport \cite{Kube1997} etc.
%%
\begin{figure}
\centering
\subfloat[Robots pulling sticks]{\includegraphics[width=\textwidth]{./photos/AGASSONOUN_S_SV.eps}} 
\hspace{0.5cm}\\
\subfloat[Probabilistic FSM]{\includegraphics[width=\textwidth]{./photos/agassonoun-pfsm.eps}}
\caption{ (a) Stick-pulling experiments by a group of Khephera robots equipped with gripper turrets and (b) Probabilistic finite state machie (PFSM) of robot controllers. From \protect\citeasnoun{Agassounon+2004}.}
\label{fig:stick-pulling-expt}
\end{figure}

In 1999, \citeasnoun{Martinoli1999} proposed probabilistic modelling of swarm robotic system. This probabilistic approach often has two major aspects: controller design through probabilistic finite state machine (Fig. \ref{fig:stick-pulling-expt}) and automated parameter adaptation through genetic algorithm. This approach has been adopted by many recent swarm robotic researches \cite{Agassounon+2004,Lerman+2005,Liu+2007b}.
 
swarm robotic models can be classified into many distinct classes. Firstly, they can be classified into: spatial and non-spatial models. {\em Spatial models} keep track of the agent's trajectories and perhaps use that spatial distribution. However, in {\em non-spatial models} it is assumed that agents occupy independent, random positions at consecutive time-steps. Swarm robotic models can also be classified into embodied and non-embodied models. {\em Non-embodied models} consider agents  as points and their physical characteristics are ignored, whereas {\em embodied models} take the physical characteristics or interferences of agents into account. Spatial models with embodied agents are chosen in typical simulations.

As another distinct classification, swarm robotic models can be classified into two major groups: 1) microscopic models and 2) macroscopic models. {\em Macroscopic models} focus on individual robots and state transitions of each robot controller are updated based on the stochastically approximated robot-robot and robot-environment interactions. The probabilities of state transitions are calculated from simple geometric configurations and with few trial experiments. Here, no group-level sensing or actuation is taken into account. On the other hand, {\em macroscopic models} captures the snapshot-by-snapshot pictures of whole swarm. Each snapshot presents the total number of robots in a given state. Fig. \ref{fig:stick-pulling-expt}(b) shows the probabilistic macroscopic model where $S_{x}$ denotes a particular state $x$ and $N_{s}$ denotes the number of robots under state $S_{x}$. Here $\tau$ represents the corresponding probability density function derived from a set of Master-Equations \cite{Agassounon+2004}.

Despite a lot of attractive benefits of swarm robotic system modelling approaches, formal models of swarms, particularly probabilistic models, may not be attractive or useful for many reasons. Firstly, constructing a functional model takes time due to the need for accurate calibration of necessary parameters (which also involves running several real-experiments or simulations). 
Secondly, most of probabilistic models rely upon some simplifying assumptions, e.g. coverage of the area should be spatially uniform or the system should follow Markov properties i .e. the robot's future state depends only on its current state and how much time it has spent in that state. These can not be satisfied in many practical applications e.g. open space exploration or robots with memory. As task complexity increases the parameters space becomes large and searching good combinations of parameters by some means, e.g. genetic algorithms, becomes more complex.

Similar to the traditional multi-robot systems, swarm robotic systems face great challenges in enabling localization, communication and  interaction in group-level. For example, without the presence of any centralized localization module, such as global positioning system or indoor navigation system, it is not easy to localize precisely and locally the position of a robot with respect to other robots or environment. Recently researchers investigated  these issues and reported some  novel solutions. For example, \citeasnoun{Spears+2006} reported a novel technique based on trilateration for localization of swarm robots using ultrasonic and radio-frequency transceivers. \citeasnoun{Schmickl+2006} reported hop-count and bio-inspired strategies for collective perception or how a swarm robot can join multiple instances of individual perception to get a global picture. Because of the similarity of the problems in both traditional multi-robot system and swarm robotic system, we have presented the issues of  task-allocation and communication of both types of multi-robot system in Sec. \ref{bg:mrta} and Sec. \ref{bg:mrs-comm} respectively.

In this dissertation, we closely follow the swarm robotic system approach of designing robot group and solving related issues. We have followed the behaviour-based approach for designing robot controllers (Chapter \ref{expt-tools}) that rely on GPS-like overhead camera-based solution to fulfil their localization needs. We have modelled our real robotic system considering their spatial, embodied and microscopic properties. No macroscopic simulation or analysis of the robot group has been conducted. Our autonomous robot group meets all the criteria of a swarm robotic system mentioned by \cite{Sahin+2005} except the communication issue. Our task-allocation model does not depend on local communication strictly. Our MRTA solution and multi-robot communication and sensing strategies have been presented in Chapter \ref{afm} and Chapter \ref{local-comm}.
 %%%%%%%%%%%%%%%%%%%%%%%%%%%%%%%%%%%%%%%%%%%%%%%%%%%%%%%%%%%%%%%%%%%
\section{Multi-robot task allocation (MRTA)}
\label{bg:mrta}
Since 90s, MRTA is a common research challenge that tries to define the preferred mapping of robots to tasks in order to optimize some objective functions \cite{Gerkey+2004}. Many control architectures   have been solely designed to address this task-allocation issue from different perspectives. Based-on the high-level design of those solutions, here we have classified researches on MRTA into two major categories: 1) predefined or intentional task-allocation and 2) bio-inspired self-organized task-allocation. Fig. \ref{fig:mrta-classification} illustrates our classification. 

This classification has been adopted from \citeasnoun{Shen+2001}, but our sub-categories are different since \citename{Shen+2001} proposed the classification for multi-agent system alone that does not take the spatiality and embodiment of agents into account. Under each of our sub-categories of MRTA there are many inter-connected issues that need to be addressed by the system designer. In the following subsections, we have discussed these two major categories and their key issues with some example MRTA solutions.
%
\begin{figure}
\centering
\includegraphics[width=12cm, angle=0]
{./dia-files/ta-categories.eps}
\caption{\small Classification of MRTA}
\label{fig:mrta-classification} % Give a unique label
\end{figure}
%%---------------------------------------
\subsection{Predefined task-allocation}
\label{bg:mrta:predefined}
In most of the traditional multi-robot system, task allocation is done using well-defined models of tasks and environments. Here it is assumed that the system designer has the precise knowledge about tasks, robot-capabilities etc. Many flavours of the type of task-allocation can be found in the literature. Below we briefly discussed a few well-acknowledged works.\\ 
%% FIG: ALLIANCE motivational bh
\begin{figure}
\centering
\includegraphics[width=10cm, angle=0]
{./dia-files/alliance-motivational-bh.eps}
%figure caption is below the figure
\caption{Motivational behaviour in ALLIANCE. From \protect\citeasnoun{Parker1998}.}
\label{fig:alliance-motivation} % Give a unique label
\end{figure}
%
\subsubsection*{Knowledge-based and multi-agent based approaches}
In this approach knowledge-based techniques are used to represent tasks, robot capabilities etc. One of the early well-known MRTA architecture of this category was ALLIANCE  in which each robot models the ability of team-members to perform tasks by observing their current task performances and collecting relevant task quality statistics e.g. time to complete tasks \cite{Parker1998}. Robots use these models to select a task that benefit  the group as a whole.
As shown in Fig. \ref{fig:parker-alliance-arch}, ALLIANCE architecture, implemented in each robot, delineates several mathematically modelled behaviour sets, each of which corresponds to some high-level task-achieving function. 

In ALLIANCE, the concept of motivational behaviour was introduced  as a mechanism to choose among these high level behaviours. As shown in Fig. \ref{fig:alliance-motivation}, each motivational behaviour had a number of inputs and one output. The output, i.e. the activation level corresponding behavioural set, was activated once a predefined threshold was passed. In the same time, all other behavioural sets became inhibited for allowing that selected behavioural set to complete its task. The input of the behavioural sets was ranged from sensory reading to robot-robot broadcast communication of state-information. 

Internal behaviours e.g. {\em impatience}  and {\em acquiescence} were also used to evaluate the motivation of a robot to select a high-level behaviour set.  Impatience encouraged individual robots to perform a task that was not selected by any other robot of the team and a robot's acquiescence of a task was increased when a robot selected to perform it.  Moreover, robots had the ability to override the inhibitory signal from another robot if a task assigned to other robot was not being completed to a desired level (e.g. when a robot stalled). In case of unsatisfactory self-progress, i.e. not doing any significant progress in a task, robots were able to switch from that task to a different one. 

This system was deployed to a mock hazardous waste clean up and achieved fault-tolerant distributed task performance of the robot team. L-ALLIANCE, an extension of this system was also developed to enable robots  to learn from the observations of a set of task-performance metrics \cite{Parker1995}. 

Similar to  ALLIANCE, multi-agent based task allocation also  use both centralized and decentralized approaches for allocating tasks among  its  peers. \citeasnoun{Shen+2001} presented a detailed categorization where in a multi-agent system task allocation can be done by using various agents ranging from a central supervising agent or a few mediator agents to  all independent agents. 

In case of centralized systems, the central supervisor (or a group of mediators)  must have the necessary system knowledge, e.g. the capabilities and availabilities of all agents and the descriptions of tasks. This scheme gives a well coordinated, consistent and optimized task-allocation but reduces  the reliability, fault-tolerance and scalability of the system. On the other hand, in case of distributed task-allocation, each agent can assign a task {\em directly} to another agent provided that all of them have precise knowledge about others. This approach is very expensive for large number of agents since it requires all agents to have huge processing power and communication bandwidth which is not practical.

Alternatively, agents can know only a few agents and {\em delegate} a task to these known peers so that a suitable agent can be found who has sufficient capabilities and free resources to do this task. This task-allocation by delegation also suffers from poor performance  due to the use of time-consuming search algorithms. This approach also assumes the availability of high communication bandwidth which is not available in large systems. 
%%
\subsubsection*{Market-based approaches}
As a feasible alternative to the above common multi-agent based task-allocation techniques, many researchers have been following the market-based bidding approach \citeasnoun{Dias+2006}. Originated from the Contract-Net Protocol, market-based approach can be implemented as a centralized auctioning system or as a combination of {\em a few auctioneers -- all bidders} or, independently {\em all auctioneers -- all bidders}. For example, in a completely distributed system, when a robot needs to perform a task for which it does not have necessary expertise or resources, it broadcasts a task-announcement message, often with  a expiry time of that message. Robots, that received the message and can perform that task, return a bid message. The initiating robot or {\em  manager} selects one (or more) bidder, called as {\em contractor}, and offers the opportunity to complete the task. The choice of contractor is done by the manager with a mutual agreement with contractor that maximizes the individual profits. 

In market-based approach, high-level communication protocol is necessary to define several types of messages with structured content. In centralized market-based approach there is only one manager that can be an external supervising agent or  one of the robots. While market-based approach consume more resources it usually produces more efficient task-allocations. Anonymous robots can be selected for tasks and these can be different in each bidding cycle.
%
\subsubsection*{Role or value-based task-allocation}
In this type of task-allocation each role assumes several specific tasks and each robot selects roles that best suit their individual skills and capabilities \cite{Chaimowicz2002}. In this case, robots are typically heterogeneous, each one having variety of different sensing, computation and effector capabilities. Here robot-robot or robot-environment interactions are designed as a part of the organization. In multi-robot soccer \cite{Stone+1999}, positions played by different  robots are often defined as roles, e.g. goal-keeper, left/right defender, left/right forwarder etc. The robot, best suited and in closest proximity to available roles/positions, selects to perform that role.
%%
\subsubsection*{Control-theoretic approaches:}
In this type of task-allocation systems, a model of the system is usually developed that converts the task specification into an objective function to be optimized. This model typically  uses  the rigid  body dynamics of the robots assuming the masses and other parameters well-known. Control laws of individual robots are derived either by analytically or by run-time iterations. Unlike most other approaches where task-allocation problem is taken as discrete, control-theoretic approaches can produce continuous  solutions. The formalisms of these systems allow system designer to check the system's controllability, stability and related other properties.  These systems typically use some degree of centralization, e.g. choosing a leader robot.  Example of control-theoretic approach include: multi-robot formation control \cite{Belta+2004}, multi-robot box-pushing \cite{Pereira+2003}  etc.

Predefined task-allocation through few other approaches are also present in the literature. For example, inspired by the vacancy chain phenomena in nature, \citeasnoun{Dahl+2004} proposed a vacancy chain scheduling algorithm for a restricted class of MRTA problems in spatially classifiable domains.
%----------------------------------------------------------------
\subsection{Bio-inspired self-organized task-allocation}
\label{bg:mrta:self-organized}
Task performance in self-organized approaches relies on the collective behaviours resulted from the local interactions of many simple and mostly homogeneous (or interchangeable) agents. Robots choose their tasks independently using the principles of self-organization, e.g. positive and negative feedback mechanisms, randomness (recall Sec. \ref{bg:def:self-reg}). Moreover interaction among individuals and their environment are modulated by the stigmergic, local and broadcast communications (more in Sec. \ref{bg:mrs-comm}).  Among many variants of self-organized task-allocation, most common type is threshold-based task-allocation \cite{Bonabeau+1999}. 

In this approach, a robot's decision to select a particular task depends largely on its perception of stimulus (demand for a task) and its corresponding response threshold for that task. Below, we describe most common forms of threshold-based task-allocation:  deterministic response-threshold and probabilistic response-threshold techniques. Both of them can use the fixed values of response-thresholds or they can adapt their response-thresholds over time based on a suitable learning or adaptation mechanisms.
\begin{equation}
\label{eqn:fixed-response-th1}
\sigma (r,e) = \frac{1}{d(r,e)}
\end{equation}
\begin{equation}
\label{eqn:fixed-response-th2}
\theta_{e} = \frac{1}{\mid D_{r} \mid}
\end{equation}
%%
\subsubsection*{Deterministic response-threshold approach}
In this approach, each robot has a fixed or deterministic activation threshold for each task that needs to be performed. It continuously perceives or monitors the stimulus of all tasks that reflect the relative urgencies of tasks. When a particular task-stimuli exceeds a predefined  threshold the robot starts working on that task and gradually decreases this stimuli. When the task-stimuli falls below the fixed threshold the robot abandons that task. This type of approach has been effectively applied in foraging \cite{Krieger+2000,Liu+2007}, aggregation \cite{Agassounon+2002}. This fixed response-threshold can initially be same for all robots \cite{Jones+200}, or they can be different according robot capabilities or configuration of the system \cite{Krieger+2000}.

In event-handling domain \citeasnoun{kalra+2007} show how task-stimulus can be encoded in mathematical equations. Eq. \ref{eqn:fixed-response-th1} encodes the stimuli of robot $r$ for task-urgency perception event $e$, ($\sigma (r,e)$)  as inversely proportional to the distance between the robot and the event occurring place. Eq. \ref{eqn:fixed-response-th2} gives the threshold value $\theta_{e}$ (based on a predefined distance value $D_{r}$) under which robots select this particular task or event. 

%Adaptive Response-Threshold:\\
Unlike maintaining a fixed response-threshold, adaptive response threshold model changes or adapts the threshold over time. Response-threshold decreases often due to performance of a task and this enables a robot  to select that particular task more frequently or in other words it learns about that task \cite{Bonabeau+1999,Agassounon+2002}.
\begin{equation}
\label{eqn:probl-response-th}
p_{e} = \frac{\sigma (r,e)^n}{\sigma (r,e)^n + \theta_{e}^n}
\end{equation}
%%
\subsubsection*{Probabilistic Response-Threshold}
Unlike deterministic approach, where robots always respond to a task-stimuli that has a largest stimulus above the threshold,  probabilistic approach offers a selection process based-on a probability distribution. For example, in probabilistic response, robots can respond to an event $e$ with probability $p_{e}$, as outlined in Eq. \ref{eqn:probl-response-th}, where $\theta_{e}$ is the threshold and $n$ is the non-linearity of the response. Thus, robots  always have small nonzero probabilities  for all tasks.

In this dissertation we have closely followed this approach with an on-line adaptation mechanism which has been outlined in Chapter \ref{afm}.                                                                                                                                                                                                                                                                                                                                                                                                                                                                                                                                                                                                                                                                                                                            
%------------------------------------------------------------------
\subsection{Comparisons of MRTA solutions}
\begin{figure}
\centering
\includegraphics[width=12cm, angle=0]
{./dia-files/mrta-lines.eps}
%figure caption is below the figure
\caption{ Three major axes of complexities in MRTA}
\label{fig:mrta-complexities} % Give a unique label
\end{figure}
%%
From the vast amount of literature on MRTA, we can easily infer the level of complexities exist in MRTA. In fact researchers generally agree that the MRTA is a {\em NP-hard} problem where optimal solutions can not be found quickly for large problems \cite{Gerkey+2004,Parker2008}. But why do we find so many variants of MRTA solutions? 

In order to answer this question, first we need to look into the contexts from where these MRTA solutions are originated. Most predefined task-allocation solutions are proposed within the context of a known or controlled environment where the modelling of tasks, robots, environments etc. becomes feasible. Note that here tasks can be arbitrarily complex that often require relatively higher sensory and processing abilities of robots. Robot-team can be consists of homogeneous or heterogeneous individuals, having different capabilities based on the variations in their hardware, software etc. But the uncertainty of the environment is assumed to be minimum. 

On the other hand, bio-inspired self-organized MRTA solutions are free from extensive modelling of environment, tasks or robot capabilities. Most of the existing research considers very simple form of one global task e.g. foraging, area cleaning, box-pushing etc. This is due to the fact that major focus of this approach is limited mainly to design individual robot controllers in such a way that a few simple  or {\em specific} tasks can be accomplished. More research is needed to verify the capabilities of self-organized approach in doing multiple complex tasks. At this moment, the bottom line remains as ``select simple robots for simple tasks (self-organized approach) and complex robots for complex tasks (predefined approach)''.

Both of the above task-allocation approaches expose their relative strengths and weaknesses when they are put under real-time experiments with variable number of robots and dynamic tasks. In an arbitrary event handling domain, \citeasnoun{kalra+2007}  compared between self-organized and predefined market-based task-allocation,  where they found that predefined  task-allocation was more efficient when the information was accurate, but threshold-based  approach offered similar quality of allocation at a fraction of cost  under noisy environment.  

\citeasnoun{Gerkey+2003} presented a comparative study of  the complexity and optimality of key architectures, e.g.  ALLIANCE \cite{Parker1998}, BLE \cite{Werger2001}, M+ \cite{Botelho+1999}, MURDOCH \cite{Gerkey+2002}, First piece auctions \cite{Zlot+2002} and Dynamic role assignment \cite{Chaimowicz2002}, all of them relied upon predefined task-allocation methods. The computational and communication requirements of these MRTA solutions were expressed in terms of number of robots and tasks. Although this study does not explicitly measures the scalability of those key architectures, it clearly shows us that many predefined task-allocation solutions will fail to scale well in challenging environments  when the number of  robots and tasks will increase, under the given limited overall communication bandwidth and processing power of individual robots. 

From above discussions we can see that, self-organized task-allocation methods are advantageous as they can provide fully distributed, scalable and robust MRTA solutions through redundancy and parallelism in task-executions. Moreover, the interaction and communication requirements of robots can also be kept under a minimum limit.  Thus  we can say that for large multi-robot system, self-organized task-allocation methods  can potentially be selected, if a system designer can divide his complex tasks into simple pieces that can be carried out by multiple simple robots in parallel with limited communication and interaction requirements.
%%--------------------------------------------------
\subsection{A three-axes taxonomy of MRTA solutions}
\label{bg:mrta:3a-taxonomy}
In order to characterize both predefined and self-organized approaches in terms of their deployment, we propose three distinct axes: 1) organization of task-allocation (X), 2) degree of interaction (Y) and 2) degree of communication (Z). Fig. \ref{fig:mrta-complexities} depicts these axes with a reference point $O$. These axes can be used to measure the complexities involved in various kinds of MRTA problems and the design of their solutions. 

In Fig. \ref{fig:mrta-complexities}, X axis represents the number of active nodes that provides the task-allocation to the group. For example, in any predefined  task-allocation approach, we can use one external centralized entity or one of the robots (aka leader) to manage the task-allocation. In many predefined methods, e.g. in market-based systems,  multiple nodes can act as mediators or task-allocators that we have discussed before. Under predefined task-allocation approach,  a small number of robots can have fully distributed task-allocation where each robot acts as an independent task-allocator (e.g. as discussed before in ALLIANCE architecture). 

Most of the self-organized task-allocation methods are fully distributed, i.e. they allocate their tasks independently without the help of a centralized entity. However, they might be dependent on external entities for getting status or descriptions of tasks. Recent studies on swarm-robotic flocking by \citeasnoun{Celikkanat+2008} show that a swarm can be guided to a target by a few informed individuals (or leaders) while  maintaining the self-organizing principles of task-allocation. Task-allocation of a swarm of robots  just by one central entity may be rare since one of the major spirits of swarm robotic system is to become fully distributed.

In Fig. \ref{fig:mrta-complexities},  Y axis corresponds to the level of robot-robot interaction present in the system. As we have mentioned before in Sec. \ref{bg:mrs:taxonomies}, interaction can be classified into various levels: collective, cooperative, coordinative and collaborative. The presence of interaction can be due to the nature of the problem, e.g. cooperation is necessary in co-operative transport tasks. Alternately, this interaction can be a design choice where interaction can improves the performance of the team, e.g. cooperation in cleaning a work-site is not necessary but it can help to improve the  efficiency of this task. 

Y axis can also be used to refer to the degree of coupling present in the system.  In case of collective interaction, robots merely co-exist, i.e.  they may not be aware of each other except treating others as obstacles. Many other multi-robot systems are loosely-coupled where robots can indirectly infer some states of the environment from their team-mates' actions.  But in many cases, e.g. in  co-operative transport, robots not only recognize others as their team-mates, but also they coordinate their actions. Thus they form a  tightly coupled system. This level of interaction and coupling also gives us the information about potential side-effects of failure of an individual robot. Tightly coupled systems with high degrees of interactions among the robots  suffer from the performance loss if some of the robots removed from the system.

The Z axis of Fig. \ref{fig:mrta-complexities} represents the communication overhead of the system. This can be the result of the interactions  of robots under a given task-allocation method. As we have discussed before various task-allocation methods rely upon variable degrees of robot-robot communications.  On the other hand, the communication capabilities of individual robots can limit (or expand) the level of interaction can be made  in a given group. Thus in one way, considering the interaction requirements of a MRTA problem, the system designer can  select suitable communication strategies that both minimizes the communication overhead and maximizes the performance of the group. And in other way, the communication capabilities of robots can guide a system designer to design interaction rules of robot teams, e.g. the specification of robot's on-board camera  can determine the degree of possible visual interactions among robots. The suitable trade-offs between these two axes: communication and interaction can give us a balanced design of our MRTA solutions.

The central issue of this thesis is to determine the role of communication and sensing strategies under an adaptive response-threshold task-allocation method. So we have focused to examine the benefits of traversing along the various axes of Fig. \ref{fig:mrta-complexities}. In this dissertation, we are interested on two distinct lines: 1) distributed task-allocation, with no direct robot-robot interaction and communication, say line $OX_{n}$ ($n$ being the number of robots)  and 2) distributed task-allocation, with no direct robot-robot interaction, but varying degrees of local communications, say line $X_{n}Z_{l}$  ($Z_{l}$ being a local broadcast communication strategy that involves $l$ number of peers in communication). Our MRTA experiments along $OX_{n}$ and $X_{n}Z_{l}$ can be found in Chapter \ref{afm} and Chapter \ref{local-comm}. The issue of  communication in a multi-robot system is presented in more detail in next section.
%=======================================================================
\section{Communication in multi-robot system}
\label{bg:mrs-comm}
Communication plays an important role for any high-level interaction (e.g. cooperative or coordinative) among a multi-robot team \cite{Arkin1998}. This is not a prerequisite for the group to be functioning, but often act as a useful component of multi-robot system. The characteristics of communication in multi-robot system can be presented in terms of these issues:
\begin{itemize}
\item Rationale of communication: {\em why do the robots communicate?} 
\item Message content: {\em what do they communicate?} 
\item Communication modalities: {\em How do they communicate?} 
\item Target recipients: {\em With whom do they communicate?}
\end{itemize}
Below we have described the above issues with a focus on how communication  can lead to produce effective MRTA solutions.
%-----------
\subsection{Rationale of communication}
From three kinds of communication experiments: indirect stigmergic communication, direct robot-robot state communication, and goal communication, \citeasnoun{Balch2005}  found that in some tasks communication provided performance improvements while others did not. Most of the robotic researchers generally agree that communication in multi-robot system usually provides several major benefits listed below.
\begin{enumerate}
\item \textbf{Improved perception: }
Robots can exchange potential information (as discussed below) based on their spatial position and knowledge of past events. This, in turn, leads to improve perception over a distributed region without directly sensing it.
\item \textbf{Synchronization of actions: }
In order to perform (or stop performing) certain tasks simultaneously or in a particular order, robots need to communicate, or signal, to each other. 
\item \textbf{Enabling interactions and negotiations: }
Communication can help a lot to influence each-other in a team that, in turn, enables robots to interact and negotiate their actions effectively.
\end{enumerate}
%%
\begin{figure}
\begin{minipage}[t]{0.48\linewidth}
\centering
\includegraphics[width=6cm, height=4cm, angle=0]
{./photos/s-bots-comm-evolve-300x214.eps}
\caption{A team of s-bots communicating by light signals.\protect\newline From http://lis.epfl.ch, last seen on 01/06/2010.}
\label{fig:robots-comm-light}
\end{minipage}
\hspace{0.5cm}
\begin{minipage}[t]{0.48\linewidth}
\centering
\includegraphics[width=6cm,height=4cm, angle=0]{./photos/robots_cs_utk_edu_balajee.eps}
\caption{A fleet of robots relying on camera (vision) for search operation. From http://www.cs.utk.edu, last seen on 01/06/2010.}
\label{fig:robots-comm-camera} 
\end{minipage}
\end{figure}
%----------------- 
\subsection{Information content}
Although communication provides several benefits for team-work it is costly to provide communication support in terms of hardware, firmware as well as run-time energy spent in communication. So robotic researchers carefully minimize the necessary information content in communications by using suitable communication protocols and high-level abstractions. For example in foraging, grazing and consuming experiments \citeasnoun{Balch+1994} introduced state and goal communications. In state communication, a single bit is transmitted indicating the current state of robot (e.g. 0 transmitted when robot was in {\em Wander} state and 1 transmitted when robot was in {\em Acquire} or {\em Retrieve} states). In case of goal communication, the location of task was also transmitted. Here is a brief summary of potential information contents used in communication among robots.
%%
\begin{itemize}
\item \textbf{Individual state:} ID number, battery level, task-performance statistics, e.g. number of tasks done.
\item \textbf{Goal:} Location of target task or all tasks discovered.
\item \textbf{Task-related state:} The amount of task completed, number of other robots present there etc.
\item \textbf{Environmental state:} Free and blocked paths, level of interference found, any urgent event or dangerous changes found in the environment.
\item \textbf{Intentions:} Detail plan for doing a task, sequences of selected actions etc.
\end{itemize}
%% 
 Since a multi-robot system can be comprised of robots of various computation and communication capabilities, these information contents can vary greatly based on their individual communication modules and channel capacities.
%------------------------------------------
\subsection{Communication modalities}
\begin{figure}
\centering
\subfloat[ E-puck robots with table-lamps]{\includegraphics[width=6cm, height=4cm]{./photos/distributed_table_lamp_triangle.eps}} 
\hspace{0.25cm}
\subfloat[Sniffing Khepera III]{\includegraphics[width=6cm, height=4cm]{./photos/odor_loc_khepera3odorprototype300x169.eps}}
\caption{ (a) A fleet of mobile ``lighting" robots moving on a large table, such that the swarm of robots form a distributed table light and (b) Distributed odour source localization by Khephera robot equipped with volatile organic compound sensor and an anemometer (wind sensor). From http://http://disal.epfl.ch, last seen~01/06/2010.}
\label{fig:epfl-disal}
\end{figure}
%%
%%
Robotic researchers typically use robot's on-board wireless radio, infra-red (IR), vision and sound hardware modules for robot-robot and robot-host communication. The reduction in price of wireless radio hardware chips e.g. wifi (ad-hoc WLAN 802.11 network) or Bluetooth\footnote{www.bluetooth.com}makes it possible to use wireless radio communication widely. In-expensive IR communication module is also typically built into almost all mobile robots due to its low-cost and suitability for ambient light and obstacle detection. IR can also be used for low bandwidth communication in short-range, e.g. keen-recognition. Most robots can also produce basic sound waves and detect it with their built-in speakers and suitable configuration of on-board microphones. Although speech-recognition is not commonly found in mobile robots yet, detection of pre-recorded sound waves can be feasible.

Most of the mobile robots come with a series of LEDs, and tiny camera that can emit light signals and detect it with camera. Fig. \ref{fig:robots-comm-light} shows the robot-robot communication through the red and green coloured LEDs. Many robots can also detect blobs of colours and can recognize peers of other objects through the use of a color-coded markers. Fig. \ref{fig:robots-comm-camera} shows a team of robots with colour-coded markers attached on it that can be detected by the on-board camera of other robots. Some other researchers also tried to establish communication among robots solely relying on vision \cite{Kuniyoshi1994}. Although a lot of researches have been carried out to design robot skin and tactile communication system, we do not know any instance of tactile communication used in multi-robot team. In terms of chemical communication, \citeasnoun{Lochmatter+2007} showed  limited success in odour-source localization, a form of detecting chemical signals.
%-----------------------------
\subsection{Communication strategies}
\begin{figure}
\centering
\includegraphics[width=9cm, angle=0]
{./dia-files/mrs-comm-strategies.eps}
%figure caption is below the figure
\caption{Three aspects of communication in multi-robot system}
\label{fig:mrs-comm-strategies} % Give a unique label
\end{figure}
Whatever be the communication modalities of a multi-robot system, suitable strategies are required to disseminate information in a timely manner to a target audience that maximizes the effective task-completion and minimizes delays and conflicts. A review of various communication strategies in social system has been presented in Sec. \ref{bg:def:comm}. Here, in order to discuss the complexities of communication strategies we have selected three independent scales: organization, expressiveness and range of communication. With these independent aspects, we can measure the level of complexities in communication and classify a multi-robot system according to its communication characteristics. Fig. \ref{fig:mrs-comm-strategies} outlines these scales and we have described them  below.
%%
\subsection*{Centralized and decentralized communications}
Similar to the organization of task-allocation discussed in Sec. \ref{bg:mrta:3a-taxonomy}, communication in a multi-robot system can be organized using an external/internal central entity (e.g. a server PC, or a leader robot) or, a few leader robots, or by using decentralized or local schemes where every robot has the option to communicate with every other robot of the team. From a recent study of multi-robot flocking \citeasnoun{Celikkanat+2008} have shown that a mobile robot flock can be steered toward a desired direction through externally guiding some of its members, i.e. the flock relies on multiple leaders or information repositories. Note that here task-allocation is fully decentralized i.e. each robot selects its task, but the communication structure is hybrid; robots communicate with each other and with a centralized entity.
%%--------------------------------
\subsection*{Explicit and implicit communications}
Communication in a multi-robot system can also be characterized its expressiveness or the degree of explicitness. In one extreme it can be fully implicit, e.g. stigmergic, or on the other end, it can be fully explicit where communication is done by a rich vocabulary of symbols and meanings. Researchers generally tend to stay in either end based on the robotic architecture and task-allocation mechanism used. However, both of these approaches can be tied together under any specific application. They are highlighted below.
%%
\begin{enumerate}
\item \textbf{Explicit or direct communication: }
This is also known as intentional communication. This is done purposefully by usually using suitable modality e.g. wireless radio, sound, LEDs. Because explicit communication is costly in terms of both hardware and software, robotic researchers always put extra attention to design such a system by analysing strict requirements such as communication necessity, range, content, reliability of communication channel (loss of message) etc.
%
\item \textbf{Implicit or indirect communication:} 
This is also known as indirect stigmergic communication. This is a powerful way of communication where individuals leave information in the environment. This method was adopted from the social insect behaviour, such as stigmergy of ants (leaving of small amount of pheromone or chemicals behind while moving in a trail).
\end{enumerate}
%%
%%--------------------------
\subsection*{Local and Broadcast communications}
The target recipient selection or determining the communication range or sometimes called radius of communication is an interesting issue in multi-robot system research. Researchers generally tries to maximize the information gain by using larger range. However, transmission power and communication interference among robots play a major role to limit this range. The following major instances of this strategy can be used.
\begin{itemize}
\item \textbf{Global broadcast communication:} where all robots in the team can receive the message.
\item \textbf{Local broadcast communication:} where a few robots in local neighbourhood can receive the message.
\item \textbf{Publish-subscribe communication:} where only the previously subscribed robots can receive the message.
\item \textbf{P2P communication:} where only the closest peer robot can receive the message.
\end{itemize}
%----------------------------------------------------------
\subsection{Key issues in multi-robot communication}
\label{bg:mrs-comm:key-issues}
Several important issues related to communication among a multi-robot team have identified since last decade. Some of the major issues are discussed here.
\subsubsection*{Determination of local neighbourhood}
Most researchers in the area of swarm-robotic system, who use algorithms based on local-neighbourhood of communication, face this problem of defining the range of local neighbourhood. \citeasnoun{Agah+1995} presented that larger communication range is not always optimum for some types of tasks e.g. exploration where a large number of recipient robots decreased the performance of exploration task. \citeasnoun{Yoshida+2000} provided a design of optimal communication range of homogeneous robots based on their spatial and temporal analyses of information diffusion within the context of cooperative tasks in a manufacturing shop-floor. Spatial design tried to minimize the time for information transmission and temporal design tried to minimize the information announcing time to avoid excessive information diffusion. Eq. \ref{egn:yoshida-range} describes their optimal range $\chi_{optimal}$ as a function of information acquisition capacity of robots ($c$) and the probability of information output of a robot ($p$). Here $c$ is an integer representing the upper-limit of the number of robots that can be the target recipients at any time without the loss of information and $\chi_{optimal}$ gives the average number of robots within the output range.
\begin{equation}
\chi_{optimal} = \frac{\sqrt [c] {c!}}{p}
\label{egn:yoshida-range}
\end{equation}
%%
\subsubsection*{Kin recognition}
Kin recognition refers to the ability of a robot to recognize immediate family members by implicit or explicit communication or sensing \cite{Mataric2007}. In case of multi-robot system, this can be as simple as identifying other robots from objects and environment or as finding team-mates in a multi-robot soccer. This is an useful ability that helps interaction, such as cooperation among team members avoiding opponents. 
%%
\subsubsection*{Representation of languages}
In case of effective communication several researchers also focused on representation of languages and grounding of these languages in physical world. Implicit communication generally has no or very little necessity of symbol grounding. In foraging experiments \citeasnoun{Liu+2007} used certain cues or events to adapt the response-thresholds  of robots. For example, successfully retrieving food makes a robot keep foraging and colliding with other robots makes a robot more likely to rest. These simple cues are based on dynamic events but they can hardly be adopted in complex tasks and environments. Explicit communication often uses custom hand-crafted messages by a set of symbols. Based on a shared vocabulary, they can provide the necessary meaning for the robots. For new kinds of messages one always needs to modify the symbols and shared vocabulary. In such cases knowledge representation techniques and tools can be useful to some extent \cite{Parker2008}.
%%
\subsubsection*{Fault-tolerance and reliability}
Since every communication channel is not free from noise and corruption of messages significant attention has been also given to manage these no communication situations, such as by setting up and maintaining communication network, managing reliability and adaptation rules when there is no communication link available. In terms of guaranteeing communication, researchers also tried to find ways for a deadlock free communication methods, such as signboard communication method \cite{Wang1989}.
%%-------------------------------------------------
\subsection{Role of communication in MRTA}
Although researchers in the field of multi-robot system have been adopting various communication strategies for achieving MRTA solutions in different task domains, very few studies correlate the role of communication with the effectiveness of MRTA. This is due to the fact that researchers usually adopt a certain task-allocation method and they limit their use of communication strategy to either explicit global/local broadcast (in most predefined task-allocation researches) or  implicit/no communication (in most self-organized task-allocation researches). In the former one, communication becomes the part and parcel of the robot-robot interactions that enable them to exchange variety of information as discussed before. But in the latter one, the environment serves as a shared memory for all robots to access information or sense the current state of the environment, mostly locally. Here we have attempted to scrutinize how MRTA solutions have been affected by the variations in communication strategies.

As mentioned in Sec. \ref{bg:mrta:self-organized}, \citeasnoun{kalra+2007} empirically studied the comparative performance of MRTA under both predefined and self-organized approaches with event-driven simulations. They found that the accuracy of information is crucial for predefined market-based approach where every robot communicated with every other robot (i.e. global broadcast). In case of unreliable link or absence of communication, threshold-based approach performed same as market-based approach, but with less computational overhead. This indicates the dependence of predefined approach on reliable communication links.However their global broadcast strategy is not feasible for large teams of real-robots. In case of varying robot's communication range,  they found that market-based approach performed well for a short communication-range where robots were able to communicate with less than a third of the total number of team-mates. Since the events are handled based-on their spatial locations only, it is not clear how this strategy  will perform in other task-domains.  
  
In order to pursue MRTA, robots can receive information from a centralised source \cite{Krieger+2000} or from their local peers \cite{Agassounon+2002}. This centralized communication system is easy to implement. It simplifies the overall design of a robot controller. However as we mentioned before, this system has disadvantage of a single point of failure and it is not scalable. The increased number of robots and tasks cause inevitable increase in communication load and transmission delay. Consequently, the overall system performance degrades. On the other hand, uncontrolled reception of information from decentralized or local sources is also not free from drawbacks. If a robot exchanges signals with all other robots, it might get the global view of the system quickly and can select an optimal or near optimal task. This can produce a great improvement in overall performance of some types of tasks e.g., in area coverage \cite{Rutishauser+2009}. But this is also not practical and scalable for a typically large multi-robot system due to the limited communication and computational capabilities of robots and limited available communication bandwidth of this type of system.

A potential alternate solution to this problem can be obtained by decreasing the number of message recipients on the basis of a local communication range. This means that robots are allowed to communicate only with those peers who are physically located within a pre-set distance. When this strategy is used for sharing task information among peers, MRTA can be more robust to the dynamic changes in the environment and energy- efficient \cite{Agassounon+2002}. Simailar to this, \citeasnoun{Pugh+2009} reported a distributed multi-robot learning scenario with two cases: 1) robots were allowed to communicate with any two other robots ({\em Model A}) and 2)  robots were allowed to communicate with all robots in a fixed radius ({\em Model B}). In simulation and real robotic experiments with 10 robots and communication ranges of 0.3 m, 1.0 m and 3.3 m, they showed that Model B performed better in intermediate communication range. However, these learning process of individual robot controller were  conducted  in static environment.  So it is not clear why intermediate communication range performs better than other ranges. 

Many robotic researchers tried to use some forms of adaptation rules in local communication to avoid saturation of the communication channel, e.g. based on robot densities in a given area.  As mentioned before,  \citeasnoun{Yoshida+2000} tried to formalize the suitable communication range based on spatial and temporal properties of information diffusion of a given communication channel. The major focus of this type of research is to measure the cost of communication based on some metrics, e.g.  transmission time and collisions with other robots, and then regulate communication strategies or ranges dynamically ranging from global broadcast to local P2P or not doing communication at all when a huge cost is involved. These ideas are attractive to maximize information gain in dynamic environment, but there is no point of doing communication if there is little or no task-requirement. Thus we find this approach, i.e. maximizing information gain,  is not always useful or necessary for effective MRTA.

\cite{Oca+2005} also acknowledged the above fact within the context of their ant-based clustering experiments. They used two simple communication strategies: 1) simple memory sharing by robots (shared memory access) and 2) shared use of environment maps (global sensing). In both of these cases, it was found that communication is only useful when some initial random clustering phase was passed. The accuracy of shared information in highly dynamic environment was poor and did not carry any significant advantage. In case of local memory sharing by robots, they showed that  sharing information within a limited number of robots produced more efficient  clusters, rather than  not sharing information at all in stigmergic communication mode. However, sharing memory in a large group is not a feasible communication strategy because of the huge latencies and interferences involved in the communication channel.
%=====================================================================
\section{Application of MRS in automation industry}
\label{bg:mrs-industry}
Automation industry, particularly automated manufacturing domain, provides an excellent area where MRTA problem can be studied and applied. Most of the research in this area is inspired by intelligent multi-agent technology \cite{Shen+2006}. A few other researchers also tried to apply the concepts of biological self-organization \cite{Ueda2006,Lazinica+2007}. In this section we have reviewed these concepts and technologies mainly focusing on physical embodiment of agents, i.e., the use of multiple mobile robots or  AGVs. 
%%
\subsection{Multi-agent based approaches}
Since early 80s, researchers have been applying agent technology to manufacturing enterprise integration, manufacturing process planning, scheduling and shop floor control, material handing and so on. An agent is a software system that communicates and cooperates with other software systems to solve a complex problem that is beyond the capability of each individual agent's software system \cite{Shen+2001}. Most notable capabilities of agents are autonomous, adaptive, cooperative and proactive. There exists many different extensions of agent-based technologies such as holonic manufacturing system \cite{Bussmann+2004}. Here, {\em holon} refers to an autonomous and cooperative unit of manufacturing system for transporting, transforming, sorting and/or validating information and physical objects.
%% FIG: Kiva systems
\begin{figure}
\centering
\includegraphics[width=12cm, angle=0]
{./photos/Kiva-Systems.eps}
\caption{A multi-robot material handling system from Kiva systems Inc. From http://www.kivasystems.com/, last seen on 06/06/2010.}
\label{fig:kiva-systems}
\end{figure}

Agent based technologies have addressed many of the problems encountered by the traditional centralized manufacturing methods. They can respond to the dynamic changes and disturbances through local decision making. The autonomy of individual resource agents and loosely coupled network architecture provide better fault-tolerance. The inter agent distributed communication and negotiation also eliminate the problem of having a single point of failure of a centralized system. These facilitate a manufacturing enterprise to reduce their response time to market demands in globally competitive market. 

Despite having so many advantages, agent-based systems are still not widely implemented in the manufacturing industry comparing to the other similar technologies, such as distributed objects and web-based technologies due to the lack of integration of this systems with other existing systems particularly real-time data collection system, e.g., radio frequency identification system. Another barrier is the increased cost of investment in exchange of some additional flexibility and throughput \cite{Schild+2007}.
%%
\subsubsection*{A case study:  Kiva material handling system.}
 As a notable exception in multi-robot automation application, the Kiva material handling system\footnote{http://www.kivasystems.com/}  has revolutionized the traditional warehouse order-processing jobs by replacing the old-style relatively expensive AGVs with cheaper mobile robots \cite{Wurman+2008}. As shown in Fig. \ref{fig:kiva-systems} hundreds of mobile robots are working in a warehouse where thousands of customer-orders are handled in real-time. The major advantages of this system over traditional material handling systems are as follows.
 \begin{itemize}
\item Multiple orders are handled in parallel with random access to all items. This greatly simplifies the ware-house operations.
\item Each worker does his job independently. No inter-dependency among workers exists which is very common in a sequential order processing system.
\item Instead of having 5-10 expensive AGVs, now a company can have 30-50 mobile robots  that greatly increases the productivity of the warehouse and reduces operational cost significantly.
\item Real-time order processing removes any need for batch processing.
\item All major benefits of a distributed system including: no single point of failure, spatial flexibility and better expandability.
\end{itemize}
%
Kiva material handling system has been implemented using Java-based multi-agent and AI techniques. It uses a centralized  resource and task-allocation unit which assigns jobs to a mobile robot's drive-unit agent and resources to a worker's inventory-station agent using utility-based heuristics, similar to the predefined task allocation approach. Here utility is measured in terms of the cost to the warehouse owner. Spatial location of robots and inventory stations, specific task skills of certain stations, commonalities among orders etc. are taken into account to find the lowest utility cost and to make allocation decisions on-the-fly. 

The control architecture of robotic drive units follows typical three layer hybrid control strategy \cite{Simmons+2002}. In order to locate resources and collectively\footnote{Recall the collective interaction  from Sec. \ref{bg:mrs:taxonomies} where agents are unaware of each-other, share common goals and their actions are beneficial to their team-mates.} process customer orders, robotic agents communicate with each other via XML messages. About 100 types of messages have been designed to meet this communication need.

At the  time of writing this dissertation,  the base price of this system with 30-50 robotic drive units has been estimated as \$1M US dollars and this system has been installed about 10-20 warehouses with more than a thousand robotic units being operational \footnote{http://www.roboticstrends.com/, last accessed on 10/06/2010.}.  Interested readers may find more on Kiva material handling system and its technical implementation in \citeasnoun{Wurman+2008}.
%-----------------------------------------------------------
\subsection{Biology-inspired approaches}
The insightful findings from biological studies on insects and organisms have directly inspired many researchers to solve problems of manufacturing industries in a biological way. These can be categorized into two groups: one that allocates task with explicit  potential fields and another that allocate tasks without specifying any potential field. Below we have discussed both types of \acf{BMS}.
%%
\subsubsection*{Explicit potential filed based BMS}
The biological evidences of the existence of potential field between a task and an individual worker such as, a flower and a bee, a food source and an ant, inspired some researchers to conceptualize the assigning of artificial potential field between two manufacturing resources. For example, potential field is assumed between a machine that produce a material part and a worker robot or AGV that manipulates the raw materials and finished products.

\citeasnoun{Ueda2006} conceptualized this potential field as the attractive and repulsive forces based on machine capabilities and product requirements. Task allocation is carried out based on the local matching between machine capabilities and product requirements. Each machine generates an attractive field based on its capabilities and each robot can sense and matches this attractive filed according to the requirements of a product. potential field is a function of distance between entities. Here, self-organization of manufacturing resources occurred by the process of matching the machine capabilities and requirements of moving robots. Through computer simulations and a prototype implementation of a line-less car chassis welding \cite{Ueda2006} found that this system was providing higher productivity and cost-effectiveness of manufacturing process where frequent reconfiguration of factory layout was a major requirement. This approach, was also extended and implemented in a supply chain network and in a simulated ant system model where individual agents were rational agents who selected tasks based on their imposed limitations on sensing.
%% 
\subsubsection*{BMS without explicit potential fields}
Several other researchers did not express the above potential field for task allocation among manufacturing resources explicitly, rather they stressed on task selection of robots based on the task-capability broadcasts from the machines to the worker robots. In case of \cite{Lazinica+2007}, task capabilities are expressed as the required time to finish a task in a specific machine. They used assigned priority levels to accomplish the assembly of different kinds of products in the computer simulation of their bionic manufacturing system. In another earlier computer simulated implementation of swarm robotic material handing of a manufacturing work-cell, \cite{Doty+1993} pointed out several pitfalls of such a BMS system, such as dead-lock in manufacturing in inter-dependant product parts, unpredictability of task completion, energy wastage of robots wandering for tasks etc. Although most of these problems remain unsolved researchers are still exploring the concepts BMS in order to achieve a higher level of robustness, flexibility and operational efficiency in a highly decentralized, flexible, and globally competent next generation automated manufacturing system.\\
%%
In this dissertation, we have considered a manufacturing shop-floor scenario where robots are required to attend mock production and machine-maintenance jobs in different machines in an arena using their {\em homing} behaviours. Our study significantly differs from the above works since our generic framework of self-regulation considers, not only the spatial distances to tasks and their dynamic urgencies but also the learning and forgetting of agents about tasks. Moreover, unlike the above deterministic approaches, our approach uses a probabilistic method for task allocation. Detail features of our generic framework of self-regulated DOL is discussed in Chapter \ref{afm}. 