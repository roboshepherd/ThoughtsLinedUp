\chapter{Background}
%%%%%%%%%%%%%%%%%%%%%
\section{Definition of Key Issues and Basic Terms}
\subsection{Self-regulation}
Self-regulation (SR) primarily refers to the exercise of control over oneself to bring the self into line with preferred standards \cite{Baumeister+2007}. One of the most notable self-regulatory process is the human body's homoeostatic process where the human body's inner process seeks to return to its regular temperature when it gets overheated or chilled. Baumeister et al. has referred self-regulation to goal-directed behaviour or feedback loops, whereas self-control may be associated with conscious impulse control. Since regulation carries  the meaning of  control with a hint of regularity.  

SR is not only the regulation of the self, but also it carries the meaning of regulation by the self (consciously and/or unconsciously). In psychology, SR denotes the strenuous actions to resist temptation or to overcome anxiety.  SR is also divided into two categories: 1) conscious and 2) unconscious SR.  Conscious SR puts emphasis on conscious, deliberate efforts in self-regulation. On the other hand, unconscious self-regulation refers to the automatic self-regulatory process  that is although not nearly as labour intensive, but operate in harmony with unpredictable, unfolding events in the environment, using and transforming the available informational input in ways that help to attain an activated goal.

The concepts of SR is also commonly used in cybernetic theory where SR in inanimate mechanisms shows that they can regulate themselves by making adjustments according to programmed goals or set standards. A common example of this kind can be found in a thermostat that controls a heating  and cooling system to maintain a desired temperature in a room. In that sense, SR covers both aspects of monitoring ones own state in relation to the goal and making adjustments with respect to the changes found. 

In physics, chemistry, biology and some other branches of natural sciences, the concept of SR is centered around the study of self-organizing individuals. The term self-organization (SO) has a broader and somewhat different meaning in different places of literature. 
%!!!!
For example, Bonabeau et al. \cite{Bonabeau+1999} has summarized the ingredients and properties of  self-organization modelling after the social insects. The four ingredients of self-organization are:
positive feedback or amplification, 
negative feedback, 
reliance on amplification of fluctuations ( randomness of  tasks, error etc.) and 
reliance on multiple interactions. 
In many cases, the concepts of self-organization and self-regulation share many properties, especially when they are considered in the context of division of labor, formation of emergent structures and behaviours etc. For example,  SO is observed in ant colonies and other similar biological species. In the process of SO, we can find self-regulating properties of the insects that give rise to the spatio-temporal  structures, or nests, existence of multiple stable states and bifurcation with respect to the dynamic changes of the environment. In the process of SO, we can find self-regulating properties of the insects that give rise to the spatio-temporal structures, or nests, existence of multiple stable states and bifurcation with respect to the dynamic changes of the environment.
%% In SECTION <>, a more closer look on SO has been given to find out interesting phenomena of emergent self-regulation in nature.   

SR has also been studied in the context of social systems where it originates from the division of social labor that creates SO process that has self-regulating effects \cite{Kppers+1990}. Two types of SR have been reported in many places of literature of sociology: 1) SR from SO and 2) SR from activities of components in a heterarchical organization. It is interesting to note that self-regulation in biological species provides the similar evidences of bottom-up approach of self-regulation of heterarchical organization through interaction of individuals or the absence of strict hierarchy. In the study of sustainable or viable system, self-regulation gets much attention in organizational cybernetics. Many researchers are trying to put Stafford Beer's viable system model \cite{Beer1981} into social organizations for a bottom-up self regulation of an organization.  
%% SECTION <> focuses on self-regulation in human social systems.
It is interesting to note that self-regulation in biological species provides similar evidences of bottom-up approach of self-regulation of heterarchical organization through interaction of individuals in the absence of or in parallel with strict hierarchy. 

From the above discussion, it has been shown that the term self-regulation carries a  wide range of meaning in different branches of knowledge. In psychology and cognitive neuroscience point of view, self-regulation is discussed in an individual's perspective whereas, in biological and social contexts the SR is discussed in a context of group of individuals or the society as a whole.  From this initial study, it can be said that  to study emergent SR,  the latter context is more appropriate where emergence is evident.  For a better understanding of emergent self-regulation, we need to look into the process of emergence from the view point of interactions of large individuals in a society.  
% SECTION <> deals with this topic.  

\subsection{Emergence}
In order to understand emergent self-regulation, it is necessary to find out how emergence and emergent properties are described in the literature. According to Encyclopedia Britannica emergence is the rise of a system that cannot be predicted or explained from antecedent conditions. Emergence occurs when interactions among objects at one level gives rise to different types of objects at another level. More precisely, a phenomenon is emergent if it requires new categories to describe the behavior of the underlying components \cite{Andriani+2004}, \cite{Kppers+1990}. 

Emergence is the central concept to the theory of complex systems which consists of a large number of elements, non-linear dynamic interactions among elements and their environment, feedback loops etc. Paul Cilliers \cite{Andriani+2004} summarized the properties of  complex systems as follows:
Complex systems consist of large number of elements such that a system of differential equation becomes impractical.
The elements have to interact and this interactions must be dynamic, changing with time. 
The interaction is fairly rich so that elements are influenced by each other. These interactions are non-linear where small causes can have large results and vice versa.
The interactions usually have a fairly short range, i.e. Information is received primarily from immediate neighbors. Long range interaction is also possible which can be modulated along the way.
There are loops in the interactions. The effect of any activity can feed back onto itself, sometimes directly, sometimes after a number of intervening stages. The feedback can be positive (enhancing, stimulating) or negative (detracting, inhibiting).
Complex systems are usually open systems, i.e. they interact with their environment.
Complex systems operate under conditions far from equilibrium. There has to be constant flow of  energy to maintain the organization of the system and to ensure its survival.
Complex systems have histories. No only do they evolve through time, but also their past is co-responsible for their present behavior.
Each element in the system is ignorant of the behavior of the system as a whole, it responds only to the information that is available to it locally. The complexity emerges as resultant of the patterns of interaction between the elements.


The common characteristics of emergence in complex systems \cite{Corning2002} are typically described as follows: 
\begin{enumerate}
\item radical novelty or features not previously observed in systems; 
\item coherence or correlation or meaning integrated wholes that maintain themselves over some period of time; 
\item  A global or macro "level" i.e. there is some property of "wholeness"; 
\item it is the product of a dynamical process that evolves; and 
\item it can be perceived.  
\end{enumerate}

Therefore, the concepts of emergent SR can be referred to the bottom-up or decentralized control by which large number of individuals perform robust SO through local non-linear interactions that give rise to the novel and coherent structures, patterns and properties during the SO process. 

\subsection{Division of Labour}
Encyclopaedia Britannica serves the definition of division of labour as the separation of a work process into a number of tasks, with each task performed by a separate person or group of persons. Originated from economics and sociology the term division of labour is widely used in many branches of knowledge. As mentioned by Adam Smith (1776).
The great increase of the quantity of work which, in consequence of the division of labour, the same number of people are capable of performing, is owing to three different circumstances; first, to increase the dexterity in every particular workman; secondly, to the saving of the time which is commonly lost in passing from one species of work to another; and lastly, to the invention of a great number of machines which facilitate and abridge labour, and   enable one man to do the work of many. (Adam Smith (1776) in \cite{Sendova-Franks+1999})

In the study of biological insects, a worker usually does not perform all tasks, but rather specializes in a set of tasks, according to its morphology, age, or chance \cite{Bonabeau+1999}. This division of labour among nest-mates, whereby different activities are performed simultaneously by groups of specialized individuals, is believed to be more efficient than if tasks were performed sequentially by unspecialised individuals. Division of  labour has a great plasticity where the removal of one class of workers is quickly compensated for by other workers.

In sociology, division of labour usually denotes the work specialization \cite{Sayer+1992}. Basically it answers three questions:
\begin{description}
\item[1)What task?] i.e., the description of the tasks to be done, service to be rendered or products to be manufactured.
\item[2)Why dividing it to individuals?] i.e., the underlying social standards for this division, such as task appropriateness based on class, gender, age, skill etc.
\item[3)How to divide it?] i.e.,the method or process of separating the whole task into small pieces of tasks that can be performed easily. 
\end{description}
% 
%%%%%%%%%%%%%%%%%%%%%%%%%%%%%%%%%%%%%%%%%%%%%%%%%%%%%%%%%%%%%%%%%%%
\section{Overview of Multi-robot Systems}

%%%%%%%%%%%%%%%%%%%%%%%%%%%%%%%%%%%%%%%%%%%%%%%%%%%%%%%%%%%%%%%%%%%
\section{Task-Allocation in Multi-robot Systems}
Since 90s multi-robot task allocation (MRTA) is a common research challenge that tries to define the preferred mapping of robots to tasks in order to optimize some objective functions  \cite{Gerkey+2004}. Many MRS control architectures have been solely designed to address this task-allocation issue. In 2003 Gerkey et al. formally analysed the complexity and optimality of key architectures (e.g., ALLIANCE, BLE, M+, MURDOCH, First piece auctions and Dynamic role assignment) for this MRTA issue and it has been found that MRTA is an instance of the so-called optimal assignment problem \cite{Gerkey+2003} and generally known as NP-hard where optimal solutions can not be found quickly for large problems \cite{Gerkey+2004}. 
If we look the MRTA problem from multi-agent system's perspective we can find it is broadly divided into two major categories \cite{Shen+2001}: 
\begin{enumerate}
\item Predefined (off-line) task-allocation and 
\item Emergent (real-time) task-allocation. 
\end{enumerate}
% (Insert a Fig)
Usually predefined task allocation method uses either centralized coordination or distributed task-allocation approach. Distributed predefined task-allocation approach is again subdivided into three subcategories: 
\begin{enumerate}
\item Direct allocation, 
\item Task allocation by delegation 
\item Task allocation through bidding
\end{enumerate}
In MRS domain, early research on predefined distributed task-allocation approach has been dominated mainly by intentional coordination \cite{Gerkey+2004,Parker1998}, the use of dynamic role assignment e.g., \cite{Chaimowicz2002}, and market-based bidding approach \cite{Dias+2006}. In intentional coordination e.g., \cite{Parker1998}, robots uses direct allocation method to communicate and to negotiate for assigning tasks. This is preferred approach among MRS research community since it is easily understood, easier to design, implement and analysis formally. Task allocation through bidding is mainly based on the Contract Net Protocol \cite{Davis1988+}. Predefined Task allocation through other approaches are also present in literature. For example, inspired by the vacancy chain phenomena in nature, \cite{Dahl+2003} proposed a vacancy chain scheduling (VCS) algorithm for a restricted class of MRTA problems in spatially classifiable domains.

On the other hand emergent task-allocation approach relies on the emergent group behaviours e.g., \cite{Kube+1993}, such as emergent cooperation \cite{Lerman+2006}, adaptation rules \cite{Liu+2007} etc., that lead to task allocation with local sensing, local interactions. It typically uses little or no explicit communication or negotiations between robots. They are more scalable to large team size and more robust via parallelism and redundancy.

MRTA problem can be addressed in many different ways depending upon the paradigm selected to abstract the problem and its relevant constraints and requirements \cite{Parker2008}. Firstly, in emergent task allocation in bioinspired swarms paradigm MRTA homogeneous robots are employed to perform mostly similar tasks only by local sensing and indirect stigmergic ( or no) communication. Secondly, in organizational and social paradigm MRTA can follow one of the two major approaches: 1) task allocation by making use of roles and 2) task allocation through bidding. For example, in multi-robot soccer, each role encompasses several specific tasks and heterogeneous robots select their roles based on their position and capabilities. In market-based approach, robots can negotiate with other team-mates to collectively solve a set of tasks. Finally, in knowledge-based approach, also known as intentional coordination, MRTA is done through the modelling of team-mate capabilities, such as by observing the performance of other team-members performance with or without explicit communication.
%%%%%%%%%%%%%%%%%%%%%
\section{Communication in Self-regulated Biological Social Systems}
Communication plays a central role in self-regulated division of labour of biological societies.In this section communication among biological social insects are briefly reviewed within the context of self-regulated  division of labour.

%[Purposes]
Communication in biological societies serves many closely related social purposes. Most peer-to-peer (P2P) communication include: recruitment to a new food source or nest site, exchange of food particles, recognition of individuals, simple attraction, grooming, sexual communication etc. In addition to that colony-level broadcast communication include: alarm signal, territorial and home range signals and nest markers, communication for achieving certain group effect such as, facilitating or inhibiting  a group activity \cite{Holldobler1990}.
%
%[Modalities and Ranges]
Biological social insects use different modalities to establish social communication, such as, sound, vision, chemical, tactile,  electric and so forth.  Sound waves can travel a long distance and thus they are suitable for advertising signals. They are also best for transmitting complicated information quickly \cite{Slater1986}. Visual signals can travel more rapidly than sound but they are limited by the physical size or line of sight of an animal. They also do not travel around obstacles. Thus they are suitable for short-distance private signals such as in courtship display. 
In ants and some other social insects chemical communication is dominant. Any kind of chemical substance that is used for communication between intra-species or inter-species is termed as semiochemical \cite{Holldobler1990}. A pheromone is a semiochemical, usually a glandular secretion, used for communication within species. One individual releases it as a signal and others responds it after tasting or smelling it. Using pheromones individuals can code quite complicated messages in smells. For example a typical an ant colony operates with somewhere between 10 and 20 kinds of signals \cite{Holldobler1990}. Most of these are chemical in nature. If wind and other conditions are favourable,  this type of signals emitted by such a tiny species can be detected from several kilometres away. Thus chemical signals are extremely economical of their production and transmission. But they are quite slow to diffuse away. But ants and other social insects manage to create sequential and compound messages either by a graded reaction of different concentrations of same substance or by blends of signals.
Tactile communication is also widely observed in ants and other species typically by using their body antennae and forelegs. It is observed that in ants touch is primarily used  for receiving information rather than informing something. It is usually found as an invitation behaviour in worker recruitment process. When an ant intends to recruit a nestmate for foraging or other tasks it runs upto a nestmate and beats her body very lightly with  antennae and forelegs. The recruiter then runs to a recently laid pheromone trail or lays a new one. In this form of communication limited amount of information is exchanged. 
In underwater environment some fishes and other species also communicate through electric signals where there nerves and muscles work as batteries. They use continuous or intermittent pulses with  different frequencies learn about environment and to convey their identity and aggression messages.
%
%[Signal Active Space]
The concept of active space is widely used to describe the propagation of signals by species. In a network environment of signal emitters and receivers, active space is defined as the area encompassed by the signal  during the course of transmission \cite{Mcgregor2000}. In case of long-range signals, or even in case of short-range signals, this area include several individuals where their social grouping allows them to stay in cohesion. The concept of active space is described somewhat differently in case some social insects. In case of ants, this active space is defined as a zone within which the concentration of pheromone (or any other behaviourally active chemical substances) is  at or above threshold concentration \cite{Holldobler1990}. Mathematically this is denoted by a ratio:
The amount of pheromone emitted (Q)
The threshold concentration at which the receiving animal responds (K)
Q is measured in number of molecules released in a burst or in per unit of time whereas K is measured in molecules per unit of volume.
The adjustment of this ratio enables individuals to gain a shorter fade-out time and permits signals to be more sharply pinpointed in time and space by the receivers. In order to transmit the location of the animal in the signal, the rate of information transfer can be increased by either by lowering the rate of emission of Q or by increasing K, or both.  For alarm and trail systems a lower value of this ratio is used. Thus, according to need, individuals regulate their active space by making it large or small, or by reaching their maximum radius quickly or slowly, or by enduring briefly or for a long period of time. For example, in case of alarm, recruitment or sexual communication signals where encoding the location of an individual is needed, the information in each signal increases as the logarithm of the square of distance over which the signal travels. From the precise study of pheromones it has been found that active space of alarm signal is consists of  a concentric pair of hemispheres. (FIG). As the ant enters the outer zone she is attracted inward toward the point source; when she next crosses into the central hemisphere she become alarmed. It is also observed that ants can release pheromones with different active spaces.

Active space has strong role in modulating the behaviours of ants. For example, when workers of {\em Acanthomyops claviger} ants produce alarm signal due to an attack by a rival or insect predator, workers sitting a few millimetres away begin to react within seconds. However those ants sitting a few centimetres away take a minute or longer to react. In many cases ants and other social insects exhibit modulatory communication within their active space where many individuals involve in many different tasks. For example, while retrieving the large prey, workers of {\em Aphaeonogerter} ants produce chirping sounds (known as stridulate) along with releasing poison gland pheromones. These sounds attract more workers and keep them within the vicinity of the dead prey  to protect it from their competitors. This communication amplification behaviour can increase the active space to a maximum distance of 2 meters.
%%%%%%%%%%%%%%%%%%%%%
\section{Communication in Self-regulated Multi-robot Systems}
Communication between robots is an  important issue in MRS \cite{Arkin1998}. This is not a prerequisite for the group to be functioning, but often useful component of MRS \cite{Mataric2007}. Let us now investigate why communication is important, how this is usually archived in MRS and related other issues.

Researchers generally agree that communication in MRS usually provides several major benefits, such as:

\begin{description}
\item[Exchange of information and improving perception:]
Robots  can exchange potential information (as discussed below) based on their spatial position and knowledge of past events. This, in turn, leads to improve perception over a distributed region without directly sensing it.
\item[Synchronization of actions:]
In order to perform (or stop performing) certain tasks simultaneously or in a particular order robots need to communicate, or signal, to each other. 
\item[Enabling interactions:]
Communication is not strictly necessary for coordinating team actions. But  communication can help a lot to interact (and hence influence) each-other in a team that, in turn, enables robots to coordinate and negotiate their actions.
\end{description}

Since a MRS  can be comprised of robots of various computation and communication capabilities, it is also necessary to define the communication content and range \cite{Arkin1998,Mataric2007}. Usually robots can communicate about various states (e.g., task-related, individual, environmental etc.), their individual intentions and goals. 
 
%[How]
Robots communicate in a number of ways available under a specific application. This communication methods can be divided into two major categories:

\paragraph{Explicit communication:}
This is also known as intentional communication. This is done purposefully and usually using wireless radio. Based on the number of recipients of message, the communication process is termed differently. Such as,
Broadcast communication: where all other robots receive the message.
Peer-to-peer communication: where only a single robot receive the message.
Publish-subscribe communication: where only a selected (previously subscribed) number of robots receive the message.
Because explicit communication is costly in terms of both hardware and software, robotic researchers always put extra attention to design such a system by analysing strict requirements such as communication necessity, range, content, reliability of communication channel (loose of message) etc.

\paragraph{Implicit Communication:} 
This is also known as indirect stigmergic communication. This is a powerful way of communication where individuals leave information in the environment. This method was adopted from the social insect behaviour, such as stigmergy of ants (leaving of small amount of pheromone or chemicals behind while moving in a trail). Some researchers also tried to establish communication among robots through vision \cite{Kuniyoshi1994}.

% Issues
In multi-robot communication researchers have identified several issues. Some of the major issues are discussed here.
Kin Recognition
Kin recognition refers to the ability of a robot to recognize immediate family members by implicit or explicit communication or sensing. In case of MRS, this can be as simple as identifying other robots from objects and environment or as finding team-mates in a robotic soccer. This is an useful ability that helps interaction, such as cooperation among team members. 

\subsubsection*{Representation of Languages}
In case of effective communication several researchers also focused on representation of languages and grounding of these languages in physical world.

\subsubsection*{Fault-tolerance, Reliability and Adaptation}  
Since every communication channel is not free from noise and corruption of messages significant attention has been also given to manage these no communication situations, such as by setting up and maintaining communication network, managing reliability and adaptation rules when there is no communication link available. In terms of guaranteeing communication, researchers also tried to find ways for a deadlock free communication methods \cite{Arkin1998}, such as signboard communication method \cite{Wang1989}.

%%%%%%%%%%%%%%%%%%%%%