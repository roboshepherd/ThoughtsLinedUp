\chapter{Introduction}
\section{Multi-robot task allocation (MRTA)}
Robotic researchers generally agree that multiple robots can perform complex and distributed tasks more conveniently. A multi-robot system (MRS) can provide improved performance, fault-tolerance and robustness in those tasks through parallelism and redundancy \cite{Arkin1998,Parker+2006,Mataric2007}. In last two decades, many robotic researchers have conducted many challenging researches on various aspects of MRS. Results of those researches have been applied   in many real robotic applications. The list of those applications has now become too long to be enumerated here. For some examples, MRS can be used in: large space monitoring tasks, e.g underwater environment monitoring, dangerous tasks e.g., robotic de-mining, tasks that scale-up or scale-down over time, e.g. factory automation, and so forth \cite{Sahin+2005}.\\
%%
In order to get potential benefits of MRS in any application domain, we need to solve a common research challenge. \textit{How can we allocate tasks among multiple robots ?} In robotic literature,  this issue is identified as the \textit{multi-robot task allocation} (MRTA) problem. This can also be treated as the \textit{division of labour} (DOL) among robots, analogous to the DOL in biological  and human societies. Although the term ``division of labour'' is often used in biological literature and the term ``task-allocation'' is primarily used in multi-agent literature, both of these terms carry similar kinds of meanings with slightly different semantics. In this dissertation, we have used these terms interchangeably.\\
%%
MRTA is generally identified as the question of assigning tasks in an appropriate time to the appropriate robots considering the changes of the environment and/or the performance of other team members \cite{Gerkey+2004}. This is a {\em NP-hard} optimal assignment problem where optimum solutions can not be found quickly for large and complex problems \cite{Parker2008}.\\
%%
The complexities of the distributed MRTA arise from the fact that there is no central planner or coordinator for task assignments, and in a large MRS, generally robots have limited capabilities to sense, to communicate and to interact locally. None of them has the complete knowledge of the past, present or future actions of other robots. Moreover, they don't have the complete view of the world state. The computational and communication bandwidth requirements also restrict the solution quality of the problem \cite{Gerkey+2004}.\\
%%
Researchers from multi-robot or multi-agent systems, operations research and other disciplines have approached the issue of MRTA or task-allocation in multi-agent systems in many different ways. Traditionally, task allocation in a multi-agent systems has been divided into two major categories: 1) Predefined and 2) Bio-inspired self-organized task-allocation \cite{Shen+2001}.\\
%%
Early research on predefined task-allocation was dominated by intentional coordination, use of dynamic role assignment \cite{Parker2008} and market-based bidding approach \cite{Dias+2006}. Under these approaches, robots use direct task-allocation method to communicate and to negotiate tasks. These approaches are intuitive, comparatively straightforward to design and implement and can be analysed formally. However, these approaches typically works well only when the number of robots are small ($\leq 10$) \cite{Lerman+2006}.\\
%%
On the other hand, self-organized task-allocation approach relies on the emergent group behaviours e.g., \cite{Kube+1993}, such as emergent cooperation \cite{Lerman+2006}, adaptation rules \cite{Liu+2007} etc. They are more robust and scalable to large team sizes. However most of the robotic researchers found that self-organized task-allocation approach is difficult to design, to analyse formally and to implement in real robots. The solutions from these systems are also sub-optimal. It is also difficult to predict exact behaviours of robots and overall system performance.\\
%%
Within the context of the Engineering and Physical Sciences Research Council (EPSRC) project, ``Defying the Rules: How Self-regulatory Systems Work'', we have proposed to solve the above mentioned MRTA problem in an alternate way \cite{Arcaute+2008}. Our approach is inspired from the studies of emergence of task-allocation in both biological and human social systems. These studies show that a large number of species grow, evolve and generally continue functioning well by the virtue of their individual self-regulatory DOL systems.\\
%%
The amazing abilities of biological organisms to change, to respond to unpredictable environments, and to adapt over time lead them to sustain life through biological functions such as self-recognition, self-recovery, self-growth etc. It is interesting to note that in animal societies task-allocation has been accomplished years after years without a central authority or an explicit planning and coordinating element. Direct and indirect communication strategies are used to exchange information among individuals \cite{Camazine+2001}.\\
%%
The decentralized self-growth of Internet and its bottom-up interactions of millions of users around the globe present us similar evidences of task-allocation in human social systems \cite{Andriani+2004}. These interactions of individuals happen in the absence of or in parallel with strict hierarchy. Moreover from the study of sociology e.g., \cite{Sayer+1992}, cybernetics e.g., \cite{Beer1981}, strategic management e.g., \cite{Kogut2000} and related other disciplines we have found that decentralized self-regulated systems exist in nature and in man-made systems which can grow and achieve self-regulated division of labour over time.\\
%% 
From the above mentioned multi-disciplinary studies of various complex systems, we have proposed four generic rules that generically explains self-regulation in social systems. These four rules are: continuous flow of information, concurrency, learning and forgetting, all of them will be explained in Chapter \ref{afm}. Primarily these rules  deal with the issue of deriving local control laws of an individual's task-allocation system that can facilitate the DOL of the entire group. In order employ these rules in the individual level, we have developed a formal model of self-regulated DOL, called the attractive field model (AFM).\\
%%
AFM provides an abstract framework for self-regulatory DOL in social systems. However, without a proper validation in a real robotic system, we can not claim its applicability to the above MRTA problem. Particularly, we have got the following two questions:
%%
\begin{enumerate}
\item Can AFM  be used to solve the above MRTA problem for a relatively large group of real robots ?
\item Can this abstract model be used  in complex multi-tasking environments, preferably under an industrial scenario, rather than emulating the canonical biological tasks, e.g. foraging?
\end{enumerate}
%%
The first question demands to prove the very basic capabilities of AFM, i.e. whether it can provide the basic form of DOL but with a large group size. As we know doing real robotic experiments with large number of robots is time consuming and resource intensive, under this study, we use the term ``large group of robots'' to mean a group with at least 10 or more robots \footnote{Although this is very difficult to specify a certain number for defining ``large'', we find a consensus with some other researchers, e.g. \protect\cite{Lerman+2006} that a large group of robots currently  stands for roughly 10 or more robots.}. Thus we have hypothesized that AFM can be used as a distributed task-allocation mechanism for a large group of robots. It should be capable of providing all the basic task-allocation facilities, e.g. plasticity in task-allocation, task-specialization, as detailed in Sec. \ref{bg:def:dol}.\\
%%
The second question put AFM under a big challenge. Traditionally, robotic researchers, who use self-organized task-allocation methods, limit their experiments to test  if their models can sufficiently imitate biological tasks, e.g. foraging \citeaffixed{Krieger+2000}{e.g.}, pre-retrieval citeaffixed{Labella2007}{e.g.}. These experiments simply match the robotic task-allocation models with their biological counterparts. However, we have hypothesized that AFM can work under real task settings where multiple tasks  should be served by multiple-robots concurrently. We believe that even if we do not perform specific complex industrial tasks by our our robots, we can use AFM to find the high-level task-allocation performance by mimicking any suitable industrial task. This  high-level task-allocation performance can be used as an indication of the validity of AFM for putting it in real-world applications.\\
%%
The outcome of our research can be applied to solve generic task-allocation problem in numerous multi-agent systems. Generally AFM can serve as a generic framework for DOL in various social systems and particularly, our implementation can guide engineers to design practical MRS for doing realistic industrial tasks. As an example, our technique can be useful in automated manufacturing  which faces all the existing challenges of traditional centralized and sequential manufacturing processes \cite{Shen+2006}. We believe that our approach can help many other automation industries to overcome many of their day-to-day challenging issues, e.g. changing the manufacturing plant layouts on-the-fly, adapting for high variation in product styles, quantities, and active manufacturing resources e.g., robots, AGVs etc.
% 
%%%%%%%%%%%%%%%%%%%%%%%
\section{Communication \& sensing strategies for MRTA}
In biological social systems, communications among the group members, as well as sensing the work-in-progress, are two key components of self-organized DOL. In robotics, existing self-organized task-allocation methods heavily rely upon local  sensing and local communication of individuals for achieving self-organized task-allocation. However,  AFM differs significantly in this point by avoiding the strong dependence on the local communications and interactions found in many existing approaches to MRTA. AFM only requires a system-wide continuous flow of information about tasks, agent states etc. But this can be achieved by using both centralized and decentralized communication modes under explicit and implicit communication strategies. Chapter \ref{bg} reviews these communication modes and strategies with necessary details.\\
%%
In order to enable continuous flow of information in our MRS, we have implemented two types sensing and communication strategies inspired by the self-regulated DOL found in two types of social wasps: 1){\em Polistes} and 2)  {\em Polybia} \cite{Jeanne1999}. Depending on the group size, these species follows different strategies of sensing and communication.   Polistes wasps are called  ``independent founders'' (IF) in which reproductive females establish colonies alone or in small groups (in the order of $10^2$), but independent of any sterile workers. On the other hand, polybia wasps are called ``swarm founders'' (SF) where swarm of workers and queens initiate colonies consisting of several hundreds to millions of individuals.\\
%%
The most notable difference in the organization of work of these two social wasps is: IF does not rely on any cooperative task performance while SF interact with each-other locally to accomplish their tasks. The work mode of IF can be considered as {\em global sensing – no communication (GSNC)} where individuals sense the task requirements throughout a small colony and do these tasks without communicating with each other. On the other hand, the work mode of SF can be treated as {\em local sensing – local communication (LSLC)} where individuals can only sense locally due to large colony-size and they communicate locally to exchange information, e.g. task-requirements (although their exact mechanism is unknown).\\
%%
In this study, we have used these two sensing and communication strategies to compare the performance of self-regulated DOL of our MRS. %Chapter \ref{afm} uses GSNC strategy for enabling system-wide continuous flow of information through a centralized communication mode. Chapter \ref{afm} realizes LSLC strategy among robot controllers by emulating decentralized communication mode with suitable inter-process communication schemes.
Our research questions behind putting these strategies into action are as follows.
\begin{enumerate}
\item Is  task-allocation through GSNC strategy limited to only small group of robots ?
\item 
Does the task-allocation performance  in large groups  significantly vary between GSNC and LSLC strategies ?
\end{enumerate}
%%
If the self-regulated DOL in a large group of robots can not be achieved by following GSNC strategy, we can  conclude that this strategy is inappropriate for large group. The second question quantitatively investigates the variations of performance due to the use of different sensing and communication strategies.
%%
The findings from this  comparative study of different bio-inspired  sensing and communication strategies can be very useful  for designing efficient MRTA solutions. Thus we can be able to conclude, both quantitatively and qualitatively, whether local communication and local interaction should be considered as the prerequisites to achieve self-regulated DOL.
%%
%%%%%%%%%%%%%%%%%%%%%%%%%%%
%%In MRS research, robotic researchers have been using various forms of communications e.g., \cite{Bonabeau+1999,Labella2007}. Two widely used forms of communications are: 1) direct or explicit communication and 2) indirect or implicit communication. {\em Direct communication} is an intentional communicative act of message passing that aims at one or more particular receiver(s) \cite{Mataric1998}. It typically exchanges information through physical signals. In contrast, indirect communication, sometimes termed as {\em stigmergic} in biological literature, happens as a form of modifying the environment (e.g., pheromone dropping by ants) \cite{Bonabeau+1999}. In ordinary sense, this is an observed behaviour and many robotic researchers call it as {\em no communication} \cite{Labella2007}. In order to avoid ambiguity, by the term {\em self-regulated MRTA} (or {\em MRTA} for short) we refer to those MRS where robots can exhibit most common self-regulatory properties \cite{Bonabeau+1999} in their task-allocation process. Also in this thesis, by the term {\em communication}, we always refer to direct communication and we confine our discussion on MRTA within the context of direct communication only.\\
%%
%%In the process of pursuing self-regulated MRTA, robots can receive information from a centralised source \cite{Krieger+2000} or from their local peers \cite{Agassounon+2004}. In \cite{Sarker+2010robotic}, we reported a steady-state convergence of MRTA in a practical MRS using a centralized information source. This centralized communication system is easy to implement. It simplifies the overall design of a robot controller. However this system has disadvantage of a single point of failure and it is not scalable. The increased number of robots and tasks cause inevitable increase in communication load and transmission delay. Consequently, the overall system performance degrades. On the other hand, uncontrolled reception of information from decentralized or local sources is also not free from drawbacks. If a robot exchanges signals with all other robots (hereafter called as {\em peers}), it might get the global view of the system quickly and can select an optimal or near optimal task. This can produce a great improvement in overall performance of some types of tasks e.g., in area coverage \cite{Rutishauser+2009}. But this is also not practical and scalable for a typically large MRS due to the limited communication and computational capabilities of robots and limited available communication bandwidth of this type of system.\\
%%
%%A potential alternate solution to this problem can be obtained by decreasing the number of message recipient peers on the basis of a local communication radius ($r_{comm}$). This means that robots are allowed to communicate only with those peers who are physically located within a pre-set distance. When this strategy is used for sharing task information among peers, MRTA can be more robust and efficient \cite{Agassounon+2004}. However it is not well-defined how the selection of communication range can be made despite the significant differences in various implementation of MRS. In case of biological social insects, the concept of {\em active space} explains how each individual set their dynamic communication radius \cite{Holldobler1990,Mcgregor2000} (see Section \ref{bg:comm-biology}). In this thesis, we present a  locality based dynamic P2P communication model that design a desired communication range by considering both biological inspirations and geometric relationships of the environment particularly, the shapes and communication capabilities of robots. Along with a practical insight for selecting $r_{comm}$ value, various other design issues have been tackled. The recursion-free design of local communication channels is also achieved by a dynamic publish/subscribe model of communication. We also compare this system with our baseline centralized communication based MRS in terms of convergence of MRTA, communication load, robot motions and their task specializations.
%%%%%%%%%%%%%%%%%%%%%%%%
\section{Contributions}
The main contributions of this thesis are as follows:
\begin{itemize}
\item Introduction of attractive field model (AFM), an  inter-disciplinary generic model of division of labour, as a basic mechanism of  self-regulated MRTA.
\item Validation of the model through experiments with reasonably large number of real robots (i.e. 16 e-puck robots).
\item Comparisons of the performances of two bio-inspired sensing and communication strategies in achieving self-regulated MRTA.
\item Development of flexible multi-robot control architecture using D-Bus inter-process communication technology.
\item Classification of MRTA solutions based on three major axes: organization of task-allocation, interaction and communication. 
\end{itemize}
%%
%%%%%%%%%%%%%%%%%%%%%%%%
\section{Thesis outline and relevant publications}
This dissertation has been organized as follows.
Chapter 2  reviews the reviews the related literature on general terms, key issues of MRS and MRTA. This also includes the review of sensing and  communication strategies  for self-regulated task-allocation in biological societies. This chapter concludes by discussing the related work on communication for self-regulated MRTA.\\
Chapter 3 describes our research methodology with a review of experimental tools and technologies used in this dissertation.\\
Chapter 4 describes the validation of attractive filed model in our MRS under GSNC strategy.\\
Chapter 5 presents the extended work on using AFM under LSLC strategy and analyse the  performance of self-regulated task-allocation under both GSNC and LSLC strategies.\\
Chapter 6 concludes this dissertation  with a summary and future research directions.\\
%%
The following two publications were relevant to this dissertation.
\begin{itemize}
\item Sarker, M. \& Dahl, T.\textit{ A Robotic Validation of the Attractive Field Model: An Inter-Disciplinary Model of Self-Regulatory Social Systems}. In Proc. of the Seventh International Conference on Swarm Intelligence (ANTS 2010) to appear, 2010.
\item Sarker, M. \& Dahl, T. \textit{Flexible Communication in Multi-robotic Control System Using HEAD: Hybrid Event-driven Architecture on D-Bus}. In Proc. of the UKACC International Conference on Control 2010 (CONTROL 2010), to appear, 2010.
\end{itemize}