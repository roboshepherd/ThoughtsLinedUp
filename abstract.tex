\begin{Large}
\noindent
Abstract\\
\end{Large}
\newline
To deploy a large group of autonomous robots in dynamic multi-tasking environments,  a suitable multi-robot task-allocation (MRTA) solution is required that must be scalable to variable number of robots and tasks. Recent studies show that  biology-inspired self-organized approaches can effectively handle MRTA issue  of  relatively large multi-robot systems.  However most existing MRTA researches have overlooked the role of different communication and sensing strategies, commonly followed by various biological societies in producing self-regulated division of labour under varying group sizes.

This dissertation proposes to solve the MRTA problem using a set of  previously published generic rules for division of labour derived from the observation of ant, human and robotic social systems. These bottom-up rules describe the phenomenon of self-regulated division of labour in terms of attractive fields between robots and tasks. The concrete form of these rules, the \textit{attractive filed model} (AFM), provides sufficient abstraction to hide the dependencies  on local communication and sensing found in many existing approaches to MRTA. 

This dissertation validates the effectiveness of AFM to address the above MRTA issue using two bio-inspired communication and sensing strategies: 1) \textit{global sensing with no communication} and 2) \textit{local sensing with local communication}.  The former has been realized using a centralized communication system and the later has been emulated under a peer-to-peer local communication scheme. Both of them have been applied to achieve self-regulated MRTA among a group of 16 e-puck robots under a realistic manufacturing shop-floor scenario. A robotic interpretation of AFM is presented  that shows us how various generic parameters of AFM can be mapped with the properties of a  manufacturing shop-floor. 

In order to integrate various heterogeneous robotic software frameworks, a flexible multi-robot control architecture, \textit{hybrid event-driven architecture on D-Bus}, has been outlined. This is a pioneering software architecture that uses D-Bus interprocess communication protocol to decouple the communication  process among different pieces of robotic software components. 

In this study, a concise taxonomy of MRTA solutions has been proposed to remove the ambiguities found in describing the existing MRTA solutions. This taxonomy  considers the organization of task-allocation, communication and interaction among robots, as the three major axes of classification.  Besides, a set of domain independent metrics, e.g. plasticity, task-specialization, energy usage, has been formalized to compare  the performances  of self-regulated MRTA under the above two communication and sensing strategies. The presented comparisons are intended to extend our general understanding of the role of information exchange strategies in designing the distributed task-allocation solutions for various group sizes in a wide range of social systems.