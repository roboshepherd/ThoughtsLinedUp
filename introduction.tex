\chapter{Introduction}
\section{Multi-robot task allocation (MRTA)}
Robotic researchers generally agree that multiple robots can perform complex and distributed tasks more conveniently. Multi-robot systems (MRS) can provide improved performance, fault-tolerance and robustness through parallelism and redundancy \cite{Arkin1998,Parker+2006,Mataric2007}. However, in order to get potential benefits of MRS in any application domain, we need to solve a common research challenge i.e., \textit{multi-robot task allocation} (MRTA) \cite{Gerkey+2004}. MRTA can also be called as \textit{division of labour} (DoL) analogous to DoL in biological social insect and human societies (hereafter the term MRTA is used to denote an instance of social DoL). It is generally identified as the question of assigning tasks in an appropriate time to the appropriate robots considering the changes of the environment and/or the performance of other team members. This is a NP-hard optimal assignment problem where optimum solutions can not be found quickly for large complex problems \cite{Gerkey+2003,Parker2008}.
%
The complexities of MRTA arise from the fact that there is no central planner or coordinator for task assignments and the robots are limited to sense, to communicate and to interact locally. None of them has the complete knowledge of the past, present or future actions of other robots. Moreover, they don't have the complete view of the world state. The computational and communication bandwidth requirements also restrict the solution quality of the problem \cite{Lerman+2006}.\\
% 
Researchers from multi-robot or multi-agent systems, operations research and other disciplines have approached the MRTA or task-allocation in multi-agents issue in many different ways. Traditionally task allocation in a multi-agent systems has been divided into two major categories: 1) Predefined (off-line) and 2) Emergent (real-time) task-allocation \cite{Shen+2001}. However predefined task- allocation approach fails to scale well as the number of tasks and robots becomes large, e,g,, more than 10 \cite{Lerman+2006}. On the other hand emergent task-allocation approach relies on the emergent group behaviours e.g., \cite{Kube+1993}, such as emergent cooperation \cite{Lerman+2006}, adaptation rules \cite{Liu+2007} etc. They are more robust and scalable to large team size. However most of the robotic researchers found that emergent task-allocation approach is difficult to design, to analyse formally and to implement in real robots. The solutions from these systems are also sub-optimal. It is also difficult to predict exact behaviours of robots and overall system performance.\\
%
Within the context of the Engineering and Physical Sciences Research Council (EPSRC) project, ``Defying the Rules: How Self-regulatory Systems Work'', we have proposed to solve the above mentioned MRTA problem in a new way \cite{Arcaute+2008}. Our approach is inspired from the studies of emergence of task-allocation in both biological insect societies and human social systems. Biological studies show that a large number of animal as well as human social systems grow, evolve and generally continue functioning well by the virtue of their individual self-regulatory task-allocation systems. The amazing abilities of biological organisms to change, to respond to unpredictable environments, and to adapt over time lead them to sustain life through biological functions such as self-recognition, self-recovery, self-growth etc. It is interesting to note that in animal societies task-allocation has been accomplished years after years without a central authority or an explicit planning and coordinating element. Direct peer-to-peer (P2P) and indirect communication such as stigmergy is  used to exchange information among individuals \cite{Camazine+2001}. The decentralized self-growth of Internet and its bottom-up interactions of millions of users around the globe present us similar evidences of task-allocation in human social systems \cite{Andriani+2004}. These interactions of individuals happen in the absence of or in parallel with strict hierarchy. Moreover from the study of sociology e.g., \cite{Sayer+1992}, cybernetics e.g., \cite{Beer1981}, strategic management e.g., \cite{Kogut2000} and related other disciplines we have found that decentralized self-regulated systems exist in nature and in man-made systems which can grow and achieve self-regulated division of labour over time.\\
% 
From the above mentioned multi-disciplinary studies of various complex systems, we believe that a set of generic rules can govern the self-regulated task-allocation in MRS. Primarily these rules should deal with the issue of deriving local control rules for facilitating the task-allocation of an entire robot team.\\
% 
The outcome of our research can be applied to solve generic task-allocation problem in numerous multi-agent systems. As an example, our technique can be useful in automated manufacturing (AM) which faces all the existing challenges of traditional centralized and sequential manufacturing processes such as, insufficiently flexible to respond and adapt changes in production styles of high-mix low-volume production environments \cite{Shen+2006}. We believe that our approach can help AM industries to overcome many of these challenging issues, such as flexibility to change the manufacturing plant layouts on-the-fly, adaptability for high variation in product styles, quantities, and active manufacturing resources e.g., robots, AGVs etc.
%
%%%%%%%%%%%%%%%%%%%%%%%
\section{Communication for self-regulated task-allocation}
In MRS research, robotic researchers have been using various forms of communications e.g., \cite{Bonabeau+1999,Labella2007}. Two widely used forms of communications are: 1) direct or explicit communication and 2) indirect or implicit communication. {\em Direct communication} is an intentional communicative act of message passing that aims at one or more particular receiver(s) \cite{Mataric1998}. It typically exchanges information through physical signals. In contrast, indirect communication, sometimes termed as {\em stigmergic} in biological literature, happens as a form of modifying the environment (e.g., pheromone dropping by ants) \cite{Bonabeau+1999}. In ordinary sense, this is an observed behaviour and many robotic researchers call it as {\em no communication} \cite{Labella2007}. In order to avoid ambiguity, by the term {\em self-regulated MRTA} (or {\em MRTA} for short) we refer to those MRS where robots can exhibit most common self-regulatory properties \cite{Bonabeau+1999} in their task-allocation process. Also in this thesis, by the term {\em communication}, we always refer to direct communication and we confine our discussion on MRTA within the context of direct communication only.\\
%
In the process of pursuing self-regulated MRTA, robots can receive information from a centralised source \cite{Krieger+2000} or from their local peers \cite{Agassounon+2004}. In \cite{Sarker+2010robotic}, we reported a steady-state convergence of MRTA in a practical MRS using a centralized information source. This centralized communication system is easy to implement. It simplifies the overall design of a robot controller. However this system has disadvantage of a single point of failure and it is not scalable. The increased number of robots and tasks cause inevitable increase in communication load and transmission delay. Consequently, the overall system performance degrades. On the other hand, uncontrolled reception of information from decentralized or local sources is also not free from drawbacks. If a robot exchanges signals with all other robots (hereafter called as {\em peers}), it might get the global view of the system quickly and can select an optimal or near optimal task. This can produce a great improvement in overall performance of some types of tasks e.g., in area coverage \cite{Rutishauser+2009}. But this is also not practical and scalable for a typically large MRS due to the limited communication and computational capabilities of robots and limited available communication bandwidth of this type of system.\\
%
A potential alternate solution to this problem can be obtained by decreasing the number of message recipient peers on the basis of a local communication radius ($r_{comm}$). This means that robots are allowed to communicate only with those peers who are physically located within a pre-set distance. When this strategy is used for sharing task information among peers, MRTA can be more robust and efficient \cite{Agassounon+2004}. However it is not well-defined how the selection of communication range can be made despite the significant differences in various implementation of MRS. In case of biological social insects, the concept of {\em active space} explains how each individual set their dynamic communication radius \cite{Holldobler1990,Mcgregor2000} (see Section \ref{bg:comm-biology}). In this thesis, we present a  locality based dynamic P2P communication model that design a desired communication range by considering both biological inspirations and geometric relationships of the environment particularly, the shapes and communication capabilities of robots. Along with a practical insight for selecting $r_{comm}$ value, various other design issues have been tackled. The recursion-free design of local communication channels is also achieved by a dynamic publish/subscribe model of communication. We also compare this system with our baseline centralized communication based MRS in terms of convergence of MRTA, communication load, robot motions and their task specializations.
%%%%%%%%%%%%%%%%%%%%%%%%
\section{Contributions}
The main contributions of this thesis are as follows:
\begin{itemize}
\item Introduction of attractive field model (AFM), an  inter-disciplinary generic model of division of labour, as a basic mechanism of  self-regulated MRTA.
\item Validation of the model through experiments with reasonably large number of real robots.
\item Development of a centralized and a local P2P communication model and their respective implementation algorithms that satisfy the requirement of system-wide continuous flow of information for self-regulated task-allocation.
\item Comparisons of performances of both communication models in achieving similar self-regulated MRTA.
\item Development of a point-to-point signal based multi-robot control architecture using D-Bus inter-process communication technology.
\end{itemize}
%%%%%%%%%%%%%%%%%%%%%%%%
\section{Thesis outline and relevant publications}
This report has been organized as follows.
Chapter 2  reviews the reviews the related literature on general terms, key issues of MRS and MRTA. This also includes the review of communication  for self-regulated task-allocation in biological societies. This chapter concludes by discussing the related work on communication for self-regulated MRTA.
Chapter 3 describes the attractive filed model in details.
Chapter 4 presents the our centralized and local communication models and analyse it  from the geometric and biological view-point. 
Chapter 5 includes experiment tools used in this research.
Chapter 6 describes the design of our experiments.
Chapter 7 describes the results of our experiments.
Chapter 8 concludes this thesis  with a summary and future research directions.




