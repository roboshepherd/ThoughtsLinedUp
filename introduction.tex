\chapter{Introduction}
\label{intro}
\section{Issues in multi-robot task allocation (MRTA)}
\label{intro:mrta}
Robotic researchers generally agree that multiple robots can perform complex and distributed tasks more conveniently. A \acf{MRS} can provide improved performance, fault-tolerance and robustness in those tasks through parallelism and redundancy \cite{Arkin1998,Parker+2006,Mataric2007}. In last two decades, many robotic researchers have conducted many challenging researches on various aspects of MRS. Results of those researches have been applied   in many real robotic applications. For example, MRS can be used in: large space monitoring tasks, e.g underwater environment monitoring, dangerous tasks e.g., robotic de-mining, tasks that scale-up or scale-down over time, e.g. factory automation, and so forth.\\
%%
In order to get potential benefits of MRS in any application domain, we need to solve a common research challenge. \textit{How can we allocate tasks among multiple robots dynamically ?} In robotic literature,  this issue is identified as the \acfi{MRTA}. This problem can also be treated as the \acfi{DOL} among robots, analogous to the DOL in biological  and human social systems. Although the term ``division of labour'' is often used in biological literature and the term ``task-allocation'' is primarily used in multi-agent literature, both of these terms carry similar kinds of meanings with slightly different semantics. In this dissertation, we have used these terms interchangeably.\\
%%
MRTA is generally identified as the question of assigning tasks in an appropriate time to the appropriate robots considering the changes of the environment and/or the performance of other team members \cite{Gerkey+2003}. This is a {\em NP-hard} optimal assignment problem where optimum solutions can not be found quickly for large and complex problems \cite{Parker2008}.\\
%%
The complexities of the distributed MRTA problem arise from the fact that there is no central planner or coordinator for task assignments, and in a large MRS, generally robots have limited capabilities to sense, to communicate and to interact locally. None of them has the complete knowledge of the past, present or future actions of other robots. Moreover, they do not have the complete view of the world state. The computational and communication bandwidth requirements also restrict the solution quality of the problem \cite{Gerkey+2004}.\\
%%
Researchers from multi-robot and multi-agent systems, operations research and other disciplines have approached the issue task-allocation in many different ways. Traditionally, task allocation in a multi-agent systems has been divided into two major categories: 1) Predefined and 2) Bio-inspired self-organized task-allocation \cite{Shen+2001}.\\
%%
Early research on predefined task-allocation was dominated by intentional coordination \cite{Parker2008}, use of dynamic role assignment \cite{Chaimowicz2002} and market-based bidding approach \cite{Dias+2006}. Under these approaches, robots use direct task-allocation method, often to communicate with group members for negotiating on tasks. These approaches are intuitive, comparatively straightforward to design and implement and can be analysed formally. However, these approaches typically works well only when the number of robots are small ($\leq 10$) \cite{Lerman+2006}.\\
%%
On the other hand, self-organized task-allocation approach relies on the emergent group behaviours, such as emergent cooperation \citeaffixed{Kube+1993}{e.g.}, adaptation rules \citeaffixed{Liu+2007}{e.g.} etc. They are more robust and scalable to large team sizes. However, most of the robotic researchers found that self-organized task-allocation approach is difficult to design, to analyse (formally) and to implement in real robots. The solutions from these systems are also sub-optimal. It is also difficult to predict exact behaviours of robots and overall system performance.\\
%%
Within the context of the \acfi{EPSRC} project, ``Defying the Rules: How Self-regulatory Systems Work'', we have proposed to solve the above mentioned self-regulated DOL problem in an alternate way \cite{Arcaute+2008}. Our approach is inspired from the studies of emergence of task-allocation in both biological and human social systems. These studies show that a large number of species grow, evolve and generally continue functioning well by the virtue of their individual self-regulatory DOL systems.\\
%%
The amazing abilities of biological organisms to change, to respond to unpredictable environments, and to adapt over time lead them to sustain life through biological functions such as self-recognition, self-recovery, self-growth etc. It is interesting to note that in animal societies task-allocation has been accomplished years after years without a central authority or an explicit planning and coordinating element. Direct and indirect communication strategies are used to exchange information among individuals \cite{Camazine+2001}.\\
%%
The decentralized self-growth of Internet and its bottom-up interactions of millions of users around the globe present us similar evidences of task-allocation in human social systems \cite{Andriani+2004}. These interactions of individuals happen in the absence of, or in parallel with, strict hierarchy. Moreover from the study of sociology \citeaffixed{Sayer+1992}{e.g.}, cybernetics  \citeaffixed{Beer1981}{e.g.}, strategic management  \citeaffixed{Kogut2000}{e.g.} and related other disciplines we have found that decentralized self-regulated systems exist in nature and in man-made systems which can grow and achieve self-regulated DOL over time by the virtue of their common bottom-up rules of self-regulation.\\
%% 
From the above mentioned multi-disciplinary studies of various complex systems, we have proposed four generic rules to explain self-regulation in those social systems. These four rules are: \textit{continuous flow of information}, \textit{concurrency}, \textit{learning} and \textit{forgetting}, all of them will be explained in Chapter \ref{afm}. Primarily these rules  deal with the issue of deriving local control laws for regulating an individual's task-allocation behaviour that can facilitate the DOL in the entire group. In order employ these rules in the individual level, we have developed a formal model of self-regulated DOL, called the \acfi{AFM}.\\
%%
AFM provides an abstract framework for self-regulatory DOL in social systems. However, without a proper validation in a real robotic system, we can not claim its applicability to the above MRTA problem. Particularly, we have got the following two questions:
%%
\begin{enumerate}
\item Can the AFM  be adopted to solve the above MRTA problem for a relatively large group of real robots?
\item Can the AFM be made relevant for industrial scenarios as well as for the emulation of biological tasks?
\end{enumerate}
%%
The first question demands that we evaluate the very basic capabilities of AFM, i.e. whether it can provide the basic form of DOL in large groups. As we know that conducting real robotic experiments with large number of robots is time consuming and resource intensive, in this study we use the term ``large group of robots'' to mean a group with at least 10 or more robots \footnote{Although this is very difficult to specify a certain number for defining ``large'', we find a consensus with some other researchers, e.g. \protect\citeaffixed{Lerman+2006}{e.g.} that a large group of robots currently  stands for roughly 10 or more robots.}. We have hypothesized that AFM can be used as a distributed task-allocation mechanism for a large group of robots. It should be capable of providing all the basic task-allocation facilities, e.g. plasticity in task-allocation, task-specialization, as discussed in detail in Sec. \ref{bg:def:dol}.\\
%%
The second question put AFM under a big challenge. Traditionally, researchers in the area of robotics, who use self-organized task-allocation methods, limit their experiments to test  if their models can sufficiently imitate biological tasks, e.g., foraging \cite{Krieger+2000} or pre-retrieval \cite{Labella2007}. These experiments simply match the biological DOL models with their robotic counterparts. However, we have hypothesized that AFM can work under real task settings where multiple tasks  should be served by multiple-robots concurrently. We believe that even if we do not perform specific complex industrial tasks with our robots, we can use AFM to find the high-level task-allocation performance by mimicking any suitable industrial task. This  high-level task-allocation performance can be used as an indication of the validity of AFM for putting it in real-world applications.\\
%%
The outcome of our research can be applied to solve generic task-allocation problem in numerous multi-agent systems. Generally AFM can be used as a generic framework for DOL in various social systems and particularly, our results are relevant to the design of MRTA solution of large MRS in order to perform real industrial tasks. As an example, our technique contributes to making MRS more useful for automated material handling tasks in warehouses or in manufacturing  industries which face all the existing challenges of traditional centralized and sequential manufacturing processes e.g. changing the manufacturing plant layouts on-the-fly, adapting for high variation in product styles/quantities, and dynamic allocation of active manufacturing resources e.g., robots and automated guided vehicles \cite{Shen+2006}.
% 
%%%%%%%%%%%%%%%%%%%%%%%
\section{Communication \& sensing strategies for MRTA}
\label{intro:comm}
In biological social systems, communications among the group members, as well as sensing the task-in-progress, are two key components of self-organized DOL. In robotics, existing self-organized task-allocation methods rely heavily upon local  sensing and local communication of individuals for achieving self-organized task-allocation. However,  AFM differs significantly in this point by avoiding the strong dependence on the local communications and interactions found in many existing approaches to MRTA. AFM requires a system-wide continuous flow of information about tasks, agent states etc. but this can be achieved by using both centralized and decentralized communication modes under explicit and implicit communication strategies. Chapter \ref{bg} reviews these communication modes and strategies with necessary details.\\
%%
In order to enable continuous flow of information in our MRS, we have implemented two types of sensing and communication strategies inspired by the self-regulated DOL found in two types of social wasps: {\em polistes} and  {\em polybia} \cite{Jeanne1999}. Depending on the group size, these species follow different strategies for  communication and sensing of tasks.  Polistes wasps are called the \acfi{IF} in which reproductive females establish colonies alone or in small groups (in the order of $10^2$), but independent of any sterile workers. On the other hand, polybia wasps are called the \acfi{SF} where a swarm of workers and queens initiate colonies consisting of several hundreds to millions of individuals.\\
%%
The most notable difference in the organization of work of these two social wasps is: IF does not rely on any cooperative task performance while SF interact with each-other locally to accomplish their tasks. The work mode of IF can be considered as {\em global sensing - no communication (GSNC)} where the individuals sense the task requirements throughout a small colony and do these tasks without communicating with each other. On the other hand, the work mode of SF can be treated as {\em local sensing - local communication (LSLC)} where the individuals can only sense tasks locally due to large colony-size and they can communicate locally to exchange information, e.g. task-requirements (although their exact mechanism is unknown).\\
%%
In this study, we have used these two sensing and communication strategies to compare the performance of the self-regulated DOL of our robots under AFM. %Chapter \ref{afm} uses GSNC strategy for enabling system-wide continuous flow of information through a centralized communication mode. Chapter \ref{afm} realizes LSLC strategy among robot controllers by emulating decentralized communication mode with suitable inter-process communication schemes.
Our research questions behind putting these strategies into action are as follows.
\begin{enumerate}
\item Is it the case that task-allocation through GSNC strategy should be limited only to  small group of robots, like polistes wasps?
\item 
Does the task-allocation performance  in large groups  significantly vary under GSNC and LSLC strategies?
\end{enumerate}
%%
Regarding the first question, if the self-regulated DOL in a large group of robots can not be achieved by following GSNC strategy, we can  conclude that this strategy is inappropriate for a large group. The second question quantitatively investigates the variations of performance due to the use of different  communication and sensing strategies.
%%
The findings from this  comparative study of different bio-inspired  communication and sensing strategies can be very useful  for designing efficient MRTA solutions. Thus we can be able to conclude, both quantitatively and qualitatively, whether local communication and local interaction should be considered as the prerequisites to achieve self-regulated DOL in large group of robots.
%%
% In order to avoid ambiguity, by the term {\em self-regulated MRTA} (or {\em MRTA} for short) we refer to those MRS where robots can exhibit most common self-regulatory properties \cite{Bonabeau+1999} in their task-allocation process. 
%%
%%%%%%%%%%%%%%%%%%%%%%%%
\section{Summary of contributions}
The main contributions of this thesis are as follows:
\begin{itemize}
\item Interpretation of AFM, an  inter-disciplinary generic model of division of labour, as a basic mechanism of  self-regulated MRTA.
\item Validation of the model through experiments with reasonably large number of real robots i.e., 16 e-puck robots.
\item Comparisons of the performances of two bio-inspired sensing and communication strategies in achieving self-regulated MRTA.
\item Development of a flexible multi-robot control architecture using D-Bus inter-process communication technology.
\item Classification of MRTA solutions based on three major axes: organization of task-allocation, interaction and communication.
\end{itemize}
%%
%%%%%%%%%%%%%%%%%%%%%%%%
\section{Thesis outline and relevant publications}
This dissertation has been organized as follows.
Chapter 2  reviews the related literature on general terms and key issues of MRS. This chapter presents a novel taxonomy of MRTA solutions. This chapter also discusses the related work on communication in MRS within the context of MRTA.
Chapter 3 reviews experimental tools and technologies used in this dissertation.
Chapter 4 describes the validation of AFM in our MRS under the GSNC strategy and discusses the findings from our self-regulated MRTA experiments.
Chapter 5 presents our work on using AFM under the LSLC strategy and analyses the  performance of self-regulated task-allocation under both GSNC and LSLC strategies.
Chapter 6 draws conclusions, summaries and presents the possibilities for future work.
%%
The following two publications are relevant to this dissertation.
\begin{itemize}
\item Sarker, M. \& Dahl, T.\textit{ A Robotic Validation of the Attractive Field Model: An Inter-Disciplinary Model of Self-Regulatory Social Systems}. In Proc. of the Seventh International Conference on Swarm Intelligence (ANTS 2010) to appear, 2010.
\item Sarker, M. \& Dahl, T. \textit{Flexible Communication in Multi-robotic Control System Using HEAD: Hybrid Event-driven Architecture on D-Bus}. In Proc. of the UKACC International Conference on Control 2010 (CONTROL 2010), to appear, 2010.
\end{itemize}