\chapter{General Background}
\label{bg-gen}
%%%%%%%%%%%%%%%%%%%%%
\section{Definition of key terms}
\subsection{Self-regulation}
\label{bg:def:self-reg}
Animals and flying beings, that live on or above earth, form social communities similar to human societies \cite{SIHQ1995}. In recent years, the study of biological social insects and other animals reveals us that simple individuals of these self-organized  societies can solve various complex and large everyday-problems with a few behavioural rules, relying on their minimum sensing and communication abilities \cite{Camazine+2001}. Some common tasks of these biological societies include: dynamic foraging, building amazing nest structures, maintaining division of labour among workers \cite{Bonabeau+1999}. These tasks are done by colonies  ranging from a few animals to thousands or millions of individuals. Despite their huge colony size, they easily achieve surprising efficiency in those tasks with many common features, e.g. robustness, flexibility, synergy (for an example in ants, see Fig. \ref{fig:self-org-ants}).

Today, these findings have inspired scientists and engineers to use this knowledge of biological self-organization in developing solutions for various problems of artificial systems, such as  routing traffics in telecommunication and vehicle networks, designing control algorithms for large groups of autonomous robots, automating industrial shop-floor tasks and so forth \cite{Garnier+2007}.
%%----- FIG fig:weaver-green-tree-ants---------
\begin{figure}[htp]
  \centering
  \subfloat[Weaver ants]{\label{fig:weaver-ants}\includegraphics[width=6cm, height=4cm]{./photos/ants-hy27c.eps}}                
  \hspace{0.25cm}
  \subfloat[Green-tree ants]{\label{fig:green-tree-ants}\includegraphics[width=6cm, height=4cm]{./photos/ants_spacecollective.org_greenTreeAnt.eps}}
  \caption{(a) During nest construction, weaver ants combine two leaves by pulling them from two sides \protect\citeasnoun{Yahya2000}.
(b) Green-tree ants retrieve a large-prey. http://www.spacecollective.org, last seen on 01/05/2010.}
%
  \label{fig:self-org-ants}
\end{figure}
% FIG: self-org
\begin{figure}[htp]
\centering
\includegraphics[height=8cm, angle=0]
{./images/dia-files/self-org-1}
\caption{Self-organization viewed from four (A-D) inseparable perspectives. Adopted from \protect\citeasnoun{Camazine+2001}.}
\label{fig:self-org-1} % Give a unique label
\vspace*{0.25cm}
\centering
\includegraphics[height=5cm, angle=0]{./images/dia-files/self-org-agent}
\caption{ Three major interfaces of a self-regulated agent.}
\label{fig:self-org-agent} % Give a unique label
\end{figure}

Self-organization, in biological and  other systems, is often characterized in terms of four major ingredients: 1) positive feedback, 2) negative feedback, 3) presence of multiple interactions among individuals and their environment, and 4) amplification of fluctuations  e.g., random walks, errors, random task-switching \cite{Camazine+2001}. An external observer, as if looking through transparent glasses, can recognize a self-organized system by observing the individual interactions of that system from four interlinked perspectives (Fig. \ref{fig:self-org-1}). 

The first perspective is the {\em  positive feedback} or amplification that can be resulted from the execution of simple behavioural ``rules of thumb''. For example, recruitment to a food source through trail laying and trail following in some ants  is due to the positive feedback that attract other ants to follow the trail and to lay more pheromones over time. The second perspective is the {\em negative feedback} that counterbalances positive feedback. This usually occurs to stabilize collective patterns, e.g., crowding at the food sources (saturation), competition between paths to food sources etc. The third perspective is the {\em presence of multiple  interactions} that can be direct \acf{P2P}, broadcast or indirect {\em stigmergic}, e.g. ants' pheromone laying. Finally, the fourth  perspective is the {\em amplification of fluctuations} that comes from various stochastic events. For example, errors in trail following of some ants may lead some foragers to get lost and later on, to find new and unexploited food sources, and recruit other ants to those sources.

%% FIG: fig:honey-bee-nest-comb
\begin{figure}
\centering
\subfloat[Honey-bee nest]{\includegraphics[width=6cm, height=4cm]{./photos/honey-bee-nest-hy103.eps}} 
\hspace{0.25cm}
\subfloat[Construction of honey-combs]{\includegraphics[width=6cm, height=4cm]{./photos/honey-bee-comb-building-knol-google.eps}}
\caption{(a) A Honey-bee colony has built a nest on a tree-branch. From http://www.harunyahya.com, last seen on 01/05/2010. (b) Honey-bees are constructing honey combs. From http://knol.google.com, last seen on 01/05/2010.}
\label{fig:honey-bee-nest}
\end{figure}
% 
In a self-organized system, an individual agent may have limited cognitive, sensing and communication capabilities. But they are collectively capable of solving complex and large problems, e.g. coordinated nest construction of honey-bees (Fig. \ref{fig:honey-bee-nest}), collective defence of school fishes from a predator attack (Fig. \ref{fig:school-of-fish}), ordered homing of bats (Fig. \ref{fig:bats-colony}).  Since the discovery of these collective behavioural patterns of self-organized societies, scientists observed modulation or adaptation of behaviours in the individual level \cite{Garnier+2007}. For example, in order to prevent a life-threatening humidity-drop in the colony, cockroaches maintain a locally sustainable humidity level by increasing their tendency to aggregate, i.e. by regulating their individual aggregation behaviours. 

As shown in Fig. \ref{fig:self-org-agent}, the  self-regulation of an individual agent is depicted through a triangle where its base-arm represents the simple behavioural rules of thumb (in this case, intense aggregation of cockroaches in low humidity). This is supported by two side-arms: local communication and local sensing. This local sensing is sometimes also referred to as sensing or information gathering from the work in progress, e.g.stigmergy and this local communication is an instance of direct communication with neighbours  \cite{Camazine+2001}.

%% FIG: School of Fish
\begin{figure}[htp]
\centering
\subfloat[A moving school of fish]{\includegraphics[width=6cm, height=4cm]{./photos/adventureaquariumschooloffish.eps}} 
\hspace{0.25cm}
\subfloat[A shark attacked a school of fish]{\includegraphics[width=6cm, height=4cm]{./photos/schoo_of_fish_shark-bigtiger.eps}}
\caption{(a) Group cohesion of a school of fish. (b) When a shark attacks a school of fish, they misguide the attacker by swift random movements. From http://www.travelblog.org last seen on 01/05/2010.}
\label{fig:school-of-fish}
\end{figure}
%%
Self-regulation has been studied in many other branches of knowledge. In most places of literature, self-regulation refers to the exercise of control over oneself to bring the self into line with preferred standards \cite{Baumeister+2007}. One of the most notable self-regulatory process is the human body's homoeostatic process where the human body's inner process seeks to return to its regular temperature when it gets overheated or chilled. \citeasnoun{Baumeister+2007} has referred self-regulation to goal-directed behaviour or feedback loops, whereas self-control may be associated with conscious impulse control.

In psychology, self-regulation denotes the strenuous actions to resist temptation or to overcome anxiety. self-regulation is also divided into two categories: 1) conscious and 2) unconscious self-regulation. Conscious self-regulation puts emphasis on conscious and deliberate efforts in self-regulation. On the other hand, unconscious self-regulation refers to the automatic self-regulatory process that is not labour intensive but operate in harmony with unpredictable and unfolding events in the environment. This process uses the available informational input in ways that help to attain an activated goal.
%% FIG: Bats navigation
\begin{figure}[htp]
\centering
\subfloat[A bat colony]{\includegraphics[width=6cm, height=4cm]{./photos/bats_hy2.eps}} 
\hspace{0.25cm}
\subfloat[Navigation of bats]{\includegraphics[width=6cm, height=4cm]{./photos/bats_hy3.eps}}
\caption{(a) One of the largest bat colony with about 50 million bats, (b) These bats can show amazing navigation abilities: they always fly back to their nest on a straight route from wherever they are, reproduced from \protect\citeasnoun{Tuttle1995}.}
\label{fig:bats-colony}
\end{figure}

The concepts of self-regulation is also common in cybernetic theory where self-regulation in inanimate mechanisms shows that they can regulate themselves by making adjustments according to pre-programmed goals or set standards. A common example of this kind can be found in a thermostat that controls a heating and cooling system to maintain a desired temperature in a room. In physics, chemistry, biology and some other branches of natural sciences, the concept of self-regulation is centred around the study of self-organizing individuals. 

Self-regulation has also been studied in the context of human social systems. Here it is originated from the division of social labour that creates self-organized processes with self-regulating effects \cite{Kppers+1990}. Two types of self-regulation have been reported in many places of sociology literature: 1) self-regulation from self-organization and 2) self-regulation from activities of components in a heterarchical organization. It is interesting to note that self-regulation in biological species provides the similar evidences of bottom-up approach of self-regulation of heterarchical organization through the interaction of individuals in the absence, or in parallel, of strict hierarchy \cite{Beer1981}.

From the above discussion, we see that the term {\em self-regulation} carries a wide range of meanings in different branches of knowledge. In psychology and cognitive neuroscience, self-regulation is discussed solely within an individual's perspective whereas, in biology and social sciences, self-regulation is discussed within the context of a group of individuals or society as a whole. In  this thesis, the latter context is more appropriate where  self-regulation focuses on monitoring an individual's own state and environmental changes, in relation to the communal goal and, in turn, this leads to adjust the behaviours of that individual with respect to the changes found.
%%%%======================================================================
\subsection{Communication} 
\label{bg:def:comm}
%%FIG:  Comm defined
\begin{figure}
\centering
\includegraphics[width=8.5cm, angle=0]
{./images/dia-files/comm-defined.eps}
\caption{ General models of communication, adopted from \protect\citeasnoun{West+2003}.}
\label{fig:gen-comm-defined} % Give a unique label
\end{figure}
%%
\textbf{What is Communication?} Defining {\em communication} can be challenging due to the use of this term in several disciplines with somewhat different meanings. This has been potrayed in the writing of Sarah Trenholm \cite{West+2003} who describes communication as a piece of luggage overstuffed with all manners of odd ideas and meanings. \citeasnoun{West+2003} defines communication  as:
\begin{quote}
\ssp
``A social process where individuals employ symbols to establish and interpret meaning in their environment.''
\end{quote}
\sdp
 The notion of being a ``social process'' involves, at least two or more individuals engaged in dynamic and ongoing interactions. Moreover, symbols can be simply some sort of arbitrary labels given to a phenomena and they can represent  concrete objects or an abstract ideas. Encyclopaedia Britannica also defines communication as ``the exchange of meanings between individuals through a common system of symbol''. But since this definition lacks the notion of sociality we find this definition incomplete.In our view, communication process takes place within the context of  symbol or message exchange between two or more parties with a clear intent to influence each others' behaviours.

According to a biological model of communication (Fig. \ref{fig:bio-comm-defined}), communication is a biological process where an  individual (sender) intentionally transmits encoded message though a physical signal and that, on being received and decoded by another individual of same species (receiver), influences receiver's behaviour \cite{Frings1997}. Note that, here individuals are of same species and thus they have a  shared message vocabulary and mechanism of message encoding/decoding. Although  this definition has not included the dynamics of a communication process, it is more precise for low-level biological and artificial systems. It accounts for the behavioural changes during communication process. These changes can be tracked though observing states of individuals.
   
%% FIG bio-comm defined
\begin{figure}
\centering
\includegraphics[width=8cm,height=4cm, angle=0]{./images/dia-files/animal-comm-defined.eps}
\caption{ A biological model of communication, adopted from \protect\citeasnoun{Frings1997}.}
\label{fig:bio-comm-defined} 
\end{figure}
%%
\textbf{Models of Communication.} If we explore the elements of communication, we can grab the whole picture involved in an individual's communication and sensing process. This can be explained through the study of the models  of communication. There exists a plenty of models of communication. For this thesis, here we briefly discuss three prominent models: 1) linear model, 2) interaction model and 3) transaction model.

Fig. \ref{fig:gen-comm-defined} embeds first two models inside a single circle. The linear model is shown by dashed lines. This model was introduced by Claude Shanon and Warren Weaver in 1949. Here communication is a one way process where a {\em message} is sent from a {\em source} to a {\em receiver} through a {\em channel}.

In Fig. \ref{fig:gen-comm-defined}, around the linear model, interaction model has been drawn. This model, proposed by Wilber Schramm in 1954, views communication as a two-way process that uses an additional {\em feedback} element  linking both source and receiver. This feedback is a response given to the source by the receiver to confirm how the message is being understood. Here, during message passing, both source and receiver utilize their individual {\em field of experiences} that describe the overlap of their common experiences, cultures etc. 

Unlike separate filed of experiences and discrete sending and receiving of messages, in transactional model, introduced by Barnlund in 1970, the sending and receiving of messages are done simultaneously. Here, the  field of experiences of source and receiver can overlap to some degrees. In all of the above three models, {\em noise} or a common message distorting element is present in the communication process. This noise can be occurred from the linguistic influences, i.e. message semantics,  physical or bodily influences, cognitive influences or even from biological
or physiological influences e.g., anger or shouting voice while talking.

The above models of communication describe the incremental complexities of message exchanging in the communication process. Surely, the transactional model is the most elegant model that prescribes adjusting the sender's message content while receiving an implicit or explicit feedback in real-time. For example, while advising her son to read a story book, a mother may alter her verbal message as she simultaneously ``reads'' the non-verbal message from her child's face. 

However, in case of a multi-robot system, such sophistication in communication may not be required or realizable by the current state-of-the-art in multi-robot communication technology. In this study, we follow the simple linear model that can meet the necessary communication requirements of a multi-robot system. The feedback has not been considered as we have assumed that all robots of our artificial multi-robot system have a common shared vocabulary such that a message is understood as it is sent.

%% TABLE: COMM-CATEGORIES
\begin{table}
\caption{General characteristics of common communication modes}
\label{table:comm-categories}
\begin{center}
\begin{tabular}{|l|l|l|}
\hline \textbf{Type} & \textbf{Indirect or} & \textbf{Direct or }\\
& \textbf{implicit communication} & \textbf{explicit communication}\\
\hline Centralized & Typically a central entity   & Both global \& local broadcast  \\
Communication & modifies the environment. & communication are commonly\\
 & It facilitates passive forms  &  used. P2P  communication can \\
  &  of communications, i.e.     & also occur. Here, exchange\\
&  communication without   &    of messages  occurs through a\\
& specific target recipient. &  central entity.\\
\hline Decentralized & All individuals are free to & P2P and local-broadcast \\
or Local & modify the environment &  are  most common. \\
Communication & and convey information &  Global broadcast occurs \\
 & to others. & to handle an emergency. \\
 & & All communications are local\\
& & without having a central entity.\\
\hline
\end{tabular}
\end{center}
\end{table}
%%
%% [COMM classification]
\begin{figure}
\centering
\includegraphics[width=10cm, angle=0]
{./dia-files/bio-comm-strategies.eps}
\caption{Common communication strategies observed in social systems.}
\label{fig:comm-strategies} 
\end{figure}
%%
\textbf{Communication modes and strategies.} The communication  structure of a system can broadly be classified into two major categories: centralized communication and decentralized or local communication. A centralized communication system generally has a central entity, e.g. a gateway,  that routes all incoming and outgoing communications. Individual nodes of this system often do not communicate each other directly. But they can send and receive messages through this central gateway.  This central entity can play other roles, such as access control, resource allocation and so on. On the other hand, in local communication, there is no central entity and each node can independently route messages to others.

%%
\begin{figure}
\centering
\includegraphics[width=10cm, angle=0]
{./dia-files/bio-comm-strategies-peers.eps}
%figure caption is below the figure
\caption{Number of recipients involved in various communication strategies.}
\label{fig:comm-strategies-peers}  % Give a unique label
\end{figure}
%%
In both biological and robotic literature two basic types of communication are often discussed: 1) direct or explicit communication and 2) indirect or implicit communication. {\em Direct communication} is an intentional communicative act of message passing that aims at one or more particular receiver(s) \cite{Mataric1998}. It typically exchanges information through physical signals. 

In contrast, {\em indirect communication}, sometimes termed as {\em stigmergy} in biological literature, happens as a form of modifying the environment, e.g. pheromone dropping by ants \cite{Bonabeau+1999}. In an ordinary sense, this is an observed behaviour and many robotic researchers call it as {\em no communication} \cite{Labella2007}. In order to avoid ambiguity, in this dissertation, by the term {\em communication}, we always refer to the {\em direct} communication.

Direct or explicit communication can be limited by a communication range and thus by a number of target recipients. Under both centralized and local communication, nodes can select a certain number of target recipients of their messages. This process specifies {\em to whom} a node intends to communicate. In this thesis, we have denoted this mechanism of target recipient(s) selection as {\em communication strategy}.  

Fig. \ref{fig:comm-strategies} shows the most common communication strategies found in a  social system.  In the simplest case, when only two nodes can communicate we call this P2P communication. When nodes can spread information to a limited number of peers of their locality, the communication takes the form of {\em local broadcast}, i.e. one sender and a few receivers within a certain locality. For example, when a foraging honey-bee gives the information of flower sources to a number of peers through various dances, it conveys this information to a few peers through a local broadcast. However, giving the sample of nectar through tactile or taste to its peers can be considered as an instance of P2P communication. The {\em global broadcast} strategy can be found in almost all social species to handle emergency situations, e.g. emitting alarm signal in danger.

Table \ref{table:comm-categories}  shows the relationship between various communication modes and their ways of adopting different strategies. Fig. \ref {fig:comm-strategies-peers} shows a typical count of average number of peers in various communication strategies. The actual number of peers under local broadcast strategy depends on a particular social system and it changes over time in different levels of interactions among individuals. Sec. \ref{bg:bio-comm} and \ref{bg:mrs-comm} reviews communication in biological social system  and multi-robot system respectively.
%%-------------------------------------------------------------------
\subsection{Division of labour or task-allocation}
\label{bg:def:dol}
Encyclopaedia Britannica serves the definition of {\em division of labour} as the ``separation of a work process into a number of tasks, where each task is performed by a separate person or a group of persons''. Originated from economics and sociology, the term division of labour is widely used in many branches of knowledge. As mentioned by the Scottish philosopher Adam Smith, the founder of modern economics:
\begin{quote}
\ssp 
The great increase of the quantity of work, which in consequence of the division of labour the same number of people are capable of performing, is owing to three different circumstances; firstly, to increase the dexterity in every particular workman; secondly, to the saving of the time which is commonly lost in passing from one species of work to another; and lastly, to the invention of a great number of machines which facilitate and abridge labour, and enable one man to do the work of many.
(Adam Smith (1776) in \citeasnoun{Sendova-Franks+1999})
\end{quote} 
\sdp
In sociology, DOL usually denotes the work specialization \cite{Sayer+1992}. Basically, it answers three major questions:
\begin{enumerate}
\item {\em What task?} This gives us the description of the tasks to be done, service to be rendered or products to be manufactured.
\item {\em Why dividing it to individuals?} This states the underlying social standards for this division, such as task appropriateness based on class, gender, age, skill etc.
\item {\em How to divide it?} This relates to the method or process of separating the whole task into small pieces of tasks that can be performed easily. 
\end{enumerate}
%% DoL: termites, skyscrapper
%% FIG: Termite nest
\begin{figure}[htp]
\centering
\subfloat[A termite nest]{\includegraphics[width=6cm, height=4cm]{./photos/termites_nest.eps}} 
\hspace{0.25cm}
\subfloat[Two Skyscrapers]{\includegraphics[width=6cm, height=4cm]{./photos/skyscraper.eps}}
\caption{(a) A termite colony constructs their nest through bottom-up approach, i.e. without a central planner. (b) Humans construct skyscrapers using a top-down plan. From http://www.harunyahya.com, last seen on 01/05/2010.}
\label{fig:termite-nest}
\end{figure}
%%
From the study of social insects and other biological societies, we can find that two major metrics of DOL have been established in literature: 1) task-specialization and 2) plasticity. {\em Task-specialization} is an integral part of DOL where a worker usually does not perform all tasks, but rather specializes in a set of tasks, according to its morphology, age, or chance \cite{Bonabeau+1999}. This DOL among nest-mates, whereby different activities are performed simultaneously by groups of specialized individuals, is believed to be more efficient than if tasks are performed sequentially by unspecialised individuals.

DOL is also characterized by {\em plasticity} which means that the removal of one class of workers is quickly compensated for by other workers. Thus distributions of workers among different concurrent tasks keep changing according to the external (environmental) and internal conditions of a colony \cite{Garnier+2007}.

In artificial social systems, like multi-agent or multi-robot system, the term ``division of labour'' is often found synonymous to ``task-allocation'' \cite{Shen+2001}. However, some researchers  argued to distinguish these terms due to the origin and particular contextual use of these terms \cite{Labella2007}. Particularly, DOL adopts the biological notion of collective task performance with little or no communication. On the other hand, task allocation follows the meaning of assigning task(s) to particular robot(s) based on individual robot capabilities, typically through explicit communication, such as {\em intentional cooperation} \cite{Parker1998}. Generally, the former is considered by swarm robotic system and latter is done under {\em traditional multi-robot system}. Sec. \ref{bg:mrs:overview} covers both of these approaches and Sec. \ref{bg:mrta} provides critical reviews on DOL under these approaches.

In this dissertation, to define DOL, we have followed the biological approach that emphasizes on having task-specialization and plasticity among workers. However, we do not put any restriction on the use of communication. In fact, we view DOL as a group-level phenomenon which occurs due to the individual agent's self-regulatory task-allocation behaviour. Our model of self-regulated DOL provides us necessary abstractions to the communication and sensing processes of individuals that are discussed in detail in Chapter \ref{afm} and \ref{local-comm}. 
%------------------------------------------------------------------
%%%%%%%%%%%%%%%%%%%%%
\section{Communication in biological social systems}
\label{bg:bio-comm}
Communication plays a central role in self-regulated DOL of biological social systems.In this section, communication among  social insects are briefly reviewed.
%
\subsection{Purposes, modalities and ranges}
Communication in biological societies serves many closely related social purposes. Most P2P communication include: recruitment to a new food source or nest site, exchange of food particles, recognition of individuals, simple attraction, grooming, sexual communication etc. In addition to that colony-level broadcast communication include: alarm signal, territorial and home range signals and nest markers, communication for achieving certain group effect such as, facilitating or inhibiting  a group activity \cite{Holldobler1990}.
\begin{table}
\caption{Common communication modalities in biological social systems}
\label{table:bio-comm-modalities}
\begin{center}
\begin{threeparttable}
\begin{tabular}{|l|l|l|}
\hline \textbf{Modality} & \textbf{Range} & \textbf{Information type}\\
\hline Sound & Long\tnote{a} & Advertising about food  source,  danger etc. \\                                                                                                                                               
\hline Vision & Short\tnote{b}  & Private, e.g. courtship display. \\
\hline Chemical  & Short/long & Various messages, e.g. food location, alarm etc.\\
\hline Tactile & Short & Qualitative info, e.g. quality of flower,\\ & & peer identification etc.\\
\hline Electric & Short/long & Mostly advertising types, e.g. aggression messages.\\
\hline
\end{tabular}
\begin{tablenotes}
\item [a]Depending on the type of species, long range signals can reach from a few metres to several kilometres.
\item [b]Short range typically covers from few mm to about a metre or so.
\end{tablenotes}
\end{threeparttable}
\end{center}
\end{table}

%[Modalities and Ranges]
Biological social insects use different modalities to establish social communication, such as, sound, vision, chemical, tactile,  electric and so forth (Table \ref{table:bio-comm-modalities}).  Sound waves can travel a long distance and thus they are suitable for advertising signals. They are also best for transmitting complicated information quickly \cite{Slater1986}. Visual signals can travel more rapidly than sound, but they are limited by the physical size or line of sight of an animal. They also do not travel around obstacles. Thus they are suitable for short-distance private signals.

%% FIG. Fireflies
\begin{figure}[htp]
\centering
\subfloat[Flashing fireflies]{\includegraphics[width=6cm, height=4cm]{./photos/fire-flies.eps}} 
\hspace{0.25cm}
\subfloat[A firefly emitting light]{\includegraphics[width=6cm, height=4cm]{./photos/firefly-light-under.eps}}
\caption{(a) Flashing lights of fireflies displaying their synchronous behaviours (b) A firefly can produce light to signal other fireflies.\protect\newline From http://www.letsjapan.markmode.com, last seen on 01/06/2010.}
\label{fig:fireflies}
\end{figure}
%
In ants and some other social insects, chemical communication is predominant. Any kind of chemical substance that is used for communication between intra-species or inter-species is termed as {\em semiochemical} \cite{Holldobler1990}. A pheromone is a semiochemical, usually a type of glandular secretion, used for communication within species. One individual releases it as a signal and others respond to it after tasting or smelling. Using pheromones individuals can code quite complicated messages in smells. For example, a typical an ant colony operates with somewhere between 10 and 20 kinds of signals. Most of these are chemical in nature. If wind and other conditions are favourable,  this type of signals emitted by such a tiny species can be detected from several kilometres away. Thus chemical signals are extremely economical of their production and transmission. But they are quite slow to diffuse away. But ants and other social insects manage to create sequential and compound messages either by a graded reaction of different concentrations of same substance or by blends of signals.

Tactile communication is also widely observed in ants and other species typically by using their body antennae and forelegs. It is observed that in ants touch is primarily used  for receiving information rather than informing something. It is usually found as an invitation behaviour in worker recruitment process. When an ant intends to recruit a nest-mate for foraging or other tasks it runs upto a nest-mate and beats her body very lightly with  antennae and forelegs. The recruiter then runs to a recently laid pheromone trail or lays a new one. In this form of communication limited amount of information is exchanged. In underwater environment some fishes and other species also communicate through electric signals where their nerves and muscles work as batteries. They use continuous or intermittent pulses with  different frequencies to learn about environment and to convey their identity and aggression messages.
%%
%%%%%%%%%%%%%%%%%%%%%%%%%%%%%%%%%%%%%%%%%%
\subsection{Signal active space and locality}
%bio-comm-ants-active-space
\begin{figure}
\centering
\includegraphics[width=12cm, angle=0]
{./dia-files/bio-comm-ants-active-space.eps}
%figure caption is below the figure
\caption{Pheromone active space observed in ants, reproduced from \protect\citeasnoun{Holldobler1990}.}
\label{fig:ants-active-space} % Give a unique label
\end{figure}
The concept of active space (AS) is widely used to describe the propagation of signals by species. In a network environment of signal emitters and receivers, active space is defined as the area encompassed by the signal during the course of transmission \cite{Mcgregor2000}. In case of long-range signals, or even in case of short-range signals, this area include several individuals where their social grouping allows them to stay in cohesion. The concept of active space is described somewhat differently in case some social insects. In case of ants, this active space is defined as a zone within which the concentration of pheromone (or any other behaviourally active chemical substances) is at or above threshold concentration \cite{Holldobler1990}. Mathematically this is denoted by a ratio:
\begin{equation}
AS = \frac{\textit{Amount of pheromone emitted (Q)}}{\textit{Threshold concentration at which the receiving ant responds (K)}}
\end{equation}
Here, Q is measured in number of molecules released in a burst or in per unit of time whereas K is measured in molecules per unit of volume. 
Fig. \ref{fig:ants-active-space} shows the use of active spaces of two species of ants: (a) {\em Atta texana} and (b) {\em Myrmicaria eumenoides}.  The former one uses two different concentrations of {\em 4-methyl-3-heptanone} to create attraction and alarm signals, whereas the latter one uses two different chemicals: {\em Beta-pinene} and {\em Limonene} to create similar kinds signals, i.e. alerting and circling.
 
The adjustment of this ratio enables individuals to gain a shorter fade-out time and permits signals to be more sharply pinpointed in time and space by the receivers. In order to transmit the location of the animal in the signal, the rate of information transfer can be increased by either lowering the rate of emission of Q or by increasing K, or both. For alarm and trail systems a lower value of this ratio is used. Thus, according to need, individuals regulate their active space by making it large or small, or by reaching their maximum radius quickly or slowly, or by enduring briefly or for a long period of time. For example, in case of alarm, recruitment and sexual communication signals where encoding the location of an individual is needed, the information in each signal increases as the logarithm of the square of distance over which the signal travels. From the precise study of pheromones it has been found that active space of alarm signal is consists of a concentric pair of hemispheres (Fig. \ref{fig:ants-active-space}). As an ant enters the outer zone, she is attracted inward toward the point source; when she next crosses into the central hemisphere she become alarmed. It is also observed that ants can release pheromones with different active spaces.

% indirect & b/c comm
\begin{figure}
\begin{minipage}[t]{0.48\linewidth}
\centering
\includegraphics[width=6cm, height=4cm, angle=0]
{./photos/ants_group_comm_bioteams_com.eps}
\caption{A group of ants following pheromone-trail. \protect\newline From http://www.bioteams.com, last seen on 01/06/2010.}
\label{fig:ant-indirect} % Give a unique label
\end{minipage}
\hspace{0.5cm}
\begin{minipage}[t]{0.48\linewidth}
\centering
\includegraphics[width=6cm,height=4cm, angle=0]{./photos/honey-bee-waggle-dance-knol-google.eps}
\caption{ A dancing honey-bee (\protect{\em centre}) and its followers. \protect\newline From http://knol.google.com, last seen on 01/06/2010.}
\label{fig:honey-bee-local-bc} % Give a unique label
\end{minipage}
\end{figure}
Active space has strong role in modulating the behaviours of ants. For example, when workers of {\em Acanthomyops claviger} ants produce alarm signal due to an attack by a rival or insect predator, workers sitting a few millimetres away begin to react within seconds. However, those ants sitting a few centimetres away take a minute or longer to react. In many cases, ants and other social insects exhibit modulatory communication within their active space where many individuals involve in many different tasks. For example, while retrieving the large prey, workers of {\em Aphaeonogerter} ants produce chirping sounds (known as \textit{stridulate}) along with releasing poison gland pheromones. These sounds attract more workers and keep them within the vicinity of the dead prey to protect it from their competitors. This communication amplification behaviour can increase the active space to a maximum distance of 2 meters.
\begin{table}
\caption{Common communication strategies in biological social systems}
\label{table:bio-comm-strategy}
\begin{center}
\begin{tabular}{|l|l|}
\hline 
\textbf{Communication strategy} & \textbf{Common modalities used}\\
\hline 
Indirect & Chemical and electric \\
%\hline 
P2P &  Vision and tactile\\
%\hline 
Local broadcast &  Sound, chemical and vision\\
%\hline 
Global broadcast & Sound, chemical and electric\\
\hline
\end{tabular}
\end{center}
\end{table}
%%%%%%%%%%%%%%%%%%%%%
\subsection{Common communication strategies}
\label{bg:bio-comm:strategies}
%%%
% FIG. Honey bee dance language
\begin{figure}
\centering
\subfloat[Honey-bee's waggle dance]{\includegraphics[width=6cm, height=4cm]{./photos/honey-bee-round-dance.eps}} 
\hspace{0.25cm}
\subfloat[Honey-bee's round dance]{\includegraphics[width=6cm, height=4cm]{./photos/honey-bee-waggle-dance.eps}}
\caption{Examples of local broadcast communication of honey-bees: (a) Honey-bees show waggle-dance (making figure of 8) when food is far and (b) they show round-dance without any waggle when food is closer (within about 75m of hive). From \protect\citeasnoun{Slater1986}.}
\label{fig:honey-bee-dances}
\end{figure}
%%
In biological social systems, we can find all different sorts of communication strategies ranging from indirect pheromone trail laying to local and global broadcast of various signals. Sec. \ref{bg:def:comm} discusses the most common four communication strategies in natural and artificial world, i.e. indirect, P2P, local and global broadcast communication strategies. Table \ref{table:bio-comm-strategy} lists the use of various communication modalities under different communication strategies. Here we give a few real examples of those strategies from biological social systems. In biological literature, the pheromone trail laying is one of the most discussed indirect communication strategy among various species of ants. Fig. \ref{fig:ant-indirect} shows a pheromone trail following of a group of foraging ants. This indirect communication strategy effectively helps ants to find a better food source among multiple sources, find shorter distance to a food source, marking nest site and move there etc. \cite{Hughes2008}. Direct P2P communication strategy is also very common among most of the biological species. Fig. \ref{fig:bees-ants-p2p-comm} shows P2P communication of ants and honey-bees. This tactile form of communication is very effective to exchange food item, flower nectar with each-other or this can be useful even in recruiting nest-mates to a new food source or nest-site.
%%%%%%%%%%%%%%%%%%%%%%
\begin{figure}
\centering
\subfloat[Two honey-bees]{\includegraphics[width=6cm, height=4cm]{./photos/honey-bee-p2p-hy23.eps}} 
\hspace{0.25cm}
\subfloat[Two ants ]{\includegraphics[width=6cm, height=4cm]{./photos/ants-p2p-hy14.eps}}
\caption{Example of P2P tactile communication: (a) Honey-bees exchange nectar samples by close contact (b) ants also exchange food or information via tactile communication. \protect\newline  From http://www.harunyahya.com/ last seen 01/05/2010.}
\label{fig:bees-ants-p2p-comm}
\end{figure}
%%
%%-----------------------------------------------------
\subsection{Roles of communication in task-allocation}
\label{bg:bio-comm:comm-role}
Communication among nest-mates and sensing of tasks are the integral parts of the self-regulated DOL process in biological social systems. They create necessary  preconditions for switching from one tasks to another or to attend dynamic urgent tasks. Suitable communication strategies favour individuals to select a better tasks. For example, \citeasnoun{Garnier+2007} reported two worker-recruitment experiments on black garden ants and honey-bees. The scout ants of {\em Lasius niger}  recruit uninformed ants to food source using a well-laid pheromone trails. {\em Apis mellifera} honey-bees also recruit nest-mates to newly discovered distant flower sources through waggle-dances. In the experiments,  poor food sources were given first to both ants and honey-bees. After some time,  rich food sources were introduced  to them. It was found that only honey-bees were able to switch from poor source to a rich source using their sophisticated dance communication.

%%
\begin{table}
\caption{Self-regulation of communication behaviours in biological social systems}
\label{table:bio-comm-task-urgency}
\begin{center}
\begin{tabular}{|l|l|l|}
\hline \textbf{Example event} & \textbf{Strategy} & \textbf{Modulation of communication}\\
&  &  \textbf{upon sensing tasks}\\
\hline Ant's alarm signal &  Global  & High concentration of pheromones\\
by pheromones & broadcast &  increase aggressive alarm-behaviours \\                                                                                                                                               
\hline Honey-bee's  & Local  &  High quality of nectar source increases \\
round dance & broadcast & dancing and foraging bees\\
\hline Ant's tandem run     & P2P & High quality of nest \\
for nest selection & &   increases traffic flow\\
\hline Ant's pheromone   & Indirect & Food source located at shorter distance\\
trail-laying to   & &  gets higher priority as less pheromone \\
food sources & & evaporates and more ants joins\\
\hline
\end{tabular}
\end{center}
\end{table}
%%%%%%%%%%%%%%%%%
Table \ref{table:bio-comm-task-urgency} presents the link between sensing the task and self-regulation of communication behaviours among ants and honey-bees. Here, we can see that communication is modulated based on the perception of  task-urgency irrespective of the communication strategy of a particular species. Under indirect communication strategy of ants, i.e. pheromone trail-laying, we can see that the principles of self-organization, e.g. positive and negative feedbacks take place due to the presence of different amount of pheromones for different time periods. Initially, food source located at shorter distance gets relatively more ants  as the ants take less time to return nest. So, more pheromone deposits can be found in this path as a result of positive feedback process.  Thus, the density of pheromones or the strength of indirect communication link reinforces ants to follow this particular trail.

%%
\begin{figure}
\centering
\includegraphics[width=6cm, angle=-90]
{./images/ch2/honey-bee-dance-stat.eps}
%figure caption is below the figure
\caption{Self-regulation in honey-bee's dance communication behaviours, produced after the results of \protect\possessivecite{Von1967} honey-bee round-dance experiment performed on 24 August 1962.}
\label{fig:honey-bee-dance-stat}  % Give a unique label
\end{figure}
%%
Similarly, perception of task-urgency influences the P2P and broadcast communication strategies. {\em Leptothorax albipennis} ant take lees time in assessing a relatively better nest site and quickly return home to recruit its nest-mates \cite{Pratt+2002}. Here, the quality of nest directly influences its intent to make more ``tandem-runs'' or to do tactile communication with nest-mates. We have already discussed about the influences of the quality of  flower sources to honey-bee dance.  Fig. \ref{fig:honey-bee-dance-stat} shows this phenomena more vividly. It has been plotted using the data from the honey-bee round-dance experiments of \citeasnoun[p. 45]{Von1967}. In this plot, Y1 line refers to the concentration of sugar solution. This solution was kept in a bowl  to attract honey-bees and the amount of this solution was varied from $\frac{3}{16}$M to 2M (taken as 100\%). In this experiment, the variation of this control parameter influenced honey-bees' communication behaviours while producing an excellent self-regulated DOL.

In Fig. \ref{fig:honey-bee-dance-stat} Y2 line represents the number of collector bees that return home. The total number of collectors was 55 (taken as 100\%). Y3 line plots the percent of collectors displaying round dances. We can see that the fraction of dancing collectors is directly proportional to the concentration of sugar solution or the sensing of task-urgency. Similarly, the average duration of dance per bee  is plotted in Y4 line. The maximum dancing period was 23.8s (taken as 100\%). Finally, from Y5 line we can see the outcome of the round-dance communication as the number of newly recruited bees to the feeding place. The maximum number of recruited bees was 18 (taken as 100\%). So, from an overall observation, we can see that bees sense the concentration of food-source  as the task-urgency and they self-regulate their round-dance communication behaviour according to their perception of this task-urgency. Thus, this self-regulated dancing behaviour of honey-bees attracts an optimal number of inactive bees to work.

Broadcast communication is one of the classic ways to handle dynamic and urgent tasks in biological social systems. It can be commonly observed in birds, ants, bees and many other species. Table \ref{table:bio-comm-task-urgency} mentions about the alarm communication of ants. Similar to the honey-bee's dance communication, ants has a rich language of chemical communication that can produce words through blending of different glandular secretions in different concentrations. Fig. \ref{fig:ants-active-space} shows how ants can use different concentrations of chemicals to make different stimulus for other ants. From the study of ants, it is clear to us that taking defensive actions, upon sensing a danger, is one of the highest-priority tasks in an ant colony. Thus, for this highly urgent task, ants almost always use their global broadcast communication strategy through their strong chemical signals and they make sure all individuals can hear about this task.  This gives us a coherent picture of the self-regulation of biological species based on their perception of task-urgency.
\begin{figure}
\centering
\subfloat[Polistes wasps]{\includegraphics[width=6cm, height=4cm]{./photos/Wasps_wikimedia.org_Polistes_nest_3_sjh.eps}}
\hspace{0.25cm}
\subfloat[Polybia wasps]{\includegraphics[width=6cm, height=4cm]{./photos/Polybia_occidentalis_I_JP6646_discoverlife.org.eps}} 
\caption{Colony founding in two types of social wasps (a) {\em Polistes}  founds colony by a few queens independently (b) {\em Polybia occidentalis}  founds colony by swarms. From http://www.discoverlife.org, last seen 01/05/2010.}
\label{fig:social-wasps}
\end{figure}
%%-----------------------------------------------
\subsection{Effect of group size on communication}
\label{bg:bio-comm:group-size}
The performance of cooperative tasks in large group of individuals also depends on the communication and sensing strategies adopted by the group. As introduced in Sec. \ref{intro:comm}, from the study of social wasps,  we can find that depending on the group size, different kinds of information flow occur in different types of social wasps \cite{Jeanne1999}. Polistes independent founders (Fig. \ref {fig:social-wasps}(a)) are species in which reproductive females establish colonies alone or in small groups with about $10^2$ individuals at maturity. Polybia swarm founders (Fig. \ref {fig:social-wasps}(b)) initiates colonies by swarm of workers and queens. They have a large number of individuals, in the order of $10^6$ and 20\% of them can be queen. 
\begin{figure}[htp]
\centering
\includegraphics[width=9cm, angle=0]
{./dia-files/jannae-fig10-info-flow-cmp.eps}
%figure caption is below the figure
\caption{Different patterns of information flow in two types of social wasps: polybia and polistes, reproduced from \protect\citeasnoun{Jeanne1999}.}
\label{fig:wasps-info-flow}  % Give a unique label
\end{figure}
%%
\begin{figure}[htp]
\centering
\includegraphics[width=6cm, angle=0]
{./images/ch2/jeanne-fig9-info-flow.eps}
%figure caption is below the figure
\caption{Information flow in polybia social wasps, reproduced from \protect\citeasnoun{Jeanne1999}.}
\label{figs:sf-wasps-info-flow}  % Give a unique label
\end{figure}
Fig. \ref{fig:wasps-info-flow} compares the occurrence of information flow among independent and swarm founders. In case of swarm founders information about nest-construction or broods food-demand can not reach to foragers directly. Fig. \ref{figs:sf-wasps-info-flow} shows the path of information flow among swarm founders for nest construction. The works of {\em pulp foragers} and {\em water foragers} depend largely on their communication with {\em builders}. On the other hand, in case of independent founders there is no such communication and sensing are present among individuals. In Sec. \ref{intro:comm} we have termed these two types communication and sensing strategies as GSNC (for independent founders) and LSLC (swarm founders).

\citeasnoun{Jeanne1999} explained the above phenomena of selecting different strategies in terms of task-specialization patterns and stochastic properties found in the group. In case of large colonies, many individuals repeatedly performs same tasks as this minimizes their interferences, although they still have a little probability to select a different task randomly. But because of the large group size, the queuing delay in inter-task switching keeps this task-switching probability very low. Thus, in swarm founders, task-specialization becomes very high among individuals. On the other hand, in small group of independent founders, specialization on a specific task is very costly, because this prevents individuals not to do other tasks whose task-urgencies can soon become very high. Thus these individuals tend to become generalist and do not communicate task information with each other.

\begin{figure}[htp]
\centering
\includegraphics[width=9cm, angle=0]
{./images/ch2/jeanne-fig6-group-size.eps}
%figure caption is below the figure
\caption{Productivity of social wasps shown as a function of group size, reproduced from \protect\citeasnoun{Jeanne1999}.}
\label{fig:wasps-group-productivity}  % Give a unique label
\end{figure}
%%
The above interesting findings on GSNC and LSLC in social wasps have been linked up with  the group productivity of wasps. Fig. \ref{fig:wasps-group-productivity} illustrates high group productivity in case of LSLC of swarm founders. The per capita productivity was measured as the number of cells built in the nest in (a) and the weight of dry brood in grams in (b). In case of independent founders this productivity is much lesser (max. 24 cells per queen at the time the first offspring observed) comparing to the thousands of cells produced by swarm founders.  This shows  us the direct link between high productivity of social wasps and their selection of LSLC strategy. These fascinating findings from wasp colonies have motivated us to test these communication and sensing strategies in a fairly large multi-robot system to achieve an effective self-regulated MRTA.
%====================================================
\section{Summary}