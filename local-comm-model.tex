\chapter{Local Communication for Self-regulated MRTA}
\label{local-comm}
As outlined in Sec. \ref{intro:comm}, inspired by the LSLC strategy found in a kind of social wasps, {\em polybia}, we have intended to carry out a second set of MRTA experiments.  Here we have expected that LSLC strategy can give us a necessary system-wide continuous flow of information for achieving the self-regulated DOL by AFM. In this chapter, we have discussed how we have designed our local P2P communication  model (LPCM) for MRS, based-on biological local communication strategies. We have presented our implementation algorithms.  In order to fit LPCM into our MRS implementation, as explained in Chapter \ref{afm}, we have added  certain functionalities to our previous implementation. In this chapter, we have discussed those additional features. Finally, we have compared the results from both GSNC and LSLC strategies in achieving self-regulated MRTA. 
%%%%%%%%%%%%%%%%%%%%%
\section{Motivations}
From our understanding of different kinds of communication strategies of biological social systems (Sec. \ref{bg:bio-comm}), this is obvious that an effective LSLC strategy can greatly enhance the self-regulated MRTA.  In fact, most swarm robotic researchers now acknowledge that LSLC is the one of the most critical components of a SRS, as global behaviours emerges from local interactions among the individuals and their environment. But how can we design and implement a LSLC scheme among robots ?\\
%%
In this study, we have used the concepts of pheromone active-space of ants to realize our simple LSLC scheme. As discussed in Sec. \ref{bg:bio-comm}, ants use various chemical pheromones with different active spaces (or communication ranges) to communicate different messages with their group members. Ants sitting near the source of this pheromone sense and respond quicker than others who wander in far distance. Thus both communication and sensing occurs within a small communication range\footnote{Although, generally communication and sensing are two different issues, however within the context of our self-regulated DOL, we have broadly viewed sensing as the part of communication process, either implicitly via environment, or explicitly via local peers.}. We have used this concept of communication range or locality in our LPCM.\\
%%
In order to find a suitable  range (or radius) of communication and sensing,  robotic researchers proposed various models that we have briefly discussed in Sec. \ref{bg:mrs-comm:key-issues}. Some researchers have stated that this range can be set at design time based on the capabilities of robots. Some other researchers have argued that they can be dynamically varied over time depending on the  cost of communication and sensing, e.g. density of peers, ambient noise in the communication channels, or even by aiming for maximizing information spread. In this study, we have followed the former approach as we have assumed that typically robots will not have the precise hardware to dynamically vary their communication and sensing ranges. We have left this issue of selecting best communication range as a topic of future research.
%%%%%%%%%%%%%%%%%%%%%
%% FIG: COMM-THESIS
%\begin{figure}
%\centering
%\includegraphics[width=9cm, angle=0]
%{./dia-files/robot-comm-strategies.eps}
%%figure caption is below the figure
%\caption{\small Self-regulated communication strategy}
%\label{fig:robot-comm-strategies} % Give a unique label
%\end{figure}
%%%%%%%%%%%%%%%%%%%%%
%\section{Geometrical analysis}
%%%%%%%%%%%%%%%%%%%%%
\section{A locality-based peer-to-peer communication model (LPCM)}
\label{local-comm:model}
\subsection{General characteristics}
Our LPCM relies on the local P2P communications among robots. Robots can communicate to its nearby peers within a certain communication radius, $r_{comm}$. Here by $r_{comm}$, we assume that within this distance robots can both sense tasks and exchange communication signals reliably without any significant loss of information. A robot $R_1$ is a {\em peer} of robot $R_2$, if spatial distance between $R_1$ and $R_2$ is less than its $r_{comm}$.
Similarly, when a robot comes within this $r_{comm}$ of a task, it can sense this the status of this task. Although the communication and sensing  range can be different based on robot capabilities, we have considered them same for simplicity of our implementation.\\
%%
As shown in Fig. \ref{fig:lcm}, local communication can also give robots similar task information as in centralized communication mode. It shows that it is not necessary for each robot to communicate with every other robot to get information on all tasks. Since robots can random walk and explore the environment we assume that for a reasonably high robot-to-space density, all task will be known to all robots after an initial exploration period. In order to update the urgency of a task, robots can estimate the number of robots working on a task in many ways: such as, by using their sensory perception (e.g., camera), by doing local P2P communication with others. In Fig. \ref{fig:lcm} we have shown that robots exchange both task information and self status signals to peers.\\
%%
We characterize our communication model in terms of three fundamental issues: 1) message content ({\em what to communicate}) 2) communication frequency ({\em when to communicate}) and 3) target recipients ({\em with whom to communicate}) \cite{Gerkey}. In a typical MRS, message content can be categorized into two types: 1) state of each individual robot and 2) target task (goal) information \cite{Balch}. The latter can also be subdivided into two types: 1) an individual robot's target task information and 2) information of all available tasks found in the system.\\ 
%%
Regarding the first issue, our communication model is open. Robots can communicate with their peers with any kind of message. Our model addresses the last two issues very specifically. Robots communicate only when they meet their peers within a certain communication radius ($r_{comm}$). Although in case of an environment where robots move relatively faster the peer relationships can also be changed dynamically. But this can be manipulated by setting the signal frequency and robot to space density to somewhat reasonably higher value.\\ 
%%
In terms of target recipients, our model differs from a traditional publish/subscribe communication (PSC) model by introducing the concept of dynamic subscription. In a traditional PSC model, subscription of messages happens prior to the actual message transmission. In that case prior knowledge about the subjects of a system is necessary. But in our model this is not necessary as long as all robots uses a common addressing convention for naming their incoming signal channels. In this way, when a robot meets with another robot it can infer the address of this peer robot's channel name by using a shared rule. A robot is thus always listening to its own channel for receiving messages from its potential peers or message publishers. On the other side, upon recognizing a peer a robot sends a message to this particular peer. So here neither it is necessary to create any custom subject namespace (e.g.,  as in \citeasnoun{Gerkey}) nor we need to hard-code information in each robot controller about the knowledge of their potential peers {\em a priori}. Subscription is done automatically based on their respective $r_{comm}$.
%
%%----------------------------------------------------------------
\subsection{Implementation algorithm}
Within the context of our self-regulated MRTA experiments under AFM,  here we have formalized these aspects of LPCM into an implementation algorithm. It has three major aspects: 
\begin{enumerate}
\item local sensing of peers and tasks,
\item listening to the task-information peer signals and 
\item emitting local task-information signals to peers. Here we present a typical implementation.
\end{enumerate}
%%
\textbf{\small Algorithm 1: Locality based P2P Communication Model}
\begin{algorithmic}[1]
\label{alg:lpcm}
\State $\textbf{Input: } RobotID, r_{comm}, r_{task}, TaskInfoDB, RobotPose,$\\ \hspace*{1cm}$ListeningBuffer, EmissionBuffer$
\State $\textbf{Output: }$ updated $TaskInfoDB$
\State \COMMENT {Perception of a task, if the robot is within $r_{task}$}
\State $ TaskPose \gets $ Estimate/get pose of a task within $r_{task}$
\State $ TaskUrgency \gets $ Estimate/get urgency of a task within $r_{task}$
\State $ TaskInfo \gets (TaskPose, TaskUrgency) $ 
\State $TaskInfoDB \gets$ \textbf{UpdateTaskInfoDB(}$TaskInfo$\textbf{)}
% P2P Interaction, listen signal
\State \COMMENT {Listening of LocalTaskInfo signal(s) from peers}
\State $ RobotPeers \gets $ Identify/get a list of peers within $r_{comm}$
\ForAll {peer $\in RobotPeers$}
\If {(caught a LocalTaskInfo signal from nearby peer)}
\State $ListeningBuffer \gets $ LocalTaskInfo of peer
\State $TaskInfoDB \gets$ \textbf{UpdateTaskInfoDB(}$ListeningBuffer$\textbf{)}
\EndIf
\EndFor
%%
\State \COMMENT {Emitting of own LocalTaskInfo signal to peers}
\State $ EmissionBuffer \gets TaskInfoDB$
\ForAll {peer $\in RobotPeers$}
\State \textbf{EmitLocalTaskInfoSignal(}peer, $EmissionBuffer$ \textbf{) }
\EndFor
\State $RobotPeers \gets 0$
\end{algorithmic}
%%

From Algorithm 1, we see that a robot controller is initialized with its specific $RobotID$ and default values of $r_{comm}$ and $r_{task}$ that correspond to the robot's communication and sensing range respectively. We have assumed that these values are same for all robots and for all tasks. Initially a robot has no information about tasks, i.e. {$TaskInfoDB$} is empty. It has neither received nor transmitted any information yet, i.e. corresponding data buffers, $ListeningBuffer$ and $EmissionBuffer$, are empty. Upon sensing a task, robot determines the task's pose and urgency of that task (according AFM rules described in Sec. \ref{afm:mrs-interpretation}). This is not strictly necessary as this information can also be available from alternate sources, e.g. via communicating with a TPS.\\
%%
Robot controller then executes the \textbf{UpdateTaskInfoDB()} that modified the robot's information about the corresponding task. In second step, robot senses its nearby peers located within $r_{comm}$ and add them in $RobotPeers$. The potential LocalTaskInfo signal reception from a  peer again triggers the \textbf{UpdateTaskInfoDB()} function. In the last step, robot emits its own task information as a LocalTaskInfo  signal to its peers. Finally it cleans up the current values of $RobotPeers$ for running a new cycle.
%
%%%%%%%%%%%%%%%%%%%%%%%%%%%%%%%%%%%%%%%%%%%%%%%%%%%%%%%%%%%%%%%%%%%%%%%%%%%%%%%%%
\section{Experiment Design}
\label{local-comm:expt-design}
%
Our local communication experiments follow the similar design of experimental parameters and observables of Series B experiments as outlined in Sec. \ref{afm:expt-design}. Table \ref{local-comm:expt-design} highlights the major parameters of these experiments. Here we can see that there are two new parameters in Series C and Series D experiments, i.e. communication and sensing ranges. For the sake of simplicity, both of these values are kept same in both series of experiments.\\
%% 
Series D uses larger ranges that enables our robot to capture more task information at a time. That is similar to a working ``global sensing and global communication'' of information by the robots, since in a large group of robots, it is not practical that a robot communicate with all other robots. On the other hand, Series C uses the half of the values of Series D. This enables our robots to capture less information, that mimics a true LSLC strategy. The main motivation for using two different ranges is to find any significant difference in performance from these two types of experiments. Although these differences may not be significant in many times, they can reflect the rationale behind our selection of ranges without being purely arbitrary. 
%%
\begin{table}
\caption{Two series of local communication experiments}
\label{local-comm:expt-design}
\begin{center}
\begin{tabular}{|l|c|c|}
\hline Parameter & \hspace*{0.2cm} Series C $\mid$ Series D\\
\hline task-perception range ($r_{task}$) & $0.5 m \mid 1 m$\\
\hline Communication range ($r_{comm}$) & $0.5 m \mid 1 m$\\
\hline Total number of robots ($N$) & 16 \\
\hline Total number of tasks ($M$) & 4 \\
\hline Experiment area ($A$) & 4 $m^2$\\
%\hline Intial task urgency ($\Phi_{INIT}$) & 0.5\\
%\hline Task urgency increase rate ($\Delta\phi_{INC}$) & 0.005\\
%\hline Task urgency decrease rate ($\Delta\phi_{DEC}$) & 0.0025\\
%\hline Intial sensitization ($K_{INIT}$) & 0.1\\
%\hline Sensitization increase rate ($\Delta k_{INC}$) & 0.03\\
%\hline Sensitization decrease rate ($\Delta k_{DEC}$) & 0.01\\
%\hline A very small distance ($\delta$)& 0.000001\\
%\hline Task info update interval ($\Delta TS_{u}$) & 5s\\
%\hline Task info signal emission interval ($ \Delta TS_{e}$)& 2.5s\\
%\hline Robot's task time-out interval ($\Delta RT_{to} $)& 10s\\
\hline
\end{tabular}
\end{center}
\end{table}
%
%%========================================================================
\section{Implementation}
\label{sec:impl}
\begin{figure}
\centering
\includegraphics[width=9cm, angle=0]
{./dia-files/RIL-Expt-Setup2.eps}
%figure caption is below the figure
\caption{\small Hardware and software setup under local communication experiments.}
\label{fig:local-setup} 
\end{figure}
On top  of the base implementation of our MRS, we have extended our MRS so that we can fit our LPCM into it. The additional communication interfaces and changes in RCC in TPS implementation are mentioned here. The details  of our MRS implementation can be found in Sec. \ref{afm:impl}. 
%%%
\subsection*{Additional communication interfaces under LPCM}
We have added three new D-Bus signal interfaces in this version of our MRS. They are briefly discussed below.\\
\textbf{RobotPeers: }This signal is emitted by SwisTrack to each robot's signal path in a common ``ac.uk.newport.ril.SwisTrack'' interface. The payload of this signal contains the list of IDs of peers of each robot within it's $r_{comm}$.  These peer-IDs are extracted by analysing the captured image frame.  This signal is caught by RCCs and used in emitting the LocalTaskInfo signal (discussed below).\\
\textbf{TaskNeighbohood: }This signal is emitted by SwisTrack for TPS and the payload of this signal contains the list of IDs of robots that are co-located within a task's perception range, $r_{task}$. TPS catches one signal per task and based-on the robot IDs, it emits the TaskInfo signal to those robot's RCC path (as discussed in Sec. \ref{afm:impl}).\\ 
\textbf{LocalTaskInfo: }This signal is emitted by RCCs to their peers that are co-located within the same communication range $r_{comm}$. The payload of this signal contains the known task information of a  RCC. This task-information can be gathered either through task-perception (via TaskInfo signal received from TPS) or via communications i.e. through LocalTaskInfo signals from its peers.
%%
\subsection*{Modifications in RCC and TPS}
As shown in Fig. \ref{fig:local-setup}, RCCs disseminate task information to each other by \textit{LocalTaskInfo} D-Bus signal. The contents of task information are physical locations of tasks and their urgencies. However as our robots are incapable of sensing task directly. So it relies on TPS for a task-perception signal, TaskInfo, that contains the actual task location and urgency. When a robot $r_i$ comes within $r_{task}$ of a task $j$, SwisTrack reports it to TPS (by {\em TaskNeighbors} signal) and TPS then gives it current task information to $r_i$ by {\em TaskInfo} signal. As we have discussed, TPS catches feedback signals from robots via \textit{RobotStatus} signal. This can be used to inform TPS about a robot's current task id, its device status and so on. TPS uses this information to update relevant part of task information such as, task-urgency. This up-to-date information is encoded in next TaskInfo signal.\\
%%
From the above discussions, we can see that our base implementation of RCC's task-allocation (i.e. TaskSelector) is almost unchanged in this local implementation. The only thing we need to extend is the D-Bus signal reception and emission interfaces that catches the above signals and put the signal payload in DataManager. For the sake of simplicity, we have merges the perceived TaskInfo with communicated LocalTaskInfo into one so that TaskAllocator always gets all available task information. We have not made any major change in other modules of RCC, i.e. DeviceController. The full Python implementation of this RCC is available for download from the author's GitHub repository\footnote{http://github.com/roboshepherd/EpuckDistributedClient }.\\ 
%%
In our base implementation TPS periodically broadcasts TaskInfo signals to all robots. But in this local version, TPS only emits TaskInfo signal when a robot is within a task's perception range, $r_{task}$ based-on the TaskNeighbour signal from SwisTrack. For this additional interface, we have extended D-Bus SignalListener to catch this additional D-Bus signal.  No major changes ared done in the other part of TPS implementation. The full Python implementation of TPS is available for download from this GitHub repository\footnote{http://github.com/roboshepherd/DitributedTaskServer}.
%%%%%%%%%%%%%%%%%%%%%%%%%%%%%%%%%%%%%%%%%%%%%%%%%%%%%%%%%%%%%%%%%%%%%%
\section{Results}
\label{sec:results}
%%
The results from Series C and Series D experiments are organized by following the organization of results in our previous chapter. The sample raw task-urgencies of Series C and Series D experiments are shown in Fig. \ref{fig:raw-urgencies-SC-SD}. In case of Series D, we can see that an unattended task, Task4, was not served by any robot for a long period and later it was picked up by some of the robots. 
%%% raw urgencies
\begin{figure}
\centering
\subfloat[Series C]
{\includegraphics[width=6cm, height=4cm]{images/local-500cm/PlotUrgencyLog-2010Feb15-171017.eps}}
\subfloat[Series D]
{\includegraphics[width=6cm, height=4cm]{images/local-1m/PlotUrgencyLog-2010Feb17-112141.eps}}
\caption{\small Task-urgencies observed from (a) Series C and (b) Series D experiments}
\label{fig:raw-urgencies-SC-SD} 
\end{figure}
%%
%% – Workload
\begin{figure}
\centering
\subfloat[Series C]
{\includegraphics[width=6cm, height=4cm]{images/local-500cm/TaskUrgencyStat-SC.eps}}
\subfloat[Series D]
{\includegraphics[width=6cm, height=4cm]{images/local-1m/TaskUrgencyStat-SD.eps}}
\caption{\small Shop-floor work-load history from (a) Series C and (b) Series D experiments}
\label{fig:workload-SC-SD} 
\end{figure}
%%%
The dynamic shop floor work-load is shown in Fig. \ref{fig:workload-SC-SD}. These plots shows similar work-load as experienced in Series A and Series B experiments. Here, we can also see that initially the sum of changes of task urgencies are towards negative direction. This implies that tasks are being served by a high number of robots. When the task urgencies stabilize near zero the fluctuations in urgencies become minimum. Since robots chose tasks stochastically, there will always be a small changes in task urgencies.\\
%
% Plasticity
\begin{figure}
\centering
\subfloat[Series C]
{\includegraphics[width=6cm, height=4cm]{images/local-500cm/Local50cm-Plasticity-SC.eps}}
\subfloat[Series D]
{\includegraphics[width=6cm, height=4cm]{images/local-1m/Local1m-Plasticity-SD.eps}}
\caption{\small Self-organized allocation of workers in (a) Series C and (b) Series D experiments}
\label{fig:plasticity-SC-SD} 
\end{figure}
%
The active worker ratios of both Series C and Series D experiments are plotted in Fig. \ref{fig:plasticity-SC-SD}. Series C data shows us a large variation in this active worker ratios.
%%
%% Shop-Task performance
The task-performance of our manufacturing shop-floor scenario under both Series C and Series D experiments are plotted in Fig. \ref{fig:apcd-SC-SD} and Fig. \ref{fig:apmw-SC-SD}. 
\begin{figure}
\centering
\subfloat[APCD]
{\includegraphics[width=6cm, height=4cm]{images/apcd-SC-SD.eps}}
\subfloat[APMW]
{\includegraphics[width=6cm, height=4cm]{images/apmw-SC-SD.eps}}
\caption{\small task-performance of our shop-floor manufacturing scenario (a) APCD and (b) APMW} 
\end{figure}
For Series C we have got average production completion time 121 time-steps (605s) with SD = 36 time-steps (180s). For Series D,  average production completion time is 123 time-steps (615s) with SD = 40 time-steps (200s). According to Eq. \ref{eqn:min-pmm}, our theoretical minimum production completion time is 50 time-steps (250s) as discussed in Sec \ref{afm:results}.  The values of APCD are as follows. For Series C, $\zeta$ = 1.42 and for Series D, $\zeta$ = 1.46. For both series of experiments APCD values are very close.\\
%%
For APMW, Series C experiments give us an average time length of 359 time-steps (1795s). In this period we calculated APMW and it is 5 time-steps with SD = 17 time-steps and $\chi$ = 0.023420. For Series D experiments, from the average 357 time-steps (1575s) of maintenance activity of our robots per experiment run, we have got APMW, $\chi$ = 0.005359 which corresponds to the pending work of 2 time-steps (10s) where SD = 7 time-steps.
%%%%%%%%%%%%%%%%%%%%%%%%%%%%%%%%%%%%%%%%%%%%%%%%%%%%%%%%%%%%%%%%%%%%%%%%%%%%%%%%
%\section{Related Works}
%\label{sec:bg}
%Krieger and Billeter presented MRTA in a team of 12 robots in ants' foraging scenario where a central server sent broadcast message containing energy level of the colony to all robots \cite{Krieger}. They also applied a form of indirect communication in order to share information about discovered food sources among robots. Although this P2P communication did not happen in real time, it improved the robustness of peer recruitment when compared with centralized communication only. Agassounon and Martinoli compared three task-allocation algorithms using simulations and reported that algorithm that used information sharing among local peers became robust in worker allocation \cite{Agassounon}. However, environment condition should be known {\em a priori} to optimize some parameters. By modulating the transmission power of local on-board wireless device of E-puck robots, Cianci {\em et al.} enabled them to communicate in different local radii \cite{Cianci}. In a self-organized decision making experiment, decision of 15 robots converged faster when local communication radius was relatively higher. Since the number of tasks was only two (left or right wall following), robots got global view of the system with increased communication radius. Thus they agreed quickly in one task.
%%%%%%%%%%%%%%%%%%%%%%%%%%%%%%%%%%%%%%%%%%%%%%%%%%%%%%%%%%%%%%%%%%%%%%%%%%%%%%%%
\section{Summary}
\label{loacl-comm:summary}
We presented a locality based dynamic P2P communication model that achieves similar or better self-regulated multi-robot task allocation (MRTA) than its centralized counterpart. Particularly the reduction in robot movement states that local model is preferable when we need to minimize robot energy usage. Our model assumes no prior knowledge of the environment and it also does not depend on the number of the robots. We presented an abstract algorithm that can be used with various P2P communication technologies. We reported our implementation of this algorithm   by using  D-Bus technology in Linux. Comparative results from local mode experiments shows us that robots relatively perform better when the radius of communication is such small that most of the robots only communicate with their closest peers. This is contrary to the findings from \cite{Rutishauser, Epuck} where increased amount of information help to get desired task-allocation quickly. In our case, more information exchange with larger communication radius or with centralized communication increased task switching among robots. In case of local information exchange, robots have more chances to select local tasks by the virtue of self-regulating principles. Thus in local communication mode our system converged better and significantly reduced robot motions.