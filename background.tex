\chapter{Background and Related Work}
\label{bg}
%%%%%%%%%%%%%%%%%%%%%
\section{Definition of key terms}
\subsection{Self-regulation}
Animals and flying beings that live on or above earth form social communities like human society \cite{SIHQ1995}. In recent years, the biological study of social insects and other animals reveals 
us that individuals of these self-organized  societies can solve various complex and large everyday-problems with a few simple behavioural rules relying on their minimum sensing and communication abilities \cite{Garnier+2007,Camazine+2001}. Some common tasks of these biological societies include: dynamic foraging, building amazing nest structures, division of labour among workers and so forth. These tasks are done by colonies,  ranging from a few animals to thousands or millions of individuals, that exhibit surprising efficiency in their tasks with both robustness and flexibility. Today these findings have inspired scientists and engineers to use this knowledge of biological self-organization in developing solutions for various problems of our man-made artificial systems, such as traffic routing in telecommunication and vehicle networks, design control algorithms for groups of autonomous robots and so forth.
%%
\begin{figure}[htp]
\begin{minipage}[t]{0.48\linewidth}
\centering
\includegraphics[width=6cm, height=4cm, angle=0]
{./photos/ants-hy27c.eps}
\caption{ Ants leaf nest construction}
\label{fig:ant} % Give a unique label
\end{minipage}
\hspace{0.5cm}
\begin{minipage}[t]{0.48\linewidth}
\centering
\includegraphics[width=6cm,height=4cm, angle=0]{./photos/ants_spacecollective.org_greenTreeAnt.eps}
\caption{ Ants prey retrieval}
\label{fig:self-org-agent} % Give a unique label
\end{minipage}
\end{figure}
%
\begin{figure}[htp]
%\begin{minipage}[t]{0.68\linewidth}
\centering
\includegraphics[height=8cm, angle=0]
{./images/dia-files/self-org-1}
%figure caption is below the figur
\caption{ Self-organization viewed from four (A-D) inseparable perspectives}
\label{fig:self-org-1} % Give a unique label
%\end{minipage}
\hspace{0.5cm}
%\begin{minipage}[t]{0.28\linewidth}
\centering
\includegraphics[height=5cm, angle=0]{./images/dia-files/self-org-agent}
\caption{ Three major parts of a self-regulated agent}
\label{fig:self-org-agent} % Give a unique label
%\end{minipage}
\end{figure}
Self-organization in biological and  other systems are often characterized in terms of four major ingredients: 1) Positive feedback, 2) Negative feedback, 3) Presence of multiple interactions among individuals and their environment and 4) Amplification of fluctuations  (random walks, errors, random task-switching etc.) \cite{Bonabeau+1999,Camazine+2001}. As illustrated in Fig. \ref{fig:self-org-1} an external observer can recognize a self-organized system by observing the individual interactions of that system from these four perspectives. The first perspective is a positive feedback or amplification that results from the execution of simple behavioural ``rules of thumb''. For example, recruitment to a food source through trail laying and trail following in some ants  is a positive feedback that creates the conditions for the emergence of trail network at the global ( or an outside observer) level. The second perspective is negative feedback that counterbalances positive feedback usually to stabilize a collective patterns, e.g., crowding at the food sources (saturation), competition between paths to food sources etc. The third perspective is the presence of multiple direct peer-to-peer or indirect stigmergic (e.g., pheromone dropping in ants) interactions. The former is the main concern of this thesis and is discussed latter in detail. Finally, the fourth  perspective is the amplification of fluctuations that comes from the stochastic events. For example errors in trail following of some ants may lead some foragers to get lost and later to find new, unexploited food sources and then recruit others. 
%%
\begin{figure}[htp]
\begin{minipage}[t]{0.48\linewidth}
\centering
\includegraphics[width=6cm, height=4cm, angle=0]
{./photos/honey-bee-nest-hy103.eps}
\caption{ Honey-bee nest on a tree-branch}
\label{fig:ant} % Give a unique label
\end{minipage}
\hspace{0.5cm}
\begin{minipage}[t]{0.48\linewidth}
\centering
\includegraphics[width=6cm,height=4cm, angle=0]{./photos/honey-bee-comb-building-knol-google.eps}
\caption{ Honey bee comb construction}
\label{fig:self-org-agent} % Give a unique label
\end{minipage}
\end{figure}
%
In a self-organized system an individual agent may have limited cognitive, sensing and communication capabilities, but they are collectively capable of solving complex and large problems.  Since the discovery of these collective behavioural patterns of self-organized  societies scientists observed modulation or adaptation of behaviours in the individual level. For example, in order to prevent a life-threatening humidity-drop in the colony, cockroaches maintain a locally sustainable humidity level by increasing their tendency to aggregate, i.e., by regulating their individual aggregating behaviours \cite{Garnier+2007}. As shown in Fig. \ref{fig:self-org-agent} this  self-regulation (SR) of an individual agent is depicted through a triangle where it's base-arm of simple behavioural rules of thumb (e.g., intense aggregation in low humidity in the previous example) is supported by two side-arms: local communication and local sensing. This local sensing is sometimes also referred to as sensing or information gathering from the work in progress (e.g., stigmergy) and the local communication mentioned here is directly linked with peer-to-peer (P2P) communication with neighbours  \cite{Camazine+2001}.

%% School of Fish
\begin{figure}[htp]
\begin{minipage}[t]{0.48\linewidth}
\centering
\includegraphics[width=6cm, height=4cm, angle=0]
{./photos/School_of_Fish_group_cohesion.eps}
\caption{ School of fish group cohesion}
\label{fig:ant} % Give a unique label
\end{minipage}
\hspace{0.5cm}
\begin{minipage}[t]{0.48\linewidth}
\centering
\includegraphics[width=6cm,height=4cm, angle=0]{./photos/schoo_of_fish_shark-bigtiger.eps}
\caption{ School of fish group defence }
\label{fig:self-org-agent} % Give a unique label
\end{minipage}
\end{figure}

SR has been studied in many other branches of knowledge. In most places of literature, SR refers to the exercise of control over oneself to bring the self into line with preferred standards \cite{Baumeister+2007}. One of the most notable self-regulatory process is the human body's homoeostatic process where the human body's inner process seeks to return to its regular temperature when it gets overheated or chilled. Baumeister et al. has referred self-regulation to goal-directed behaviour or feedback loops, whereas self-control may be associated with conscious impulse control.  In psychology, SR denotes the strenuous actions to resist temptation or to overcome anxiety. SR is also divided into two categories: 1) conscious and 2) unconscious SR. Conscious SR puts emphasis on conscious, deliberate efforts in self-regulation. On the other hand, unconscious self-regulation refers to the automatic self-regulatory process that is although not nearly as labour intensive, but operate in harmony with unpredictable, unfolding events in the environment, using and transforming the available informational input in ways that help to attain an activated goal.
%% Bats navigation
\begin{figure}[htp]
\begin{minipage}[t]{0.48\linewidth}
\centering
\includegraphics[width=6cm, height=4cm, angle=0]
{./photos/bats_hy2.eps}
\caption{ A bat colony with about 50 million bats}
% Ref: Merlin D. Tuttle, "Saving North America’s Beleaguered Bats", National Geographic, August 1995, p. 40. 
\label{fig:ant} % Give a unique label
\end{minipage}
\hspace{0.5cm}
\begin{minipage}[t]{0.48\linewidth}
\centering
\includegraphics[width=6cm,height=4cm, angle=0]{./photos/bats_hy3.eps}
\caption{ Amazing navigation ability of bats: always fly back to nest on a straight route from wherever they are}
\label{fig:self-org-agent} % Give a unique label
\end{minipage}
\end{figure}

The concepts of SR is also commonly used in cybernetic theory where SR in inanimate mechanisms shows that they can regulate themselves by making adjustments according to programmed goals or set standards. A common example of this kind can be found in a thermostat that controls a heating and cooling system to maintain a desired temperature in a room. In physics, chemistry, biology and some other branches of natural sciences, the concept of SR is centred around the study of self-organizing individuals. SR has also been studied in the context of human social systems where it originates from the division of social labor that creates SO process that has self-regulating effects \cite{Kppers+1990}. Two types of SR have been reported in many places of literature of sociology: 1) SR from SO and 2) SR from activities of components in a heterarchical organization. It is interesting to note that self-regulation in biological species provides the similar evidences of bottom-up approach of self-regulation of heterarchical organization through interaction of individuals or the absence of strict hierarchy {Beer1981}. 

From the above discussion, we see that the term {\em self-regulation} carries a wide range of meaning in different branches of knowledge. In psychology and cognitive neuroscience point of view, self-regulation is discussed in an individual's perspective whereas, in biological and social contexts the SR is discussed in a context of a group of individuals or the society as a whole. In  this thesis, the latter context is more appropriate where  SR covers both aspects of monitoring ones own state and environmental changes in relation to the communal goal and thus making adjustments of self behaviours with respect to the changes found. 
%%%%
\subsection{Communication} 
\label{bg:def:comm}
%% Comm defined
\begin{figure}
\centering
\includegraphics[width=10cm, angle=0]
{./images/dia-files/comm-defined.eps}
\caption{ General models of communication}
\label{fig:gen-comm-defined} % Give a unique label
\end{figure}
%%
Defining {\em communication} can be challenging. Due to the use of this term in several disciplines with somewhat different meanings. This has been potrayed in the writing of Sarah Trenholm (\cite{West+2003}) who describes communication as piece of luggage overstuffed with all manner of odd ideas and meanings. This dissertation closely follows the definition of \cite{West+2003} where communication is defined as:
\begin{quotation}
``A social process where individuals employ symbols to establish and interpret meaning in their environment.''
\end{quotation}
The notion of being a ``social process'' involves (two or more) individuals and interactions that is dynamic and ongoing. Moreover symbols are simply some sort of arbitrary labels given to a phenomena and they can represent  concrete objects or an idea or thought. Encyclopaedia Britannica also defines communication as ``the exchange of meanings between individuals through a common system of symbol''. But since it ;lacks the notion of sociality I consider it incomplete for our purpose. There are many other debates related to communication, such as  the intentionality debate \cite{West+2003}, symbol grounding and so on. However, in order to draw some tractable boundaries , I consider communication process within the context of  symbol or message exchange between two or more parties with a clear intent  to influence each others' behaviours.\\
The elements of communication can give us us the whole picture involved in communication process and this can be explained through the study of the models  of communication. There exists a plenty of models of communication. For the purpose of this study, here I  discuss three prominent models: 1) linear model, 2) interaction model and 3) transaction model.  
Fig. \ref{fig:comm-model-1} combines first two models in a single diagram. In linear model, as introduced by Claude Shanon and Warren Weaver (1949), communication is a one way process where a message is sent from a source to a receiver through a channel. On top of linear view, in interactional model proposed by Wilber Schramm (1954), communication is a two-way process with an additional feedback element  that links both source and receiver. This feedback is a response given to the source by the receiver to confirm how the message is being understood. Here, during message passing, both source and receiver utilize their individual field of experiences that describe the overlap of their common experiences, cultures etc. Unlike separate filed of experiences and discrete sending and receiving of message in interactional model, in transactional model, introduced by Barnlund(1970), the sending and receiving of message is done simultaneously and their  field of experiences also overlaps to some degree. In all of the above three models a common message distorting element, i.e., noise is present. This noise can be occurred from the linguistic influences (message semantics),  physical or bodily influences, cognitive influences or even from biological
or physiological influences (e.g., anger or shouting voice while talking) and so on.\\
According to a biological model of communication (Fig. \ref{fig:bio-comm-defined}), communication is a biological process where an  individual (sender) intentionally transmits encoded message though physical signal and that, on being received and decoded by another individual of same species (receiver), influences receiver's behaviour \cite{Frings1997}. Note that, here individuals are of same species and thus they have a  shared message vocabulary and mechanism of message encoding/decoding. Although  this definition has not included the dynamics of a communication process, it is more precise for low-level biological and artificial systems. It accounts for the behavioural changes during communication process. These changes can be tracked though observing states of individuals.\\   
%% bio-comm defined
\begin{figure}
\centering
\includegraphics[width=8cm, angle=0]{./images/dia-files/animal-comm-defined.eps}
\caption{ A biological model of communication }
\label{fig:bio-comm-defined} 
\end{figure}
The above models of communication describe the incremental complexities of message exchanging in a communication process. Surely the transactional model is comparatively the most sophisticated model that prescribes adjusting the sender's message content while receiving an implicit or explicit feedback in real-time. For example, while speaking with her son for advising to read a story book, a mother may alter her verbal message as he simultaneously ``reads'' the non-verbal message of her child. However, in case of MRS, such sophistication may not be required or realizable by the current state of art in communication technology. In this study we follows the simple linear model that gives us the ease of implementation. The feedback is not accounted as we assumed that our artificial robotic system has a same shared vocabulary so that a message is understood as it is sent. Sender never waits for an additional feedback to end sending a message. 

Following the linear model of communication, the amount of communicated information associated with a certain random variable X can be calculated by the concept of {\em Shanon entropy}. Adopting the notation of Feldman \cite{Feldman1997}, and indicating a discrete random variable with the capital letter X, which can take values $x \in \chi$, the information entropy is defined as:
\begin{equation}
\label{eq:entropy}
H[X] = - \sum_{x \in \chi } p(x) \cdot  \log_{2} p(x)
\end{equation}
where p(x) is the probability that X will take the value of x. H[X] is also called the {\em marginal} entropy of X, since it depends on only the marginal probability of one random variable. The marginal entropy of the random variable X is zero if X always assumes the same value with p(X =x′)= 1, and maximum if X assumes all possible states with equal probability.

For example, in order to measure information flow in an elementary communication system, let {\em bit} be the unit amount of information needed to make a choice between two equiprobable alternates. If {\em n} alternates are present, a choice provides the following quantity of information: $H = log_{2} \hspace{1mm} n $. Thus sending of n equiprobable messages reduces $log_{2} \hspace{1mm} n $ amount of uncertainty and thus the amount of information is  $log_{2} \hspace{1mm} n $ bit.  Similarly, according to Eq. \ref{eq:entropy}, the value of H[X] depends on the discretization of x. For instance, if the value of random variable x is discretized into 4, then p(x) becomes $\frac{1}{4}$  leading to H[X] = - $ 4 \cdot \frac{1}{4} \cdot log_{2} \hspace{2mm} \frac{1}{4}$  = 2.\\
%% TABLE: COMM-CLASSES
\begin{table}
\caption{General characteristics of common communication modes}
\label{table:comm-classify}
\begin{center}
\begin{tabular}{|l||l|l|}
\hline \textbf{Type} & \textbf{Indirect or} & \textbf{Direct or }\\
& \textbf{implicit strategy} & \textbf{explicit strategy}\\
\hline Centralized & Typically a central entity   & Both global and local broadcast  \\
Communication & modifies the environment. & communications are commonly\\
Mode & It facilitates passive forms  &  used. P2P  communication can \\
(CCM)  &  of communications, i.e.     & also occur. Here, exchange\\
&  communication without   &    of messages  occurs through a\\
& specific target recipient. &  central entity.\\
\hline Decentralized & All individuals are free to & P2P and local-broadcast \\
or Local & modify the environment &  are  most commonly used forms. \\
Communication & and convey information &  Global broadcast typically  occurs \\
Mode & to others. & to handle emergency situations. \\
(LCM) & & All communications are local and\\
& & no central entity is required.\\
\hline
\end{tabular}
\end{center}
\end{table}
%%
%% [COMM classification]
\begin{figure}
\centering
\includegraphics[width=10cm, angle=0]
{./dia-files/bio-comm-strategies.eps}
\caption{\small Common communication strategies observed in social systems}
\label{fig:comm-strategies} 
\end{figure}
%%
\begin{figure}
\centering
\includegraphics[width=10cm, angle=0]
{./dia-files/bio-comm-strategies-peers.eps}
%figure caption is below the figure
\caption{\small Number of recipients involved in various communication strategies}
\label{fig:comm-strategies-peers}  % Give a unique label
\end{figure}
%%
The communication  structure of a multi-agent system can broadly be classified into two major categories: centralized communication mode (CCM) and decentralized or local communication mode (LCM). A centralized communication system generally has a central entity, e.g. gateway,  that routes all incoming and outgoing communications of the system. Individual nodes of this system often do not communicate each other directly. But they can send and receive messages through this central gateway.  Central gateway can play many roles such as, access control, resource allocation and so on. On the other hand, in LCM there is no central entity and each node can independently route message to each other. \\
Under both CCM and LCM, nodes can select a certain number of target recipients of their messages. This process specifies {\em to whom} a node intends to communicate. In this thesis, we have denoted this mechanism of target recipient selection as {\em communication strategy}.  Fig. \ref{fig:comm-strategies} shows the most common communication strategies found in a  social system. In biological and MRS literature two basic communication strategies are often are reported: 1) direct or explicit communication and 2) indirect or implicit communication. As defined in \cite{Mataric1998}, {\em direct communication} is an intentional communicative act of message passing that aims at one or more particular receiver(s). It typically exchanges information through physical signals. In contrast, {\em indirect communication}, sometimes termed as {\em stigmergic} in biological literature, happens as a form of modifying the environment (e.g., pheromone dropping by ants) \cite{Bonabeau+1999}. In our ordinary sense, this is an observed behaviour and many robotic researchers call it as {\em no communication} \cite{Labella2007}. In order to avoid ambiguity, in this dissertation, by the term {\em communication}, we always refer to direct communication. Sec. \ref{bg:bio-comm} and \ref{bg:mrs-comm} reviews communication in biological social system and MRS respectively.\\
Direction or explicit communication can be limited by a communication range and thus by a number of target recipients. In the simplest case, when only two nodes can communicate we call this peer-to-peer (P2P) communication. When nodes can spread information to a limited number of peers of their locality the communication takes the form of local broadcast, i.e. one sender and a few receivers within a certain locality. For example, when honey-bee gives the information of flower sources through various dances it conveys this information to a few peers through a local broadcast. However, giving the sample of nectar through tactile or taste to its peers can be considered as a P2P communication. The global broadcast strategy can be found in almost all social species to handle emergency situations, e.g. alarm signal in danger. Table \ref{table:comm-classify}  shows the relationship between various communication modes and their adoption of different strategies. Certainly, we can see that the presence and absence of a central entity becomes the key characteristics of both modes. Fig. \ref {fig:comm-strategies-peers} shows a typical count of average peers in various communication strategies. The actual number of peers under local broadcast strategy is dependent on a particular social system and it changes over time in different level of interactions of individuals. 
%%
\subsection{Division of labour or task-allocation}
Encyclopaedia Britannica serves the definition of division of labour as the separation of a work process into a number of tasks, with each task performed by a separate person or group of persons. Originated from economics and sociology the term division of labour is widely used in many branches of knowledge. As mentioned by Scottish philosopher Adam Smith, the founder of modern economics :
\begin{quotation} 
The great increase of the quantity of work which, in consequence of the division of labour, the same number of people are capable of performing, is owing to three different circumstances; first, to increase the dexterity in every particular workman; secondly, to the saving of the time which is commonly lost in passing from one species of work to another; and lastly, to the invention of a great number of machines which facilitate and abridge labour, and enable one man to do the work of many.\\
(Adam Smith (1776) in \cite{Sendova-Franks+1999})
\end{quotation} 
In sociology, division of labour usually denotes the work specialization \cite{Sayer+1992}. Basically it answers three questions:
\begin{enumerate}
\item {\em What task?} i.e., the description of the tasks to be done, service to be rendered or products to be manufactured.
\item  {\em Why dividing it to individuals?} i.e., the underlying social standards for this division, such as task appropriateness based on class, gender, age, skill etc.
\item {\em How to divide it?} i.e.,the method or process of separating the whole task into small pieces of tasks that can be performed easily. 
\end{enumerate}
%% DoL: termites, skyscrapper
\begin{figure}
\begin{minipage}[t]{0.48\linewidth}
\centering
\includegraphics[width=6cm, height=4cm, angle=0]
{./photos/termites_nest.eps}
\caption{ A termite nest}
% Ref: Merlin D. Tuttle, "Saving North America’s Beleaguered Bats", National Geographic, August 1995, p. 40. 
\label{fig:ant} % Give a unique label
\end{minipage}
\hspace{0.5cm}
\begin{minipage}[t]{0.48\linewidth}
\centering
\includegraphics[width=6cm,height=4cm, angle=0]{./photos/skyscraper.eps}
\caption{ A Skyscraper }
\label{fig:self-org-agent} % Give a unique label
\end{minipage}
\end{figure}
From the study of biological social insects, two major metrics of division of labour have been established in literature: 1) task-specialization and 2) plasticity. {\em Task-specialization} is an integral part of division of labour where a worker usually does not perform all tasks, but rather specializes in a set of tasks, according to its morphology, age, or chance \cite{Bonabeau+1999}. This division of labour among nest-mates, whereby different activities are performed simultaneously by groups of specialized individuals, is believed to be more efficient than if tasks were performed sequentially by unspecialised individuals. Division of labour has a great {\em plasticity} where the removal of one class of workers is quickly compensated for by other workers. Thus distribution of workers among different concurrent tasks keep changing according to the external (environmental) and internal conditions of a colony \cite{Garnier+2007}.\\
In artificial social systems, like multi-agent or MRS, the term ``division of labour'' is often found  synonymous to ``task-allocation'' . However,  some researchers (e.g. \cite{Labella2007}) argued to distinguish these terms due to the origin and particular contextual use of these terms. Particularly,  division of labour adopts the biological notion of collective task  performance with little or no communication. On the other hand, task allocation follows  the meaning of assigning task(s) to particular robot(s) based on individual robot capabilities, typically through explicit communication, such as {\em intentional cooperation} \cite{Parker1998}.  The former is considered under {\em swarm robotics (SR)} paradigm and latter is done under {\em traditional MRS}. Sec. \ref{bg:mrs} covers both of these approaches in more detail. 

In this dissertation, I closely follow the SR approach for the defining division of labour, but I do not put any restriction on the use of communication. In fact, I view division of labour as a group-level phenomenon which occur due to the individual agent's  self-regulatory task selection behaviour. But, unlike  SR approach that view communication as expensive and hence try to find solutions avoiding it, I do not advocate for restricting the  use of communication. Rather, for the following reasons, along with our generic mechanism of division of labour, i.e. AFM (Chapter \ref{afm}), I propose some self-regulatory communication strategies to vary  communication load dynamically (Chapter  \ref{local-comm}). 

Firstly, from our understaning of different kinds of communication strategies of biological social systems (Sec. \ref{bg:bio-comm}), this is obvious that the role of communication can not be ignored for achieving division of labour in MRS. Instead of being too much addicted to communication-less algorithms, perhaps due to the limitation of current communication  technology (such as mimicking biological stigmergy), we need to exploit the existing state of the art  in communication technology for developing functional and robust division of labour mechanisms for future MRS.  By selecting a suitable communication strategy and enabling robots to self-regulate their  certain behaviours,  we can significantly reduce the communication load of a MRS.

Secondly, Whatever be the objectives of a target MRS under any of the above approaches, e.g., maximizing robustness, scalability and/or task performance, the issue of task-allocation of an individual robot remains same, i.e., what task should it select at a particular time point considering dynamically changing task requirements, choices of peer robots and environmental conditions. In this issue, unlike traditional MRS approach, I emphasize on maintaining the overall group level performance and robustness, not just focusing on the instantaneous maximum benefit (or minimum cost) of a robot by performing a particular task.

Finally, by combining the above two points, I define division of labour in MRS as a self-regulated task allocation process of a group of robots, where  robots can dynamically select suitable tasks, or can switch from one task to another based on  continuously sensed  and communicated information of tasks, states etc. through their respective sensory and communication  channels. Thus, by adopting this self-regulated task selection and communication strategies, some of the robots can have chances to specialize on some particular tasks, and as a whole, the system can maintain a level of plasticity without producing unnecessary communication burden on the system.
%------------------------------------------------------------------
%%%%%%%%%%%%%%%%%%%%%
\section{Communication in biological social systems}
\label{bg:bio-comm}
Communication plays a central role in self-regulated division of labour of biological societies.In this section communication among biological social insects are briefly reviewed within the context of self-regulated  division of labour.

\subsection{Purposes, modalities and ranges}
Communication in biological societies serves many closely related social purposes. Most peer-to-peer (P2P) communication include: recruitment to a new food source or nest site, exchange of food particles, recognition of individuals, simple attraction, grooming, sexual communication etc. In addition to that colony-level broadcast communication include: alarm signal, territorial and home range signals and nest markers, communication for achieving certain group effect such as, facilitating or inhibiting  a group activity \cite{Holldobler1990}.\\
\begin{table}
\caption{Common communication modalities in biological social systems}
\label{table:bio-comm-modalities}
\begin{center}
\begin{threeparttable}
\begin{tabular}{|l|l|l|}
\hline \textbf{Modality} & \textbf{Range} & \textbf{Information type}\\
\hline Sound & Long\tnote{a} & Advertising about food  source,  danger etc. \\                                                                                                                                               
\hline Vision & Short\tnote{b}  & Private, e.g. courtship display \\
\hline Chemical  & Short/long & Various messages, e.g. food location, alarm etc.\\
\hline Tactile & Short & Qualitative info, e.g. quality of flower,\\ & & peer identification etc.\\
\hline Electric & Short/long & Mostly advertising types, e.g. aggression messages\\
\hline
\end{tabular}
\begin{tablenotes}
\item [a]Depending on the type of species, long range signals can reach from a few metres to several kilometres.
\item [b]Short range typically covers from few mm to about a metre or so.
\end{tablenotes}
\end{threeparttable}
\end{center}
\end{table}
%[Modalities and Ranges]
Biological social insects use different modalities to establish social communication, such as, sound, vision, chemical, tactile,  electric and so forth.  Sound waves can travel a long distance and thus they are suitable for advertising signals. They are also best for transmitting complicated information quickly \cite{Slater1986}. Visual signals can travel more rapidly than sound but they are limited by the physical size or line of sight of an animal. They also do not travel around obstacles. Thus they are suitable for short-distance private signals such as in courtship display.\\
% 
In ants and some other social insects chemical communication is dominant. Any kind of chemical substance that is used for communication between intra-species or inter-species is termed as semiochemical \cite{Holldobler1990}. A pheromone is a semiochemical, usually a glandular secretion, used for communication within species. One individual releases it as a signal and others responds it after tasting or smelling it. Using pheromones individuals can code quite complicated messages in smells. For example a typical an ant colony operates with somewhere between 10 and 20 kinds of signals \cite{Holldobler1990}. Most of these are chemical in nature. If wind and other conditions are favourable,  this type of signals emitted by such a tiny species can be detected from several kilometres away. Thus chemical signals are extremely economical of their production and transmission. But they are quite slow to diffuse away. But ants and other social insects manage to create sequential and compound messages either by a graded reaction of different concentrations of same substance or by blends of signals.
Tactile communication is also widely observed in ants and other species typically by using their body antennae and forelegs. It is observed that in ants touch is primarily used  for receiving information rather than informing something. It is usually found as an invitation behaviour in worker recruitment process. When an ant intends to recruit a nest-mate for foraging or other tasks it runs upto a nest-mate and beats her body very lightly with  antennae and forelegs. The recruiter then runs to a recently laid pheromone trail or lays a new one. In this form of communication limited amount of information is exchanged. In underwater environment some fishes and other species also communicate through electric signals where there nerves and muscles work as batteries. They use continuous or intermittent pulses with  different frequencies learn about environment and to convey their identity and aggression messages.
\begin{figure}
\begin{minipage}[t]{0.48\linewidth}
\centering
\includegraphics[width=6cm, height=4cm, angle=0]
{./photos/fire-flies.eps}
\caption{ Fire-flies}
\end{minipage}
\hspace{0.5cm}
\begin{minipage}[t]{0.48\linewidth}
\centering
\includegraphics[width=4cm,height=4cm, angle=0]{./photos/moth_evil_eye.eps}
\caption{ Moth evil eye }
\label{fig:self-org-agent} % Give a unique label
\end{minipage}
\end{figure}
%%%%%%%%%%%%%%%%%%%%%%%%%%%%%%%%%%%%%%%%%%
\subsection{Signal active space and locality}
%bio-comm-ants-active-space
The concept of active space is widely used to describe the propagation of signals by species. In a network environment of signal emitters and receivers, active space is defined as the area encompassed by the signal during the course of transmission \cite{Mcgregor2000}. In case of long-range signals, or even in case of short-range signals, this area include several individuals where their social grouping allows them to stay in cohesion. The concept of active space is described somewhat differently in case some social insects. In case of ants, this active space is defined as a zone within which the concentration of pheromone (or any other behaviourally active chemical substances) is at or above threshold concentration \cite{Holldobler1990}. Mathematically this is denoted by a ratio:
\begin{equation}
\frac{\textit{The amount of pheromone emitted (Q)}}{\textit{The threshold concentration at which the receiving ant responds (K)}}
\end{equation}
Q is measured in number of molecules released in a burst or in per unit of time whereas K is measured in molecules per unit of volume. 
Fig. \ref{fig:ants-active-space} shows the use of active spaces of two species of ants: (a) {\em Atta texana} and (b) {\em Myrmicaria eumenoides}.  The former one uses two different concentrations of {\em 4-methyl-3-heptanone} to create attraction and alarm signals while the latter one uses two different chemicals: {\em Beta-pinene} and {\em Limonene} two create similar kinds signals, i.e. alerting and circling.\\ 
The adjustment of this ratio enables individuals to gain a shorter fade-out time and permits signals to be more sharply pinpointed in time and space by the receivers. In order to transmit the location of the animal in the signal, the rate of information transfer can be increased by either by lowering the rate of emission of Q or by increasing K, or both. For alarm and trail systems a lower value of this ratio is used. Thus, according to need, individuals regulate their active space by making it large or small, or by reaching their maximum radius quickly or slowly, or by enduring briefly or for a long period of time. For example, in case of alarm, recruitment or sexual communication signals where encoding the location of an individual is needed, the information in each signal increases as the logarithm of the square of distance over which the signal travels. From the precise study of pheromones it has been found that active space of alarm signal is consists of a concentric pair of hemispheres (see Fig. \ref{fig:ants-active-space}). As the ant enters the outer zone she is attracted inward toward the point source; when she next crosses into the central hemisphere she become alarmed. It is also observed that ants can release pheromones with different active spaces.\\
%%
\begin{figure}
\centering
\includegraphics[width=12cm, angle=0]
{./dia-files/bio-comm-ants-active-space.eps}
%figure caption is below the figure
\caption{\small Pheromone active space observed in ants}
\label{fig:ants-active-space} % Give a unique label
\end{figure}
Active space has strong role in modulating the behaviours of ants. For example, when workers of {\em Acanthomyops claviger} ants produce alarm signal due to an attack by a rival or insect predator, workers sitting a few millimetres away begin to react within seconds. However those ants sitting a few centimetres away take a minute or longer to react. In many cases ants and other social insects exhibit modulatory communication within their active space where many individuals involve in many different tasks. For example, while retrieving the large prey, workers of {\em Aphaeonogerter} ants produce chirping sounds (known as stridulate) along with releasing poison gland pheromones. These sounds attract more workers and keep them within the vicinity of the dead prey to protect it from their competitors. This communication amplification behaviour can increase the active space to a maximum distance of 2 meters.
%%%%%%%%%%%%%%%%%%%%%
\subsection{Common communication strategies}
\label{bg:bio-comm:strategies}
% indirect & b/c comm
In biological social systems, we can find all different sorts of communication strategies ranging from indirect pheromone trail laying to local and global broadcast of various signals. Sec. \ref{bg:def:comm} discusses the most common four communication strategies in natural and artificial world, i.e. indirect, P2P, local and global broadcast communication strategies. Table \ref{table:bio-comm-strategy} lists the use of various communication modalities under different communication strategies. Here we give a few real examples of those strategies from biological social systems. In biological literature, the pheromone trail laying is one of the most discussed indirect communication strategy among various species of ants. Fig. \ref{fig:ant-indirect} shows a pheromone trail following of a group of foraging ants. This indirect communication strategy effectively helps ants to find a better food source among multiple sources, find shorter distance to a food source, marking nest site and move there etc. \cite{Hughes2008 }.\\
Direct P2P communication strategy is also very common among most of the biological species. Fig. \ref{fig:ant-p2p} and Fig. \ref{fig:honey-bee-p2p} shows P2P communication of ants and honey-bees respectively. This tactile form of communication is very effective to exchange food item, flower nectar with each-other or this can be useful even in recruiting nest-mates to a new food source or nest-site.\\
\begin{figure}
\begin{minipage}[t]{0.48\linewidth}
\centering
\includegraphics[width=6cm, height=4cm, angle=0]
{./photos/ants_group_comm_bioteams_com.eps}
\caption{ Ant pheromone trail}
\label{fig:ant-indirect} % Give a unique label
\end{minipage}
\hspace{0.5cm}
\begin{minipage}[t]{0.48\linewidth}
\centering
\includegraphics[width=6cm,height=4cm, angle=0]{./photos/honey-bee-waggle-dance-knol-google.eps}
\caption{ Honey bee's local communication }
\label{fig:honey-bee-local-bc} % Give a unique label
\end{minipage}
\end{figure}

% Honey bee dance language
\begin{figure}
\begin{minipage}[t]{0.48\linewidth}
\centering
\includegraphics[width=6cm, height=4cm, angle=0]
{./photos/honey-bee-round-dance.eps}
\caption{ Honey bee round dance}
\label{fig:round-dance} % Give a unique label
\end{minipage}
\hspace{0.5cm}
\begin{minipage}[t]{0.48\linewidth}
\centering
\includegraphics[width=6cm,height=4cm, angle=0]{./photos/honey-bee-waggle-dance.eps}
\caption{ Honey bee waggle dance }
\label{fig:waggle-dance} % Give a unique label
\end{minipage}
\end{figure}
%%%%%%%%%%%%%%%%%%%%%%
\begin{table}
\caption{Common communication strategies in biological social systems}
\label{table:bio-comm-strategy}
\begin{center}
\begin{tabular}{|l||l|}
\hline \textbf{Communication strategy} & \textbf{Common modalities used}\\
\hline Indirect & Chemical and electric \\
\hline Peer-to-peer (P2P) &  Vision and tactile\\
\hline Local broadcast &  Sound, chemical and vision\\
\hline Global broadcast & Sound, chemical and electric\\
\hline
\end{tabular}
\end{center}
\end{table}
%%
% p2P comm
\begin{figure}
\begin{minipage}[t]{0.48\linewidth}
\centering
\includegraphics[width=6cm, height=4cm, angle=0]
{./photos/honey-bee-p2p-hy23.eps}
\caption{ Honey bee P2P comm}
\label{fig:honey-bee-p2p} % Give a unique label
\end{minipage}
\hspace{0.5cm}
\begin{minipage}[t]{0.48\linewidth}
\centering
\includegraphics[width=6cm,height=4cm, angle=0]{./photos/ants-p2p-hy14.eps}
\caption{ Ants p2p comm }
\label{fig:ant-p2p} % Give a unique label
\end{minipage}
\end{figure}
%%
\subsection{Roles of communication in task-allocation}
%%
\begin{table}
\caption{Modulation of communication behaviours based on task urgency perception}
\label{table:bio-comm-task-urgency}
\begin{center}
\begin{tabular}{|l||l|l|}
\hline \textbf{Example event} & \textbf{Strategy} & \textbf{Modulation of communication}\\
&  &  \textbf{based-on task-urgency}\\
\hline Ant's alarm signalling &  Global  & High concentration of pheromones\\
by pheromones & broadcast &  leads to increased aggressive alarm-behaviours \\                                                                                                                                               
\hline Honey-bee's  & Local  &  High quality of nectar source increases the duration\\
round dance & broadcast & of dances and increases the number of foraging bees\\
\hline Ant's tandem run for    & Peer-to-Peer & High quality of nest decreases the assessment time\\
new nest selection & &  and increases traffic flow\\
\hline Ant's pheromone trail-  & Indirect & Food source located at shorter distance\\
laying to multiple  & &  gets higher priority as less pheromone evaporates\\
food sources & & and the more ants joins the trail\\
\hline
\end{tabular}
\end{center}
\end{table}
%%
\begin{figure}
\centering
\includegraphics[width=6.5cm, angle=-90]
{./images/ch2/honey-bee-dance-stat.eps}
%figure caption is below the figure
\caption{\small Self-regulation of honey-bee's communication behaviours}
\label{fig:honey-bee-dance-stat}  % Give a unique label
\end{figure}

\subsection{Information flow in communication}
%% social wasps
\begin{figure}
\begin{minipage}[t]{0.48\linewidth}
\centering
\includegraphics[width=6cm, height=4cm, angle=0]
{./photos/Polybia_occidentalis_I_JP6646_discoverlife.org.eps}
\caption{ Polybia}
\end{minipage}
\hspace{0.5cm}
\begin{minipage}[t]{0.48\linewidth}
\centering
\includegraphics[width=6cm,height=4cm, angle=0]{./photos/Wasps_wikimedia.org_Polistes_nest_3_sjh.eps}
\caption{ Polistes }
\label{fig:self-org-agent} % Give a unique label
\end{minipage}
\end{figure}
%%
\begin{figure}
\centering
\includegraphics[width=6cm, angle=0]
{./images/ch2/jeanne-fig9-info-flow.eps}
%figure caption is below the figure
\caption{\small Information flow in social wasps}
\label{fig:wasps-info-flow}  % Give a unique label
\end{figure}
\subsection{Group size and communication strategy}
%%
\begin{figure}
\centering
\includegraphics[width=9cm, angle=0]
{./images/ch2/jeanne-fig6-group-size.eps}
%figure caption is below the figure
\caption{\small Productivity of social wasps as a function of group size}
\label{fig:wasps-info-flow}  % Give a unique label
\end{figure}
%%
\begin{figure}
\centering
\includegraphics[width=9cm, angle=0]
{./dia-files/jannae-fig10-info-flow-cmp.eps}
%figure caption is below the figure
\caption{\small Different Information flow in different kinds of social wasps}
\label{fig:wasps-info-flow}  % Give a unique label
\end{figure}
%%%%%%%%%%%%%%%%%%%%%%%%%%%%%%%%%%%%%%%%%%%%%%%%%%%%%%%%%%%%%%%%%%%
\section{Overview of multi-robot systems (MRS)}
\label{bg:mrs}
Historically the concept of multi-robot system comes almost after the introduction of behaviour-based robotics paradigm \cite{Brooks1986,Arkin1990}. In 1967, using the traditional sense-plan-act or hierarchical approach \cite{Murphy2000}, the first Artificially Intelligent (AI) robot, Shakey, was created at the Stanford Research Institute. In late 80s, Rodney A. Brooks revolutionized this entire field of mobile robotics who outlined a layered, behaviour based approach that acted significantly differently than the hierarchical approach \cite{Brooks1986}. At the same time, Valentino Braitenberg described a set of experiments where increasingly complex vehicles are built from simple mechanical and electrical components \cite{Braitenberg1984}. Around the same time and with similar principle, Reynolds developed a distributed behavioural model for a bird in a flock that assumed that a flock is simply the result of the interactions among the individual birds \cite{Reynolds1987}. Early research on multi-robot systems also include the concept of cellular robotic system \cite{Fukuda+1987}, \cite{Beni1988} multi-robot motion planning \cite{Arai+1989,Premvuti+1990,Wang1989} and architectures for multi-robot cooperation \cite{Asama+1989}.\\
%
From the beginning of the behaviour based paradigm, the biological inspirations influenced many cooperative robotics researchers to examine the social characteristics of insects and animals and to apply them to the design multi-robot systems \cite{Arkin1998}. The underlying basic idea is to use the simple local control rules of various social species, such as ants, bees, birds etc., to the development of similar behaviours in multi-robot systems. In multi-robot literature, there are many examples that demonstrate the ability of multi-robot teams to aggregate, flock, forage, follow trails etc. \cite{Bonabeau+1999,Mataric1994}. The dynamics of ecosystem, such as cooperation, has also been applied in multi-robot systems that has presented the emergent cooperation among team members \cite{Mcfarland1994}, \cite{Martinoli+1996}. On the other hand, the study of competitive behaviours among animal and human societies has also been applied in multi-robot systems, such as that found in multi-robot soccer \cite{Asada+1999}.\\
%%% Earliest MRS: nerd herd
\begin{figure}
\begin{minipage}[t]{0.48\linewidth}
\centering
\includegraphics[width=6cm, height=4cm, angle=0]
{./photos/Nerd_Herd.eps}
\caption{ Mataric's Nerd Herd}
\end{minipage}
\hspace{0.5cm}
\begin{minipage}[t]{0.48\linewidth}
\centering
\includegraphics[width=6cm,height=4cm, angle=0]{./photos/kube_box_pushing.eps}
\caption{ Kube's box pushing experiments }
\label{fig:self-org-agent} % Give a unique label
\end{minipage}
\end{figure}
 \subsection{MRS research paradigms}
\label{bg:mrs:paradigms}
As discussed above, there are several research groups who follow different approaches to handle multi-robot research problems. Parker \cite{Parker2008} has summarized most of the recent research approaches into three paradigms:
\begin{enumerate}
\item Bioinspired, emergent swarms paradigm,
\item Organizational and social paradigm and
\item Knowledge-based, ontological and semantic paradigm
\end{enumerate}
\subsubsection*{Bioinspired, emergent swarms paradigm}
In bio-inspired, emergent swarms paradigm local sensing and local interaction forms the basis of collective behaviors of swarms of robots. Many researchers addressed the issues of local interaction, local communication (i.e., stigmergy) and other issues of this paradigm \cite{Mataric1995}, \cite{Kube+1993}. Today, this paradigm has been emerged as a sub-field of robotics called swarm robotics \cite{Sahin+2005}. This is a powerful paradigm for those applications that require performing shared common tasks over distributed workspace, redundancy or fault-tolerance without any complex interaction of entities. Some examples include flocking, herding, searching, chaining, formations, harvesting, deployment, coverage etc. \\
% FIG: Multi-robot in emergency entertainment
\begin{figure}
\begin{minipage}[t]{0.48\linewidth}
\centering
\includegraphics[width=6cm, height=4cm, angle=0]
{./photos/burning-oil-rig-explosion-fire-photo11.eps}
\caption{ Multi-robot emergency disaster recovery}
\end{minipage}
\hspace{0.5cm}
\begin{minipage}[t]{0.48\linewidth}
\centering
\includegraphics[width=6cm,height=4cm, angle=0]{./photos/robocup_bbc_41132545_robotbody2.eps}
\caption{ Multi-robot soccer: challenge and fun }
\label{fig:self-org-agent} % Give a unique label
\end{minipage}
\end{figure}
%%
% FIG: Multi-robot indoor and outdoor operations
\begin{figure}
\begin{minipage}[t]{0.48\linewidth}
\centering
\includegraphics[width=6cm, height=4cm, angle=0]
{./photos/centibot_demo3-11.eps}
\caption{ Centibots indoor}
\end{minipage}
\hspace{0.5cm}
\begin{minipage}[t]{0.48\linewidth}
\centering
\includegraphics[width=6cm,height=4cm, angle=0]{./photos/pioneer_robots_610x455.eps}
\caption{ Pioneer in outdoor }
\label{fig:self-org-agent} % Give a unique label
\end{minipage}
\end{figure}
%% Re arrange
Swarm robotics is a relatively new branch of robotics where a large number of collective robots are studied from the inspiration of the observation of social insects ants, termites wasps and bees \cite{Sahin+2005}. The term swarm intelligence was first coined by Gerado Beni \cite{Beni2005} in late 1980s and during recent years the term swarm robotics emerged as an application of swarm intelligence to multi-robot systems with emphasis on physical embodiment of entities and realistic interactions among the entities and between the entities and their environment. In order to distinguish swarm robotics from other branches of robotics such as collective robotics, distributed robotics, robot colonies and so forth, Sahin proposed a formal definition and a set of criteria for swarm robotics research \cite{Sahin+2005}. According to him, swarm robotics is the study of how large number of relatively simple physically embodied agents can be designed such that a desired collective behaviour emerges from the local interactions among agents and between the agents and the environment. And the notable criteria of swarm robotics research are listed as follows.
\begin{description}
\item[Autonomous robots]
that exclude the sensor networks and may include metamorphic robotic system without having no centralized planning and control element.
\item[Large number of robots,]
usually $\geq$ 10 robots, or at least having provision for scalability if the group size is below this number.
\item[Mostly homogeneous groups of robots]
that typically exclude the multi-robot soccer teams having heterogeneous robots.
\item[Relatively incapable of inefficient robots]
that is the task complexity enforces either cooperation among robots or increased performance or robustness without putting no restriction on individual robot's hardware/software complexity.
\item[Robots with local sensing and communication capabilities]
that does not use global coordination channel to coordinate among themselves, rather enforces distributed coordination.
\end{description}
%%%
%% Swarm algorithm testing
% FIG: Multi-robot indoor and outdoor operations
\begin{figure}
\begin{minipage}[t]{0.48\linewidth}
\centering
\includegraphics[width=6cm, height=4cm, angle=0]
{./photos/jamesMcLurkin_SwarmbotAndSwarm.eps}
\caption{ Swarmbot}
\end{minipage}
\hspace{0.5cm}
\begin{minipage}[t]{0.48\linewidth}
\centering
\includegraphics[width=6cm,height=4cm, angle=0]{./photos/james_fig-BoundaryDetection-Robots.eps}
\caption{ Swarmbot boundary detection }
\label{fig:self-org-agent} % Give a unique label
\end{minipage}
\end{figure}
%% modelling swarm: TODO: to be summarized
Modelling the swarms is a key issue in swarm robotics. This is aimed for investigating suitable models and algorithms for control and task-allocation of swarms of robots. A review of models and approaches for coordination and control of dynamic multi-agent systems by \cite{Gazi+2006} presented that a number of approaches can be used to model swarm robotic systems with the specific focus on major issues like stability, performance, robustness and scalability. Based on the relevance to our study, we have discussed them as follows.
\paragraph*{Behaviour-based approaches:}
The ease of implementation of a behaviour-based robotic system has inspired researchers to follow behaviour-based approaches for modelling swarm robotic systems using variety of specific swarm behaviours. Early research of Reynold provided the example of a behaviour-based approach for swarm coordination such as, flocking of birds \cite{Reynolds1987}. Recent studies on behaviour-based approaches include the work of \cite{Balch+1998} where they have evaluated the formation acquisition and stabilization of multi-robot systems. Several other researchers used other techniques, such as use of adaptation rules \cite{Liu+2007}, collective behaviours \cite{Cianci2007} etc., for implementing a behaviour-based swarm robotic system.
\paragraph*{Probabilistic approaches:}
Probabilistic approaches and Markov models also present attractive alternatives for modelling of swarm behaviour. They typically use the population level swarming dynamics in a non spatial way in terms of frequency distributions of groups of various size \cite{Gazi+2006}. A recent review of probabilistic approaches for swarm modelling is presented in \cite{Lerman+2005}.
\paragraph*{Asynchronous swarm model based approaches:}
Asynchronous multi-agent dynamic systems are difficult to tract for analysis and are not widely found in literature. One of the pioneer study by \cite{Beni+1996} provided sufficient conditions for the asynchronous convergence of linear swarm to a synchrously achievable configuration. Some other recent studies can be found in \cite{Gazi+2006}.
\paragraph*{Control theoretic approaches:}
Control theoretic approaches include potential field, feedback linearisation, sliding mode, and various non-linear control approaches, e.g., Fuzzy, Neural nets, Knowledge-based/Rule-based, Lyapunov analysis etc. In recent years, combined or hybrid approaches, e.g., neuro-fuzzy, are also being adopted for modelling swarm behaviours and learning parameter settings of a system \cite{Sahin+2007}. 
\paragraph*{Artificial Physics based approaches:}
Artificial physics based approaches use the fundamental laws of physics such as the Newton's laws of motion to model swarm robotic systems. In a pioneering work by \cite{Spears+1999}, this approach has been illustrated and since then many development has been taken place under this framework to address the issue of formation stabilization, surveillance, coverage of a region etc.
\paragraph*{Multi-agent based and other approaches:}
Since the field of swarm intelligence and swarm robotics is expanding contentiously many researchers are putting efforts to bring newer swarm models based on multi-agent based other techniques which are not reviewed here explicitly.
%%%
%%% FIG: Swarmbot indoor, outdoor
% FIG: Multi-robot indoor and outdoor operations
\begin{figure}
\begin{minipage}[t]{0.48\linewidth}
\centering
\includegraphics[width=6cm, height=4cm, angle=0]
{./photos/swarm-bots-crossing-canal.eps}
\caption{ Swarmbot crossing rough terrain}
\end{minipage}
\hspace{0.5cm}
\begin{minipage}[t]{0.48\linewidth}
\centering
\includegraphics[width=6cm,height=4cm, angle=0]{./photos/swarmbots-pulling-child-725972.eps}
\caption{ Swarmbot pulling a child }
\label{fig:self-org-agent} % Give a unique label
\end{minipage}
\end{figure} 
\subsubsection*{Organizational and social paradigm}
Organizational and social paradigms are typically based on organization theory derived from human systems that reflects the knowledge from sociology, economics, psychology and other related fields. To solve complex problems this paradigm usually follows the cooperative and collaborative forms of distributed intelligence. In multi-robot systems the example of this paradigm is found in two major formats: 
the use of roles and value system and
market economics.
In multi-robot applications under this paradigm, an easy division of labor is achieved by assigning roles depending on the skills and capabilities in individual team member. For example, in multi-robot soccer \cite{Stone+1999,Asada+1999} positions played by different robots are usually considered as defined roles. On the contrary, in market economics approach \cite{Gerkey+2002,Dias+2006} task allocation among multiple robots are done via market economics theory that enables the selection of robots for specific tasks according to their individual capabilities determined by a bidding process.
\subsubsection*{Knowledge-based, ontological and semantic paradigm}
The third paradigm, commonly used for developing multi-agent systems, is knowledge-based, ontological and semantic paradigm. Here knowledge is defined as ontology and shared among robots/agents from disparate sources. It reduces the communication overhead by utilizing the shared vocabulary and semantics. Due to low bandwidth, limited power, limited computation and noise and uncertainty in sensing/actuation, the use of this approach is usually restricted in multi-robot systems. \\
Although this approximate classification includes most of the research directions it is very hard to specifically categorize all diverse researches on multi-robot systems. However, most of the researchers select a suitable paradigm to abstract the problem from an specific perspective with a fundamental challenge to determine how best to achieve global coherence from the interactions of entities at the local level. \\
%
Whatever the principle characteristics of a MRS, e.g., homogeneity, coupling, communication methods etc., each MRS must address some basic problems to some degree. For example, usually every MRS adopts a control architecture under a specific paradigm. Similarly every MRS address the issues of communication, localization, interaction in a way specific to the application and underlying design principles (or philosophies). In the following subsections, we have attempted to summarize the key MRS research issues that would influence the selection and implementation our research. In this initiative we have deliberately omitted the non-central or very specific issues, such as collaborative transport or reconfigurable MRS, that does not directly relate to our research.
%%%
\subsection{MRS taxonomies}
\label{bg:mrs:taxonomies}
%%%%
\subsection{Traditional MRS}
\label{bg:mrs:tmrs}
%%%
\subsection{Swarm robotic systems}
\label{bg:mrs:sr}
\subsection*{Architecture and control}
\label{bg:mrs:arch}
In MRS, two high-level control strategies are very common: 1) centralized and 2) decentralized or distributed. Under a specific control strategy, traditionally three basic system architectures are widely adopted: deliberative, reactive and hybrid. Deliberative systems based on central planning are well suited for the centralized control approach. The single controller makes a plan from its Sense-Plan-Act (SPA) loop by gathering the sensory information and each robot performs its part. Reactive systems are widely used in distributed control where each robot executes its own controller maintaining a tight coupling between the system's sensors and actuators, usually through a set of well-designed behaviours. Here, various group behaviour emerges from the interactions of individuals that communicate and cooperate when needed. Hybrid systems are usually the mixture of the two above approaches; where each robot can run its own hybrid controller with the help of a plan with necessary information from all other robots. \cite{Mataric2007} described behaviour-based control architecture as a separate category of distributed control architecture where each robot behaves according to a behaviour-based controller and can learn, adapt and contribute to improve and optimize the group-level behaviour. 
Although most of the MRS control architectures share some common characteristics (such as distributed and behaviour-based control strategy) based on their difference of underlying design principles we have put them into three groups:
\begin{enumerate}
\item Behaviour-based classical architectures
\item Market-based architectures
\item Multi-agent based architectures
\end{enumerate}
Due to the overwhelming amount of literature on MRS architectures it is not possible to include most of them. However, below some representative key architectures strictly designed for MRS are described. 
\subsubsection*{Behaviour-based classical architectures}
The ALLIANCE architecture \cite{Parker1998} is one of the earliest behaviour-based fully distributed architectures. This architecture has used the mathematically modelled behaviour sets and motivational system. The primary mechanism for task selection of a robot is to activate the motivational behaviour partly based on the estimates of other robots behaviour. This architecture was designed for heterogeneous teams of robots performing loosely coupled tasks with fault-tolerance and co-operative control strategy. Broadcast of local eligibility (BLE) \cite{Werger2001} is another behaviour-based architecture that uses port-attributed behaviour technique through broadcast communication method. It was demonstrated to perform coordinated tasks, such as multi-target observation tasks. Major differences between this two behaviour-based systems include the need in ALLIANCE for motivational behaviours to store information about other individual robots, the lack of uniform inter-behaviour communication, and ALLIANCE's monitoring of time other robots have spent performing behaviours rather than BLE's local eligibility estimates. Similar to the above two architectures, many other researchers proposed and implemented many variants of behaviour-based architectures. Some of them used the classic three layer (plan-sequence-execute) approach, e.g., \cite{Simmons+2002} used a Layered Architecture where each layer interact directly to coordinate actions at multiple levels of abstraction. 
\subsubsection*{Market-based architectures}
Using the theory of marker economics and well-known Contract Net Protocol (CNP) \cite{Davis1988+}, these architectures solve the task-allocation problem by auction or bidding process. Major architectures following market-based approaches include MURDOCH \cite{Gerkey+2002}, M+ system \cite{Botelho+1999}, first-piece auction \cite{Zlot+2002}, dynamic role assignments \cite{Chaimowicz2002} among others. 
\subsubsection*{Multi-agent based architectures}
Some MRS architectures are influenced by multi-agent systems (MAS). For example, CHARON is a hierarchical behaviour-based architecture that rely on the notion of agents and modes. Similarly CAMPOUT is another distributed behaviour-based architecture that provide high-level functionality by making use of basic low-level behaviours in downward task decomposition of a multi-agent planner. It is comprised of five different architectural mechanisms including, behaviour representation, behaviour composition, behaviour coordination, group coordination and communication behaviours.
%%%%%
\subsection*{Interaction and learning}
\label{bg:mrs:learn}
\subsubsection*{Interaction}
According to the Oxford Dictionary of English the term interaction means reciprocal action or influence. In MRS research, such as in \cite{Mataric1994}, interaction is referred to as mutual influence on behaviour. Following this definition, it is obvious that objects in the world do not interact with agents, although they may affect on their behaviour. The presence of an object affects the agent, but the agent does not affects the object since objects, by definition, do not behave, only agents do. However many other researchers acknowledge that interactions of robots with their environment (as found in stigmergic communication) have a great impact on their behaviours. Therefore, we adopt the broad meaning of interaction that is reciprocal action or influence among robots and their environment.
From the above review of MRS system architecture, task allocation and communication, it is obvious that interaction among robots and their environment is the core of the dynamics of MRS.
Without this interaction, it can not be a functioning MRS. 
% Insert Fig
While analysing the role and application of distributed intelligence on MRS, Parker \cite{Parker2008} presented an excellent classification of interactions of entities of MRS. She viewed the interactions along three different axes:
\begin{enumerate}
\item the types of goals of entities (either shared goal such as, cleaning a floor, or, individual goal)
\item whether entities have awareness of others on the team (either aware such as, in cooperative transport, or, unaware such as, in a typical foraging)
\item whether the action of one entity advances the goal of others (e.g., one robot's floor cleaning helps other robots not to clean that part of the floor)
\end{enumerate}
Based on this approximate observation Parker classified interactions into four categories:
% see Fig
\paragraph{Collective interaction:}
Entities are not aware of others on the team, yet do share goals and their actions are beneficial to team-mates. Mostly, swarm-robotic work of many researchers follow this kind of interaction to perform biologically-relevant tasks, such as foraging, swarming, formation keeping and so forth.
\paragraph{Cooperative interaction:}
Entities are aware of others on the team, they share goals and their actions are beneficial to their team-mates. This type of interaction is used to reason about team-mates capabilities multiple robots works together, usually in shared workspace, such as cleaning a work-site, pushing a box, performing search and rescue, extra-planetary exploration and so forth. 
\paragraph{Collaborative interaction:}
Having individual goals (and even individual capabilities), entities aware of their team-mates and their actions are beneficial to their team-mates. One example of this kind of interaction is a team of collaborative robots where each must reach a unique goal position by sharing sensory capabilities to all members such as illustrated as coalition formation in \cite{Parker+2006}.
\paragraph{Coordinative interaction:}
Entities are aware of each other, but they do not share a common goal and their actions are not helpful to other team members. For example, in a common workspace robots try to minimize interference by coordinating their actions as found in multi-robot path planning techniques, traffic control techniques and so on.
Beyond this four most common types of interactions Parker also described another kind of interaction in adversarial domain where entities effectively work each other such as multi-robot soccer. Here entities have individual goals, they are aware of each other, but their actions have a negative affect on others goal.
%%
\subsubsection*{Learning}
A great deal of research on multi-robot learning has been carried out since the inception of MRS \cite{Mataric+2001,Yang+2004,Parker1995}. Learning, identified as the ability to acquire new knowledge or skills and improve one's performance, is useful in MRS due to the necessity of robots to know about itself, its environment and other team-members \cite{Mataric2007}. Learning can improve performance since robot controllers are not perfect by design and robots are required to work in an uncertain environment that all possible states or actions can not be predicted in advance. Besides learning a new skill or piece of knowledge it is also important to forget learned things that are no longer needed or correct as well as, to make room for new things to be learned and stored in a finite memory space of a robot. \\
Several learning techniques are available in robotics domain, such as reinforce or unsupervised learning, supervised learning and learning by imitation \cite{Mataric2007}. Although reinforce learning, or learning based on environmental or peer feedback, is a good option for MRS, it has been found that in large teams the ability to lean in this way is restricted due to large continuous state and action space \cite{Yang+2004}. Several other learning techniques are also available to explore in MRS domain including Markov models, Q-learning, fuzzy logic, neural nets, game theory, probabilistic or Bayesian theory among others. \\
%% SRS
Based on a specific model of swarm behaviours researchers generally adopt similar communication methods to enable interaction of swarms as discussed in Section \ref{subsec:mrs-comm}. In \cite{Balch2005} three kinds of communication including, indirect stigmergic communication, direct robot to robot state communication, and goal communication were performed and it was found that in some tasks communication provided performance improvements while others did not. Since then, researchers emphasize on both the necessity and cost of communication in a swarm robotic system. Indirect communication approaches, e.g. virtual pheromone \cite{Payton+2005,Hamann+2006} by which mobile robots communicated through directional infrared messaging or LEDs, are mainly tried for large teams with the spatially distributed applications such as search and rescue, de-mining etc. More recently, \cite{Cianci2007} reported a IEEE 802.15.4-compatible radio-communication module in e-puck robot for achieving multiple interactions simultaneously, as demonstrated in their collective decision making scenario. 
Although research on learning in swarm robotic teams was not explored widely, \cite{Balch2005} presented an example of reinforcement-based learning in multi-robot soccer and foraging tasks. He concluded that in a team of homogeneous robots with diverse behaviours, communication, interaction and learning are well interconnected and depending on the selection of global or local learning means, learning can be effectively employed in a swarm robotic system.
%
\subsubsection*{Conflict resolution}
In MRS, conflicts occur if a resource is required by or, a unique single task is distributed to, more than one robot at any given time. Several resources such as bandwidth, space etc. may be needed by more than one robot. The space sharing problem was treated as traffic control problem in urban areas, but the robots are never restricted in road networks in case of behaviour-based control application \cite{Cao+1997}. In explicit communication mode in MRS, the sharing of bandwidth among robots is a great problem in case of applications like multi-robot mapping \cite{Konolige+2003}. In large multi-robot team such as in Centibots system \cite{Ortiz+2005}, task interference and high bandwidth communication between 100s of robots appear as a significant research challenge.

\subsection*{Localization and exploration}
\label{bg:mrs:loc}
Mobile robot systems highly rely on precise localization for performing their autonomous activities in indoor or outdoor. Localization is the determination of exact pose (position and orientation) with respect to some relative or absolute coordinate system. This can be done by using proprioceptive sensors that monitor motion of a robot or exteroceptive sensors that provide information of world representation, such as global positioning system (GPS) or indoor navigation system (INS). Many other methods are also available, such as landmark recognition, cooperative positioning and other visual methods.\\
Localization issue of MRS also invites researchers to examine specific areas like exploration and map generation. In exploration problem, robots need to minimize the time needed to explore the given area. Many researchers uses various kinds of exploration algorithms for solving this NP-hard problem, such as line-of-sight constrained exploration algorithm \cite{Arkin+2002}, collaborative multi-robot exploration \cite{Burgard+2000} and so on. In mapping problem, mostly inaccurate localization information from teams of robots are accumulated and combined to generate a map by various techniques, such as probabilistic approaches \cite{Thurn+2000}.
%% SRS
Similar to in a MRS, localization is one of the hardest problem in swarm robotics. Without the presence of any centralized localization module, such as GPS or INS, it is not easy to localize precisely and locally the position of a robot with respect to other robots or environment. \cite{Spears+2006} presented a novel technique based on trilateration for localization of swarm robots using ultrasonic and RF transceivers without relying on global information from GPS, beacons, landmarks or maps. This system localize a robot with respect to other nearby robots and this is done using ultrasonic and RF signals. \cite{Schmickl+2006} reported hop-count and bio-inspired strategies for collective perception or how a swarm robot can join multiple instances of individual perception to get a global picture. Distributed mapping is another important application using swarms. \cite{Rothermich+05} presented a collaborative localization algorithm using landmark based localization technique.

\subsection*{Applications of MRS}
\label{bg:app}
MRS systems have been put to numerous application domains that all can not be listed together. Rather than listing all of areas explored by researchers, below we have included few major areas that have received highest attention in the MRS research community. Sahin also listed a set of promising applications for swarm robots including spatially distributed tasks (e.g., environment monitoring), dangerous tasks (e.g., robotic de-miner), tasks that scale-up or scale-down over time, and tasks that require redundancy \cite{Sahin+2005}.
\subsubsection*{Object Transport}
Cooperative transport of large objects (that one robot is unable to handle) by multi-robots was investigated by many researchers such as, following a formal model of cooperative transport in ants \cite{Kube+1993}, box-pushing by six-legged robots \cite{Mataric+1995}. Another kind of object transport problem include clustering objects into piles e.g., \cite{Beckers+1994}, collecting waste or trash e.g., \cite{Parker1994}, sorting coloured objects e.g., \cite{Melhuish+1998}, constructing a building site collectively \cite{Wawerla+2002} and so on. 
\subsubsection*{Mining} 
It has also been observed that multi-robot teams as micro or mini machines are helpful to improve the control and efficiency of mining and its processing operations \cite{Dunbar+2002}.
\subsubsection*{Military and Space Applications}
Many researchers address MRS research issues under the requirements of a military or space application. Behaviour-based formation control \cite{Balch+1998}, landmine detection \cite{Franklin+1995}, multiple planetary rovers for various missions \cite{Huntsberger2004} and so forth, all are the examples of this areas.
%%%%%%%%%%%%%%%%%%%%%%%%%%%%%%%%%%%%%%%%%%%%%%%%%%%%%%%%%%%%%%%%%%%
\section{Task-allocation in MRS}
\label{bg:mrta}
Since 90s multi-robot task allocation (MRTA) is a common research challenge that tries to define the preferred mapping of robots to tasks in order to optimize some objective functions  \cite{Gerkey+2004}. Many MRS control architectures have been solely designed to address this task-allocation issue. In 2003 Gerkey et al. formally analysed the complexity and optimality of key architectures (e.g., ALLIANCE, BLE, M+, MURDOCH, First piece auctions and Dynamic role assignment) for this MRTA issue and it has been found that MRTA is an instance of the so-called optimal assignment problem \cite{Gerkey+2003} and generally known as NP-hard where optimal solutions can not be found quickly for large problems \cite{Gerkey+2004}. 
If we look the MRTA problem from multi-agent system's perspective we can find it is broadly divided into two major categories \cite{Shen+2001}: 
\begin{enumerate}
\item Predefined (off-line) task-allocation and 
\item Emergent (real-time) task-allocation. 
\end{enumerate}
% (Insert a Fig)
\subsection{Predefined task-allocation}
Usually predefined task allocation method uses either centralized coordination or distributed task-allocation approach. Distributed predefined task-allocation approach is again subdivided into three subcategories: 
\begin{enumerate}
\item Direct allocation, 
\item Task allocation by delegation 
\item Task allocation through bidding
\end{enumerate}
In MRS domain, early research on predefined distributed task-allocation approach has been dominated mainly by intentional coordination \cite{Gerkey+2004,Parker1998}, the use of dynamic role assignment e.g., \cite{Chaimowicz2002}, and market-based bidding approach \cite{Dias+2006}. In intentional coordination e.g., \cite{Parker1998}, robots uses direct allocation method to communicate and to negotiate for assigning tasks. This is preferred approach among MRS research community since it is easily understood, easier to design, implement and analysis formally. Task allocation through bidding is mainly based on the Contract Net Protocol \cite{Davis1988+}. Predefined Task allocation through other approaches are also present in literature. For example, inspired by the vacancy chain phenomena in nature, \cite{Dahl+2003} proposed a vacancy chain scheduling (VCS) algorithm for a restricted class of MRTA problems in spatially classifiable domains.

\subsection{Emergent task-allocation}
On the other hand emergent task-allocation approach relies on the emergent group behaviours e.g., \cite{Kube+1993}, such as emergent cooperation \cite{Lerman+2006}, adaptation rules \cite{Liu+2007} etc., that lead to task allocation with local sensing, local interactions. It typically uses little or no explicit communication or negotiations between robots. They are more scalable to large team size and more robust via parallelism and redundancy.

MRTA problem can be addressed in many different ways depending upon the paradigm selected to abstract the problem and its relevant constraints and requirements \cite{Parker2008}. Firstly, in emergent task allocation in bioinspired swarms paradigm MRTA homogeneous robots are employed to perform mostly similar tasks only by local sensing and indirect stigmergic ( or no) communication. Secondly, in organizational and social paradigm MRTA can follow one of the two major approaches: 1) task allocation by making use of roles and 2) task allocation through bidding. For example, in multi-robot soccer, each role encompasses several specific tasks and heterogeneous robots select their roles based on their position and capabilities. In market-based approach, robots can negotiate with other team-mates to collectively solve a set of tasks. Finally, in knowledge-based approach, also known as intentional coordination, MRTA is done through the modelling of team-mate capabilities, such as by observing the performance of other team-members performance with or without explicit communication.

\subsection{Key issues in MRS task-allocation}
%-----------------------------------------------------------------------
\section{Communication in MRS}
\label{bg:mrs-comm}
Communication between robots is an  important issue in MRS \cite{Arkin1998}. This is not a prerequisite for the group to be functioning, but often useful component of MRS \cite{Mataric2007}. Let us now investigate why communication is important, how this is usually archived in MRS and related other issues.

Researchers generally agree that communication in MRS usually provides several major benefits, such as:

\begin{description}
\item[Exchange of information and improving perception:]
Robots  can exchange potential information (as discussed below) based on their spatial position and knowledge of past events. This, in turn, leads to improve perception over a distributed region without directly sensing it.
\item[Synchronization of actions:]
In order to perform (or stop performing) certain tasks simultaneously or in a particular order robots need to communicate, or signal, to each other. 
\item[Enabling interactions:]
Communication is not strictly necessary for coordinating team actions. But  communication can help a lot to interact (and hence influence) each-other in a team that, in turn, enables robots to coordinate and negotiate their actions.
\end{description}

Since a MRS  can be comprised of robots of various computation and communication capabilities, it is also necessary to define the communication content and range \cite{Arkin1998,Mataric2007}. Usually robots can communicate about various states (e.g., task-related, individual, environmental etc.), their individual intentions and goals. 
% [Organization]
\subsection{Centralized and decentralized communications}

% [ To Whom] 
\subsection{Local and Broadcast communications}
%[How]
Robots communicate in a number of ways available under a specific application. This communication methods can be divided into two major categories:
% FIG: MRS comm. eg.
\begin{figure}
\begin{minipage}[t]{0.48\linewidth}
\centering
\includegraphics[width=6cm, height=4cm, angle=0]
{./photos/s-bots-comm-evolve-300x214.eps}
\caption{ Swarmbot communicating by light signals}
\end{minipage}
\hspace{0.5cm}
\begin{minipage}[t]{0.48\linewidth}
\centering
\includegraphics[width=6cm,height=4cm, angle=0]{./photos/robots_cs_utk_edu_balajee.eps}
\caption{Larger robots can have on board communication module}
\label{fig:self-org-agent} % Give a unique label
\end{minipage}
\end{figure}
%%
\subsection{Explicit and implicit communications}
\subsection*{Explicit or direct communication}
This is also known as intentional communication. This is done purposefully and usually using wireless radio. Based on the number of recipients of message, the communication process is termed differently. Such as,
Broadcast communication: where all other robots receive the message.
Peer-to-peer communication: where only a single robot receive the message.
Publish-subscribe communication: where only a selected (previously subscribed) number of robots receive the message.
Because explicit communication is costly in terms of both hardware and software, robotic researchers always put extra attention to design such a system by analysing strict requirements such as communication necessity, range, content, reliability of communication channel (loose of message) etc.

\subsection*{Implicit or indirect communication} 
This is also known as indirect stigmergic communication. This is a powerful way of communication where individuals leave information in the environment. This method was adopted from the social insect behaviour, such as stigmergy of ants (leaving of small amount of pheromone or chemicals behind while moving in a trail). Some researchers also tried to establish communication among robots through vision \cite{Kuniyoshi1994}.

\subsection{Key issues in MRS communication}
In multi-robot communication researchers have identified several issues. Some of the major issues are discussed here.
Kin Recognition
Kin recognition refers to the ability of a robot to recognize immediate family members by implicit or explicit communication or sensing. In case of MRS, this can be as simple as identifying other robots from objects and environment or as finding team-mates in a robotic soccer. This is an useful ability that helps interaction, such as cooperation among team members. 

\subsubsection*{Representation of Languages}
In case of effective communication several researchers also focused on representation of languages and grounding of these languages in physical world.

\subsubsection*{Fault-tolerance, Reliability and Adaptation}  
Since every communication channel is not free from noise and corruption of messages significant attention has been also given to manage these no communication situations, such as by setting up and maintaining communication network, managing reliability and adaptation rules when there is no communication link available. In terms of guaranteeing communication, researchers also tried to find ways for a deadlock free communication methods \cite{Arkin1998}, such as signboard communication method \cite{Wang1989}.

%%%%%%%%%%%%%%%%%%%%%
\section{Application of MRS in automation industry}
In order to examine the feasibility of our approach of emergent DoL, we have selected the distributed automated manufacturing application domain. Most of the research in this area is inspired by intelligent multi-agent technology \cite{Shen+2001}. A few other researchers also tried to apply the concepts of biological self-organization  \cite{Ueda2006,Lazinica+2007}. In this section we have reviewed these concepts and technologies mainly focusing on physical embodiment of agents, i.e., the use of multiple mobile robots or automated guided vehicles (AGV).
% The use of static robots (or manipulators) or scheduling of static processes or resources are excluded from this review.
%
\subsection{Multi-agent based approaches}
Since early 80s researchers have been applying agent technology to manufacturing enterprise integration,  manufacturing process planning, scheduling and shop floor control, material handing and so on\cite{Shen+2006}. An agent as a software system that communicates and cooperates with other software systems to solve a complex problem that is beyond the capability of each individual software system \cite{Shen+2001}. Most notable capabilities of agents are autonomous, adaptive, cooperative and proactive. There exists many different extensions of agent-based technologies such as Holonic Manufacturing System (HMS) \cite{Bussmann+2004}. A holon is an autonomous and cooperative unit of manufacturing system for transporting, transforming, sorting and/or validating information and physical objects. 
%% FIG: Kiva systems
\begin{figure}
\centering
\includegraphics[width=10cm, angle=0]
{./photos/Kiva-Systems.eps}
\caption{ KIVA systems revolutionary material handling system}
\end{figure}
%
Agent based technologies have addressed many of the problems encountered by the traditional centralized method. It can respond to the dynamic changes and disturbances through local decision making. The autonomy of individual resource agents and loosely coupled network architecture provide better fault-tolerance. The inter agent distributed communication and negotiation also eliminate the problem of having a single point of failure of a centralized system. These facilitate a manufacturing enterprise to reduce their response time to market demands in globally competitive market. Despite having so many advantages, agent-based systems are still not widely implemented in the manufacturing industry comparing to the other similar technologies, such as distributed objects and web-based technologies due to the lack of integration of this systems with other existing systems particularly real-time data collection system, e.g., RFID (radio frequency identification), SCADA (supervisory control and data acquisition) etc \cite{Shen+2006}. Another barrier is the increased cost of investment in exchange of some additional flexibility and throughput \cite{Schild+2007}. 
%
\subsection{Biology-inspired approaches}
The insightful findings from biological studies on insects and organisms have directly inspired many researchers to solve problems of manufacturing industries in a biological way. These can be categorized into two groups: one that allocates task with explicit potential fields (PF) and another that allocate tasks without specifying any PF. Below we have discussed both types of BMS.
\subsubsection*{Explicit potential filed based BMS}
The biological evidences of the existence of PF between a task and an individual worker such as, a flower and a bee, a food source and an ant,  inspired some researchers to conceptualize the assigning of artificial PF between two manufacturing resources. For example, PF is assumed between a machine that produce a material part and a worker robot (or AGV) that manipulates the raw materials and finished products. \cite{Ueda2006} conceptualized this PF as the attractive and repulsive forces based on machine capabilities and product requirements. Task allocation is carried out based on the local matching between machine capabilities and product requirements. Each machine generates an attractive field based on its capabilities and each robot can sense and matches this attractive filed according to the requirements of a product. PF is a function of distance between entities. Here, self-organization of manufacturing resources occurred by the process of matching the machine capabilities and requirements of moving robots.  Through computer simulations and a prototype implementation of a line-less car chassis welding \cite{Ueda2006} found that this system was providing higher productivity and cost-effectiveness of manufacturing process where frequent reconfiguration of factory layout was a major requirement. This approach, was also extended and implemented in a supply chain network and in a simulated ant system model where individual agents were rational agents who selected tasks based on their imposed limitations on sensing. 
\subsubsection*{BMS without explicit potential fields}
Several other researchers did not express the above PF for task allocation among manufacturing resources explicitly, rather they stressed on task selection of robots based on the task-capability broadcasts from the machines to the worker robots. In case of \cite{Lazinica+2007}, task capabilities are expressed as the required time to finish a task in a specific machine. They used assigned priority levels to accomplish the assembly of different kinds of products in the computer simulation of their bionic manufacturing system. In another earlier computer simulated implementation of swarm robotic material handing of a manufacturing work-cell, \cite{Doty+1993} pointed out several pitfalls of such a BMS system, such as dead-lock in manufacturing in inter-dependant product parts, unpredictability of task completion, energy wastage of robots wandering for tasks etc. Although most of these problems remain unsolved researchers are still exploring the concepts BMS in order to achieve a higher level of robustness, flexibility and operational efficiency in a highly decentralized, flexible, and  globally competent next generation automated manufacturing system.