\chapter{Background and Related Work}
\label{bg}
%%%%%%%%%%%%%%%%%%%%%
\section{Definition of key terms}
\subsection{Self-regulation}
\label{bg:def:self-reg}
Animals and flying beings, that live on or above earth, form social communities similar to human societies \cite{SIHQ1995}. In recent years, the biological study of social insects and other animals reveals 
us that simple individuals of these self-organized  societies can solve various complex and large everyday-problems with a few behavioural rules, relying on their minimum sensing and communication abilities \cite{Camazine+2001}. Some common tasks of these biological societies include: dynamic foraging, building amazing nest structures, maintaining division of labour among workers \cite{Bonabeau+1999}. These tasks are done by colonies  ranging from a few animals to thousands or millions of individuals. Despite their huge colony size, they easily achieve surprising efficiency in those tasks with many common features, e.g. robustness, flexibility, synergy (for an example in ants, see Fig. \ref{fig:self-org-ants}). Today, these findings have inspired scientists and engineers to use this knowledge of biological self-organization in developing solutions for various problems of artificial systems, such as  routing traffics in telecommunication and vehicle networks, designing control algorithms for large groups of autonomous robots, automating industrial shop-floor tasks and so forth \cite{Garnier+2007}.\\
%%----- FIG fig:weaver-green-tree-ants---------
\begin{figure}[htp]
  \centering
  \subfloat[Weaver ants]{\label{fig:weaver-ants}\includegraphics[width=6cm, height=4cm]{./photos/ants-hy27c.eps}}                
  \hspace{0.25cm}
  \subfloat[Green-tree ants]{\label{fig:green-tree-ants}\includegraphics[width=6cm, height=4cm]{./photos/ants_spacecollective.org_greenTreeAnt.eps}}
  \caption{(a) During nest construction, weaver ants combine two leaves by pulling them from two sides, reproduced from \protect\citeasnoun{Yahya2000}.
(b) Green-tree ants retrieve a relatively large prey. From http://www.spacecollective.org, last seen on 01/05/2010. }
  \label{fig:self-org-ants}
\end{figure}
% FIG: self-org
\begin{figure}[htp]
\centering
\includegraphics[height=8cm, angle=0]
{./images/dia-files/self-org-1}
\caption{Self-organization viewed from four (A-D) inseparable perspectives. Adopted from \protect\citeasnoun{Camazine+2001}.}
\label{fig:self-org-1} % Give a unique label
\vspace*{0.25cm}
\centering
\includegraphics[height=5cm, angle=0]{./images/dia-files/self-org-agent}
\caption{ Three major interfaces of a self-regulated agent.}
\label{fig:self-org-agent} % Give a unique label
\end{figure}
%%
Self-organization (SO) in biological and  other systems are often characterized in terms of four major ingredients: 1) positive feedback, 2) negative feedback, 3) presence of multiple interactions among individuals and their environment, and 4) amplification of fluctuations  e.g., random walks, errors, random task-switching \cite{Camazine+2001}. An external observer, as if looking through transparent glasses, can recognize a self-organized system by observing the individual interactions of that system from four interlinked perspectives (Fig. \ref{fig:self-org-1}). The first perspective is the {\em  positive feedback} or amplification that can be resulted from the execution of simple behavioural ``rules of thumb''. For example, recruitment to a food source through trail laying and trail following in some ants  is due to positive feedback that attract other ants to follow the trail and to lay more pheromones over time. The second perspective is the {\em negative feedback} that counterbalances positive feedback. This usually occurs to stabilize collective patterns, e.g., crowding at the food sources (saturation), competition between paths to food sources etc. The third perspective is the {\em presence of multiple  interactions} that can be direct peer-to-peer (P2P) or indirect {\em stigmergic}, e.g. ants pheromone laying. Finally, the fourth  perspective is the {\em amplification of fluctuations} that comes from various stochastic events. For example, errors in trail following of some ants may lead some foragers to get lost and later on to find new, unexploited food sources and recruit other ants to those sources.\\
%% FIG: fig:honey-bee-nest-comb
\begin{figure}
\centering
\subfloat[Honey-bee nest]{\includegraphics[width=6cm, height=4cm]{./photos/honey-bee-nest-hy103.eps}} 
\hspace{0.25cm}
\subfloat[Construction of honey-combs]{\includegraphics[width=6cm, height=4cm]{./photos/honey-bee-comb-building-knol-google.eps}}
\caption{(a) A Honey-bee colony has built a nest on a tree-branch. From http://www.harunyahya.com, last seen on 01/05/2010. (b) Honey-bees are constructing honey combs. From http://knol.google.com, last seen on 01/05/2010.}
\label{fig:honey-bee-nest}
\end{figure}
% 
In a self-organized system, an individual agent may have limited cognitive, sensing and communication capabilities. But they are collectively capable of solving complex and large problems, e.g. coordinated nest construction of honey-bees (Fig. \ref{fig:honey-bee-nest}), collective defence of school fishes upon predator attack (Fig. \ref{fig:school-of-fish}), ordered homing of bats (Fig. \ref{fig:bats-colony}).  Since the discovery of these collective behavioural patterns of self-organized societies, scientists observed modulation or adaptation of behaviours in the individual level \cite{Garnier+2007}. For example, in order to prevent a life-threatening humidity-drop in the colony, cockroaches maintain a locally sustainable humidity level by increasing their tendency to aggregate, i.e. by regulating their individual aggregation behaviours . As shown in Fig. \ref{fig:self-org-agent}, this  self-regulation (SR) of an individual agent is depicted through a triangle where its base-arm of simple behavioural rules of thumb (e.g. in this case, intense aggregation of cockroaches in low humidity) is supported by two side-arms: local communication and local sensing. This local sensing is sometimes also referred to as sensing or information gathering from the work in progress, e.g.stigmergy, and the local communication mentioned here is an instance of direct communication with neighbours  \cite{Camazine+2001}.\\
%% FIG: School of Fish
\begin{figure}[htp]
\centering
\subfloat[A moving school of fish]{\includegraphics[width=6cm, height=4cm]{./photos/adventureaquariumschooloffish.eps}} 
\hspace{0.25cm}
\subfloat[A shark attacked a school of fish]{\includegraphics[width=6cm, height=4cm]{./photos/schoo_of_fish_shark-bigtiger.eps}}
\caption{(a) Group cohesion of a school of fish. (b) When a shark attacks a school of fish, they misguide the attacker by swift random movements. From http://www.travelblog.org last seen on 01/05/2010.}
\label{fig:school-of-fish}
\end{figure}
%%
SR has been studied in many other branches of knowledge. In most places of literature, SR refers to the exercise of control over oneself to bring the self into line with preferred standards \cite{Baumeister+2007}. One of the most notable self-regulatory process is the human body's homoeostatic process where the human body's inner process seeks to return to its regular temperature when it gets overheated or chilled. \citeasnoun{Baumeister+2007} has referred self-regulation to goal-directed behaviour or feedback loops, whereas self-control may be associated with conscious impulse control.  In psychology, SR denotes the strenuous actions to resist temptation or to overcome anxiety. SR is also divided into two categories: 1) conscious and 2) unconscious SR. Conscious SR puts emphasis on conscious, deliberate efforts in self-regulation. On the other hand, unconscious self-regulation refers to the automatic self-regulatory process that is not labour intensive but operate in harmony with unpredictable, unfolding events in the environment. This uses the available informational input in ways that help to attain an activated goal.\\
%% FIG: Bats navigation
\begin{figure}[htp]
\centering
\subfloat[A bat colony]{\includegraphics[width=6cm, height=4cm]{./photos/bats_hy2.eps}} 
\hspace{0.25cm}
\subfloat[Navigation of bats]{\includegraphics[width=6cm, height=4cm]{./photos/bats_hy3.eps}}
\caption{(a) One of the largest bat colony with about 50 million bats, (b) These bats can show amazing navigation abilities: they always fly back to their nest on a straight route from wherever they are, reproduced from \protect\citeasnoun{Tuttle1995}.}
\label{fig:bats-colony}
\end{figure}
%%
The concepts of SR is also commonly used in cybernetic theory where SR in inanimate mechanisms show that they can regulate themselves by making adjustments according to pre-programmed goals or set standards. A common example of this kind can be found in a thermostat that controls a heating and cooling system to maintain a desired temperature in a room. In physics, chemistry, biology and some other branches of natural sciences, the concept of SR is centred around the study of self-organizing individuals. SR has also been studied in the context of human social systems where it originates from the division of social labour that creates self-organized process that has self-regulating effects \cite{Kppers+1990}. Two types of SR have been reported in many places of literature of sociology: 1) SR from SO and 2) SR from activities of components in a heterarchical organization. It is interesting to note that SR in biological species provides the similar evidences of bottom-up approach of SR of heterarchical organization through interaction of individuals or the absence of strict hierarchy \cite{Beer1981}.\\
%%
From the above discussion, we see that the term {\em self-regulation} carries a wide range of meaning in different branches of knowledge. In psychology and cognitive neuroscience, SR is discussed within an individual's perspective whereas, in biology and social sciences, SR is discussed within the context of a group of individuals or society as a whole. In  this thesis, the latter context is more appropriate where  SR covers both aspects of monitoring one's own state and environmental changes in relation to the communal goal and thus making adjustments in self behaviours with respect to the changes found.
%%%%======================================================================
\subsection{Communication} 
\label{bg:def:comm}
%%FIG:  Comm defined
\begin{figure}
\centering
\includegraphics[width=8.5cm, angle=0]
{./images/dia-files/comm-defined.eps}
\caption{ General models of communication, adopted from \protect\citeasnoun{West+2003}.}
\label{fig:gen-comm-defined} % Give a unique label
\end{figure}
%%
\textbf{What is Communication ?} Defining {\em communication} can be challenging due to the use of this term in several disciplines with somewhat different meanings. This has been potrayed in the writing of Sarah Trenholm \cite{West+2003} who describes communication as piece of luggage overstuffed with all manner of odd ideas and meanings. \citeasnoun{West+2003} defines communication  as:
\begin{quotation}
``A social process where individuals employ symbols to establish and interpret meaning in their environment.''
\end{quotation}
The notion of being a ``social process'' involves, at least two or more individuals, and interactions that are both dynamic and ongoing. Moreover, symbols can be simply some sort of arbitrary labels given to a phenomena and they can represent  concrete objects or an abstract ideas. Encyclopaedia Britannica also defines communication as ``the exchange of meanings between individuals through a common system of symbol''. But since this definition lacks the notion of sociality we find it  incomplete. There are many other debates related to communication, e.g.  the intentionality debate \cite{West+2003}, symbol grounding \cite{Mataric2007} and so on. However, in order to draw some tractable boundaries , we consider communication process within the context of  symbol or message exchange between two or more parties with a clear intent  to influence each others' behaviours.\\
According to a biological model of communication (Fig. \ref{fig:bio-comm-defined}), communication is a biological process where an  individual (sender) intentionally transmits encoded message though physical signal and that, on being received and decoded by another individual of same species (receiver), influences receiver's behaviour \cite{Frings1997}. Note that, here individuals are of same species and thus they have a  shared message vocabulary and mechanism of message encoding/decoding. Although  this definition has not included the dynamics of a communication process, it is more precise for low-level biological and artificial systems. It accounts for the behavioural changes during communication process. These changes can be tracked though observing states of individuals.\\   
%% FIG bio-comm defined
\begin{figure}
\centering
\includegraphics[width=10cm,height=4cm, angle=0]{./images/dia-files/animal-comm-defined.eps}
\caption{ A biological model of communication, adopted from \protect\citeasnoun{Frings1997}.}
\label{fig:bio-comm-defined} 
\end{figure}
%%
\textbf{Models of Communication.} If we explore the elements of communication, we can grab the whole picture involved in the communication process. This can be explained through the study of the models  of communication. There exists a plenty of models of communication. For this thesis, here we briefly discuss three prominent models: 1) linear model, 2) interaction model and 3) transaction model. Fig. \ref{fig:gen-comm-defined} embeds first two models inside a single circle. In linear model, as introduced by Claude Shanon and Warren Weaver in 1949, communication is a one way process where a {\em message} is sent from a {\em source} to a {\em receiver} through a {\em channel}. In Fig. \ref{fig:comm-model-1}, around this linear view, interaction model has been drawn. This model, proposed by Wilber Schramm in 1954, views communication as a two-way process with an additional {\em feedback} element  that links both source and receiver. This feedback is a response given to the source by the receiver to confirm how the message is being understood. Here, during message passing, both source and receiver utilize their individual {\em field of experiences} that describe the overlap of their common experiences, cultures etc. Unlike separate filed of experiences and discrete sending and receiving of message, in transactional model, introduced by Barnlund in 1970, the sending and receiving of message is done simultaneously. Here, the  field of experiences of source and receiver can overlap to some degrees. In all of the above three models, {\em noise} or a common message distorting element is present in the communication process. This noise can be occurred from the linguistic influences, i.e. message semantics,  physical or bodily influences, cognitive influences or even from biological
or physiological influences e.g., anger or shouting voice while talking.\\ 
The above models of communication describe the incremental complexities of message exchanging in the communication process. Surely, the transactional model is comparatively the most sophisticated model that prescribes adjusting the sender's message content while receiving an implicit or explicit feedback in real-time. For example, while speaking with her son for advising to read a story book, a mother may alter her verbal message as she simultaneously ``reads'' the non-verbal message of her child from his face. However, in case of a MRS, such sophistication may not be required or realizable by the current state of the art in multi-robot communication technology. In this study, we follow the simple linear model that meets the most of communication requirements of a MRS. The feedback has not been considered as we have assumed that all robots of our artificial MRS have a same shared vocabulary such that a message is understood as it is sent. Source never waits for an additional feedback from receiver to terminate sending a message.\\
\textbf{Measuring communicated information.} Following the linear model of communication, the amount of transferred information associated with a certain random variable X can be calculated by the concept of {\em Shanon entropy}. Adopting the notation of Feldman \cite{Feldman1997}, and indicating a discrete random variable with the capital letter X, that can take values $x \in \chi$, the information entropy is defined as:
\begin{equation}
\label{eq:entropy}
H[X] = - \sum_{x \in \chi } p(x) \cdot  \log_{2} p(x)
\end{equation}
where $p(x)$ is the probability that $X$ will take the value of $x$. $H[X]$ is also called the {\em marginal} entropy of $X$ , since it depends on only the marginal probability of one random variable. The marginal entropy of the random variable $X$ is zero if $X$  always assumes the same value with $p(X =x′)= 1$, and maximum if $X$  assumes all possible states with an equal probability.\\
For example, in order to measure information flow in an elementary communication system, let {\em bit} be the unit amount of information needed to make a choice between two equiprobable alternates. If $n$ alternates are present, a choice provides the following quantity of information: $H = log_{2} \hspace{1mm} n $. Thus sending of $n$ equiprobable messages reduces $log_{2} \hspace{1mm} n $ amount of uncertainty and thus the amount of information is  $log_{2} \hspace{1mm} n $ bit.  Similarly, according to Eq. \ref{eq:entropy}, the value of $H[X]$ depends on the discretization of $x$. For instance, if the value of random variable $x$ is discretized into 4, then $p(x)$ becomes $\frac{1}{4}$  leading to $H[X]$ = - $ 4 \cdot \frac{1}{4} \cdot log_{2} \hspace{2mm} \frac{1}{4}$  = 2.\\
%%
%% TABLE: COMM-CATEGORIES
\begin{table}
\caption{General characteristics of common communication modes}
\label{table:comm-categories}
\begin{center}
\begin{tabular}{|l|l|l|}
\hline \textbf{Type} & \textbf{Indirect or} & \textbf{Direct or }\\
& \textbf{implicit communication} & \textbf{explicit communication}\\
\hline Centralized & Typically a central entity   & Both global and local broadcast  \\
Communication & modifies the environment. & communications are commonly\\
Mode & It facilitates passive forms  &  used. P2P  communication can \\
(CCM)  &  of communications, i.e.     & also occur. Here, exchange\\
&  communication without   &    of messages  occurs through a\\
& specific target recipient. &  central entity.\\
\hline Decentralized & All individuals are free to & P2P and local-broadcast \\
or Local & modify the environment &  are  most commonly used forms. \\
Communication & and convey information &  Global broadcast occurs \\
Mode & to others. & to handle emergency situations. \\
(LCM) & & All communications are local\\
& & without requiring a central entity.\\
\hline
\end{tabular}
\end{center}
\end{table}
%%
%% [COMM classification]
\begin{figure}
\centering
\includegraphics[width=10cm, angle=0]
{./dia-files/bio-comm-strategies.eps}
\caption{Common communication strategies observed in social systems.}
\label{fig:comm-strategies} 
\end{figure}
%%
\textbf{Communication modes and strategies.} The communication  structure of a system can broadly be classified into two major categories: centralized communication mode (CCM) and decentralized or local communication mode (LCM). A centralized communication system generally has a central entity, e.g. gateway,  that routes all incoming and outgoing communications of the system. Individual nodes of this system often do not communicate each other directly. But they can send and receive messages through this central gateway.  Central gateway can play many roles such as, access control, resource allocation and so on. On the other hand, in LCM, there is no central entity and each node can independently route message to each other. \\
%%
\begin{figure}
\centering
\includegraphics[width=10cm, angle=0]
{./dia-files/bio-comm-strategies-peers.eps}
%figure caption is below the figure
\caption{Number of recipients involved in various communication strategies.}
\label{fig:comm-strategies-peers}  % Give a unique label
\end{figure}
%%
In both biological and robotic literature two basic types of communication are often discussed: 1) direct or explicit communication and 2) indirect or implicit communication. As defined in \cite{Mataric1998}, {\em direct communication} is an intentional communicative act of message passing that aims at one or more particular receiver(s). It typically exchanges information through physical signals. In contrast, {\em indirect communication}, sometimes termed as {\em stigmergy} in biological literature, happens as a form of modifying the environment, e.g. pheromone dropping by ants \cite{Bonabeau+1999}. In an ordinary sense, this is an observed behaviour and many robotic researchers call it as {\em no communication} \citeaffixed{Labella2007}{e.g. }. In order to avoid ambiguity, in this dissertation, by the term {\em communication}, we always refer to {\em direct} communication.\\
Direct or explicit communication can be limited by a communication range and thus by a number of target recipients. Under both CCM and LCM, nodes can select a certain number of target recipients of their messages. This process specifies {\em to whom} a node intends to communicate. In this thesis, we have denoted this mechanism of target recipient(s) selection as {\em communication strategy}.  Fig. \ref{fig:comm-strategies} shows the most common communication strategies found in a  social system.  In the simplest case, when only two nodes can communicate we call this {\em peer-to-peer} (P2P) communication. When nodes can spread information to a limited number of peers of their locality, the communication takes the form of {\em local broadcast}, i.e. one sender and a few receivers within a certain locality. For example, when a foraging honey-bee gives the information of flower sources to a number of peers through various dances, it conveys this information to a few peers through a local broadcast. However, giving the sample of nectar through tactile or taste to its peers can be considered as an instance of P2P communication. The {\em global broadcast} strategy can be found in almost all social species to handle emergency situations, e.g. emitting alarm signal in danger. Table \ref{table:comm-categories}  shows the relationship between various communication modes and their ways of adopting different strategies. Fig. \ref {fig:comm-strategies-peers} shows a typical count of average number of peers in various communication strategies. The actual number of peers under local broadcast strategy is dependent upon a particular social system and it changes over time in different levels of interactions among individuals. Sec. \ref{bg:bio-comm} and \ref{bg:mrs-comm} reviews communication in biological social system (BSS) and MRS respectively.
%%-------------------------------------------------------------------
\subsection{Division of labour or task-allocation}
\label{bg:def:dol}
Encyclopaedia Britannica serves the definition of {\em division of labour (DOL)} as the ``separation of a work process into a number of tasks, with each task performed by a separate person or group of persons''. Originated from economics and sociology, the term division of labour is widely used in many branches of knowledge. As mentioned by the Scottish philosopher Adam Smith, the founder of modern economics :
\begin{quotation} 
The great increase of the quantity of work which, in consequence of the division of labour, the same number of people are capable of performing, is owing to three different circumstances; first, to increase the dexterity in every particular workman; secondly, to the saving of the time which is commonly lost in passing from one species of work to another; and lastly, to the invention of a great number of machines which facilitate and abridge labour, and enable one man to do the work of many.\\
(Adam Smith (1776) in \citeasnoun{Sendova-Franks+1999})
\end{quotation} 
In sociology, DOL usually denotes the work specialization \cite{Sayer+1992}. Basically, it answers three major questions:
\begin{enumerate}
\item {\em What task?} i.e., the description of the tasks to be done, service to be rendered or products to be manufactured.
\item {\em Why dividing it to individuals?} i.e., the underlying social standards for this division, such as task appropriateness based on class, gender, age, skill etc.
\item {\em How to divide it?} i.e.,the method or process of separating the whole task into small pieces of tasks that can be performed easily. 
\end{enumerate}
%% DoL: termites, skyscrapper
%% FIG: Termite nest
\begin{figure}[htp]
\centering
\subfloat[A termite nest]{\includegraphics[width=6cm, height=4cm]{./photos/termites_nest.eps}} 
\hspace{0.25cm}
\subfloat[Two Skyscrapers]{\includegraphics[width=6cm, height=4cm]{./photos/skyscraper.eps}}
\caption{(a) A termite colony constructs their nest through bottom-up approach, i.e. without a central planner. (b) Humans construct skyscrapers using a top-down plan. From http://www.harunyahya.com, last seen on 01/05/2010.}
\label{fig:termite-nest}
\end{figure}
%%
From the study of biological social insects and other BSSes, we can find that two major metrics of DOL have been established in literature: 1) task-specialization and 2) plasticity. {\em Task-specialization} is an integral part of DOL where a worker usually does not perform all tasks, but rather specializes in a set of tasks, according to its morphology, age, or chance \cite{Bonabeau+1999}. This DOL among nest-mates, whereby different activities are performed simultaneously by groups of specialized individuals, is believed to be more efficient than if tasks were performed sequentially by unspecialised individuals. DOL also has a great {\em plasticity} where the removal of one class of workers is quickly compensated for by other workers. Thus distributions of workers among different concurrent tasks keep changing according to the external (environmental) and internal conditions of a colony \cite{Garnier+2007}.\\
%%
In artificial social systems, like multi-agent or MRS, the term ``division of labour'' is often found synonymous to ``task-allocation'' \cite{Shen+2001}. However, some researchers (e.g. \cite{Labella2007}) argued to distinguish these terms due to the origin and particular contextual use of these terms. Particularly, DOL adopts the biological notion of collective task performance with little or no communication. On the other hand, task allocation follows the meaning of assigning task(s) to particular robot(s) based on individual robot capabilities, typically through explicit communication, such as {\em intentional cooperation} \cite{Parker1998}. Generally, the former is considered by {\em swarm robotic system (SRS)} and latter is done under {\em traditional MRS}. Sec. \ref{bg:mrs:paradigms} covers both of these approaches and Sec. \ref{bg:mrta} provides critical review on DOL under these approaches.\\
%%
In this dissertation, for defining DOL we closely follow the SR approach that emphasizes on having task-allocation and plasticity among workers. However, we do not put any restriction on the use of communication. In fact, we view DOL as a group-level phenomenon which occur due to the individual agent's self-regulatory task selection behaviour. But, unlike SR approach that view communication as expensive and hence try to find solutions avoiding it, we do not advocate for restricting the use of communication. Rather,  along with our generic mechanism of division of labour, i.e. AFM (Chapter \ref{afm}), we have proposed some self-regulatory communication strategies to vary communication load dynamically (Chapter \ref{local-comm}). 
%------------------------------------------------------------------
%%%%%%%%%%%%%%%%%%%%%
\section{Communication in biological social systems}
\label{bg:bio-comm}
Communication plays a central role in self-regulated division of labour of biological societies.In this section communication among biological social insects are briefly reviewed within the context of self-regulated  division of labour.

\subsection{Purposes, modalities and ranges}
Communication in biological societies serves many closely related social purposes. Most peer-to-peer (P2P) communication include: recruitment to a new food source or nest site, exchange of food particles, recognition of individuals, simple attraction, grooming, sexual communication etc. In addition to that colony-level broadcast communication include: alarm signal, territorial and home range signals and nest markers, communication for achieving certain group effect such as, facilitating or inhibiting  a group activity \cite{Holldobler1990}.\\
\begin{table}
\caption{Common communication modalities in biological social systems}
\label{table:bio-comm-modalities}
\begin{center}
\begin{threeparttable}
\begin{tabular}{|l|l|l|}
\hline \textbf{Modality} & \textbf{Range} & \textbf{Information type}\\
\hline Sound & Long\tnote{a} & Advertising about food  source,  danger etc. \\                                                                                                                                               
\hline Vision & Short\tnote{b}  & Private, e.g. courtship display \\
\hline Chemical  & Short/long & Various messages, e.g. food location, alarm etc.\\
\hline Tactile & Short & Qualitative info, e.g. quality of flower,\\ & & peer identification etc.\\
\hline Electric & Short/long & Mostly advertising types, e.g. aggression messages\\
\hline
\end{tabular}
\begin{tablenotes}
\item [a]Depending on the type of species, long range signals can reach from a few metres to several kilometres.
\item [b]Short range typically covers from few mm to about a metre or so.
\end{tablenotes}
\end{threeparttable}
\end{center}
\end{table}
%[Modalities and Ranges]
Biological social insects use different modalities to establish social communication, such as, sound, vision, chemical, tactile,  electric and so forth.  Sound waves can travel a long distance and thus they are suitable for advertising signals. They are also best for transmitting complicated information quickly \cite{Slater1986}. Visual signals can travel more rapidly than sound but they are limited by the physical size or line of sight of an animal. They also do not travel around obstacles. Thus they are suitable for short-distance private signals such as in courtship display.\\
% 
In ants and some other social insects chemical communication is dominant. Any kind of chemical substance that is used for communication between intra-species or inter-species is termed as semiochemical \cite{Holldobler1990}. A pheromone is a semiochemical, usually a glandular secretion, used for communication within species. One individual releases it as a signal and others responds it after tasting or smelling it. Using pheromones individuals can code quite complicated messages in smells. For example a typical an ant colony operates with somewhere between 10 and 20 kinds of signals \cite{Holldobler1990}. Most of these are chemical in nature. If wind and other conditions are favourable,  this type of signals emitted by such a tiny species can be detected from several kilometres away. Thus chemical signals are extremely economical of their production and transmission. But they are quite slow to diffuse away. But ants and other social insects manage to create sequential and compound messages either by a graded reaction of different concentrations of same substance or by blends of signals.
Tactile communication is also widely observed in ants and other species typically by using their body antennae and forelegs. It is observed that in ants touch is primarily used  for receiving information rather than informing something. It is usually found as an invitation behaviour in worker recruitment process. When an ant intends to recruit a nest-mate for foraging or other tasks it runs upto a nest-mate and beats her body very lightly with  antennae and forelegs. The recruiter then runs to a recently laid pheromone trail or lays a new one. In this form of communication limited amount of information is exchanged. In underwater environment some fishes and other species also communicate through electric signals where there nerves and muscles work as batteries. They use continuous or intermittent pulses with  different frequencies learn about environment and to convey their identity and aggression messages.
%% FIG. Fireflies
\begin{figure}[htp]
\centering
\subfloat[Flashing fireflies]{\includegraphics[width=6cm, height=4cm]{./photos/fire-flies.eps}} 
\hspace{0.25cm}
\subfloat[A firefly emitting light]{\includegraphics[width=6cm, height=4cm]{./photos/firefly-light-under.eps}}
\caption{(a) Flashing lights of fireflies displaying their synchronous behaviours (b) A firefly can produce light to signal other fireflies. From http://www.letsjapan.markmode.com, last seen on 01/06/2010.}
\label{fig:fireflies}
\end{figure}
%%
%%%%%%%%%%%%%%%%%%%%%%%%%%%%%%%%%%%%%%%%%%
\subsection{Signal active space and locality}
%bio-comm-ants-active-space
The concept of active space is widely used to describe the propagation of signals by species. In a network environment of signal emitters and receivers, active space is defined as the area encompassed by the signal during the course of transmission \cite{Mcgregor2000}. In case of long-range signals, or even in case of short-range signals, this area include several individuals where their social grouping allows them to stay in cohesion. The concept of active space is described somewhat differently in case some social insects. In case of ants, this active space is defined as a zone within which the concentration of pheromone (or any other behaviourally active chemical substances) is at or above threshold concentration \cite{Holldobler1990}. Mathematically this is denoted by a ratio:
\begin{equation}
\frac{\textit{The amount of pheromone emitted (Q)}}{\textit{The threshold concentration at which the receiving ant responds (K)}}
\end{equation}
Q is measured in number of molecules released in a burst or in per unit of time whereas K is measured in molecules per unit of volume. 
Fig. \ref{fig:ants-active-space} shows the use of active spaces of two species of ants: (a) {\em Atta texana} and (b) {\em Myrmicaria eumenoides}.  The former one uses two different concentrations of {\em 4-methyl-3-heptanone} to create attraction and alarm signals while the latter one uses two different chemicals: {\em Beta-pinene} and {\em Limonene} two create similar kinds signals, i.e. alerting and circling.\\ 
The adjustment of this ratio enables individuals to gain a shorter fade-out time and permits signals to be more sharply pinpointed in time and space by the receivers. In order to transmit the location of the animal in the signal, the rate of information transfer can be increased by either by lowering the rate of emission of Q or by increasing K, or both. For alarm and trail systems a lower value of this ratio is used. Thus, according to need, individuals regulate their active space by making it large or small, or by reaching their maximum radius quickly or slowly, or by enduring briefly or for a long period of time. For example, in case of alarm, recruitment or sexual communication signals where encoding the location of an individual is needed, the information in each signal increases as the logarithm of the square of distance over which the signal travels. From the precise study of pheromones it has been found that active space of alarm signal is consists of a concentric pair of hemispheres (see Fig. \ref{fig:ants-active-space}). As the ant enters the outer zone she is attracted inward toward the point source; when she next crosses into the central hemisphere she become alarmed. It is also observed that ants can release pheromones with different active spaces.\\
%%
\begin{figure}
\centering
\includegraphics[width=12cm, angle=0]
{./dia-files/bio-comm-ants-active-space.eps}
%figure caption is below the figure
\caption{Pheromone active space observed in ants, reproduced from \protect\citeasnoun{Holldobler1990}.}
\label{fig:ants-active-space} % Give a unique label
\end{figure}
Active space has strong role in modulating the behaviours of ants. For example, when workers of {\em Acanthomyops claviger} ants produce alarm signal due to an attack by a rival or insect predator, workers sitting a few millimetres away begin to react within seconds. However those ants sitting a few centimetres away take a minute or longer to react. In many cases ants and other social insects exhibit modulatory communication within their active space where many individuals involve in many different tasks. For example, while retrieving the large prey, workers of {\em Aphaeonogerter} ants produce chirping sounds (known as stridulate) along with releasing poison gland pheromones. These sounds attract more workers and keep them within the vicinity of the dead prey to protect it from their competitors. This communication amplification behaviour can increase the active space to a maximum distance of 2 meters.
%%%%%%%%%%%%%%%%%%%%%
\subsection{Common communication strategies}
\label{bg:bio-comm:strategies}
% indirect & b/c comm
In biological social systems, we can find all different sorts of communication strategies ranging from indirect pheromone trail laying to local and global broadcast of various signals. Sec. \ref{bg:def:comm} discusses the most common four communication strategies in natural and artificial world, i.e. indirect, P2P, local and global broadcast communication strategies. Table \ref{table:bio-comm-strategy} lists the use of various communication modalities under different communication strategies. Here we give a few real examples of those strategies from biological social systems. In biological literature, the pheromone trail laying is one of the most discussed indirect communication strategy among various species of ants. Fig. \ref{fig:ant-indirect} shows a pheromone trail following of a group of foraging ants. This indirect communication strategy effectively helps ants to find a better food source among multiple sources, find shorter distance to a food source, marking nest site and move there etc. \cite{Hughes2008 }.\\
Direct P2P communication strategy is also very common among most of the biological species. Fig. \ref{fig:ant-p2p} and Fig. \ref{fig:honey-bee-p2p} shows P2P communication of ants and honey-bees respectively. This tactile form of communication is very effective to exchange food item, flower nectar with each-other or this can be useful even in recruiting nest-mates to a new food source or nest-site.\\
\begin{figure}
\begin{minipage}[t]{0.48\linewidth}
\centering
\includegraphics[width=6cm, height=4cm, angle=0]
{./photos/ants_group_comm_bioteams_com.eps}
\caption{A group of ants following pheromone-trail. From http://www.bioteams.com, last seen on 01/06/2010.}
\label{fig:ant-indirect} % Give a unique label
\end{minipage}
\hspace{0.5cm}
\begin{minipage}[t]{0.48\linewidth}
\centering
\includegraphics[width=6cm,height=4cm, angle=0]{./photos/honey-bee-waggle-dance-knol-google.eps}
\caption{ A dancing honey-bee (\protect{\em centre}) and its followers. From http://knol.google.com, last seen on 01/06/2010.}
\label{fig:honey-bee-local-bc} % Give a unique label
\end{minipage}
\end{figure}
%%%%%%%%%%%%%%%%%%%%
% FIG. Honey bee dance language
\begin{figure}
\centering
\subfloat[Honey-bee's waggle dance]{\includegraphics[width=6cm, height=4cm]{./photos/honey-bee-round-dance.eps}} 
\hspace{0.25cm}
\subfloat[Honey-bee's round dance]{\includegraphics[width=6cm, height=4cm]{./photos/honey-bee-waggle-dance.eps}}
\caption{Examples of local broadcast communication of honey-bees: (a) Honey-bees show waggle-dance (making figure of 8) when food is far and (b) they show round-dance without any waggle when food is closer (within about 75m of hive). From \protect\citeasnoun{Slater1986}.}
\label{fig:honey-bee-dances}
\end{figure}
%%
%%%%%%%%%%%%%%%%%%%%%%
\begin{table}
\caption{Common communication strategies in biological social systems}
\label{table:bio-comm-strategy}
\begin{center}
\begin{tabular}{ll}
\hline 
\textbf{Communication strategy} & \textbf{Common modalities used}\\
\hline 
Indirect & Chemical and electric \\
%\hline 
Peer-to-peer (P2P) &  Vision and tactile\\
%\hline 
Local broadcast &  Sound, chemical and vision\\
%\hline 
Global broadcast & Sound, chemical and electric\\
\hline
\end{tabular}
\end{center}
\end{table}
%%
\begin{figure}
\centering
\subfloat[Two honey-bees]{\includegraphics[width=6cm, height=4cm]{./photos/honey-bee-p2p-hy23.eps}} 
\hspace{0.25cm}
\subfloat[Two ants ]{\includegraphics[width=6cm, height=4cm]{./photos/ants-p2p-hy14.eps}}
\caption{Example of P2P tactile communication: (a) Honey-bees exchange nectar samples by close contact (b) ants also exchange food or information via tactile communication  From http://www.harunyahya.com/ last seen 01/05/2010 .}
\label{fig:bees-ants-p2p-comm}
\end{figure}
%%
%%-----------------------------------------------------
\subsection{Roles of communication in task-allocation}
\label{bg:bio-comm:comm-role}
Communication is an integral part of the DOL process in biological social systems. It creates necessary  preconditions for switching from one tasks to another or to attend dynamic urgent tasks. Suitable communication strategies favour individuals to select a better tasks. For example, \citeasnoun{Garnier+2007} has reported two worker-recruitment experiments on black garden ants and honey-bees. The scout ants of {\em Lasius niger}  recruit uninformed ants to food source using a well-laid pheromone trails. {\em Apis mellifera} honey-bees also recruit nest-mates to newly discovered distant flower sources through waggle-dances. In the experiments,  poor food sources are given first to both ants and honey-bees. After some time,  rich food has been introduced  to both of them. It has been found that only honey-bees can switch from poor source to a rich source. The sophisticated dance communication of honey-bees favours them to get a better solution. \\
%%
\begin{table}
\caption{Self-regulation of communication behaviours based on task-urgency perception}
\label{table:bio-comm-task-urgency}
\begin{center}
\begin{tabular}{|l|l|l|}
\hline \textbf{Example event} & \textbf{Strategy} & \textbf{Modulation of communication}\\
&  &  \textbf{based-on task-urgency}\\
\hline Ant's alarm signal &  Global  & High concentration of pheromones\\
by pheromones & broadcast &  increase aggressive alarm-behaviours \\                                                                                                                                               
\hline Honey-bee's  & Local  &  High quality of nectar source increases \\
round dance & broadcast & dancing and foraging bees\\
\hline Ant's tandem run     & P2P & High quality of nest \\
for nest selection & &   increases traffic flow\\
\hline Ant's pheromone   & Indirect & Food source located at shorter distance\\
trail-laying to   & &  gets higher priority as less pheromone \\
food sources & & evaporates and more ants joins\\
\hline
\end{tabular}
\end{center}
\end{table}
%%%%%%%%%%%%%%%%%
Table \ref{table:bio-comm-task-urgency} presents the link between the task-urgency perception and self-regulation of communication behaviours in biological social systems. Here, we can see that communication is modulated based on the perception of  task-urgency irrespective of the communication strategy of a particular species. This dissertation takes this biological evidence as the baseline of our hypothesis that communication behaviours of a self-regulated system must be linked to the task-requirements of the society. Under indirect communication strategy of ants, i.e. pheromone trail-laying, we can see the that principles of self-organization, e.g. positive and negative feedbacks take place due to the presence of different amount of pheromones for different time periods. Initially, food source located at shorter distance gets relatively more ants  as the ants take less time to return nest. So, more pheromone deposits can be found in this path as a result of positive feedback process.  Thus, the density of pheromones or the strength of indirect communication link reinforces ants to follow this particular trail.\\
\begin{figure}
\centering
\includegraphics[width=6.4cm, angle=-90]
{./images/ch2/honey-bee-dance-stat.eps}
%figure caption is below the figure
\caption{Self-regulation in honey-bee's dance communication behaviours, produced after the results of \protect\possessivecite{Von1967} honey-bee round-dance experiment performed on 24 August 1962.}
\label{fig:honey-bee-dance-stat}  % Give a unique label
\end{figure}
Similarly, perception of task-urgency influences the P2P and broadcast communication strategies. {\em Leptothorax albipennis} ant take lees time in assessing a relatively better nest site and quickly return home to recruit its nest-mates \cite{Pratt+2002}. Here, the quality of nest directly influences its intent to make more ``tandem-runs'' or to do tactile communication with nest-mates. We have already discussed about the influences of the quality of  flower sources to honey-bee dance.  Fig. \ref{fig:honey-bee-dance-stat} shows this phenomena more vividly. It has been plotted using the data from the honey-bee round-dance experiments of \citeasnoun[p. 45]{Von1967}. In this plot, Y1 line refers to the concentration of sugar solution. This solution was kept in a bowl  to attract honey-bees and the amount of this solution was varied from $\frac{3}{16}$M to 2M (taken as 100\%).  The variation of this control parameter influenced honey-bees communication behaviours and thus they produce varying degree of division of labour. Y2 line in Fig. \ref{fig:honey-bee-dance-stat} represents the number of collector bees that return home. The total number of collectors was 55 (taken as 100\%). Y3 plots the percent of collectors displaying round dances. We can see this dancing collectors is directly proportional to the concentration of sugar solution or task-urgency in this case. Similarly the average duration of dance per bee  is plotted in Y4 line. The maximum dancing period was 23.8s (taken as 100\%). Finally, from Y5 line we can see the outcome of the round-dance communication as the number of newly recruited bees to the feeding place. The maximum number of recruited bees was 18 (taken as 100\%). So, from an overall observation, we can see that bees sense the concentration of food-source  as the task-urgency and they self-regulate their round-dance behaviour according to this task-urgency. Thus, this self-regulated dancing behaviour of honey-bees attracts an optimal number of inactive bees to work.\\ 
Broadcast communication is one of the classic way to handle dynamic urgent tasks in biological social systems. It can be commonly observed in birds, ants, bees and many other species. Table \ref{table:bio-comm-task-urgency} mentions about the alarm communication of ants. Similar to the honey-bee's dance communication, ants has a rich language of chemical communication that can produce words through blending different glandular secretions in different concentrations. Fig. \ref{fig:ants-active-space} shows how ants can use different concentrations of chemicals to make different stimulus for other ants. From the study of ants, it is clear to us that taking defensive actions, upon sensing a danger, is one of the highest-priority tasks in an ant colony. Thus, in this highly urgent task, ants almost always use their global broadcast communication strategy through their strong chemical signals and they make sure all individuals can join in this task.  This gives us a coherent picture of self-regulation of animal communication based on the perception of task urgency.
%%-----------------------------------------------
%\subsection{Information flow in communication}
\subsection{Effect of group size on communication}
\label{bg:bio-comm:group-size}
The performance of cooperative tasks in large group of individuals also depends on communication and information sharing. From the study of social wasps,  \citeasnoun{Jeanne1999} has reported that, depending on the group size, different kinds of information flow occur in two groups of social wasps: 1) independent founders of {\em Polistes} (Fig. \ref {fig:polistes}) 2) swarm founders of {\em Polybia} (Fig. \ref {fig:polybia}). Independent founders (IF) are species in which reproductive females establish colonies alone or in small groups, but independent of any sterile workers and in the range of $10^2$ individuals at maturity. Swarm founders (SF) initiates colonies by swarm of workers and queens. They have a large number of individuals, in the order of $10^6$ and 20\% of them can be queen. The most notable difference in the  organization of work of these two social wasps is IF does not rely on any cooperative task performance while SF interact with each-other locally to accomplish their tasks. The work mode of IF can be considered as {\em global sensing – no communication (GSNC)} where individuals sense the task-urgencies throughout a small colony and do these tasks without communicating with each other. On the other hand, the work mode of SF can be treated as {\em local sensing – local communication (LSLC)} where individuals can only sense locally due to large colony-size and they communicate locally to exchange information, e.g. task-urgency (although their exact mechanism is unknown).\\
\begin{figure}
\centering
\subfloat[Polybia wasps]{\includegraphics[width=6cm, height=4cm]{./photos/Polybia_occidentalis_I_JP6646_discoverlife.org.eps}} 
\hspace{0.25cm}
\subfloat[Polistes wasps]{\includegraphics[width=6cm, height=4cm]{./photos/Wasps_wikimedia.org_Polistes_nest_3_sjh.eps}}
\caption{Colony founding in two types of social wasps (a) {\em Polybia occidentalis}  founds colony by swarms (b) {\em Polistes}  founds colony by a few queens independently. From http://www.discoverlife.org, last seen 01/05/2010.}
\label{fig:social-wasps}
\end{figure}
%%
Fig. \ref{fig:wasps-info-flow} compares the occurrence of information flow among IF and SF. In case of SF information about nest-construction or broods food-demand can not reach to foragers directly. Fig. \ref{figs:sf-wasps-info-flow} shows the path of information flow among SF for nest construction. The works of {\em pulp foragers} and {\em water foragers} depend largely on their communication with {\em builders}. On the other hand, in case of IF there is no such communication present among individuals. This phenomena raises the question of how these individuals can select the best work modes from GSNC and LSLC.\\
%%
\begin{figure}[htp]
\centering
\includegraphics[width=9cm, angle=0]
{./dia-files/jannae-fig10-info-flow-cmp.eps}
%figure caption is below the figure
\caption{Different patterns of information flow in two types of social wasps: polybia and polistes, reproduced from \protect\citeasnoun{Jeanne1999}.}
\label{fig:wasps-info-flow}  % Give a unique label
\end{figure}
%%
\citeasnoun{Garnier+2007} tried to answer this question in terms of task-specialization. In case of large colonies, many individuals repeatedly performs same tasks as this minimizes their interferences, although they still have a little probability to select a different task randomly \cite{Jeanne1999}. But because of the large group size, the queuing delay in inter-task switching keeps this task-switching probability very low. Thus, in SF, task-specialization becomes very high among individuals. On the other hand, in small group of IF, specialization in specific tasks is costly because these prevents individuals not to do other tasks whose task-urgency becomes high. Thus they becomes generalist and do not communicate task-urgency to each other.\\
%%
\begin{figure}[htp]
\centering
\includegraphics[width=6cm, angle=0]
{./images/ch2/jeanne-fig9-info-flow.eps}
%figure caption is below the figure
\caption{Information flow in polybia social wasps, reproduced from \protect\citeasnoun{Jeanne1999}.}
\label{figs:sf-wasps-info-flow}  % Give a unique label
\end{figure}
The above interesting findings on GSNC and LSLC in social wasps have been linked up with  the group productivity of wasps. Fig. \ref{fig:wasps-group-productivity} illustrates high group productivity in case of LSLC of SF. The per capita productivity was measured as the number of cells built in the nest in (a) and the weight of dry brood in grams in (b). In case of IF this productivity is much lesser (max. 24 cells per queen at the time the first offspring observed) comparing to the thousands of cells produced by SF \cite{Jeanne1999}.  This shows  us the direct link between high productivity of social wasps and their local communication and information strategy.
%%
\begin{figure}[htp]
\centering
\includegraphics[width=9cm, angle=0]
{./images/ch2/jeanne-fig6-group-size.eps}
%figure caption is below the figure
\caption{Productivity of social wasps as a function of group size, reproduced from \protect\citeasnoun{Jeanne1999}.}
\label{fig:wasps-group-productivity}  % Give a unique label
\end{figure}
%%%%%%%%%%%%%%%%%%%%%%%%%%%%%%%%%%%%%%%%%%%%%%%%%%%%%%%%%%%%%%%%%%%
\section{Overview of multi-robot systems (MRS)}
\label{bg:mrs:overview}
%%
\subsection{MRS research paradigms}
\label{bg:mrs:paradigms}
%%% Earliest MRS
\begin{figure}
\begin{minipage}[t]{0.48\linewidth}
\centering
\includegraphics[width=6cm, height=4cm, angle=0]
{./photos/Nerd_Herd.eps}
\caption{ The Nerd-Herd. From \protect\citeasnoun{Mataric1994}}
\label{fig:mataric-nerd-herd} % Give a unique label
\end{minipage}
\hspace{0.5cm}
\begin{minipage}[t]{0.48\linewidth}
\centering
\includegraphics[width=6cm,height=4cm, angle=0]{./photos/kube_box_pushing.eps}
\caption{ A group of 10 box pushing robots. From \protect\citeasnoun{Kube1997} }
\label{fig:kube-box-pushing} 
\end{minipage}
\end{figure}
%%
Historically the concept of multi-robot system comes almost after the introduction of behaviour-based robotics paradigm \cite{Brooks1986,Arkin1990}. In 1967, using the traditional sense-plan-act or hierarchical approach \cite{Murphy2000}, the first Artificially Intelligent (AI) robot, Shakey, was created at the Stanford Research Institute. In late 80s, \citename{Brooks1986} influenced this entire field of mobile robotics by his layered, behaviour based robot-control approach that acted significantly differently than the hierarchical approach.  At the same time, \citeasnoun{Braitenberg1984} described a set of experiments where increasingly complex vehicles are built from simple mechanical and electrical components. Around the same time and with similar principles, \citeasnoun{Reynolds1987} developed a distributed behavioural model for a bird in a flock that assumed that a flock is simply the result of the interactions among the individual birds (see Sec.  \ref{bg:mrs:srs}). Early research on multi-robot systems also include the concept of cellular robotic system \cite{Fukuda+1987,Beni1988} multi-robot motion planning \cite{Arai+1989,Premvuti+1990,Wang1989} and architectures for multi-robot cooperation \cite{Asama+1989}. Fig. \ref{fig:mataric-nerd-herd} and \ref{fig:kube-box-pushing} present us the two earliest MRS system in foraging and box-pushing task domains, developed by the pioneers in this filed, \citeasnoun{Mataric1994} and \citeasnoun{Kube1997} respectively.\\
% FIG: Multi-robot in emergency
\begin{figure}
\centering
\subfloat[A disaster site: buring oil rig]{\includegraphics[width=6cm, height=4cm]{./photos/burning-oil-rig-explosion-fire-photo11.eps}} 
\hspace{0.25cm}
\subfloat[An array of Seaglider underwater robots]{\includegraphics[width=6cm, height=4cm]{./photos/seaglider.eps}}
\caption{Example of MRS working at a disaster site (a) British Petroleum's oil rig sinks in the Gulf of Mexico after explosion. From http:///news.bbc.co.uk, reported on 22/04/2010. 
(b) IRobot's Seaglider fleet that was surveying oil spill in the Gulf of Mexico at depths of up to 1,000 meters. From http:///www.irobot.com, reported on 25/05/2010.}
\label{fig:bp-oil-disaster}
\end{figure}
%% FIG: Robocup
\begin{figure}
\centering
\subfloat[Robot teams at Robocup soccer]{\includegraphics[width=6cm, height=4cm]{./photos/robocup_bbc_41132545_robotbody2.eps}} 
\hspace{0.25cm}
\subfloat[Robot team at Robocup rescue]{\includegraphics[width=6cm, height=4cm]{./photos/rugbot_robocup_rescue_osaka.eps}}
\caption{Example of MRS at sports and rescue (a) Robot dogs playing at Robocup soccer. From http:///news.bbc.co.uk, reported on 11/05/2005. 
(b)Rugbot robot team was competing at Osaka's Robocup rescue league from \protect\citeasnoun{Birk+2006}.}
\label{fig:robocup}
\end{figure}
%%
From the beginning of the behaviour based paradigm, the biological inspirations influenced many cooperative robotics researchers to examine the social characteristics of insects and animals and to apply them to the design multi-robot systems \cite{Arkin1998}. The underlying basic idea is to use the simple local control rules of various social species, such as ants, bees, birds etc., to the development of similar behaviours in multi-robot systems. In multi-robot literature, there are many examples that demonstrate the ability of multi-robot teams to aggregate, flock, forage, follow trails etc. \cite{Bonabeau+1999,Mataric1994}. The dynamics of ecosystem, such as cooperation, has also been applied in multi-robot systems that has presented the emergent cooperation among team members \cite{Mcfarland1994,Martinoli+1996}. On the other hand, the study of competitive behaviours among animal and human societies has also been applied in multi-robot systems, such as that found in multi-robot soccer \cite{Asada+1999}. From Fig. \ref{fig:bp-oil-disaster} and \ref{fig:robocup} show us that the applications of MRS can be ranging from human disaster recovery to games and entertainment.\\
%%
As discussed above, there are several research groups who follow different approaches to handle multi-robot research problems. Based on the underlying philosophies and principles, We have classified them into two broad paradigms: 1) Traditional MRS and 2) Swarm-robotic system (SRS). Below we have highlighted them briefly.
% FIG: Multi-robot indoor and outdoor operations
\begin{figure}
\begin{minipage}[t]{0.48\linewidth}
\centering
\includegraphics[width=6cm, height=4cm, angle=0]
{./photos/centibot_demo3-11.eps}
\caption{ Hundreds of Centibots robots worked at indoor search, navigation and mapping tasks. From \protect\citeasnoun{Ortiz+2005}. }
\label{fig:centibots-indoor}
\end{minipage}
\hspace{0.5cm}
\begin{minipage}[t]{0.48\linewidth}
\centering
\includegraphics[width=6cm,height=4cm, angle=0]{./photos/pioneer_robots_610x455.eps}
\caption{Pioneer robots operating at outdoor uncertain environment of Georgia Tech. Mobile Robot Lab. From http://news.cnet.com, reported on 05/04/2010.}
\label{fig:pioneers-outdoor} % Give a unique label
\end{minipage}
\end{figure}
\subsubsection*{1. Traditional MRS paradigm}
As discussed above, unlike SRS, traditional MRS does not directly  take inspiration from BSS. Rather it follows the organizational, social, knowledge-based and multi-agent based approaches to solve problems of MRS. Explicit modelling of environment, tasks, robots can be the main features of these systems. According to \citeasnoun{Parker2008},  traditional MRS can be classified into two following categories.\\ 
\textbf{Organizational and social approaches: }
Organizational and social paradigms are typically based on organization theory derived from human systems that reflects the knowledge from sociology, economics, psychology and other related fields. To solve complex problems this paradigm usually follows the cooperative and collaborative forms of distributed intelligence. In multi-robot systems the example of this paradigm is found in two major formats: 1) the use of roles and value system and 2) market economics. In multi-robot applications under this paradigm, an easy division of labour is achieved by assigning roles depending on the skills and capabilities in individual team member. For example, in multi-robot soccer \cite{Stone+1999,Asada+1999} positions played by different robots are usually considered as defined roles. On the contrary, in market economics approach \cite{Gerkey+2002,Dias+2006} task allocation among multiple robots are done via market economics theory that enables the selection of robots for specific tasks according to their individual capabilities determined by a bidding process.\\
\textbf{Knowledge-based and multi-agent based approaches: }
This paradigm, commonly used for developing multi-agent systems, is knowledge-based, ontological and semantic paradigm. Here knowledge is defined as ontology and shared among robots/agents from disparate sources. It reduces the communication overhead by utilizing the shared vocabulary and semantics. Due to low bandwidth, limited power, limited computation and noise and uncertainty in sensing/actuation, the use of this approach is usually restricted in multi-robot systems. \\
\subsubsection*{2. SRS paradigm}
In bio-inspired, emergent swarms paradigm local sensing and local interaction forms the basis of collective behaviors of swarms of robots. Many researchers addressed the issues of local interaction, local communication (i.e., stigmergy) and other issues of this paradigm \cite{Mataric1995}, \cite{Kube+1993}. Today, this paradigm has been emerged as a sub-field of robotics called swarm robotics \cite{Sahin+2005}. This is a powerful paradigm for those applications that require performing shared common tasks over distributed workspace, redundancy or fault-tolerance without any complex interaction of entities. Some examples include flocking, herding, searching, chaining, formations, harvesting, deployment, coverage etc. \\
%%
Although our approximate classification of MRS includes most of the research directions it is very hard to specifically categorize all diverse researches on multi-robot systems. However, most of the researchers select a suitable paradigm to abstract the problem from an specific perspective with common fundamental challenges of MRS discussed in the later sections.
\subsection{MRS taxonomies}
\label{bg:mrs:taxonomies}
%%%%
The vast amount of research in MRS makes it necessary to use well-established classifications or taxonomies in order to specify and design the target MRS. In Section \ref{bg:mrs:paradigms} we see that the main-stream research in MRS can be classified into two distinct paradigms. However, these paradigms have certain assumptions, often unspecified or implicit, regarding the design of MRS hardware, software, communication and interaction etc. Thus, MRS taxonomies can be useful for many purposes, e.g. to avoid ambiguities in system specification by reducing the size and complexity of possible design spaces, and to use certain trade-off among various features for achieving overall system performance.\\
%%
While earlier MRS taxonomies, e.g. one proposed by \citeasnoun{Premvuti+1990}, \citeasnoun{Cao+1997} discuss very fundamental design issues of MRS, recent taxonomies e.g. proposed by \citeasnoun{Dudek+2002}, \citeasnoun{Gerkey+2004}, \citeasnoun{Balch2002}, \citeasnoun{Farinelli+2004} etc. gives us the detail design choices for making useful system specifications. We classify these recent taxonomies into two groups: 1) generalized taxonomies and 2) specialized taxonomies. We consider taxonomies of \citeasnoun{Dudek+2002} and \citeasnoun{Parker2008} as generalized taxonomies since they  can be used to specify almost all necessary features of a MRS.   On the other hand, specialized taxonomies provide MRS specification with respect to particular system features, e.g., the taxonomy of \citename{Balch2002} is only useful in a MRS with reinforcement learning, the taxonomy of \citename{Gerkey+2004} gives us the specification of tasks in a MRTA context. Other less common taxonomies e.g., one proposed by \citeasnoun{Farinelli+2004} or another proposed by \citeasnoun{Cao+1997} are centred around the coordination (weak or strong or none), communication (implicit or explicit or none), architecture (centralized or decentralized) etc. Here we briefly describe the major axes of leading taxonomies of MRS.
%%%
\subsubsection*{Dudek's generalized taxonomy of MRS}
\citeasnoun{Dudek+2002} provides seven main axes of MRS specification. We have regrouped them into the following three major areas.\\
\textbf{1. Collective or group size, composition and reconfigurability: }  A MRS can be formed by one, two, multiple or effectively infinite number of autonomous robots. Composition refers to the homogeneity of the group members. Robots can be identical in both form and function (hardware and software), homogeneous (consisting of same physical hardware) or physically heterogeneous. Collective reconfigurability refers to the rate at which robots can spatially re-position themselves. It can be completely static, coordinated or dynamically arranged.\\
\textbf{2. Communication range, topology and bandwidth : } The maximum distance between two robots, required for effective communication, can be zero (i.e. they can not communicate directly) or infinite (i.e. all robots can communicate to any other robot) or in-between these two. Communication topology determines the style of addressing target peers e.g., through broadcast messaging, individual addressing by name or address or, following tree-like hierarchy or redundant graphs. Communication bandwidth provides the measure of costs associated with communication. This can be no cost (i.e. infinite bandwidth) or high cost (i.e. limited or no bandwidth) or something in between these two extreme cases.\\
\textbf{3. Processing abilities: } This refers to the software architectures that can be used for controllers of the robots. General models are finite state automata, a push-down automata, neural-networks or Turing machines (most common assumption). 
%%%
\subsubsection*{Specialized taxonomies of Gerkey and Balch} 
With this general speculation of MRS we can specify the tasks or rewards in the system using any specialized taxonomies. For example, taxonomy of \citename{Gerkey+2004} can be helpful to understand the target domain of application of MRS. He defined three axes of possible tasks and robot capabilities of MRS:\\
\textbf{Single-task robots (ST) {\em vs.} multi-task robots (MT)}: ST (MT) means a robot can perform one (multiple) tasks at a time.\\
\textbf{Single-robot tasks (SR) {\em vs.} multi-robot tasks (MR)}: while under SR each task requires only one robot, under MR, multi-robots may be required.\\
\textbf{Instantaneous assignment (IA) {\em vs.} time-executed assignment (TA)}: While IA refers a situation when planning for future task-allocations is not possible, under TA planning is possible.\\
\citename{Balch2002} extended this taxonomies of tasks and applied to multi-robot learning cases. His taxonomy of reward include: source of reward (internal or external or both), rewarding time (immediate or delayed), continuity of reward (discrete or continuous), locality of reward (global or local or a combination of both) relation to performance (tied to performance or based-on intuitive state-value).
%%
\subsubsection*{Parker's taxonomy of interaction} 
In addition to the above two classes of taxonomies, we present here \possessivecite{Parker2008} notion of {\em interaction} in a robot team in four levels: 1) collective, 2) cooperative 3) collaborative and 4) coordinative. Although \citename{Parker2008} did not claim it to be a MRS taxonomy, we found this very much useful to describe the high-level relationships of individuals of the system. Moreover this removes the ambiguities among these overused terms and makes them precise for future use. 
%% Fig interaction
\begin{figure}
\centering
\includegraphics[width=9cm, angle=0]
{./images/ch2/parker-interaction-classification.eps}
%figure caption is below the figure
\caption{\small Categorization of types of interactions in MRS, reproduced from \protect\citeasnoun{Parker2008}.}
\label{fig:parker-interactio} % Give a unique label
\end{figure}
While analysing the role and application of distributed intelligence on MRS, \citename{Parker2008} presented an excellent classification of interactions of entities of MRS. As seen in Fig. \ref{fig:parker-interaction} she viewed the interactions along three different axes:
\begin{enumerate}
\item the types of goals of entities (either shared goal such as, cleaning a floor, or, individual goal)
\item whether entities have awareness of others on the team (either aware such as, in cooperative transport, or, unaware such as, in a typical foraging)
\item whether the action of one entity advances the goal of others (e.g., one robot's floor cleaning helps other robots not to clean that part of the floor)
\end{enumerate}
Based on this approximate observation Parker classified interactions into four categories:\\
% see Fig
\textbf{1. Collective interaction: }
Entities are not aware of others on the team, yet do share goals and their actions are beneficial to team-mates. Mostly, swarm-robotic work of many researchers follow this kind of interaction to perform biologically-relevant tasks, such as foraging, swarming, formation keeping and so forth.\\
\textbf{2. Cooperative interaction: }
Entities are aware of others on the team, they share goals and their actions are beneficial to their team-mates. This type of interaction is used to reason about team-mates capabilities multiple robots works together, usually in shared workspace, such as cleaning a work-site, pushing a box, performing search and rescue, extra-planetary exploration and so forth.\\ 
\textbf{3. Collaborative interaction: }
Having individual goals (and even individual capabilities), entities aware of their team-mates and their actions are beneficial to their team-mates. One example of this kind of interaction is a team of collaborative robots where each must reach a unique goal position by sharing sensory capabilities to all members such as illustrated as coalition formation in \cite{Parker+2006}.\\
\textbf{4. Coordinative interaction: }
Entities are aware of each other, but they do not share a common goal and their actions are not helpful to other team members. For example, in a common workspace, robots try to minimize interference by coordinating their actions as found in multi-robot path planning techniques, traffic control techniques and so on.\\
%%
Beyond this four most common types of interactions Parker also described another kind of interaction in adversarial domain where entities effectively work each other such as multi-robot soccer. Here entities have individual goals, they are aware of each other, but their actions have a negative affect on other robots' goals.\\
%%
In this dissertation, we use the taxonomy of \citeasnoun{Dudek+2002} for specifying our MRS (Sec. \ref{expt-tools:mrs-design}. The taxonomy of \citeasnoun{Parker2002} is used to analyse the dependence of MRS on various levels of interactions in Sec. \ref{bg:mrta}. 
%-----------------------------------------------------------------
\subsection{Traditional MRS}
\label{bg:mrs:mrs}
%%
%Challenges of MRS
MRS not only shares the problem of controlling a single robot but also it amplifies the problem to several orders or magnitude. Below we list a few major challenges of any MRS:\\
\textbf{Increased uncertainty about environment:}
When multiple robots work in a partially observable world, the environmental view becomes severely restricted due to both in terms of noisy sensor readings and frequent obstacle detections. Thus, in MRS, the uncertainty about the environment increased in may folds.\\
\textbf{Increased dynamic changes of the environment:}
Since many robots work in a shared environment, the dynamic movements and physical interferences among the robots becomes more frequent and robots are required to change their course of action more frequently.\\
\textbf{Decreased communication throughput:}
Interference in communication is inescapable for a team of robots. Since the typical bandwidth of a communication channel is fixed, adding more robots reduces the effective communication throughout and thus increased latency in robot-robot or robot-computer communications. If the robots are required to coordinate their action then the saturation of the communication channel affects the overall team-performance.\\
\textbf{Decreased real-time performance:}
In a functional MRS, autonomous mobile robots need to do some tasks in real-time, e.g. identifying their current poses ({\em localization}) to determine next motions, avoiding obstacles etc. However, when the number of robots increases the real-time performance can be poor due to the above factors.\\
\textbf{Increased sensor failures and break-downs:}
This is also common in a MRS that the real-time interaction of large number robots can decrease the life of their hardware as they become subject to more collisions and interferences. Thus overall reliability of the MRS can be decreased gradually.\\
%%
Despite the above big challenges, robotic researchers design and operate MRS successfully using a number of intelligent solutions since the last few decades. In the previous subsections we have seen how the researches are classified into distinctive paradigms and can be specified by precise taxonomies. Here we list a number of typical issues that any MRS may encounter from its inception to implementation.
\begin{itemize}
\item {\em Motion control}. How to use sensor values to produce real-time motions avoiding obstacles ?
\item {\em Localization}. How to find out the self-position in the world so that reaching to a specified target becomes possible ?
\item {\em Navigation/map-building}. How to integrate sensor values to build maps or representation of the environment for further exploration ?
\item {\em Task-selection}. How to plan/predict and select a particular high-level task (e.g. find a red object or picking up a stick) provided that a number of tasks present in the environment ? 
\item {\em Interaction and communication}.How to interact or communicate with other robots for cooperating, collaborating or coordination in doing tasks ?
\item {\em Adaptation/learning} How to remember things so that future robot actions or behaviours become improved ?
\end{itemize}
%%
Not all of the above issues are present in all MRS. Many MRSes do not use any form of navigation, or communication or learning and yet they do some useful tasks. However it is important to understand how these issues can be solved in a structured, modular and timely manner. Integrating the solutions of these issues and resolving the conflicts among them also appear to be the major functions of multi-robot control architectures. For example, conflicts occur if a resource is required by or, a unique single task is distributed to, more than one robot at any given time. Several resources such as bandwidth, space etc. may be needed by more than one robot \cite{Cao+1997}. The sharing of bandwidth among robots is a great problem in case of applications like multi-robot mapping \cite{Konolige+2003}. As shown in Fig. \ref{{fig:centibots-indoor}}, in large multi-robot team such as in Centibots system \cite{Ortiz+2005}, task interference and high bandwidth communication between 100s of robots appear as a significant research challenge.\\
%%
Whatever the principle characteristics of a MRS, e.g., homogeneity, coupling, communication methods etc., each MRS must address some degree to those problems. For example, usually every MRS adopts a control architecture under a specific paradigm to organize its hardware, software and communication system. Similarly every MRS address the issues of communication, localization, interaction in a way specific to the application and underlying design principles (or philosophies). In the following subsections, we have attempted to summarize the key MRS research issues that would influence the selection and implementation our research. In this initiative we have deliberately omitted the non-central or very specific issues, such as collaborative transport or reconfigurable MRS, that does not directly relate to our research.
\subsection*{Architecture and control}
\label{bg:mrs:arch}
%% FIG: three layer arch
\begin{figure}
\begin{center}
\includegraphics[width=5cm,height=4cm]{./images/ch2/three-layered.eps} % The printed column width is 8.4 cm.
\caption{A typical hybrid robot control architecture, adopted from \protect\citeasnoun{Mataric2007}.} 
\label{fig:three-layer-arch}
\end{center}
\end{figure}
%%
In MRS, two high-level control strategies are very common: 1) centralized and 2) decentralized or distributed. Under a specific control strategy, traditionally three basic system architectures are widely adopted: deliberative, reactive and hybrid \cite{Mataric2007,Arkin1998}. Deliberative systems based on central planning are well suited for the centralized control approach. The single controller makes a plan from its Sense-Plan-Act (SPA) loop by gathering the sensory information and each robot performs its part. Reactive systems are widely used in distributed control where each robot executes its own controller maintaining a tight coupling between the system's sensors and actuators, usually through a set of well-designed behaviours. Here, various group behaviour emerges from the interactions of individuals that communicate and cooperate when needed. Hybrid systems are usually the mixture of the two above approaches; where each robot can run its own hybrid controller with the help of a plan with necessary information from all other robots. Behaviour-based control architecture can also be considered as a separate category of distributed control architecture \cite{Mataric2007}, where each robot behaves according to a behaviour-based controller and can learn, adapt and contribute to improve and optimize the group-level behaviour.\\ 
Although most of the MRS control architectures share some common characteristics (such as distributed and behaviour-based control strategy) based on their difference of underlying design principles we have put them into three groups:
\begin{enumerate}
\item Behaviour-based classical architectures
\item Market-based architectures
\item Multi-agent based architectures
\end{enumerate}
Due to the overwhelming amount of literature on MRS architectures it is not possible to include all of them. However, below some representative key architectures, strictly designed for MRS, are described. 
%% FIG: ALLIANCE
\begin{figure}
\centering
\includegraphics[width=12cm, angle=0]
{./images/ch2/parker-alliance-arch.eps}
%figure caption is below the figure
\caption{ALLIANCE architecture. From \protect\citeasnoun{Parker1998}.}
\label{fig:parker-alliance-arch} % Give a unique label
\end{figure}
%% MOTIVATIONAL BH
\begin{figure}
\centering
\includegraphics[width=10cm, angle=0]
{./images/ch2/parker-motivational-bh.eps}
%figure caption is below the figure
\caption{Motivational behaviour in ALLIANCE. From \protect\citeasnoun{Parker1998}.}
\label{fig:parker-motivational-bh} % Give a unique label
\end{figure}
%%
\subsubsection*{Behaviour-based classical architectures}
The ALLIANCE architecture \cite{Parker1998} is one of the earliest behaviour-based fully distributed architectures (Fig. \ref{fig:alliance-arch}). This architecture has used the mathematically modelled behaviour sets and motivational system (Fig. \ref{fig:parker-motivational-bh}). The primary mechanism for task selection of a robot is to activate the motivational behaviour partly based on the estimates of other robots behaviour. This architecture was designed for heterogeneous teams of robots performing loosely coupled tasks with fault-tolerance and co-operative control strategy. Broadcast of local eligibility (BLE) \cite{Werger2001} is another behaviour-based architecture that uses port-attributed behaviour technique through broadcast communication method. It was demonstrated to perform coordinated tasks, such as multi-target observation tasks. Major differences between this two behaviour-based systems include the need in ALLIANCE for motivational behaviours to store information about other individual robots, the lack of uniform inter-behaviour communication, and ALLIANCE's monitoring of time other robots have spent performing behaviours rather than BLE's local eligibility estimates. Similar to the above two architectures, many other researchers proposed and implemented many variants of behaviour-based architectures. Some of them used the classic three layer (plan-sequence-execute) approach, \citeaffixed{Simmons+2002,Gat1997} {e.g.} used a {\em layered architecture} where each layer interact directly to coordinate actions at multiple levels of abstraction. 
%% FIG: foraging fsa
\begin{figure}
\begin{center}
\includegraphics[width=6cm,height=4cm]{./dia-files/foraging-fsa.eps} % The printed column width is 8.4 cm.
\caption{Finite state machine for foraging task, reproduced from \protect\citeasnoun{Arkin1998}.} 
\label{fig:foraging-fsa}
\end{center}
\end{figure}
%%
\subsubsection*{Market-based architectures}
Using the theory of marker economics and well-known Contract Net Protocol (CNP) \cite{Davis1988+}, these architectures solve the task-allocation problem by auction or bidding process. Major architectures following market-based approaches include MURDOCH \cite{Gerkey+2002}, M+ system \cite{Botelho+1999}, first-piece auction \cite{Zlot+2002}, dynamic role assignments \cite{Chaimowicz2002} among others.
%% 
\subsubsection*{Multi-agent based architectures}
Some MRS architectures are influenced by multi-agent systems (MAS). For example, CHARON is a hierarchical behaviour-based architecture that rely on the notion of agents and modes. Similarly CAMPOUT is another distributed behaviour-based architecture that provide high-level functionality by making use of basic low-level behaviours in downward task decomposition of a multi-agent planner. It is comprised of five different architectural mechanisms including, behaviour representation, behaviour composition, behaviour coordination, group coordination and communication behaviours.\\
%% Our approach
In this dissertation, we closely follow the behaviour-based hybrid architecture with an event-driven mechanism for activating behaviours. Our architecture and robot controllers are illustrated in Sec. \ref{expt-tools:arch}.
%%%%%---------------------------------------
\subsection*{Interaction and learning}
\label{bg:mrs:learn}
According to the Oxford Dictionary of English the term interaction means reciprocal action or influence. In MRS research, such as in \cite{Mataric1994}, interaction is referred to as mutual influence on behaviour. Following this definition, it is obvious that objects in the world do not interact with agents, although they may affect on their behaviour. The presence of an object affects the agent, but the agent does not affects the object since objects, by definition, do not behave, only agents do. However many other researchers acknowledge that interactions of robots with their environment (as found in stigmergic communication) have a great impact on their behaviours. Therefore, we adopt the broad meaning of interaction that is reciprocal action or influence among robots and their environment. From the above review of MRS system architecture, task allocation and communication, it is obvious that interaction among robots and their environment is the core of the dynamics of MRS. Without this interaction, it can not be a functioning MRS. We have split our discussions of MRS interaction into multi-robot learning and communication. Multi-robot learning is described here and communication in MRS is discussed in Sec. \ref{bg:mrs-comm}.
%%---------------------------------------------
%\subsubsection*{Learning}
A great deal of research on multi-robot learning has been carried out since the inception of MRS \cite{Mataric+2001,Yang+2004,Parker1995}. Learning, identified as the ability to acquire new knowledge or skills and improve one's performance, is useful in MRS due to the necessity of robots to know about itself, its environment and other team-members \cite{Mataric2007}. Learning can improve performance since robot controllers are not perfect by design and robots are required to work in an uncertain environment that all possible states or actions can not be predicted in advance. Besides learning a new skill or piece of knowledge it is also important to forget learned things that are no longer needed or correct as well as, to make room for new things to be learned and stored in a finite memory space of a robot. \\
%%
Several learning techniques are available in robotics domain, such as reinforce or unsupervised learning, supervised learning and learning by imitation. Although reinforce learning, or learning based on environmental or peer feedback, is a good option for MRS, it has been found that in large teams the ability to lean in this way is restricted due to large continuous state and action space \cite{Yang+2004}. Several other learning techniques are also available to explore in MRS domain including Markov models, Q-learning, fuzzy logic, neural nets, game theory, probabilistic or Bayesian theory among others \cite{Mataric2007}. 
%%
\subsection*{Localization and exploration}
\label{bg:mrs:loc}
Mobile robot systems highly rely on precise localization for performing their autonomous activities in indoor or outdoor. Localization is the determination of exact pose (position and orientation) with respect to some relative or absolute coordinate system. This can be done by using proprioceptive sensors that monitor motion of a robot or exteroceptive sensors that provide information of world representation, such as global positioning system (GPS) or indoor navigation system (INS). Many other methods are also available, such as landmark recognition, cooperative positioning and other visual methods \cite{Arkin+2002}.\\
Localization issue of MRS also invites researchers to examine specific areas like exploration and map generation. In exploration problem, robots need to minimize the time needed to explore the given area. Many researchers uses various kinds of exploration algorithms for solving this NP-hard problem, such as line-of-sight constrained exploration algorithm , collaborative multi-robot exploration \cite{Burgard+2000}. In mapping problem, mostly inaccurate localization information from teams of robots are accumulated and combined to generate a map by various techniques, such as probabilistic approaches \cite{Thurn+2000}.
% FIG: Multi-robot indoor and outdoor operations
\begin{figure}
\centering
\subfloat[Swarmbot SRS]{\includegraphics[width=6cm, height=4cm]{./photos/jamesMcLurkin_SwarmbotAndSwarm.eps}} 
\hspace{0.25cm}
\subfloat[Robots detecting boundary]{\includegraphics[width=6cm, height=4cm]{./photos/james_fig-BoundaryDetection-Robots.eps}}
\caption{ (a) A Swarmbot and the Swarms 
 (b) Swarmbots detecting boundary using distributed algorithms, from \protect\citeasnoun{Mclurkin+2009}.}
\label{fig:swarmbot-boundary-detection}
\end{figure}
%%
%%
\subsection*{Example areas of MRS research}
\label{bg:mrs:eg}
Researches on MRS have been targeted for numerous application domains that all can not be listed here altogether. Rather than listing all of areas explored by researchers, here we have included a few major areas that have received highest attention in the MRS research community. 
%\subsubsection*{Object Transport, Mining, Space/Military}
Cooperative transport of large objects (that one robot is unable to handle) by multi-robots was investigated by many researchers such as, following a formal model of cooperative transport in ants \cite{Kube+1993}, box-pushing by six-legged robots \cite{Mataric+1995}. Another kind of object transport problem include clustering objects into piles e.g., \cite{Beckers+1994}, collecting waste or trash e.g., \cite{Parker1994}, sorting coloured objects e.g., \cite{Melhuish+1998}, constructing a building site collectively \cite{Wawerla+2002} and so on. It has also been observed that multi-robot teams as micro or mini machines are helpful to improve the control and efficiency of mining and its processing operations \cite{Dunbar+2002}. Many researchers address MRS research issues under the requirements of a military or space application. Behaviour-based formation control \cite{Balch+1998}, landmine detection \cite{Franklin+1995}, multiple planetary rovers for various missions \cite{Huntsberger2004} and so forth, all are the examples of this areas.\\
Although research on MRS has been becoming more mature since last decades, it is not easy to find many industrial applications relying on multiple autonomous mobile robots. We have found one exceptional application developed by Kiva Systems in the domain of multi-robot material handling in warehouses \cite{Wurman+2008}. Along with this, Sec. {bg:mrs-industry} reviews some possible applications of MRS in automation industry.
%%-----------------------------------------------
\subsection{Swarm robotic systems (SRS)}
\label{bg:mrs:srs}
\textbf{Background of SRS:} When many traditional MRSes showed serious  scalability failures, the necessity of adopting a  new paradigm becomes obvious \cite{Lerman+2006}. Researchers of traditional MRS approach realized  that  executing their time and processing intensive algorithms in large number of real robots ($\geq 10$) could be a nightmare. Adding more robots almost exponentially amplified their inherent  problems e.g. physical and communication interferences with an ever-ending hunger of more CPU powers. Traditional approach became infeasible for some other reasons too. For example, one of the early inspirations for constructing a MRS is to own 10-20 cheaper and simple robots is preferable to manage 1-2 expensive and complex robots. But under traditional approach, when the robot-team size increases,  robots require more sensory and on-board processing power for maintaining large internal and environmental sates. This goes against the original spirit of MRS. Moreover, many traditional MRSes, that relied upon a centralized system for communication, localization and other necessary supports,  often failed under heavy-stress conditions of large MRS.\\
%% 
In early and mid-90s, many researchers found that applying biological principles of swarm intelligence effectively removed and reduced many bottlenecks of traditional MRS.  In 1995, Maja Mataric published that complex group behaviours could be produced by the appropriate combinations of more simple ``basis behaviours'' \cite{Mataric1995}. The idea of using simple biological behaviours, such as avoidance and following, to create complex flocking and foraging behaviours inspired many other researchers to search solution for controlling large MRS in this direction.  The early research of Reynold (1987) guided many others to apply biological principles of self-organization aka swam intelligence (SI) in MRS.  The term {\em swarm intelligence} was first coined by Gerado Beni \cite{Beni2005} in late 1980s and it represents the effort of designing algorithms or distributed problem solving devices inspired by the collective behaviours of social insect colonies \cite{Bonabeau+1999}. The idea of using simple robots to create complex patterns or structures was also studied under {\em cellular robotics} \cite{Fukuda+1987}. During recent years the term swarm robotics (SR) emerged as an application of swarm intelligence to multi-robot systems with the emphasis on physical embodiment of entities and realistic interactions among the entities and between the entities and their environment. These systems of swarm robotics or minimalist robotics \footnote{although both SR and minimalist robotics follow similar approaches for solving similar problems, the latter does not explicitly relate its origin to SI} can be represented by a common term swarm robotic system (SRS).\\
%%
\textbf{Advantages of SRS:} The simplicity of SRS approach inspires robotic researches to build MRS with cheap robotic hardware, to equip them with simple controllers and control them through local informations without, creating any explicit model of the environment or using any sophisticated central control. The redundancy of robots, parallelism in their task-executions and an overall distributed control architecture, support addition or removal (or failure) of any robots in run-time. Moreover the control algorithms are now decoupled from the model of the environment or other robots. Thus this system now becomes more robust, fault-tolerant, scalable and adaptive to unknown and dynamic environment.\\
%%
\textbf{Distinct features of SRS:}
In order to distinguish swarm robotics from other branches of robotics such as collective robotics, distributed robotics, robot colonies and so forth, \citeasnoun{Sahin+2005} proposed a formal definition and a set of criteria for swarm robotics research . 
\begin{quotation}
``Swarm robotics is the study of how large number of relatively simple physically embodied agents can be designed such that a desired collective behaviour emerges from the local interactions among agents and between the agents and the environment.'' 
\end{quotation}
And the notable criteria of swarm robotics research are listed as follows.
\begin{description}
\item[Autonomous robots]
that exclude the sensor networks and may include metamorphic robotic system without having no centralized planning and control element.
\item[Large number of robots,]
usually $\geq$ 10 robots, or at least having provision for scalability if the group size is below this number.
\item[Mostly homogeneous groups of robots]
that typically exclude the multi-robot soccer teams having heterogeneous robots.
\item[Relatively incapable of inefficient robots]
that is the task complexity enforces either cooperation among robots or increased performance or robustness without putting no restriction on individual robot's hardware/software complexity.
\item[Robots with local sensing and communication capabilities]
that does not use global coordination channel to coordinate among themselves, rather enforces distributed coordination.
\end{description}
%%%
% FIG: Swarmbots demo
\begin{figure}
\begin{minipage}[t]{0.48\linewidth}
\centering
\includegraphics[width=6cm, height=4cm, angle=0]
{./photos/swarm-bots-crossing-canal.eps}
\caption{ A group of Swarmbots are crossing rough terrain, from \protect\citeasnoun{Mondada+2004}.}
\label{fig:srs:rough-terrain} % Give a unique label
\end{minipage}
\hspace{0.5cm}
\begin{minipage}[t]{0.48\linewidth}
\centering
\includegraphics[width=6cm,height=4cm, angle=0]{./photos/swarmbots-pulling-child-725972.eps}
\caption{ A Swarmbot of 18 s-bots pulls a child to a safe location from \protect\citeasnoun{Mondada+2004}.}
\label{fig:srs:pulling-child} % Give a unique label
\end{minipage}
\end{figure}
%%
\textbf{Classification and application of SRS:}
SRS can broadly be classified into two distinct classes. The first class consists of simple and relatively inexpensive mobile robots that are fully autonomous and can work in isolation. For example, E-puck robots \cite{Cianci+2004} fall under this class. Other class of robots include self-reconfigurable \cite{Fukuda+1987} and self-assembling robots which can be built by coupling several identical units together, e.g. a robotic snake. In this dissertation, we have limited our focus to the former class of robots alone.\\
The potential applications of SRS include: spatially distributed tasks (e.g., environment monitoring), dangerous tasks (e.g., robotic de-miner), tasks that scale-up or scale-down over time, and tasks that require redundancy \cite{Sahin+2005}. Fig. \ref{fig:srs:rough-terrain} shows amazing abilities of Swarmbot that is crossing a rough terrain. Fig \ref{fig:srs:pulling-child} shows another interesting demonstrations of SRS that a team of 18 Swarmbots pulls a child to a safe place.\\
%% FIG: BOIDS
\begin{figure}
\centering
\subfloat[Separation]{\includegraphics[width=3cm, height=2cm]{./photos/boid-separation.eps}} 
\hspace{0.25cm}
\subfloat[Alignment]{\includegraphics[width=3cm, height=2cm]{./photos/boid-alignment.eps}}
\hspace{0.25cm}
\subfloat[Cohesion]{\includegraphics[width=3cm, height=2cm]{./photos/boid-cohesion.eps}}
\caption{ Reynold's simulated flocking of boids \protect\citeasnoun{Reynolds1987}. (a) Separation: steer to avoid crowding local flockmates, (b) Alignment: steer towards the average heading of local flockmates and (c) Cohesion: steer to move toward the average position of local flock-mates, from http://www.red3d.com/cwr/boids/, last seen on 01/06/2010.}
\label{fig:boid-rules}
\end{figure}
%%
\begin{figure}
\centering
\subfloat[A flock of birds]{\includegraphics[width=6cm, height=4cm]{./photos/bird-flocking.eps}} 
\hspace{0.25cm}
\subfloat[Simulated boids]{\includegraphics[width=6cm, height=4cm]{./photos/boid-flocking.eps}}
\caption{ (a) Real flocks of birds, from http://www.travelblog.org, and (b) simulated flock of birds, from http://www.red3d.com/cwr/boids, last seen on 01/06/2010.}
\label{fig:birds-flocking}
\end{figure}
%%
\textbf{Modelling Swarms:} 
Modelling both the behaviour of an individual robot controllers and system-level collective behaviours have become an interesting issue in SRS. This is due to the fact that, in this kind of system, collective behaviours, e.g. flocking or aggregation, can significantly be changed by of just changing one or few parameters of individual robot controllers. Thus modelling SRS can give us an early insight about a target system before its implementation before doing any time-consuming simulation or expensive real-experiment. This is very important since in real-experiments or in simulations the parameter space usually becomes huge and it could be infeasible to run significant number of iterations of these real or simulated trials. From mathematical models, it can be easy to find out which parameters of interest dominate the system performance or which resources are more important to do a particular type of tasks. In some cases, it can also predict the system performance to some degree. \\
%%
A plenty of approaches for modelling SRS exists in MRS literature \cite{Gazi+2006}. Most common modelling approaches include: behaviour-based approach, probabilistic models, potential-function based approach, asynchronous swarm models, multi-agent based swarm models. Behaviour-based approach can be found in the study of \citeasnoun{Reynolds1987} who has simulated the flocking behaviours of birds. Fig. \ref{fig:boid-rules} illustrates how three simple rules can produce a coordinated flocking motion (Fig. \ref{fig:flocking-birds}(b)). From biological observation of flocking birds  it is obvious that collective behaviours can be generated through a local control and interaction rules. Similar to this study, many other researchers tries to apply behaviours based local strategy for formation control (e.g. \cite{Balch+1998}), aggregation \citeaffixed{Mataric1995}{e.g. }, sorting (e.g. \cite{Melhuish+1998}), foraging (e.g.Liu+2007 ), cooperative transport (e.g. \cite{Kube1997}) etc.\\
%%
%%
\begin{figure}
\centering
\subfloat[Robots pulling sticks]{\includegraphics[width=6cm, height=4cm]{./photos/AGASSONOUN_S_SV.eps}} 
\hspace{0.5cm}\\
\subfloat[Probabilistic FSM]{\includegraphics[width=6cm, height=4cm]{./photos/agassonoun-pfsm.eps}}
\caption{ (a) Stick-pulling experiments by a group of Khephera robots equipped with gripper turrets and (b) Probabilistic finite state machie (PFSM) of robot controllers. From \protect\citeasnoun{Agassounon+2004}.}
\label{fig:stick-pulling-expt}
\end{figure}
%%
In 1999, \citeasnoun{Martinoli1999} proposed probabilistic modelling of SRS. This probabilistic approach often has two major aspects: controller design through probabilistic finite state machine (PFSM) (e.g. see Fig. \ref{fig:stick-pulling-expt}) and automated parameter adaptation through genetic algorithm. This approach has been adopted by many recent SRS research \citeaffixed{Agassounon+2004, Lerman+2005, Liu+2007b}{e.g.}. \\
%%
SRS models can be classified into many distinct classes. Firstly, they can be classified into: spatial and non-spatial models. {\em Spatial models} keep track of the agent's trajectories and perhaps use that spatial distribution. However, in {\em non-spatial models} it is assumed that agents occupy independent, random positions at consecutive time-steps. SRS models can also be classified into embodied and non-embodied models. Here  {\em non-embodied models} consider agents  as points and their physical characteristics are ignored, whereas {\em embodied models} take the physical characteristics or interferences of agents into account. Thus spatial models with embodied agents are chosen in typical simulations.\\
%%
As another distinct classification, SRS models can be classified into two major groups: 1) microscopic models and 2) macroscopic models. {\em Macroscopic models} focus on individual robots and state transitions of each robot controller are updated based on the stochastically approximated robot-robot and robot-environment interactions. The probabilities of state transitions are calculated from simple geometric configurations and with few trial experiments. Here, no group-level sensing or actuation is taken into account. On the other hand, {\em macroscopic models} captures the snapshot-by-snapshot pictures of whole SRS. Each snapshot presents the total number of robots in a given state. Fig. \ref{fig:stick-pulling-expt}(b) shows the probabilistic macroscopic model where $S_{x}$ denotes a particular state $x$ and $N_{s}$ denotes the number of robots under state $S_{x}$. Here $\tau$ represents the corresponding probability density function derived from a set of Master-Equations \cite{Agassounon+2004}.
%%
Despite a lot of attractive benefits of SRS modelling approaches, they have some notable shortcomings. Formal models of SRS, particularly probabilistic models, may not be attractive or useful for many reasons. Firstly, constructing a functional model takes time due to the need for accurate calibration of necessary parameters (which also involves running several real-experiments or simulations). 
Secondly, most of probabilistic models rely upon some assumptions, e.g. coverage of the area should be spatially uniform or the system should follow Markov properties i .e. the robot's future state depends only on its current state and how much time it has spent in that state. These can not be satisfied in many practical applications e.g. open space exploration or robots with memory. As task complexity increases the parameters space becomes large and searching good combinations of parameters by some means, e.g. genetic algorithms, becomes more complex.\\
%%
Similar to traditional MRS, SRS faces great challenges in enabling localization, communication and  interaction in group-level. For example, without the presence of any centralized localization module, such as GPS or indoor navigation system (INS), it is not easy to localize precisely and locally the position of a robot with respect to other robots or environment. Recently researchers investigated  these issues and reported some  novel solutions, e.g. \citeasnoun{Spears+2006} presented a novel technique based on trilateration for localization of swarm robots using ultrasonic and RF transceivers. \citeasnoun{Schmickl+2006} reported hop-count and bio-inspired strategies for collective perception or how a swarm robot can join multiple instances of individual perception to get a global picture. \cite{Rothermich+05} presented a collaborative localization algorithm using landmark based localization technique. Because of the similarity of the problems in both traditional MRS and SRS, we have presented the issues of  task-allocation and communication of both types of MRS in Sec. \ref{bg:mrta} and Sec. \ref{mrs-comm} respectively.\\
%%
In this dissertation, we closely follow the SRS approach of designing robot group and solving related issues. We have followed the behaviour-based approach for designing robot controllers (Chapter \ref{expt-tools}) that rely on GPS-like overhead camera-based solution to fulfil their localization needs. We have modelled our real robotic system considering their spatial, embodied and microscopic properties. No macroscopic simulation or analysis of the robot group has been conducted. Our autonomous robot group meets all the criteria of a SRS mentioned by \cite{Sahin+2005} except the communication issue. We do not restrict our robots always to do local communication alone for solving their MRTA problem. Our solutions to MRTA and multi-robot communication problems have been presented in Chapter \ref{afm} and Chapter \ref{local-comm}.
 %%%%%%%%%%%%%%%%%%%%%%%%%%%%%%%%%%%%%%%%%%%%%%%%%%%%%%%%%%%%%%%%%%%
\section{Multi-robot task allocation (MRTA)}
\label{bg:mrta}
Since 90s multi-robot task allocation (MRTA) is a common research challenge that tries to define the preferred mapping of robots to tasks in order to optimize some objective functions \cite{Gerkey+2004}. Many control MRS architectures   have been solely designed to address this task-allocation issue from different perspectives . Based-on the high-level design of those solutions, here we have classified researches on MRTA into two major categories: 1) predefined or intentional task-allocation and 2) Bio-inspired self-organized task-allocation. Fig. \ref{fig:mrta-classification} illustrates our classification. This classification has been adopted from \citeasnoun{Shen+2001}, but our sub-categories are different since \citename{Shen+2001} proposed the classification for multi-agent system alone that does not take the spatiality and embodiment of agents into account. Under each of our sub-categories of MRTA there are many inter-connected issues that need to be addressed by the system designer. We have put major issues into three major axes: 1) organization of task-allocation, 2) degree of communication and 3) degree of interaction. In the following subsections, we have discussed these two categories and their key issues with some example MRSes and their comparisons.
%
\begin{figure}
\centering
\includegraphics[width=12cm, angle=0]
{./dia-files/ta-categories.eps}
\caption{\small Classification of MRTA}
\label{fig:mrta-classification} % Give a unique label
\end{figure}
%%---------------------------------------
\subsection{Predefined task-allocation}
In most of the traditional MRS, task allocation is done using a well-defined models of tasks and environments. Here it is assumed that the system designer has the precise knowledge about tasks, robot-capabilities etc. Many flavours of the type of task-allocation can be found in the literature. Below we briefly discussed a few well-acknowledged works. \\
%% FIG: ALLIANCE motivational bh
\begin{figure}
\centering
\includegraphics[width=10cm, angle=0]
{./dia-files/alliance-motivational-bh.eps}
%figure caption is below the figure
\caption{Motivational behaviour in ALLIANCE. From \protect\citeasnoun{Parker1998}.}
\label{fig:alliance-motivation} % Give a unique label
\end{figure}
%
\textbf{Knowledge-based and multi-agent based approaches:}\\
In this approach knowledge-based techniques are used to represent tasks, robot capabilities etc. One of the early well-known MRTA architecture of this category was Parker's ALLIANCE \cite{Parker1998} in which each robot models the ability of team-members to perform tasks by observing their current task performances and collecting relevant task quality statistics e.g. time to complete tasks. Robots use these models to select a task that benefit  the group as a whole.
As shown in Fig. \ref{fig:parker-alliance-arch}, ALLIANCE architecture, implemented in each robot, delineates several mathematically modelled behaviour sets, each of which corresponds to some high-level task-achieving function. The concept of motivational behaviour was introduced  as a mechanism to choose among these high level behaviours. As shown in Fig. \ref{fig:alliance-motivation}, each motivational behaviour had a number of inputs and one output. The output, i.e. the activation level corresponding behavioural set, was activated once a predefined threshold was passed. In the same time, all other behavioural sets became inhibited for allowing that selected behavioural set to complete its task. The input of the behavioural sets was ranged from sensory reading to robot-robot broadcast communication of state-information. Internal behaviours e.g. {\em impatience}  and {\em acquiescence} were also used to evaluate the motivation of a robot to select a high-level behaviour set.  Impatience encouraged individual robots to perform a task that was not selected by any other robot of the team and a robot's acquiescence of a task was increased when a robot selected to perform it.  Moreover, robots had the ability to override the inhibitory signal from another robot if a task assigned to other robot was not being completed to a desired level (e.g. when a robot stalled). In case of unsatisfactory self-progress, i.e. not doing any significant progress in a task, robots were able to switch from that task to a different one. This system was deployed on a mock hazardous waste clean up and achieved fault-tolerant distributed task performance of the robot team. Later on, L-ALLIANCE, an extension of this system was also developed to enable robots  to learn from the observations of a set of task-performance metrics. \\
%%
Similar to  ALLIANCE, multi-agent based task allocation also  use both centralized and decentralized approaches for allocating tasks among  its  peers. \citeasnoun{Shen+2001} presented a detailed categorization where in a multi-agent system task allocation can be done by using various agents ranging from a central supervising agent or a few mediator agents to  all independent agents. In case of centralized systems, the central supervisor (or a group of mediators)  must have the necessary system knowledge, e.g. the capabilities and availabilities of all agents, descriptions of tasks, This system gives a well coordinated,consistent and optimized task-allocation but reduces  the reliability and fault-tolerance and scalability of the system. On the other hand, in case of distributed task-allocation, each agent can directly assign a task {\em directly} to another agent provided that all of them have precise knowledge about others. This approach is very expensive for large number of agents since it requires all agents to have huge processing power and communication bandwidth which is not practical. Alternatively, agents can know only a few agents and {\em delegate} a task to these known peers so that a suitable agent can be found who has sufficient capabilities and free resources to do this task. This task-allocation by delegation also suffers from poor performance  due to the use of time-consuming search algorithms. This approach also assumes the availability of high communication bandwidth which is not true in large systems.\\

\textbf{Market-based approaches:}\\
As a feasible alternative to the above common multi-agent based task-allocation techniques, many researchers have been following the market-based bidding approach. \citeasnoun{Dias+2006} provided a survey on it. Originated from Contract-Net Protocol, market-based approach can be implemented as a centralized auctioning system or as a combination of {\em a few auctioneers -- all bidders} or, independently {\em all auctioneers -- all bidders}. For example, in a completely distributed system, when a robot needs to solve a problem or task for which it does not have necessary expertise or resources, it broadcasts a task-announcement message, often with  a expiry time of that message. Robots that received the message and can solve that task return a bid message. The initiating robot or {\em  manager} selects one (or more) bidder, called as {\em contractor}, and offers the opportunity to complete the task. The choice of contractor is done after selection by the manager and mutual agreement that maximizes the individual profits. High-level communication protocol is necessary to define several types of messages with structured content. In centralized market-based approach there is only one manager that can be an external supervising agent or  one of the robots. While market-based approach consume more resources it usually produces more efficient task-allocations. Anonymous robots can be selected for tasks and these can be different in each bidding cycle.\\

\textbf{Role or value-based task-allocation:}\\
In this type of task-allocation each role assumes several specific tasks and each robot selects roles that best suit their individual skills and capabilities \cite{Chaimowicz2002}. In this case, robots are typically heterogeneous, each one having variety of different sensing, computation and effector capabilities. Here robot-robot or robot-environment interactions are designed as a part of the organization. In multi-robot soccer (e.g. \cite{Stone+1999}), positions played by different  robots are often defined as roles, e.g. goal-keeper, left/right defender, left/right forwarder etc. The robot, best suited and in closest proximity to available roles/positions, selects to perform that role.\\

\textbf{Control-theoretic approaches:}\\
In this type of task-allocation systems, a model of the system is usually developed that converts the task specification into an objective function to be optimized. This model typically  uses  the rigid  body dynamics of the robots assuming the masses and other parameters well-known. Control laws of individual robots are derived either by analytically or by run-time iterations. Unlike most other approaches where task-allocation problem is taken as discrete, control-theoretic approaches can produce continuous  solutions. The formalisms of these systems allow system designer to check the system's controllability, stability and related other properties.  These systems typically use some degree of centralization, e.g. choosing a leader robot.  Example of control-theoretic approach include: multi-robot formation control \cite{Belta+2004}, multi-robot box-pushing \cite{Pereira+2003 }  etc.
%%
\subsection{Bio-inspired self-organized task-allocation}
Task performance in self-organized approaches relies on the collective behaviours resulted from the local interactions of many simple and mostly homogeneous (or interchangeable) agents. Robots choose their tasks independently and asynchronously using the principles of self-organization, e.g. positive and negative feedback mechanisms, randomness (see Sec. \ref{bg:self-reg} for details). Moreover interaction among individuals and their environment are modulated by the stigmergic, local and broadcast communications (more in Sec. \ref{bg:mrs-comm}).  Among many variants of self-organized task-allocation, most common type is threshold-based task-allocation \cite{Bonabeau+1999}. Here, a robot's decision to select a particular task depends largely on its perception of stimulus or demand for a task and its corresponding response threshold for that task. Below, we describe most common forms of threshold-based task-allocation:  deterministic response-threshold and probabilistic response-threshold techniques. Both of them can use the fixed values of response-thresholds or they can adapt their response-thresholds over time based on a suitable learning or adaptation mechanisms.\\ 
\begin{equation}
\label{eqn:fixed-response-th1}
\sigma (r,e) = \frac{1}{d(r,e)}
\end{equation}
\begin{equation}
\label{eqn:fixed-response-th2}
\theta_{e} = \frac{1}{\mid D_{r} \mid}
\end{equation}
\textbf{Deterministic response-threshold:}\\
In this approach, each robot has a fixed or deterministic activation threshold for each task that needs to be performed. It continuously perceives or monitors the stimulus of all tasks that reflect the relative urgencies of tasks. When a particular task-stimuli exceeds a predefined  threshold the robot starts working on that task and gradually decreases it stimuli. When the task-stimuli falls below the fixed threshold it becomes inactive for that task. This type of approach has been effectively applied in foraging \citeaffixed{Krieger+2000, Liu+2007}{e.g.}, aggregation \citeaffixed{Agassounon+2002}{e.g.}. This fixed response-threshold can initially be same for all robots (e.g. in \cite{Jones+200}), or they can be different according robot capabilities or configuration of the system \citeaffixed{Krieger+2000}{e.g.}.\\ 
From a simple example of this approach in event-handling domain \cite{kalra+2007},  we can see how task-stimulus can be encoded in mathematical equations. For example, Eq. \ref{eqn:fixed-response-th1} encodes the stimuli of robot $r$ for task-urgency perception event $e$, ($\sigma (r,e)$)  as inversely proportional to the distance between the robot and the event occurring place. Eq. \ref{eqn:fixed-response-th2} gives the threshold value $\theta_{e}$ (based on a predefined distance value $D_{r}$) under which robot selects this particular task or event. \\
%Adaptive Response-Threshold:\\
Unlike maintaining a fixed response-threshold, adaptive response-threshold model  changes or adapts the threshold over time. Response-threshold decreases often due to performance of a task and this enables a robot  to select that particular task more frequently or in other words it learns about that task. Examples of this type of task-allocation can be found in \citeaffixed{Bonabeau+1999,Agassounon+2002}{e.g.}. \\
\begin{equation}
\label{eqn:probl-response-th}
p_{e} = \frac{\sigma (r,e)^n}{\sigma (r,e)^n + \theta_{e}^n}
\end{equation}
\textbf{Probabilistic Response-Threshold:}\\
Unlike deterministic approach, where robots always respond to a task-stimuli that has a largest stimulus above the threshold,  probabilistic approach offers a selection process based-on a probability distribution. For example, in probabilistic response, robots can respond to an event $e$ with probability $p_{e}$ as in Eq. \ref{eqn:probl-response-th} where $\theta_{e}$ is the threshold and $n$ is the non-linearity of the response. Thus, robots  always have small nonzero probabilities  for all tasks.\\
%%
In this dissertation we have closely followed this approach with an on-line adaptation mechanism which has been outlined in Chapter \ref{afm}.                                                                                                                                                                                                                                                                                                                                                                                                                                                                                                                                                                                                                                                                                                                            
%------------------------------------------------------------------
\subsection{Key issues in MRTA}
\begin{figure}
\centering
\includegraphics[width=12cm, angle=0]
{./dia-files/mrta-lines.eps}
%figure caption is below the figure
\caption{ Three major axes of complexities in MRTA}
\label{fig:mrta-complexities} % Give a unique label
\end{figure}
%%
From the vast amount of literature on MRTA, we can easily infer the level of complexities exist in MRTA. In fact researchers generally agree that the MRTA is a {\em NP-hard} problem where optimal solutions can not be found quickly for large problems \citeaffixed{Gerkey+2004,Parker2008}{e.g. See }. But why do we find so many variants of MRTA solutions ? In order to answer this question, first we need to look into the contexts from where the solutions are made. Most predefined task-allocation solutions are proposed within the context of a known or controlled environment where the modelling of tasks, robots, environments etc. becomes feasible. Note that here tasks can be arbitrarily complex that often require relatively higher sensory and processing abilities of robots. Robot-team can be consists of homogeneous or heterogeneous individuals, having different capabilities based on the variations in their hardware, software etc. But the uncertainty of the environment is assumed to be minimum. On the other hand, bio-inspired self-organized MRTA solutions are free from extensive modelling of environment, tasks or robot capabilities. Most of the existing research considers very simple form of one global task e.g. foraging, area cleaning, box-pushing etc. This is due to the fact that major focus of this approach is limited mainly to design individual robot controllers in such a way that a few simple  or {\em specific} tasks can be accomplished. More research is needed to verify the capabilities of self-organized approach in doing multiple complex tasks. At this moment, the bottom line remains as ``select simple robots for simple tasks (self-organized approach) and complex robots for complex tasks (predefined approach)''.\\
%%
Both of the above task-allocation approaches expose their relative strengths and weaknesses when they are put under real-time experiments with variable number or robots and dynamic tasks. In an arbitrary event handling domain, \citeasnoun{kalra+2007}  compared between self-organized and predefined market-based task-allocation,  where they found that predefined  task-allocation was more efficient when the information was accurate, but threshold-based  approach offered similar quality of allocation at a fraction of cost  under noisy environment.  \citeasnoun{Gerkey+2003} presented a comparative study of  the complexity and optimality of key architectures, e.g.  ALLIANCE \cite{Parker1998}, BLE \cite{Werger2001}, M+ \cite{Botelho+1999}, MURDOCH \cite{Gerkey+2002}, First piece auctions \cite{Zlot+2002} and Dynamic role assignment \cite{Chaimowicz2002}, all of them relied upon predefined task-allocation methods. The computational and communication requirements were expressed in terms of number of robots and tasks. Although this study does not explicitly measures the scalability of those key architectures, it clearly shows us that many predefined task-allocation solutions will fail to scale well in challenging environments  when the number of  robots (and tasks) will increase, under the given limited overall communication bandwidth and processing power of individual robots. In this regard, self-organized task-allocation methods are advantageous as they can provide fully distributed, scalable and robust MRTA solutions through redundancy and parallelism in task-executions. Moreover, the interaction and communication requirements of robots can also be kept under a minimum limit.  Thus  we can say that for large MRS, self-organized task-allocation methods  can potentially be selected, if a system designer can divide his complex tasks into simple pieces that can be carried out by multiple simple robots in parallel with limited communication and interaction needs.\\
%%
In order to characterize both predefined and self-organized approaches in terms of their deployment, we propose three distinct axes: 1) organization of task-allocation (X), 2) degree of interaction (Y) and 2) degree of communication (Z). Fig. \ref{fig:mrta-complexities} depicts these axes with a reference point $O$. These axes can be used to measure the complexities involved in various kinds of MRTA problems and the design of their solutions. In this  figure, X axis represents the number of active nodes that provides the task-allocation to the group. For example, in any predefined  task-allocation approach, we can use one external centralized entity or an one of the robots (aka leader) to manage the task-allocation. This can optimize the MRTA solution globally, but is subject to single point of failure. This is also not feasible for large systems where the number of robots and the descriptions of tasks are large. In many predefined methods, e.g. in market-based systems,  multiple nodes can act as mediators or task-allocators that we have discussed before. Under predefined task-allocation approach and for a small number of robots, fully distributed task-allocation can also be possible where all nodes can act as independent task-allocator, e.g. as found in ALLIANCE architecture. Most of the self-organized task-allocation methods are fully distributed, i.e. they allocate their tasks independently without the help of a centralized entity. However, they might be dependent on external entities for getting status or descriptions of tasks. Recent studies on swarm-robotic flocking by \cite{Celikkanat+2008} shows that a swarm can be guided to a target by a few informed individuals (or leaders) while  maintaining the self-organizing principles for task-allocation. Task-allocation of a swarm of robots  just by one central entity may be rare since one of the major spirits of swarm robotic system is to become fully distributed.\\
%%
The Y axis of Fig. \ref{fig:mrta-complexities},   corresponds to the level of robot-robot interaction present in the system. As we have mentioned before in Sec. \ref{bg:mrs:taxonomies}, interaction can be classified into various levels: collective, cooperative, coordinative and collaborative. The presence of interaction can be due to the nature of the problem, e.g. a cooperation is necessary in co-operative transport tasks. Alternately, this interaction can be a design choice where interaction can improves the performance of the team, e.g. cooperation in cleaning a work-site is not necessary but it can help to improve the  efficiency of this task. Y axis can also be used to refer to the degree of coupling present in the system \cite{Mataric2007}.  In case of collective interaction, robots merely co-exist, i.e.  they may not be aware of each other except treating others as obstacles. Many other MRS are loosely-coupled where robots can indirectly infer some states of the environment from their team-mates' actions.  But in many cases, e.g. in  co-operative transport, robots not only recognize others as their team-mates, but also they coordinate their actions. Thus they form a  tightly coupled system. This level of interaction and coupling also gives us the information about potential side-effects of failure of an individual robot. Tightly coupled systems where high degrees of interactions among the robots are present suffer from performance loss if some of the robots removed from the system.\\
%%
The Z axis of Fig. \ref{fig:mrta-complexities} represents the communication overhead of the system. This can be the result of the interactions  of robots under a given task-allocation method. As we have discussed before various task-allocation methods rely upon variable degrees of robot-robot communications.  On the other hand, the communication capabilities of individual robots can limit (or expand) the level of interaction can be made  in the given group. Thus in one way, considering the interaction requirements of a MRTA problem, the system designer can  select suitable communication strategies that both minimizes the communication overhead and maximizes performance of the group. And in other way, the communication capabilities of robots can guide a system designer to design interaction rules of robot teams, e.g. the specification of robot's on-board camera  can determine the degree of possible visual interactions among robots. The suitable trade-offs between these two axes: communication and interaction can give us a balanced design of our MRTA method.\\
%%
Finding suitable communication strategies under the adaptive response-threshold task-allocation method is the central issue of this thesis. So we have focused to examine the benefits of traversing along the various axes of Fig. \ref{fig:mrta-complexities}. In this dissertation, we are interested on two distinct lines: 1) distributed task-allocation, with no direct robot-robot interaction and communication, say line $OX_{n}$ ($n$ being the number of robots)  and 2) distributed task-allocation, with no direct robot-robot interaction, but varying degrees of local communications, say line $X_{n}Z_{l}$  ($Z_{l}$ being a local broadcast communication strategy that involves $l$ number of peers in communication). Our MRTA experiments along $OX_{n}$ and $X_{n}Z_{l}$ can be found in Chapter \ref{afm} and Chapter \ref{local-comm}. The issue of multi-robot communication is presented in more detail  in next section.
%=======================================================================
\section{Communication in MRS}
\label{bg:mrs-comm}
Communication is extremely important for any high-level interaction (e.g. cooperative, coordinative) among a multi-robot team \cite{Arkin1998}. This is not a prerequisite for the group to be functioning, but often useful component of MRS \cite{Mataric2007}. The characteristics of  communication in MRS can be presented in terms of these issues: rationale of communication ({\em why to communicate}), message content ({\em what to communicate}), communication modalities (how to communicate),  and target recipients ({\em with whom to communicate}) . Below we have described why communication is important, how to design this and usually how this is archived in MRS and related other issues.
%-----------
\subsection{Rationale of communication}
Researchers generally agree that communication in MRS usually provides several major benefits, such as:
\begin{description}
\item[Information exchange improves perception:]
Robots can exchange potential information (as discussed below) based on their spatial position and knowledge of past events. This, in turn, leads to improve perception over a distributed region without directly sensing it.
\item[Synchronization of actions:]
In order to perform (or stop performing) certain tasks simultaneously or in a particular order, robots need to communicate, or signal, to each other. 
\item[Enabling interactions and negotiations:]
Communication is not strictly necessary for collective interactons of a robot team. From a set of multi-robot communication experiments \citeasnoun{Arkin1998}  concluded that for certain classes of tasks, explicit communication is not strictly necessary. However,  higher level interaction, such as cooperative task-performance or coordinative actions are almost always designed with communication support built-into the robots (see Sec. \ref{bg:mrs:taxonomies}
for definition of various kinds of interactions). Thus communication can help a lot to  influence each-other in a team that, in turn, enables robots to interact and negotiate their actions effectively.
\end{description}
\begin{figure}
\begin{minipage}[t]{0.48\linewidth}
\centering
\includegraphics[width=6cm, height=4cm, angle=0]
{./photos/s-bots-comm-evolve-300x214.eps}
\caption{A team of s-bots communicating by light signals. From http://lis.epfl.ch, last seen on 01/06/2010.}
\label{fig:light-comm-robot}
\end{minipage}
\hspace{0.5cm}
\begin{minipage}[t]{0.48\linewidth}
\centering
\includegraphics[width=6cm,height=4cm, angle=0]{./photos/robots_cs_utk_edu_balajee.eps}
\caption{A fleet of robots relying on camera (vision) for search operation. From http://www.cs.utk.edu, last seen on 01/06/2010.}
\label{fig:self-org-agent} % Give a unique label
\end{minipage}
\end{figure}
%----------------- 
\subsection{Information content}
Although communication provides several benefits for team-work it is costly to provide communication support in terms of hardware, firmware as well as run-time energy spent in communication. So robotic researchers carefully  minimize the necessary information content in communications by using suitable communication protocols and high-level abstractions. For example in foraging, grazing and consuming experiments  \citeasnoun{Balch+1994} introduced state and goal communications. In state communication, a single bit is transmitted indicating the current state of robot (e.g. 0 transmitted when robot was in {\em Wander} state and 1 transmitted when robot was in {\em Acquire} or {\em Retrieve} states). In case of goal communication, the localition of task was also transmitted. From similar instances of these experiemnts below is a brief summary of information contents.\\
Individual state: ID number, battery level, task-perfomance statistics, e.g. number of tasks done\\
Goal: Location of target task or all tasks discovered\\
Task-related state: The amount of task completed, number of other robots present there etc.\\
Environmental state: Free and blocked paths, level of interference found, any urgent event or dangerous changes found in the environment.\\
Intentions: Detail plan for doing a task, sequences of selected actions etc.\\
Since a MRS can be comprised of robots of various computation and communication capabilities, it communication content can vary greatly based on their individual communication modules and channel capacities.
%------------------------------------------
\subsection{Communication modalities}
\begin{figure}
\centering
\subfloat[ E-puck robots with table-lamps]{\includegraphics[width=6cm, height=4cm]{./photos/distributed_table_lamp_triangle.eps}} 
\hspace{0.25cm}
\subfloat[Sniffing Khepera III]{\includegraphics[width=6cm, height=4cm]{./photos/odor_loc_khepera3odorprototype300x169.eps}}
\caption{ (a) A fleet of mobile ``lighting" robots moving on a large table, such that the swarm of robots form a distributed table light and (b) Distributed odor source localization by Khephera robot equipped with volatile organic compound sensor and an anemometer (wind sensor). From http://http://disal.epfl.ch, last seen 01/06/2010.}
\label{fig:epfl-disal}
\end{figure}
%%
%%
Robotic researchers typically use robot's on-board wireless radio, infrared (IR), vision and sound hardware modules for robot-robot  and robot-host communication.  The reduction in price of wireless radio hardware chips e.g. wifi (ad-hoc WLAN 802.11 network) or bluetooth makes it possible to use wireless widely. In-expensive infrared communication module is also typically built into almost all mobile robots due to its low-cost and suitability for ambient light and obstacle detection.  IR can also be used for  low bandwidth communication in short-range, e.g. keen-recognition. Most robots can also produce basic sound waves and detect it with suitable configuration of on-board microphones. Although spech-recognition is not still commonly found in mobile robots,  detection of pre-recorded sound waves can be feasible.  Most of the mobile robots come with a series of LEDs, and tiny camera that can emit light signals and detect it with camera. Fig. \ref{fig:mrs-comm-LED} shows the robot-robot communication  through the red and green coloured LEDs. Many robots can also detect blobs of colours and can recognize peers of other objects through the use of a color-coded markers.  Fig. \ref{fig:mrs-comm-marker} shows a team of robots with colour-coded markers attached on it that can be detected by other robots. Although a lot of researches have been carried out to design robot skin and tactile communication system, we do not know any instance of tactile communication system in MRS. Similarly \citeasnoun{Lochmatter+2007} showed a limited success in odor-source localization, a form of detetcing chemical signals, we do not know about any full-fledged chemical communication in any MRS. Some researchers also tried to establish communication among robots relying mainly on vision \cite{Kuniyoshi1994}.
%-----------------------------
\subsection{Communication strategies}
\begin{figure}
\centering
\includegraphics[width=9cm, angle=0]
{./dia-files/mrs-comm-strategies.eps}
%figure caption is below the figure
\caption{\small Common communication strategies found in MRS}
\label{fig:mrs-comm-strategies} % Give a unique label
\end{figure}
whatever be the communication need or modalities in a MRS, suitable strategies are required to disseminate information in a timely manner to a target audience that will maximize the effective task-completion minimizing delays and conflicts. In order to discuss the complexities of communication strategies we have selected three independent scales: organization, expressiveness and range of communication, by which we can measure the level of complexities and classify a MRS according to  its communication characteristics. Fig. \ref{fig:mrs-comm-strategies} outlines these scales and they are described in the following paragraphs.
%[Organization]
\subsection*{Centralized and decentralized communications}
Similar to the organization of MRTA,  communication in a MRS can be organized using an external/internal central entity (e.g. a server PC, or a leader robot) or, a few leader robots, or by using decentralized or local schemes where every robot has the option to communicate with every other robot of the team. From a recent study of multi-robot flocking \citeasnoun{Celikkanat+2008} shows that a mobile robot flock can be steered toward a desired direction through externally guiding some of its members, i.e. the flock relies on multiple leaders or information repositories.
%[How]
Communication in a MRS can also be characterized its expressiveness or the degree of explicity. In one extreme it can be fully implicit, e.g. stigmergic, or on the other end, it can be fully explicit where communication is done by a rich vocabulary of symbols and meanings. Researchers generally tend to stay in either end based on the  robotic architecture and task-allocation mechasim used.  However, both of these approaches can be tied together under any specific application. Below  these two major categories are described.
\subsection*{Explicit or direct communication}
This is also known as intentional communication. This is done purposefully by usually using suitable modality e.g. wireless radio, sound, LEDs. Because explicit communication is costly in terms of both hardware and software, robotic researchers always put extra attention to design such a system by analysing strict requirements such as communication necessity, range, content, reliability of communication channel (loose of message) etc.
\subsection*{Implicit or indirect communication} 
This is also known as indirect stigmergic communication. This is a powerful way of communication where individuals leave information in the environment. This method was adopted from the social insect behaviour, such as stigmergy of ants (leaving of small amount of pheromone or chemicals behind while moving in a trail). 

\subsection*{Local and Broadcast communications}
The target recipient selection or determining the communication range (or sometimes called radius of communication) is an interesting issue of research. Researchers generally tries to maximize the information gain by using larger range. However, transmission power and communication interference among robots play a major role to limit this range. In this case, the minimum nummber of peers is 1 (or just the closest neighbor) and it can vary based on the selection of a suitable targeting strategy and available bandwidth. Based on the number of recipients of message, the communication strategy is termed differently. Such as,\\
Global broadcast communication: where all other robot can receive the message.\\
Local broadcast communication: where a few robots in local neighborhood can receive the message.\\
Publish-subscribe communication: where only a selected (previously subscribed) number of robots can receive the message.\\
Peer-to-peer communication: where only a single peer robot can receive the message.\\

\subsection{Key issues in MRS communication}
In multi-robot communication researchers have identified several issues. Some of the major issues are discussed here.
Determination of local neighborhood:\\
Most swarm-robotic researchers, who use algorithms based on local-neighborhood of communication,  face this problem of defining the range of local neighborhood. \citeasnoun{Agah+1995} presented that  larger communication range is not always optimum for some types of tasks e.g. exploration where a large number of recipient robots  decreased the performance of exploration task. \citeasnoun{Yoshida+2000} provided a design of optimal communication range of homogenous robots based on their saptial and temporal analyses of information defusion  within the context of cooperative tasks in a manufacturing shop-floor. Spatial design  tried to minimize the time for information transmission and temporal design tried to minimize the information announcing time to avoid excessive information diffusion. Eq. \ref{egn:yoshida-range} descibes their optimal range $\chi_{optimal}$ as a function of information acquisition capacity of robots ($c$) and the probability of information output of a robot ($p$). Here $c$ is an integer representing the upper-limit of number of robots that can be the target recipents at any time without the loss of information and  $\chi_{optimal}$ gives the average number of robots within the output range.
\begin{equation}
\chi_{optimal} = \frac{sqrt [c] {c!}}{}
\end{equation}


Kin Recognition\\
Kin recognition refers to the ability of a robot to recognize immediate family members by implicit or explicit communication or sensing. In case of MRS, this can be as simple as identifying other robots from objects and environment or as finding team-mates in a robotic soccer. This is an useful ability that helps interaction, such as cooperation among team members. \\
Representation of Languages\\
In case of effective communication several researchers also focused on representation of languages and grounding of these languages in physical world.\\
Fault-tolerance, Reliability and Adaptation\\
Since every communication channel is not free from noise and corruption of messages significant attention has been also given to manage these no communication situations, such as by setting up and maintaining communication network, managing reliability and adaptation rules when there is no communication link available. In terms of guaranteeing communication, researchers also tried to find ways for a deadlock free communication methods \cite{Arkin1998}, such as signboard communication method \cite{Wang1989}.

%%%%%%%%%%%%%%%%%%%%%
\section{Application of MRS in automation industry}
\label{bg:mrs-industry}
In order to examine the feasibility of our approach of emergent DoL, we have selected the distributed automated manufacturing application domain. Most of the research in this area is inspired by intelligent multi-agent technology \cite{Shen+2001}. A few other researchers also tried to apply the concepts of biological self-organization  \cite{Ueda2006,Lazinica+2007}. In this section we have reviewed these concepts and technologies mainly focusing on physical embodiment of agents, i.e., the use of multiple mobile robots or automated guided vehicles (AGV).
% The use of static robots (or manipulators) or scheduling of static processes or resources are excluded from this review.
%
\subsection{Multi-agent based approaches}
Since early 80s researchers have been applying agent technology to manufacturing enterprise integration,  manufacturing process planning, scheduling and shop floor control, material handing and so on\cite{Shen+2006}. An agent as a software system that communicates and cooperates with other software systems to solve a complex problem that is beyond the capability of each individual software system \cite{Shen+2001}. Most notable capabilities of agents are autonomous, adaptive, cooperative and proactive. There exists many different extensions of agent-based technologies such as Holonic Manufacturing System (HMS) \cite{Bussmann+2004}. A holon is an autonomous and cooperative unit of manufacturing system for transporting, transforming, sorting and/or validating information and physical objects. 
%% FIG: Kiva systems
\begin{figure}
\centering
\includegraphics[width=10cm, angle=0]
{./photos/Kiva-Systems.eps}
\caption{ KIVA systems revolutionary material handling system}
\label{fig:kiva-systems}
\end{figure}
%
Agent based technologies have addressed many of the problems encountered by the traditional centralized method. It can respond to the dynamic changes and disturbances through local decision making. The autonomy of individual resource agents and loosely coupled network architecture provide better fault-tolerance. The inter agent distributed communication and negotiation also eliminate the problem of having a single point of failure of a centralized system. These facilitate a manufacturing enterprise to reduce their response time to market demands in globally competitive market. Despite having so many advantages, agent-based systems are still not widely implemented in the manufacturing industry comparing to the other similar technologies, such as distributed objects and web-based technologies due to the lack of integration of this systems with other existing systems particularly real-time data collection system, e.g., RFID (radio frequency identification), SCADA (supervisory control and data acquisition) etc \cite{Shen+2006}. Another barrier is the increased cost of investment in exchange of some additional flexibility and throughput \cite{Schild+2007}. 
%
\subsection{Biology-inspired approaches}
The insightful findings from biological studies on insects and organisms have directly inspired many researchers to solve problems of manufacturing industries in a biological way. These can be categorized into two groups: one that allocates task with explicit potential fields (PF) and another that allocate tasks without specifying any PF. Below we have discussed both types of BMS.
\subsubsection*{Explicit potential filed based BMS}
The biological evidences of the existence of PF between a task and an individual worker such as, a flower and a bee, a food source and an ant,  inspired some researchers to conceptualize the assigning of artificial PF between two manufacturing resources. For example, PF is assumed between a machine that produce a material part and a worker robot (or AGV) that manipulates the raw materials and finished products. \cite{Ueda2006} conceptualized this PF as the attractive and repulsive forces based on machine capabilities and product requirements. Task allocation is carried out based on the local matching between machine capabilities and product requirements. Each machine generates an attractive field based on its capabilities and each robot can sense and matches this attractive filed according to the requirements of a product. PF is a function of distance between entities. Here, self-organization of manufacturing resources occurred by the process of matching the machine capabilities and requirements of moving robots.  Through computer simulations and a prototype implementation of a line-less car chassis welding \cite{Ueda2006} found that this system was providing higher productivity and cost-effectiveness of manufacturing process where frequent reconfiguration of factory layout was a major requirement. This approach, was also extended and implemented in a supply chain network and in a simulated ant system model where individual agents were rational agents who selected tasks based on their imposed limitations on sensing. 
\subsubsection*{BMS without explicit potential fields}
Several other researchers did not express the above PF for task allocation among manufacturing resources explicitly, rather they stressed on task selection of robots based on the task-capability broadcasts from the machines to the worker robots. In case of \cite{Lazinica+2007}, task capabilities are expressed as the required time to finish a task in a specific machine. They used assigned priority levels to accomplish the assembly of different kinds of products in the computer simulation of their bionic manufacturing system. In another earlier computer simulated implementation of swarm robotic material handing of a manufacturing work-cell, \cite{Doty+1993} pointed out several pitfalls of such a BMS system, such as dead-lock in manufacturing in inter-dependant product parts, unpredictability of task completion, energy wastage of robots wandering for tasks etc. Although most of these problems remain unsolved researchers are still exploring the concepts BMS in order to achieve a higher level of robustness, flexibility and operational efficiency in a highly decentralized, flexible, and  globally competent next generation automated manufacturing system.