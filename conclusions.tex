\chapter{Conclusions}
%%%%%%%%%%%%%%%%%%%%%
The benefits of deploying a large number of robots in dynamic multi-tasking environment can not be fully realized without incorporating an effective communication and sensing strategy in the MRTA solution under question.  This study has focused on comparing two bio-inspired  communication and sensing strategies in producing self-regulated MRTA by our interdisciplinary model of DOL, AFM. Under the first strategy, i.e. GSNC, AFM can produce the desired self-regulated MRTA among a group of 8 and 16 robots. This gives us the evidence for applying AFM successfully to solve the MRTA issue of any complex multi-tasking environment like a manufacturing shop-floor. Under second strategy  i.e. LSLC, AFM can also produce the desired self-regulated MRTA for a large group of robots with different communication and sensing ranges. 

From our comparative results we can conclude that for large group of robots,  degradation in  task-performance and task-specialization of robots are likely to occur  under GSNC strategy that relies upon a centralized communication system. Thus GSNC strategy can give us better performance when the number of tasks and robots are relatively small. This confirms us the hypothesis made by some biologists that self-regulated DOL among small group of individuals can happen without any significant amount of local communications and interactions. However, our findings suggest that task-specialization can still be beneficial among the individuals of a small group which contradicts the claim that small groups only posses the generalist workers, but not the specialists.

On the other hand, LSLC strategy is more suitable for large group of individuals that are likely to be unable to perform global sensing and global communications with all individuals of the group. The design of communication and sensing range is still remained as a critical research issue. However, our results suggest that the idea of maximizing information gain is not appropriate under a stochastic task-allocation process, as more information causes more task-switching behaviours that lowers the level of task-specialization of the group. This might not be the case under a deterministic task-allocation scheme where more information leads to better and optimum allocations, but that is limited to a small group of individuals. Nevertheless, despite having the limited communication and sensing range, LSLC strategy helps to produce comparatively better task-allocation with increased task-specialization and significantly reduced motions or savings in energy (battery power).

Our study has experienced all the major challenges of implementing a large MRS within limited time and resource constraints. From our experiences we can say that, by following flexible and open-source hardware and software platforms, it is possible to construct a large MRS for conducting real-robotic experiments within limited time-frame. This is a good news for those who wants to test their models using a large real-robotic system without relying on simulation only. However in this case, one needs to adopt the an appropriate test-driven bottom-up development approach instead of following the traditional top-down ``model - simulate -  export-to-real-robots'' approach. 
%===================================================================
\section{Contributions}
The main contributions of this dissertation are as follows:
\begin{itemize}
\item \textbf{Introduction of AFM as a basic mechanism of self-regulated MRTA.} As we have seen in Chapter \ref{afm} how AFM can be  interpreted in a multi-robot manufacturing shop-floor scenario.  In this process we have separated out the generic part of the model such as providing learning and forgetting rate from the domain-specific part e.g. defining task-urgency increase and decrease rate. Similar interpretations can also be made in other task-domains. Our study can serve as a guide to any such future application of AFM.
\item \textbf{Validation of the model through experiments with reasonably large number of real robots i.e. 16 e-puck robots. }It can be very difficult to realize an abstract model without having a practical example implementation of it. Mere computer simulation of a model can be considered enough. In this dissertation, we have provided a rich implementation of AFM with detail algorithms and links to source code. This implementation can be replicated in solving many real-world task and resource allocation problem, e.g. vehicle and resource allocation in automation industries.
\item \textbf{Comparisons of the performances of two bio-inspired communication and sensing strategies in achieving self-regulated MRTA.} Since AFM has wrapped the communication and sensing of agents  in a system-wide continuous flow of information one need to instantiate the instance of communication system that can provide an infrastructure for this information flow. In this study we have tested two well-known bio-inspired  communication and sensing strategies: GSNC and LSLC and measured their impact on self-regulated MRTA of our MRS. This comparison enhances our understanding of information exchange strategies commonly found in various social systems.
\item \textbf{Development of a flexible multi-robot control architecture using D-Bus inter-process communication technology.} In this dissertation, we have laid the foundation of a new kind of robot control architecture that uses the state-of-the-art D-Bus IPC protocol for decoupling the communication among various software components of a MRS. This greatly helps us to reduce the bottlenecks in developing many heterogeneous software components using different programming languages and software frameworks.
\item \textbf{Classification of MRTA solutions based on three major axes: organization of task-allocation, interaction and communication.} This concise classification helps us to remove the ambiguities among various MRTA solutions found in the literature.  It is very common to be confused with terminologies used by the MRTA solutions found in existing body of literature. Many researchers claim that their MRTA solutions are decentralized. This types of claim does not explain fully in what aspects those solutions are decentralized. Are the communications among robots are decentralized ? But this question does not answer if all robots has their own task-allocation capabilities built-in their controllers. Thus these ambiguities motivates us to explain the existing MRTA solutions in terms of three major axes as mentioned above. They can enable us to precisely  explain the characteristics of most of the existing MRTA solutions and locate them within a tractable space. For example, by answering the question: \textit{``How many task-allocator does a MRTA solution propose ?''} we can be sure that if this solution is fully distributed, centralized or semi-centralized. By using similar types of probes we can find the characteristics of communication and interaction provided by a MRTA solution.
\end{itemize}
%================================================================
\section{Future work}
In this study, the interaction among robots for task-completion, under our manufacturing shop-floor scenario, has been kept as simple as possible. This is done mainly for minimizing the time and complexity of real-robotic implementation. However, in most of the instances of biological self-regulated DOL, e.g. among polybia wasps that follow LSLC strategies for DOL, several inter-dependent tasks are often performed concurrently with a high degree of interaction among individuals. Thus this study can be extended in co-operative task performance where different individuals with variety of task-skills need to interact with each other directly. \\
%% 
Our validation of AFM has been limited to a group of homogeneous robots that has initially same level of task-sensitization. Moreover no dynamic task has been introduced during the run-time of our experiments. Due to the stochastic task-allocation process, we always were able to see the  variation in task-urgencies. But AFM can be applied to a more challenging environment with suddenly appearing (and disappearing) dynamic tasks that can resemble to the real-world use-cases where any task can be pre-empted by other tasks. Moreover, some more research can be done in order to figure out how to optimize the initial experimental parameters, such as learning and forgetting rate,  task-urgency increase and decrease rate etc. without estimating them purely based-on intuitions. \\
%%
In terms of implementation of our LSLC strategy, P2P communication among robot-controllers occurred in host-PC. We did not validated the use of communication range  using real-robotic hardware. But real implementation of communication range can easily be achieved by using suitable on-board communication module, e.g. wifi or IR, with  relatively powerful robots.\\ 
%%
In our experiments, we have selected two fixed communication ranges with the approximation of LSLC strategies. Some researchers have addressed the issue of deciding the optimum communication range, \citeaffixed{Yoshida+2000}{e.g.} without any direct connection with self-regulated MRTA. However, much research is required to find optimum communication range as a property of self-regulation of individuals. For example, in nature, urgency of tasks directly influences the communication behaviours of individuals, e.g. honey-bees modulate their dancing behaviours based on the quality of flower source.\\
%%
Finally in terms of real-world implementation, we can consider to put AFM for solving many challenging industrial automation tasks. For example, AFM can solve automated material handing tasks in ware-houses and factories, or real-manufacturing tasks with suitable hardware module. In this way, our interdisciplinary model can show us its effectiveness to overcome the existing challenges in the industry.