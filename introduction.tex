\chapter{Introduction}
\section{Rationale}
Robotic researchers generally agree that multiple robots can perform complex and distributed tasks more conveniently. Multi-robot systems (MRS) can provide improved performance, fault-tolerance and robustness through parallelism and redundancy \cite{Arkin1998,Parker+2006,Mataric2007}. However, in order to get potential benefits of MRS in any application domain, we need to solve a common research challenge i.e., \textit{multi-robot task allocation} (MRTA) \cite{Gerkey+2004}. MRTA can also be called as \textit{division of labour} (DoL) analogous to DoL in insect and human societies. It is generally identified as the question of assigning tasks in an appropriate time to the appropriate robots considering the changes of the environment and/or the performance of other team members. This is a NP-hard optimal assignment problem where optimum solutions can not be found quickly for large complex problems \cite{Gerkey+2003,Parker2008}. The complexities of MRTA arise from the fact that there is no central planner or coordinator for task assignments and the robots are limited to sense, to communicate and to interact locally. None of them has the complete knowledge of the past, present or future actions of other robots. Moreover, they don't have the complete view of the world state. The computational and communication bandwidth requirements also restrict the solution quality of the problem \cite{Lerman+2006}. 

Existing researchers have approached the MRTA or DoL issue in many different ways. Traditionally task allocation in a multi-agent systems is divided into two major categories: 1) Predefined (off-line) and 2) Emergent (real-time) task-allocation \cite{Shen+2001}. However predefined DoL approach fails to scale well as the number of tasks and robots becomes large, e,g,, more than 10 \cite{Lerman+2006}. On the other hand emergent DoL approach relies on the emergent group behaviours e.g., \cite{Kube+1993}, such as emergent cooperation \cite{Lerman+2006}, adaptation rules \cite{Liu+2007} etc. They are more robust  and scalable to large team size. However most of researchers have complained that emergent DoL approach is difficult to design, to analyse formally and to implement in real robots. The solutions from these systems are also sub-optimal. It is also difficult to predict exact behaviours of robots and overall system performance.

Within the context of the Engineering and Physical Sciences Research Council (EPSRC) project, ``Defying the Rules: How Self-regulatory Systems Work'', we have proposed to solve the above mentioned MRTA problem in a new way \cite{Arcaute+2008}. Our approach is inspired from the studies of emergence of DoL in both  biological insect societies and human social systems.  Biological studies show that a large number of animal as well as human social systems grow, evolve and generally continue functioning well by the virtue of their individual self-regulatory DoL systems. The amazing abilities of biological organisms to change, to respond to unpredictable environments, and to adapt over time lead them to sustain life through biological functions such as self-recognition, self-recovery, self-growth etc. It is interesting to note that in animal societies DoL has been accomplished years after years without a central authority or an explicit planning and coordinating element. Indirect communication such as stigmergy is rather used to exchange information among individuals \cite{Camazine+2001}. The decentralized self-growth of Internet and its bottom-up interactions of millions of users around the globe present us similar evidences of DoL in human social systems \cite{Andriani+2004}. These interactions of individuals happen in the absence of or in parallel with strict hierarchy. Moreover from the study of sociology e.g., \cite{Sayer+1992}, cybernetics e.g., \cite{Beer1981}, strategic management e.g., \cite{Kogut2000} and related other disciplines we have found that decentralized self-regulated systems exist in nature and in man-made systems which can grow and achieve viability over time. 

From the above mentioned multi-disciplinary studies of various complex systems, we believe that a set of generic rules can be derived for the emergence of DoL in MRS. Primarily these rules should deal with the issue of deriving local control laws for facilitating the emergence of global team behaviour. 

We have intended to apply our research outcome in the area of automated manufacturing (AM). AM faces all the existing challenges of traditional centralized and sequential manufacturing processes such as, insufficiently flexible to respond production styles and high-mix low volume production environments \cite{Shen+2006}. We believe that, by using our emergent DoL approach, AM industries can overcome many of these challenging issues, such as flexibility to change the manufacturing plant layouts on-the-fly, adaptability for high variation in product styles, quantities, and active manufacturing resources e.g., robots, AGVs etc.

%\section{Aims and Objectives }
The aim of this study is to identify generic rules that promote emergent self-regulation leading to the emergence of division of labour in MRS. 
This manifests the following objectives of this study:
\begin{itemize}
\item To model self-regulatory dynamic task allocation behaviours of robots considering both biological and human social metaphors.
\item To set-up an experimental framework of MRS and to find out the evidences of the emergence of division of labour through simulation and experiments. 
\end{itemize}
%%%%%%%%%%%%%%%%%%%%%%%%
\section{Contributions}
\begin{itemize}
\item To model self-regulatory dynamic task allocation behaviours of robots considering both biological and human social metaphors.
\item To set-up an experimental framework of MRS and to find out the evidences of the emergence of division of labour through simulation and experiments. 
\end{itemize}
%%%%%%%%%%%%%%%%%%%%%%%%
\section{Thesis Outline}
This report has been organized as follows.\\ 
Chapter 1 rationalizes the proposed study. \\
Chapter 2 and 3 reviews the general and robotic related literature in details.\\
Chapter 4 describes the completed research work of  this year. It also critically reviewed the completed  literature reviews.\\
Chapter 5 includes anticipated contribution to knowledge along with future research and publication plans.\\
Appendix reports my completed research skill trainings in this year.



