\addcontentsline{toc}{chapter}{Abstract}
\begin{Large}
%\noindent
Abstract\\
\end{Large}
\newline
To deploy a large group of autonomous robots in dynamic multi-tasking environments, a suitable multi-robot task-allocation (MRTA) solution is required. This must be scalable to variable number of robots and tasks. Recent studies show that biology-inspired self-organized approaches can effectively handle task-allocation in large multi-robot systems. However most existing MRTA approaches have overlooked the role of different communication and sensing strategies found in self-regulated biological societies.

This dissertation proposes to solve the MRTA problem using a set of previously published generic rules for division of labour derived from the observation of ant, human and robotic social systems. The concrete form of these rules, the \textit{attractive filed model} (AFM), provides sufficient abstraction to local communication and sensing which is uncommon in existing MRTA solutions. 

This dissertation validates the effectiveness of AFM to address MRTA  using two bio-inspired communication and sensing strategies: ``global sensing - no communication'' and ``local sensing - local communication''. The former is realized using a centralized communication system and the latter is emulated under a peer-to-peer local communication scheme. They are applied in a  manufacturing shop-floor scenario using 16 e-puck robots. A robotic interpretation of AFM is presented that maps the generic parameters of AFM to the properties of a manufacturing shop-floor. A flexible multi-robot control architecture, \textit{hybrid event-driven architecture on D-Bus}, has been outlined which uses the state-of-the-art D-Bus interprocess communication  to integrate heterogeneous software components. 

Based-on the organization of task-allocation, communication and interaction among robots, a  novel taxonomy of MRTA solutions has been proposed to remove the ambiguities found in existing MRTA solutions. Besides, a set of domain-independent metrics, e.g., plasticity, task-specialization and energy usage, has been formalized to compare the performances of the above two strategies. The presented comparisons extend our general understanding of the role of information exchange strategies to achieve the distributed task-allocations among various social groups.