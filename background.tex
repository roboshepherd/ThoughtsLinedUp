\chapter{Background and Related Work}
\label{bg}
%%%%%%%%%%%%%%%%%%%%%
\section{Definition of key issues and basic terms}
\subsection{Self-regulation}
Self-regulation (SR) primarily refers to the exercise of control over oneself to bring the self into line with preferred standards \cite{Baumeister+2007}. One of the most notable self-regulatory process is the human body's homoeostatic process where the human body's inner process seeks to return to its regular temperature when it gets overheated or chilled. Baumeister et al. has referred self-regulation to goal-directed behaviour or feedback loops, whereas self-control may be associated with conscious impulse control. Since regulation carries  the meaning of  control with a hint of regularity.  

SR is not only the regulation of the self, but also it carries the meaning of regulation by the self (consciously and/or unconsciously). In psychology, SR denotes the strenuous actions to resist temptation or to overcome anxiety.  SR is also divided into two categories: 1) conscious and 2) unconscious SR.  Conscious SR puts emphasis on conscious, deliberate efforts in self-regulation. On the other hand, unconscious self-regulation refers to the automatic self-regulatory process  that is although not nearly as labour intensive, but operate in harmony with unpredictable, unfolding events in the environment, using and transforming the available informational input in ways that help to attain an activated goal.

The concepts of SR is also commonly used in cybernetic theory where SR in inanimate mechanisms shows that they can regulate themselves by making adjustments according to programmed goals or set standards. A common example of this kind can be found in a thermostat that controls a heating  and cooling system to maintain a desired temperature in a room. In that sense, SR covers both aspects of monitoring ones own state in relation to the goal and making adjustments with respect to the changes found. 

In physics, chemistry, biology and some other branches of natural sciences, the concept of SR is centered around the study of self-organizing individuals. The term self-organization (SO) has a broader and somewhat different meaning in different places of literature. 
%!!!!
For example, Bonabeau et al. \cite{Bonabeau+1999} has summarized the ingredients and properties of  self-organization modelling after the social insects. The four ingredients of self-organization are:
positive feedback or amplification, 
negative feedback, 
reliance on amplification of fluctuations ( randomness of  tasks, error etc.) and 
reliance on multiple interactions. 
In many cases, the concepts of self-organization and self-regulation share many properties, especially when they are considered in the context of division of labor, formation of emergent structures and behaviours etc. For example,  SO is observed in ant colonies and other similar biological species. In the process of SO, we can find self-regulating properties of the insects that give rise to the spatio-temporal  structures, or nests, existence of multiple stable states and bifurcation with respect to the dynamic changes of the environment. In the process of SO, we can find self-regulating properties of the insects that give rise to the spatio-temporal structures, or nests, existence of multiple stable states and bifurcation with respect to the dynamic changes of the environment.
%% In SECTION <>, a more closer look on SO has been given to find out interesting phenomena of emergent self-regulation in nature.   

SR has also been studied in the context of social systems where it originates from the division of social labor that creates SO process that has self-regulating effects \cite{Kppers+1990}. Two types of SR have been reported in many places of literature of sociology: 1) SR from SO and 2) SR from activities of components in a heterarchical organization. It is interesting to note that self-regulation in biological species provides the similar evidences of bottom-up approach of self-regulation of heterarchical organization through interaction of individuals or the absence of strict hierarchy. In the study of sustainable or viable system, self-regulation gets much attention in organizational cybernetics. Many researchers are trying to put Stafford Beer's viable system model \cite{Beer1981} into social organizations for a bottom-up self regulation of an organization.  
%% SECTION <> focuses on self-regulation in human social systems.
It is interesting to note that self-regulation in biological species provides similar evidences of bottom-up approach of self-regulation of heterarchical organization through interaction of individuals in the absence of or in parallel with strict hierarchy. 

From the above discussion, it has been shown that the term self-regulation carries a  wide range of meaning in different branches of knowledge. In psychology and cognitive neuroscience point of view, self-regulation is discussed in an individual's perspective whereas, in biological and social contexts the SR is discussed in a context of group of individuals or the society as a whole.  From this initial study, it can be said that  to study emergent SR,  the latter context is more appropriate where emergence is evident.  For a better understanding of emergent self-regulation, we need to look into the process of emergence from the view point of interactions of large individuals in a society.  
% SECTION <> deals with this topic.  

%\subsection{Emergence}
%In order to understand emergent self-regulation, it is necessary to find out how emergence and emergent properties are described in the literature. According to Encyclopedia Britannica emergence is the rise of a system that cannot be predicted or explained from antecedent conditions. Emergence occurs when interactions among objects at one level gives rise to different types of objects at another level. More precisely, a phenomenon is emergent if it requires new categories to describe the behavior of the underlying components \cite{Andriani+2004}, \cite{Kppers+1990}. 
%
%Emergence is the central concept to the theory of complex systems which consists of a large number of elements, non-linear dynamic interactions among elements and their environment, feedback loops etc. Paul Cilliers \cite{Andriani+2004} summarized the properties of  complex systems as follows:
%Complex systems consist of large number of elements such that a system of differential equation becomes impractical.
%The elements have to interact and this interactions must be dynamic, changing with time. 
%The interaction is fairly rich so that elements are influenced by each other. These interactions are non-linear where small causes can have large results and vice versa.
%The interactions usually have a fairly short range, i.e. Information is received primarily from immediate neighbors. Long range interaction is also possible which can be modulated along the way.
%There are loops in the interactions. The effect of any activity can feed back onto itself, sometimes directly, sometimes after a number of intervening stages. The feedback can be positive (enhancing, stimulating) or negative (detracting, inhibiting).
%Complex systems are usually open systems, i.e. they interact with their environment.
%Complex systems operate under conditions far from equilibrium. There has to be constant flow of  energy to maintain the organization of the system and to ensure its survival.
%Complex systems have histories. No only do they evolve through time, but also their past is co-responsible for their present behavior.
%Each element in the system is ignorant of the behavior of the system as a whole, it responds only to the information that is available to it locally. The complexity emerges as resultant of the patterns of interaction between the elements.


%The common characteristics of emergence in complex systems \cite{Corning2002} are typically described as follows: 
%\begin{enumerate}
%\item radical novelty or features not previously observed in systems; 
%\item coherence or correlation or meaning integrated wholes that maintain themselves over some period of time; 
%\item  A global or macro "level" i.e. there is some property of "wholeness"; 
%\item it is the product of a dynamical process that evolves; and 
%\item it can be perceived.  
%\end{enumerate}

%Therefore, the concepts of emergent SR can be referred to the bottom-up or decentralized control by which large number of individuals perform robust SO through local non-linear interactions that give rise to the novel and coherent structures, patterns and properties during the SO process. 

\subsection{Division of labour or task-allocation}
Encyclopaedia Britannica serves the definition of division of labour as the separation of a work process into a number of tasks, with each task performed by a separate person or group of persons. Originated from economics and sociology the term division of labour is widely used in many branches of knowledge. As mentioned by Adam Smith (1776).
The great increase of the quantity of work which, in consequence of the division of labour, the same number of people are capable of performing, is owing to three different circumstances; first, to increase the dexterity in every particular workman; secondly, to the saving of the time which is commonly lost in passing from one species of work to another; and lastly, to the invention of a great number of machines which facilitate and abridge labour, and   enable one man to do the work of many. (Adam Smith (1776) in \cite{Sendova-Franks+1999})

In the study of biological insects, a worker usually does not perform all tasks, but rather specializes in a set of tasks, according to its morphology, age, or chance \cite{Bonabeau+1999}. This division of labour among nest-mates, whereby different activities are performed simultaneously by groups of specialized individuals, is believed to be more efficient than if tasks were performed sequentially by unspecialised individuals. Division of  labour has a great plasticity where the removal of one class of workers is quickly compensated for by other workers.

In sociology, division of labour usually denotes the work specialization \cite{Sayer+1992}. Basically it answers three questions:
\begin{description}
\item[1)What task?] i.e., the description of the tasks to be done, service to be rendered or products to be manufactured.
\item[2)Why dividing it to individuals?] i.e., the underlying social standards for this division, such as task appropriateness based on class, gender, age, skill etc.
\item[3)How to divide it?] i.e.,the method or process of separating the whole task into small pieces of tasks that can be performed easily. 
\end{description}
%%%%
\subsection{Communication} 
%%
%%%%%%%%%%%%%%%%%%%%%%%%%%%%%%%%%%%%%%%%%%%%%%%%%%%%%%%%%%%%%%%%%%%
\section{Overview and key issues of multi-robot systems (MRS)}
\label{bg:mrs}
Historically the concept of multi-robot system comes almost after the introduction of behaviour-based robotics paradigm \cite{Brooks1986,Arkin1990}. In 1967, using the traditional sense-plan-act or hierarchical approach \cite{Murphy2000}, the first Artificially Intelligent (AI) robot, Shakey, was created at the Stanford Research Institute. In late 80s, Rodney A. Brooks revolutionized this entire field of mobile robotics who outlined a layered, behaviour based approach that acted significantly differently than the hierarchical approach \cite{Brooks1986}. At the same time, Valentino Braitenberg described a set of experiments where increasingly complex vehicles are built from simple mechanical and electrical components \cite{Braitenberg1984}. Around the same time and with similar principle, Reynolds developed a distributed behavioural model for a bird in a flock that assumed that a flock is simply the result of the interactions among the individual birds \cite{Reynolds1987}. Early research on multi-robot systems also include the concept of cellular robotic system \cite{Fukuda+1987}, \cite{Beni1988} multi-robot motion planning \cite{Arai+1989,Premvuti+1990,Wang1989} and architectures for multi-robot cooperation \cite{Asama+1989}.\\
%
From the beginning of the behaviour based paradigm, the biological inspirations influenced many cooperative robotics researchers to examine the social characteristics of insects and animals and to apply them to the design multi-robot systems \cite{Arkin1998}. The underlying basic idea is to use the simple local control rules of various social species, such as ants, bees, birds etc., to the development of similar behaviours in multi-robot systems. In multi-robot literature, there are many examples that demonstrate the ability of multi-robot teams to aggregate, flock, forage, follow trails etc. \cite{Bonabeau+1999,Mataric1994}. The dynamics of ecosystem, such as cooperation, has also been applied in multi-robot systems that has presented the emergent cooperation among team members \cite{Mcfarland1994}, \cite{Martinoli+1996}. On the other hand, the study of competitive behaviours among animal and human societies has also been applied in multi-robot systems, such as that found in multi-robot soccer \cite{Asada+1999}.\\
%%%
 \subsection{MRS research paradigms}
\label{bg:mrs:paradigms}
As discussed above, there are several research groups who follow different approaches to handle multi-robot research problems. Parker \cite{Parker2008} has summarized most of the recent research approaches into three paradigms:
\begin{enumerate}
\item Bioinspired, emergent swarms paradigm,
\item Organizational and social paradigm and
\item Knowledge-based, ontological and semantic paradigm
\end{enumerate}
\subsubsection*{Bioinspired, emergent swarms paradigm}
In bio-inspired, emergent swarms paradigm local sensing and local interaction forms the basis of collective behaviors of swarms of robots. Many researchers addressed the issues of local interaction, local communication (i.e., stigmergy) and other issues of this paradigm \cite{Mataric1995}, \cite{Kube+1993}. Today, this paradigm has been emerged as a sub-field of robotics called swarm robotics \cite{Sahin+2005}. This is a powerful paradigm for those applications that require performing shared common tasks over distributed workspace, redundancy or fault-tolerance without any complex interaction of entities. Some examples include flocking, herding, searching, chaining, formations, harvesting, deployment, coverage etc. \\
%% Re arrange
Swarm robotics is a relatively new branch of robotics where a large number of collective robots are studied from the inspiration of the observation of social insects ants, termites wasps and bees \cite{Sahin+2005}. The term swarm intelligence was first coined by Gerado Beni \cite{Beni2005} in late 1980s and during recent years the term swarm robotics emerged as an application of swarm intelligence to multi-robot systems with emphasis on physical embodiment of entities and realistic interactions among the entities and between the entities and their environment. In order to distinguish swarm robotics from other branches of robotics such as collective robotics, distributed robotics, robot colonies and so forth, Sahin proposed a formal definition and a set of criteria for swarm robotics research \cite{Sahin+2005}. According to him, swarm robotics is the study of how large number of relatively simple physically embodied agents can be designed such that a desired collective behaviour emerges from the local interactions among agents and between the agents and the environment. And the notable criteria of swarm robotics research are listed as follows.
\begin{description}
\item[Autonomous robots]
that exclude the sensor networks and may include metamorphic robotic system without having no centralized planning and control element.
\item[Large number of robots,]
usually $\geq$ 10 robots, or at least having provision for scalability if the group size is below this number.
\item[Mostly homogeneous groups of robots]
that typically exclude the multi-robot soccer teams having heterogeneous robots.
\item[Relatively incapable of inefficient robots]
that is the task complexity enforces either cooperation among robots or increased performance or robustness without putting no restriction on individual robot's hardware/software complexity.
\item[Robots with local sensing and communication capabilities]
that does not use global coordination channel to coordinate among themselves, rather enforces distributed coordination.
\end{description}
%%%
%% modelling swarm: TODO: to be summarized
Modelling the swarms is a key issue in swarm robotics. This is aimed for investigating suitable models and algorithms for control and task-allocation of swarms of robots. A review of models and approaches for coordination and control of dynamic multi-agent systems by \cite{Gazi+2006} presented that a number of approaches can be used to model swarm robotic systems with the specific focus on major issues like stability, performance, robustness and scalability. Based on the relevance to our study, we have discussed them as follows.
\paragraph*{Behaviour-based approaches:}
The ease of implementation of a behaviour-based robotic system has inspired researchers to follow behaviour-based approaches for modelling swarm robotic systems using variety of specific swarm behaviours. Early research of Reynold provided the example of a behaviour-based approach for swarm coordination such as, flocking of birds \cite{Reynolds1987}. Recent studies on behaviour-based approaches include the work of \cite{Balch+1998} where they have evaluated the formation acquisition and stabilization of multi-robot systems. Several other researchers used other techniques, such as use of adaptation rules \cite{Liu+2007}, collective behaviours \cite{Cianci2007} etc., for implementing a behaviour-based swarm robotic system.
\paragraph*{Probabilistic approaches:}
Probabilistic approaches and Markov models also present attractive alternatives for modelling of swarm behaviour. They typically use the population level swarming dynamics in a non spatial way in terms of frequency distributions of groups of various size \cite{Gazi+2006}. A recent review of probabilistic approaches for swarm modelling is presented in \cite{Lerman+2005}.
\paragraph*{Asynchronous swarm model based approaches:}
Asynchronous multi-agent dynamic systems are difficult to tract for analysis and are not widely found in literature. One of the pioneer study by \cite{Beni+1996} provided sufficient conditions for the asynchronous convergence of linear swarm to a synchrously achievable configuration. Some other recent studies can be found in \cite{Gazi+2006}.
\paragraph*{Control theoretic approaches:}
Control theoretic approaches include potential field, feedback linearisation, sliding mode, and various non-linear control approaches, e.g., Fuzzy, Neural nets, Knowledge-based/Rule-based, Lyapunov analysis etc. In recent years, combined or hybrid approaches, e.g., neuro-fuzzy, are also being adopted for modelling swarm behaviours and learning parameter settings of a system \cite{Sahin+2007}. 
\paragraph*{Artificial Physics based approaches:}
Artificial physics based approaches use the fundamental laws of physics such as the Newton's laws of motion to model swarm robotic systems. In a pioneering work by \cite{Spears+1999}, this approach has been illustrated and since then many development has been taken place under this framework to address the issue of formation stabilization, surveillance, coverage of a region etc.
\paragraph*{Multi-agent based and other approaches:}
Since the field of swarm intelligence and swarm robotics is expanding contentiously many researchers are putting efforts to bring newer swarm models based on multi-agent based other techniques which are not reviewed here explicitly.
%%%
\subsubsection*{Organizational and social paradigm}
Organizational and social paradigms are typically based on organization theory derived from human systems that reflects the knowledge from sociology, economics, psychology and other related fields. To solve complex problems this paradigm usually follows the cooperative and collaborative forms of distributed intelligence. In multi-robot systems the example of this paradigm is found in two major formats: 
the use of roles and value system and
market economics.
In multi-robot applications under this paradigm, an easy division of labor is achieved by assigning roles depending on the skills and capabilities in individual team member. For example, in multi-robot soccer \cite{Stone+1999,Asada+1999} positions played by different robots are usually considered as defined roles. On the contrary, in market economics approach \cite{Gerkey+2002,Dias+2006} task allocation among multiple robots are done via market economics theory that enables the selection of robots for specific tasks according to their individual capabilities determined by a bidding process.
\subsubsection*{Knowledge-based, ontological and semantic paradigm}
The third paradigm, commonly used for developing multi-agent systems, is knowledge-based, ontological and semantic paradigm. Here knowledge is defined as ontology and shared among robots/agents from disparate sources. It reduces the communication overhead by utilizing the shared vocabulary and semantics. Due to low bandwidth, limited power, limited computation and noise and uncertainty in sensing/actuation, the use of this approach is usually restricted in multi-robot systems. \\
Although this approximate classification includes most of the research directions it is very hard to specifically categorize all diverse researches on multi-robot systems. However, most of the researchers select a suitable paradigm to abstract the problem from an specific perspective with a fundamental challenge to determine how best to achieve global coherence from the interactions of entities at the local level. \\
%
Whatever the principle characteristics of a MRS, e.g., homogeneity, coupling, communication methods etc., each MRS must address some basic problems to some degree. For example, usually every MRS adopts a control architecture under a specific paradigm. Similarly every MRS address the issues of communication, localization, interaction in a way specific to the application and underlying design principles (or philosophies). In the following subsections, we have attempted to summarize the key MRS research issues that would influence the selection and implementation our research. In this initiative we have deliberately omitted the non-central or very specific issues, such as collaborative transport or reconfigurable MRS, that does not directly relate to our research.
%%%
\subsection{Architecture and control}
\label{bg:mrs:arch}
In MRS, two high-level control strategies are very common: 1) centralized and 2) decentralized or distributed. Under a specific control strategy, traditionally three basic system architectures are widely adopted: deliberative, reactive and hybrid. Deliberative systems based on central planning are well suited for the centralized control approach. The single controller makes a plan from its Sense-Plan-Act (SPA) loop by gathering the sensory information and each robot performs its part. Reactive systems are widely used in distributed control where each robot executes its own controller maintaining a tight coupling between the system's sensors and actuators, usually through a set of well-designed behaviours. Here, various group behaviour emerges from the interactions of individuals that communicate and cooperate when needed. Hybrid systems are usually the mixture of the two above approaches; where each robot can run its own hybrid controller with the help of a plan with necessary information from all other robots. \cite{Mataric2007} described behaviour-based control architecture as a separate category of distributed control architecture where each robot behaves according to a behaviour-based controller and can learn, adapt and contribute to improve and optimize the group-level behaviour. 
Although most of the MRS control architectures share some common characteristics (such as distributed and behaviour-based control strategy) based on their difference of underlying design principles we have put them into three groups:
\begin{enumerate}
\item Behaviour-based classical architectures
\item Market-based architectures
\item Multi-agent based architectures
\end{enumerate}
Due to the overwhelming amount of literature on MRS architectures it is not possible to include most of them. However, below some representative key architectures strictly designed for MRS are described. 
\subsubsection*{Behaviour-based classical architectures}
The ALLIANCE architecture \cite{Parker1998} is one of the earliest behaviour-based fully distributed architectures. This architecture has used the mathematically modelled behaviour sets and motivational system. The primary mechanism for task selection of a robot is to activate the motivational behaviour partly based on the estimates of other robots behaviour. This architecture was designed for heterogeneous teams of robots performing loosely coupled tasks with fault-tolerance and co-operative control strategy. Broadcast of local eligibility (BLE) \cite{Werger2001} is another behaviour-based architecture that uses port-attributed behaviour technique through broadcast communication method. It was demonstrated to perform coordinated tasks, such as multi-target observation tasks. Major differences between this two behaviour-based systems include the need in ALLIANCE for motivational behaviours to store information about other individual robots, the lack of uniform inter-behaviour communication, and ALLIANCE's monitoring of time other robots have spent performing behaviours rather than BLE's local eligibility estimates. Similar to the above two architectures, many other researchers proposed and implemented many variants of behaviour-based architectures. Some of them used the classic three layer (plan-sequence-execute) approach, e.g., \cite{Simmons+2002} used a Layered Architecture where each layer interact directly to coordinate actions at multiple levels of abstraction. 
\subsubsection*{Market-based architectures}
Using the theory of marker economics and well-known Contract Net Protocol (CNP) \cite{Davis1988+}, these architectures solve the task-allocation problem by auction or bidding process. Major architectures following market-based approaches include MURDOCH \cite{Gerkey+2002}, M+ system \cite{Botelho+1999}, first-piece auction \cite{Zlot+2002}, dynamic role assignments \cite{Chaimowicz2002} among others. 
\subsubsection*{Multi-agent based architectures}
Some MRS architectures are influenced by multi-agent systems (MAS). For example, CHARON is a hierarchical behaviour-based architecture that rely on the notion of agents and modes. Similarly CAMPOUT is another distributed behaviour-based architecture that provide high-level functionality by making use of basic low-level behaviours in downward task decomposition of a multi-agent planner. It is comprised of five different architectural mechanisms including, behaviour representation, behaviour composition, behaviour coordination, group coordination and communication behaviours.
%%%%%
\subsection{Interaction and learning}
\label{bg:mrs:learn}
\subsubsection*{Interaction}
According to the Oxford Dictionary of English the term interaction means reciprocal action or influence. In MRS research, such as in \cite{Mataric1994}, interaction is referred to as mutual influence on behaviour. Following this definition, it is obvious that objects in the world do not interact with agents, although they may affect on their behaviour. The presence of an object affects the agent, but the agent does not affects the object since objects, by definition, do not behave, only agents do. However many other researchers acknowledge that interactions of robots with their environment (as found in stigmergic communication) have a great impact on their behaviours. Therefore, we adopt the broad meaning of interaction that is reciprocal action or influence among robots and their environment.
From the above review of MRS system architecture, task allocation and communication, it is obvious that interaction among robots and their environment is the core of the dynamics of MRS.
Without this interaction, it can not be a functioning MRS. 
% Insert Fig
While analysing the role and application of distributed intelligence on MRS, Parker \cite{Parker2008} presented an excellent classification of interactions of entities of MRS. She viewed the interactions along three different axes:
\begin{enumerate}
\item the types of goals of entities (either shared goal such as, cleaning a floor, or, individual goal)
\item whether entities have awareness of others on the team (either aware such as, in cooperative transport, or, unaware such as, in a typical foraging)
\item whether the action of one entity advances the goal of others (e.g., one robot's floor cleaning helps other robots not to clean that part of the floor)
\end{enumerate}
Based on this approximate observation Parker classified interactions into four categories:
% see Fig
\paragraph{Collective interaction:}
Entities are not aware of others on the team, yet do share goals and their actions are beneficial to team-mates. Mostly, swarm-robotic work of many researchers follow this kind of interaction to perform biologically-relevant tasks, such as foraging, swarming, formation keeping and so forth.
\paragraph{Cooperative interaction:}
Entities are aware of others on the team, they share goals and their actions are beneficial to their team-mates. This type of interaction is used to reason about team-mates capabilities multiple robots works together, usually in shared workspace, such as cleaning a work-site, pushing a box, performing search and rescue, extra-planetary exploration and so forth. 
\paragraph{Collaborative interaction:}
Having individual goals (and even individual capabilities), entities aware of their team-mates and their actions are beneficial to their team-mates. One example of this kind of interaction is a team of collaborative robots where each must reach a unique goal position by sharing sensory capabilities to all members such as illustrated as coalition formation in \cite{Parker+2006}.
\paragraph{Coordinative interaction:}
Entities are aware of each other, but they do not share a common goal and their actions are not helpful to other team members. For example, in a common workspace robots try to minimize interference by coordinating their actions as found in multi-robot path planning techniques, traffic control techniques and so on.
Beyond this four most common types of interactions Parker also described another kind of interaction in adversarial domain where entities effectively work each other such as multi-robot soccer. Here entities have individual goals, they are aware of each other, but their actions have a negative affect on others goal.
%%
\subsubsection*{Learning}
A great deal of research on multi-robot learning has been carried out since the inception of MRS \cite{Mataric+2001,Yang+2004,Parker1995}. Learning, identified as the ability to acquire new knowledge or skills and improve one's performance, is useful in MRS due to the necessity of robots to know about itself, its environment and other team-members \cite{Mataric2007}. Learning can improve performance since robot controllers are not perfect by design and robots are required to work in an uncertain environment that all possible states or actions can not be predicted in advance. Besides learning a new skill or piece of knowledge it is also important to forget learned things that are no longer needed or correct as well as, to make room for new things to be learned and stored in a finite memory space of a robot. \\
Several learning techniques are available in robotics domain, such as reinforce or unsupervised learning, supervised learning and learning by imitation \cite{Mataric2007}. Although reinforce learning, or learning based on environmental or peer feedback, is a good option for MRS, it has been found that in large teams the ability to lean in this way is restricted due to large continuous state and action space \cite{Yang+2004}. Several other learning techniques are also available to explore in MRS domain including Markov models, Q-learning, fuzzy logic, neural nets, game theory, probabilistic or Bayesian theory among others. \\
%% SRS
Based on a specific model of swarm behaviours researchers generally adopt similar communication methods to enable interaction of swarms as discussed in Section \ref{subsec:mrs-comm}. In \cite{Balch2005} three kinds of communication including, indirect stigmergic communication, direct robot to robot state communication, and goal communication were performed and it was found that in some tasks communication provided performance improvements while others did not. Since then, researchers emphasize on both the necessity and cost of communication in a swarm robotic system. Indirect communication approaches, e.g. virtual pheromone \cite{Payton+2005,Hamann+2006} by which mobile robots communicated through directional infrared messaging or LEDs, are mainly tried for large teams with the spatially distributed applications such as search and rescue, de-mining etc. More recently, \cite{Cianci2007} reported a IEEE 802.15.4-compatible radio-communication module in e-puck robot for achieving multiple interactions simultaneously, as demonstrated in their collective decision making scenario. 
Although research on learning in swarm robotic teams was not explored widely, \cite{Balch2005} presented an example of reinforcement-based learning in multi-robot soccer and foraging tasks. He concluded that in a team of homogeneous robots with diverse behaviours, communication, interaction and learning are well interconnected and depending on the selection of global or local learning means, learning can be effectively employed in a swarm robotic system.
%
\subsubsection*{Conflict resolution}
In MRS, conflicts occur if a resource is required by or, a unique single task is distributed to, more than one robot at any given time. Several resources such as bandwidth, space etc. may be needed by more than one robot. The space sharing problem was treated as traffic control problem in urban areas, but the robots are never restricted in road networks in case of behaviour-based control application \cite{Cao+1997}. In explicit communication mode in MRS, the sharing of bandwidth among robots is a great problem in case of applications like multi-robot mapping \cite{Konolige+2003}. In large multi-robot team such as in Centibots system \cite{Ortiz+2005}, task interference and high bandwidth communication between 100s of robots appear as a significant research challenge.

\subsection{Localization and exploration}
\label{bg:mrs:loc}
Mobile robot systems highly rely on precise localization for performing their autonomous activities in indoor or outdoor. Localization is the determination of exact pose (position and orientation) with respect to some relative or absolute coordinate system. This can be done by using proprioceptive sensors that monitor motion of a robot or exteroceptive sensors that provide information of world representation, such as global positioning system (GPS) or indoor navigation system (INS). Many other methods are also available, such as landmark recognition, cooperative positioning and other visual methods.\\
Localization issue of MRS also invites researchers to examine specific areas like exploration and map generation. In exploration problem, robots need to minimize the time needed to explore the given area. Many researchers uses various kinds of exploration algorithms for solving this NP-hard problem, such as line-of-sight constrained exploration algorithm \cite{Arkin+2002}, collaborative multi-robot exploration \cite{Burgard+2000} and so on. In mapping problem, mostly inaccurate localization information from teams of robots are accumulated and combined to generate a map by various techniques, such as probabilistic approaches \cite{Thurn+2000}.
%% SRS
Similar to in a MRS, localization is one of the hardest problem in swarm robotics. Without the presence of any centralized localization module, such as GPS or INS, it is not easy to localize precisely and locally the position of a robot with respect to other robots or environment. \cite{Spears+2006} presented a novel technique based on trilateration for localization of swarm robots using ultrasonic and RF transceivers without relying on global information from GPS, beacons, landmarks or maps. This system localize a robot with respect to other nearby robots and this is done using ultrasonic and RF signals. \cite{Schmickl+2006} reported hop-count and bio-inspired strategies for collective perception or how a swarm robot can join multiple instances of individual perception to get a global picture. Distributed mapping is another important application using swarms. \cite{Rothermich+05} presented a collaborative localization algorithm using landmark based localization technique.

\subsection{Applications of MRS}
\label{bg:app}
MRS systems have been put to numerous application domains that all can not be listed together. Rather than listing all of areas explored by researchers, below we have included few major areas that have received highest attention in the MRS research community. Sahin also listed a set of promising applications for swarm robots including spatially distributed tasks (e.g., environment monitoring), dangerous tasks (e.g., robotic de-miner), tasks that scale-up or scale-down over time, and tasks that require redundancy \cite{Sahin+2005}.
\subsubsection*{Object Transport}
Cooperative transport of large objects (that one robot is unable to handle) by multi-robots was investigated by many researchers such as, following a formal model of cooperative transport in ants \cite{Kube+1993}, box-pushing by six-legged robots \cite{Mataric+1995}. Another kind of object transport problem include clustering objects into piles e.g., \cite{Beckers+1994}, collecting waste or trash e.g., \cite{Parker1994}, sorting coloured objects e.g., \cite{Melhuish+1998}, constructing a building site collectively \cite{Wawerla+2002} and so on. 
\subsubsection*{Mining} 
It has also been observed that multi-robot teams as micro or mini machines are helpful to improve the control and efficiency of mining and its processing operations \cite{Dunbar+2002}.
\subsubsection*{Military and Space Applications}
Many researchers address MRS research issues under the requirements of a military or space application. Behaviour-based formation control \cite{Balch+1998}, landmine detection \cite{Franklin+1995}, multiple planetary rovers for various missions \cite{Huntsberger2004} and so forth, all are the examples of this areas.
%%%%%%%%%%%%%%%%%%%%%%%%%%%%%%%%%%%%%%%%%%%%%%%%%%%%%%%%%%%%%%%%%%%
\section{Task-allocation in MRS}
Since 90s multi-robot task allocation (MRTA) is a common research challenge that tries to define the preferred mapping of robots to tasks in order to optimize some objective functions  \cite{Gerkey+2004}. Many MRS control architectures have been solely designed to address this task-allocation issue. In 2003 Gerkey et al. formally analysed the complexity and optimality of key architectures (e.g., ALLIANCE, BLE, M+, MURDOCH, First piece auctions and Dynamic role assignment) for this MRTA issue and it has been found that MRTA is an instance of the so-called optimal assignment problem \cite{Gerkey+2003} and generally known as NP-hard where optimal solutions can not be found quickly for large problems \cite{Gerkey+2004}. 
If we look the MRTA problem from multi-agent system's perspective we can find it is broadly divided into two major categories \cite{Shen+2001}: 
\begin{enumerate}
\item Predefined (off-line) task-allocation and 
\item Emergent (real-time) task-allocation. 
\end{enumerate}
% (Insert a Fig)
Usually predefined task allocation method uses either centralized coordination or distributed task-allocation approach. Distributed predefined task-allocation approach is again subdivided into three subcategories: 
\begin{enumerate}
\item Direct allocation, 
\item Task allocation by delegation 
\item Task allocation through bidding
\end{enumerate}
In MRS domain, early research on predefined distributed task-allocation approach has been dominated mainly by intentional coordination \cite{Gerkey+2004,Parker1998}, the use of dynamic role assignment e.g., \cite{Chaimowicz2002}, and market-based bidding approach \cite{Dias+2006}. In intentional coordination e.g., \cite{Parker1998}, robots uses direct allocation method to communicate and to negotiate for assigning tasks. This is preferred approach among MRS research community since it is easily understood, easier to design, implement and analysis formally. Task allocation through bidding is mainly based on the Contract Net Protocol \cite{Davis1988+}. Predefined Task allocation through other approaches are also present in literature. For example, inspired by the vacancy chain phenomena in nature, \cite{Dahl+2003} proposed a vacancy chain scheduling (VCS) algorithm for a restricted class of MRTA problems in spatially classifiable domains.

On the other hand emergent task-allocation approach relies on the emergent group behaviours e.g., \cite{Kube+1993}, such as emergent cooperation \cite{Lerman+2006}, adaptation rules \cite{Liu+2007} etc., that lead to task allocation with local sensing, local interactions. It typically uses little or no explicit communication or negotiations between robots. They are more scalable to large team size and more robust via parallelism and redundancy.

MRTA problem can be addressed in many different ways depending upon the paradigm selected to abstract the problem and its relevant constraints and requirements \cite{Parker2008}. Firstly, in emergent task allocation in bioinspired swarms paradigm MRTA homogeneous robots are employed to perform mostly similar tasks only by local sensing and indirect stigmergic ( or no) communication. Secondly, in organizational and social paradigm MRTA can follow one of the two major approaches: 1) task allocation by making use of roles and 2) task allocation through bidding. For example, in multi-robot soccer, each role encompasses several specific tasks and heterogeneous robots select their roles based on their position and capabilities. In market-based approach, robots can negotiate with other team-mates to collectively solve a set of tasks. Finally, in knowledge-based approach, also known as intentional coordination, MRTA is done through the modelling of team-mate capabilities, such as by observing the performance of other team-members performance with or without explicit communication.
%%%%%%%%%%%%%%%%%%%%%
\section{Communication in biological social systems}
\label{bg:comm-biology}
Communication plays a central role in self-regulated division of labour of biological societies.In this section communication among biological social insects are briefly reviewed within the context of self-regulated  division of labour.

%[Purposes]
Communication in biological societies serves many closely related social purposes. Most peer-to-peer (P2P) communication include: recruitment to a new food source or nest site, exchange of food particles, recognition of individuals, simple attraction, grooming, sexual communication etc. In addition to that colony-level broadcast communication include: alarm signal, territorial and home range signals and nest markers, communication for achieving certain group effect such as, facilitating or inhibiting  a group activity \cite{Holldobler1990}.
%
%[Modalities and Ranges]
Biological social insects use different modalities to establish social communication, such as, sound, vision, chemical, tactile,  electric and so forth.  Sound waves can travel a long distance and thus they are suitable for advertising signals. They are also best for transmitting complicated information quickly \cite{Slater1986}. Visual signals can travel more rapidly than sound but they are limited by the physical size or line of sight of an animal. They also do not travel around obstacles. Thus they are suitable for short-distance private signals such as in courtship display. 
In ants and some other social insects chemical communication is dominant. Any kind of chemical substance that is used for communication between intra-species or inter-species is termed as semiochemical \cite{Holldobler1990}. A pheromone is a semiochemical, usually a glandular secretion, used for communication within species. One individual releases it as a signal and others responds it after tasting or smelling it. Using pheromones individuals can code quite complicated messages in smells. For example a typical an ant colony operates with somewhere between 10 and 20 kinds of signals \cite{Holldobler1990}. Most of these are chemical in nature. If wind and other conditions are favourable,  this type of signals emitted by such a tiny species can be detected from several kilometres away. Thus chemical signals are extremely economical of their production and transmission. But they are quite slow to diffuse away. But ants and other social insects manage to create sequential and compound messages either by a graded reaction of different concentrations of same substance or by blends of signals.
Tactile communication is also widely observed in ants and other species typically by using their body antennae and forelegs. It is observed that in ants touch is primarily used  for receiving information rather than informing something. It is usually found as an invitation behaviour in worker recruitment process. When an ant intends to recruit a nestmate for foraging or other tasks it runs upto a nestmate and beats her body very lightly with  antennae and forelegs. The recruiter then runs to a recently laid pheromone trail or lays a new one. In this form of communication limited amount of information is exchanged. 
In underwater environment some fishes and other species also communicate through electric signals where there nerves and muscles work as batteries. They use continuous or intermittent pulses with  different frequencies learn about environment and to convey their identity and aggression messages.
%
%[Signal Active Space]
The concept of active space is widely used to describe the propagation of signals by species. In a network environment of signal emitters and receivers, active space is defined as the area encompassed by the signal  during the course of transmission \cite{Mcgregor2000}. In case of long-range signals, or even in case of short-range signals, this area include several individuals where their social grouping allows them to stay in cohesion. The concept of active space is described somewhat differently in case some social insects. In case of ants, this active space is defined as a zone within which the concentration of pheromone (or any other behaviourally active chemical substances) is  at or above threshold concentration \cite{Holldobler1990}. Mathematically this is denoted by a ratio:
The amount of pheromone emitted (Q)
The threshold concentration at which the receiving animal responds (K)
Q is measured in number of molecules released in a burst or in per unit of time whereas K is measured in molecules per unit of volume.
The adjustment of this ratio enables individuals to gain a shorter fade-out time and permits signals to be more sharply pinpointed in time and space by the receivers. In order to transmit the location of the animal in the signal, the rate of information transfer can be increased by either by lowering the rate of emission of Q or by increasing K, or both.  For alarm and trail systems a lower value of this ratio is used. Thus, according to need, individuals regulate their active space by making it large or small, or by reaching their maximum radius quickly or slowly, or by enduring briefly or for a long period of time. For example, in case of alarm, recruitment or sexual communication signals where encoding the location of an individual is needed, the information in each signal increases as the logarithm of the square of distance over which the signal travels. From the precise study of pheromones it has been found that active space of alarm signal is consists of  a concentric pair of hemispheres. (FIG). As the ant enters the outer zone she is attracted inward toward the point source; when she next crosses into the central hemisphere she become alarmed. It is also observed that ants can release pheromones with different active spaces.

Active space has strong role in modulating the behaviours of ants. For example, when workers of {\em Acanthomyops claviger} ants produce alarm signal due to an attack by a rival or insect predator, workers sitting a few millimetres away begin to react within seconds. However those ants sitting a few centimetres away take a minute or longer to react. In many cases ants and other social insects exhibit modulatory communication within their active space where many individuals involve in many different tasks. For example, while retrieving the large prey, workers of {\em Aphaeonogerter} ants produce chirping sounds (known as stridulate) along with releasing poison gland pheromones. These sounds attract more workers and keep them within the vicinity of the dead prey  to protect it from their competitors. This communication amplification behaviour can increase the active space to a maximum distance of 2 meters.
%%%%%%%%%%%%%%%%%%%%%
\section{Communication in MRS}
Communication between robots is an  important issue in MRS \cite{Arkin1998}. This is not a prerequisite for the group to be functioning, but often useful component of MRS \cite{Mataric2007}. Let us now investigate why communication is important, how this is usually archived in MRS and related other issues.

Researchers generally agree that communication in MRS usually provides several major benefits, such as:

\begin{description}
\item[Exchange of information and improving perception:]
Robots  can exchange potential information (as discussed below) based on their spatial position and knowledge of past events. This, in turn, leads to improve perception over a distributed region without directly sensing it.
\item[Synchronization of actions:]
In order to perform (or stop performing) certain tasks simultaneously or in a particular order robots need to communicate, or signal, to each other. 
\item[Enabling interactions:]
Communication is not strictly necessary for coordinating team actions. But  communication can help a lot to interact (and hence influence) each-other in a team that, in turn, enables robots to coordinate and negotiate their actions.
\end{description}

Since a MRS  can be comprised of robots of various computation and communication capabilities, it is also necessary to define the communication content and range \cite{Arkin1998,Mataric2007}. Usually robots can communicate about various states (e.g., task-related, individual, environmental etc.), their individual intentions and goals. 
 
%[How]
Robots communicate in a number of ways available under a specific application. This communication methods can be divided into two major categories:

\paragraph{Explicit communication:}
This is also known as intentional communication. This is done purposefully and usually using wireless radio. Based on the number of recipients of message, the communication process is termed differently. Such as,
Broadcast communication: where all other robots receive the message.
Peer-to-peer communication: where only a single robot receive the message.
Publish-subscribe communication: where only a selected (previously subscribed) number of robots receive the message.
Because explicit communication is costly in terms of both hardware and software, robotic researchers always put extra attention to design such a system by analysing strict requirements such as communication necessity, range, content, reliability of communication channel (loose of message) etc.

\paragraph{Implicit Communication:} 
This is also known as indirect stigmergic communication. This is a powerful way of communication where individuals leave information in the environment. This method was adopted from the social insect behaviour, such as stigmergy of ants (leaving of small amount of pheromone or chemicals behind while moving in a trail). Some researchers also tried to establish communication among robots through vision \cite{Kuniyoshi1994}.

% Issues
In multi-robot communication researchers have identified several issues. Some of the major issues are discussed here.
Kin Recognition
Kin recognition refers to the ability of a robot to recognize immediate family members by implicit or explicit communication or sensing. In case of MRS, this can be as simple as identifying other robots from objects and environment or as finding team-mates in a robotic soccer. This is an useful ability that helps interaction, such as cooperation among team members. 

\subsubsection*{Representation of Languages}
In case of effective communication several researchers also focused on representation of languages and grounding of these languages in physical world.

\subsubsection*{Fault-tolerance, Reliability and Adaptation}  
Since every communication channel is not free from noise and corruption of messages significant attention has been also given to manage these no communication situations, such as by setting up and maintaining communication network, managing reliability and adaptation rules when there is no communication link available. In terms of guaranteeing communication, researchers also tried to find ways for a deadlock free communication methods \cite{Arkin1998}, such as signboard communication method \cite{Wang1989}.

%%%%%%%%%%%%%%%%%%%%%